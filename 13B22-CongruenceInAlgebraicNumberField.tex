\documentclass[12pt]{article}
\usepackage{pmmeta}
\pmcanonicalname{CongruenceInAlgebraicNumberField}
\pmcreated{2013-03-22 18:17:11}
\pmmodified{2013-03-22 18:17:11}
\pmowner{pahio}{2872}
\pmmodifier{pahio}{2872}
\pmtitle{congruence in algebraic number field}
\pmrecord{8}{40896}
\pmprivacy{1}
\pmauthor{pahio}{2872}
\pmtype{Theorem}
\pmcomment{trigger rebuild}
\pmclassification{msc}{13B22}
\pmsynonym{congruence in number field}{CongruenceInAlgebraicNumberField}
%\pmkeywords{congruence}
\pmrelated{CongruenceRelationOnAnAlgebraicSystem}
\pmrelated{ChineseRemainderTheoremInTermsOfDivisorTheory}
\pmrelated{Congruences}
\pmdefines{residue class}

% this is the default PlanetMath preamble.  as your knowledge
% of TeX increases, you will probably want to edit this, but
% it should be fine as is for beginners.

% almost certainly you want these
\usepackage{amssymb}
\usepackage{amsmath}
\usepackage{amsfonts}

% used for TeXing text within eps files
%\usepackage{psfrag}
% need this for including graphics (\includegraphics)
%\usepackage{graphicx}
% for neatly defining theorems and propositions
 \usepackage{amsthm}
% making logically defined graphics
%%%\usepackage{xypic}

% there are many more packages, add them here as you need them

% define commands here

\theoremstyle{definition}
\newtheorem*{thmplain}{Theorem}

\begin{document}
\PMlinkescapeword{integer} \PMlinkescapeword{integers}
\textbf{Definition.}\, Let $\alpha$, $\beta$ and $\kappa$ be \PMlinkname{integers}{AlgebraicInteger} of an algebraic number field $K$ and\, $\kappa \neq 0$.\, One defines
\begin{align}
\alpha \equiv \beta \pmod{\kappa}
\end{align}
if and only if\, $\kappa \mid \alpha\!-\!\beta$,\, i.e. iff there is an integer $\lambda$ of $K$ with\, 
$\alpha\!-\!\beta = \lambda\kappa$.\\

\textbf{Theorem.}\, The congruence ``$\equiv$'' modulo $\kappa$ defined above is an equivalence relation in the maximal order of $K$.\, There are only a finite amount of the equivalence classes, the {\em residue classes modulo $\kappa$}.\\

{\em Proof.}\, For justifying the transitivity of ``$\equiv$'', suppose (1) and\, $\beta \equiv \gamma \pmod{\kappa}$; then there are the integers $\lambda$ and $\mu$ of $K$ such that\, $\alpha\!-\!\beta = \lambda\kappa$,\, 
$\beta\!-\!\gamma = \mu\kappa$.\, Adding these equations we see that\, $\alpha\!-\!\gamma = (\lambda\!+\!\mu)\kappa$\, with the integer $\lambda\!+\!\mu$ of $K$.\, Accordingly, \,$\alpha \equiv \gamma \pmod{\kappa}$.\\
Let $\omega$ be an arbitrary integer of $K$ and\, $\{\omega_1,\,\omega_2,\,\ldots,\,\omega_n\}$\, a minimal basis of the field.\, Then we can write
$$\omega = a_1\omega_1+a_2\omega_2+\ldots+a_n\omega_n,$$
where the $a_i$'s are rational integers.\, For\, $i = 1,\,2,\,\ldots,\,n$, the division algorithm determines the rational integers $q_i$ and $r_i$ with
$$a_i = \mbox{N}(\kappa)q_i+r_i, \quad 0 \leqq r_i < |\mbox{N}(\kappa)|,$$
whence
$$\omega = \mbox{N}(\kappa)(\underbrace{q_1\omega_1+q_2\omega_2+\ldots+q_n\omega_n}_{=\,\pi})
+(\underbrace{r_1\omega_1+r_2\omega_2+\ldots+r_n\omega_n}_{=\,\varrho}).$$
So we have
\begin{align}
\omega = \mbox{N}(\kappa)\pi\!+\!\varrho,
\end{align}
where $\pi$ and $\varrho$ are some integers of the field.\, If\, $\kappa^{(1)},\,\kappa^{(2)},\,\ldots,\,\kappa^{(n)}$\, are the algebraic conjugates of\, $\kappa = \kappa^{(1)}$,\, then 
$$\mbox{N}(\kappa) = \underbrace{\kappa^{(1)}}_{\mbox{integer}}\underbrace{\kappa^{(2)}\cdots\kappa^{(n)}}_{\mbox{integer}} = \kappa\kappa' \in \mathbb{Z}.$$
Hence, $\kappa$ divides $\mbox{N}(\kappa)$ in the ring of integers of $K$, and (2) implies
$$\omega \equiv \varrho \pmod\kappa.$$
Since any number $r_i$ has $|\mbox{N}(\kappa)|$ different possible values $0,\,1,\,\ldots,\,|\mbox{N}(\kappa)|\!-\!1$, there exist $|\mbox{N}(\kappa)|^n$ different ordered tuplets \,$(r_1,\,r_2,\,\ldots,\,r_n)$.\, Therefore there exist at most 
$|\mbox{N}(\kappa)|^n$ different residues and residue classes in the ring.




%%%%%
%%%%%
\end{document}
