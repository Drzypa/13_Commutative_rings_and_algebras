\documentclass[12pt]{article}
\usepackage{pmmeta}
\pmcanonicalname{AFiniteRingIsCyclicIfAndOnlyItsOrderAndCharacteristicAreEqual}
\pmcreated{2013-03-22 13:30:30}
\pmmodified{2013-03-22 13:30:30}
\pmowner{mathcam}{2727}
\pmmodifier{mathcam}{2727}
\pmtitle{a finite ring is cyclic if and only its order and characteristic are equal}
\pmrecord{12}{34090}
\pmprivacy{1}
\pmauthor{mathcam}{2727}
\pmtype{Theorem}
\pmcomment{trigger rebuild}
\pmclassification{msc}{13A99}

% this is the default PlanetMath preamble.  as your knowledge
% of TeX increases, you will probably want to edit this, but
% it should be fine as is for beginners.

% almost certainly you want these
\usepackage{amssymb}
\usepackage{amsthm}
\usepackage{amsmath}
\usepackage{amsfonts}

% used for TeXing text within eps files
%\usepackage{psfrag}
% need this for including graphics (\includegraphics)
%\usepackage{graphicx}
% for neatly defining theorems and propositions
%\usepackage{amsthm}
% making logically defined graphics
%%%\usepackage{xypic}

% there are many more packages, add them here as you need them

% define commands here
\begin{document}
{\bf \PMlinkescapetext{Lemma}.}  A finite ring is cyclic if and only if its \PMlinkname{order}{OrderRing} and characteristic are equal.  

\begin{proof}
If $R$ is a cyclic ring and $r$ is a \PMlinkname{generator}{Generator} of the additive group of $R$, then $|r|=|R|$.  Since, for every $s \in R$, $|s|$ divides $|R|$, then it follows that $\operatorname{char}~R=|R|$.  Conversely, if $R$ is a finite ring such that $\operatorname{char}~R=|R|$, then the exponent of the additive group of $R$ is also equal to $|R|$.  Thus, there exists $t \in R$ such that $|t|=|R|$.  Since $\langle t \rangle$ is a subgroup of the additive group of $R$ and $|\langle t \rangle |=|t|=|R|$, it follows that $R$ is a cyclic ring.\end{proof}
%%%%%
%%%%%
\end{document}
