\documentclass[12pt]{article}
\usepackage{pmmeta}
\pmcanonicalname{PropertiesOfAGcdDomain}
\pmcreated{2013-03-22 18:18:44}
\pmmodified{2013-03-22 18:18:44}
\pmowner{CWoo}{3771}
\pmmodifier{CWoo}{3771}
\pmtitle{properties of a gcd domain}
\pmrecord{9}{40936}
\pmprivacy{1}
\pmauthor{CWoo}{3771}
\pmtype{Result}
\pmcomment{trigger rebuild}
\pmclassification{msc}{13G05}

\usepackage{amssymb,amscd}
\usepackage{amsmath}
\usepackage{amsfonts}
\usepackage{mathrsfs}

% used for TeXing text within eps files
%\usepackage{psfrag}
% need this for including graphics (\includegraphics)
%\usepackage{graphicx}
% for neatly defining theorems and propositions
\usepackage{amsthm}
% making logically defined graphics
%%\usepackage{xypic}
\usepackage{pst-plot}

% define commands here
\newcommand*{\abs}[1]{\left\lvert #1\right\rvert}
\newtheorem{prop}{Proposition}
\newtheorem{thm}{Theorem}
\newtheorem{ex}{Example}
\newcommand{\real}{\mathbb{R}}
\newcommand{\pdiff}[2]{\frac{\partial #1}{\partial #2}}
\newcommand{\mpdiff}[3]{\frac{\partial^#1 #2}{\partial #3^#1}}
\newcommand{\GCD}{\operatorname{GCD}}
\begin{document}
Let $D$ be a gcd domain.  For any $a\in D$, denote $[a]$ the set of all elements in $D$ that are associates of $a$, $\GCD(a,b)$ the set of all gcd's of elements $a$ and $b$ in $D$, and any $S\subseteq D$, $mS:=\lbrace ms\mid s\in S\rbrace$.  Then
\begin{enumerate}
\item $\GCD(a,b)=[a]$ iff $a\mid b$.
\item $m\GCD(a,b)= \GCD(ma,mb)$.
\item If $\GCD(ab,c)=[1]$, then $\GCD(a,c)=[1]$
\item If $\GCD(a,b)=[1]$ and $\GCD(a,c)=[1]$, then $\GCD(a,bc)=[1]$.
\item If $\GCD(a,b)=[1]$ and $a\mid bc$, then $a\mid c$.
\end{enumerate}
\begin{proof}  To aid in the proof of these properties, let us denote, for $a\in D$ and $S\subseteq D$, $a|S$ to mean that every element of $S$ is divisible by $a$, and $S|a$ to mean that every element in $S$ divides $a$.
We take the following four steps:
\begin{enumerate}
\item One direction is obvious from the definition.  So now suppose $a\mid b$.  Then $a\mid\GCD(a,b)$.  But by
definition, $\GCD(a,b)\mid a$, so $[a]=\GCD(a,b)$.
\item Pick $d \in \GCD(a,b)$ and $x\in \GCD(ma,mb)$.  We want to show that $md$ and $x$ are associates.  By assumption, $d\mid a$ and $d\mid b$, so $md\mid ma$ and $md\mid mb$, which implies that $md\mid x$.  Write $x=mn$ for some $n\in D$.  Then $mn\mid ma$ and $mn\mid mb$ imply that $n\mid a$ and $n\mid b$, and therefore $n\mid d$ since $d$ is a gcd of $a$ and $b$.  As a result, $mn\mid md$, or $x\mid md$, showing that $x$ and $md$ are associates.  As a result, the map $f: m\GCD(a,b)\to \GCD(ma,mb)$ given by $f(d)=md$ is a bijection.
\item If $d\mid a$ and $d\mid c$, then $d\mid ab$ and $d\mid c$.  So $d\mid\GCD(ab,c)=[1]$, hence $d$ is a unit and
the result follows.
\item Suppose $d\mid a$ and $d\mid bc$.  Then $d\mid ab$ and $d\mid bc$ and hence $d\mid\GCD(ab,bc)=b\GCD(a,c)=[b]$.  But $d\mid a$ also, so $d\mid\GCD(a,b)=[1]$ and $d$ is a unit.
\item $\GCD(a,b)=[1]$ implies $[c]=\GCD(ac,bc)$.  Now, $a\mid ac$ and by assumption, $a\mid bc$.  Therefore,
$a\mid\GCD(ac,bc)=[c]$.
\end{enumerate}
\end{proof}

The second property above can be generalized to arbitrary integral domain: let $D$ be an integral domain, $a,b\in D$, with $\GCD(a,b)\ne \varnothing \ne \GCD(ma,mb)$, then $d\in \GCD(a,b)$ iff $md \in \GCD(ma,mb)$.
%%%%%
%%%%%
\end{document}
