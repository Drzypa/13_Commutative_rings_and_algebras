\documentclass[12pt]{article}
\usepackage{pmmeta}
\pmcanonicalname{MultiplicativelyClosed}
\pmcreated{2013-03-22 17:29:15}
\pmmodified{2013-03-22 17:29:15}
\pmowner{CWoo}{3771}
\pmmodifier{CWoo}{3771}
\pmtitle{multiplicatively closed}
\pmrecord{6}{39875}
\pmprivacy{1}
\pmauthor{CWoo}{3771}
\pmtype{Definition}
\pmcomment{trigger rebuild}
\pmclassification{msc}{13B30}
\pmclassification{msc}{16U20}
\pmsynonym{saturated}{MultiplicativelyClosed}
\pmrelated{MSystem}
\pmdefines{saturated multiplicatively closed}

\usepackage{amssymb,amscd}
\usepackage{amsmath}
\usepackage{amsfonts}
\usepackage{mathrsfs}

% used for TeXing text within eps files
%\usepackage{psfrag}
% need this for including graphics (\includegraphics)
%\usepackage{graphicx}
% for neatly defining theorems and propositions
\usepackage{amsthm}
% making logically defined graphics
%%\usepackage{xypic}
\usepackage{pst-plot}
\usepackage{psfrag}

% define commands here
\newtheorem{prop}{Proposition}
\newtheorem{thm}{Theorem}
\newtheorem{ex}{Example}
\newcommand{\real}{\mathbb{R}}
\newcommand{\pdiff}[2]{\frac{\partial #1}{\partial #2}}
\newcommand{\mpdiff}[3]{\frac{\partial^#1 #2}{\partial #3^#1}}
\begin{document}
Let $R$ be a ring.  A subset $S$ of $R$ is said to be \emph{multiplicatively closed} if $S\ne \varnothing$, and whenever $a,b\in S$, then $ab\in S$.  In other words, $S$ is a multiplicative set where the multiplication defined on $S$ is the multiplication inherited from $R$.

For example, let $a\in R$, the set $S:=\lbrace a^i, a^{i+1}, \cdots, a^n, \cdots \rbrace$ is multiplicatively closed for any positive integer $i$.  Another simple example is the set $\lbrace 1\rbrace$, if $R$ is unital.

\textbf{Remarks}.  Let $R$ be a commutative ring.
\begin{itemize}
\item
If $P$ is a prime ideal in $R$, then $R-P$ is multiplicatively closed.
\item 
Furthermore, an ideal maximal with respect to the being disjoint from a multiplicative set not containing $0$ is a prime ideal.
\item
In particular, assuming $1\in R$, any ideal maximal with respect to being disjoint from $\lbrace 1\rbrace$ is a maximal ideal.
\end{itemize}

A multiplicatively closed set $S$ in a ring $R$ is said to be \emph{saturated} if for any $a\in S$, every divisor of $a$ is also in $S$.

In the example above, if $i=1$ and $a$ has no divisors, then $S$ is saturated.  

\textbf{Remarks}.
\begin{itemize}
\item
In a unital ring, a saturated multiplicatively closed set always contains $U(R)$, the group of units of $R$ (since it contains $1$, and therefore, all divisors of $1$).  In particular, $U(R)$ itself is saturated multiplicatively closed.
\item Assume $R$ is commutative.  $S\subseteq R$ is saturated multiplicatively closed and $0\notin S$ iff $R-S$ is a union of prime ideals in $R$.  
\begin{proof}
This can be shown as follows: if let $T$ be a union of prime ideals in $R$ and $a,b\in R-T$. if $ab\notin R-T$, then $ab\in P\subseteq T$ for some prime ideal $P$.  Therefore, either $a$ or $b\in P\subseteq T$.  This contradicts the assumption that $a,b\notin T$.  So $R-T$ is multiplicatively closed.  If $ab \in R-T$ with $a\notin R-T$, then $a\in P\subseteq T$ for some prime ideal $P$, which implies $ab\in P\subseteq T$ also.  This contradicts the assumption that $ab\notin T$.  This shows that $R-T$ is saturated.  Of course, $0\notin R-T$, since $0$ lies in any ideal of $R$.

Conversely, assume $S$ is saturated multiplicatively closed and $0\notin S$.  For any $r\notin S$, we want to find a prime ideal $P$ containing $r$ such that $P\cap S=\varnothing$.  Once we show this, then take the union $T$ of these prime ideals and that $S=R-T$ is immediate.  Let $\langle r\rangle$ be the principal ideal generated by $r$.  Since $S$ is saturated, $\langle r\rangle\cap S=\varnothing$.  Let $M$ be the set of all ideals containing $\langle r\rangle$ and disjoint from $S$.  $M$ is non-empty by construction, and we can order $M$ by inclusion.  So $M$ is a poset and Zorn's lemma applies.  Take any chain $C$ in $M$ containing $\langle r\rangle$ and let $P$ be the maximal element in $C$.  Then any ideal larger than $P$ must not be disjoint from $S$, so $P$ is prime by the second remark in the first set of remarks.
\end{proof}
\item The notion of multiplicative closure can be generalized to be defined over any non-empty set with a binary operation (multiplication) defined on it.
\end{itemize}

\begin{thebibliography}{3}
\bibitem{ik} I. Kaplansky, {\it Commutative Rings}. University of Chicago Press, 1974.
\end{thebibliography}
%%%%%
%%%%%
\end{document}
