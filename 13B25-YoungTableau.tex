\documentclass[12pt]{article}
\usepackage{pmmeta}
\pmcanonicalname{YoungTableau}
\pmcreated{2013-03-22 16:48:19}
\pmmodified{2013-03-22 16:48:19}
\pmowner{mps}{409}
\pmmodifier{mps}{409}
\pmtitle{Young tableau}
\pmrecord{10}{39039}
\pmprivacy{1}
\pmauthor{mps}{409}
\pmtype{Definition}
\pmcomment{trigger rebuild}
\pmclassification{msc}{13B25}
\pmclassification{msc}{11P99}
\pmclassification{msc}{05A17}
\pmclassification{msc}{05E05}
\pmclassification{msc}{20C30}
\pmsynonym{Young tableaux}{YoungTableau}
\pmdefines{semi-standard tableau}
\pmdefines{semi-standard tableaux}
\pmdefines{standard Young tableau}
\pmdefines{standard Young tableaux}

\endmetadata

% this is the default PlanetMath preamble.  as your knowledge
% of TeX increases, you will probably want to edit this, but
% it should be fine as is for beginners.

% almost certainly you want these
\usepackage{amssymb}
\usepackage{amsmath}
\usepackage{amsfonts}

% used for TeXing text within eps files
%\usepackage{psfrag}
% need this for including graphics (\includegraphics)
%\usepackage{graphicx}
% for neatly defining theorems and propositions
%\usepackage{amsthm}
% making logically defined graphics
%%%\usepackage{xypic}

% there are many more packages, add them here as you need them
\usepackage[all,web]{xypic}

% define commands here
\def\drawsqlat{%
\begin{xy}{
0;<1.7pc,0pc>:<0pc,1.7pc>::
\xylattice{0}{9}{0}{9}}
\end{xy}}
\def\drawsq{\ar@{-}c;c+(1,0)\ar@{-}c;c+(0,1)\ar@{-}c+(1,0);c+(1,1)\ar@{-}c+(0,1);c+(1,1)}
\def\drawsqlabel#1{\save c+(0.5,0.5)*\txt<2pc>{#1} \restore}

\newcommand{\ferrers}[9]{%
\begin{renewcommand}{\latticebody}{%
\ifnum\latticeA<#1 \ifnum\latticeB=9 \drawsq\fi\fi
\ifnum\latticeA<#2 \ifnum\latticeB=8 \drawsq\fi\fi
\ifnum\latticeA<#3 \ifnum\latticeB=7 \drawsq\fi\fi
\ifnum\latticeA<#4 \ifnum\latticeB=6 \drawsq\fi\fi
\ifnum\latticeA<#5 \ifnum\latticeB=5 \drawsq\fi\fi
\ifnum\latticeA<#6 \ifnum\latticeB=4 \drawsq\fi\fi
\ifnum\latticeA<#7 \ifnum\latticeB=3 \drawsq\fi\fi
\ifnum\latticeA<#8 \ifnum\latticeB=2 \drawsq\fi\fi
\ifnum\latticeA<#9 \ifnum\latticeB=1 \drawsq\fi\fi
}
\drawsqlat
\end{renewcommand}
}
\begin{document}
Let $Y$ be a Young diagram.  A \emph{filling} of $Y$ is a labelling of the boxes in $Y$ by positive integers.  For example, consider the Young diagram with shape $\lambda = (4,4,2,1)\vdash 11$.

\begin{center}
\ferrers{4}{4}{2}{1}{0}{0}{0}{0}{0}
\end{center}

One filling of this Young diagram is

\begin{center}
\begin{renewcommand}{\latticebody}{%
\ifnum\latticeA=1 \ifnum\latticeB=4 \drawsq\drawsqlabel{2} \fi\fi
\ifnum\latticeA=2 \ifnum\latticeB=4 \drawsq\drawsqlabel{4} \fi\fi
\ifnum\latticeA=3 \ifnum\latticeB=4 \drawsq\drawsqlabel{1} \fi\fi
\ifnum\latticeA=4 \ifnum\latticeB=4 \drawsq\drawsqlabel{1} \fi\fi
\ifnum\latticeA=1 \ifnum\latticeB=3 \drawsq\drawsqlabel{5} \fi\fi
\ifnum\latticeA=2 \ifnum\latticeB=3 \drawsq\drawsqlabel{2} \fi\fi
\ifnum\latticeA=3 \ifnum\latticeB=3 \drawsq\drawsqlabel{3} \fi\fi
\ifnum\latticeA=4 \ifnum\latticeB=3 \drawsq\drawsqlabel{7} \fi\fi
\ifnum\latticeA=1 \ifnum\latticeB=2 \drawsq\drawsqlabel{1} \fi\fi
\ifnum\latticeA=2 \ifnum\latticeB=2 \drawsq\drawsqlabel{4} \fi\fi
\ifnum\latticeA=1 \ifnum\latticeB=1 \drawsq\drawsqlabel{1} \fi\fi
}
\drawsqlat
\end{renewcommand}
\end{center}

A filling is a \emph{Young tableau} if it includes each label from $1$ to $n$ exactly once.  One Young tableau with shape $\lambda$ is

\begin{center}
\begin{renewcommand}{\latticebody}{%
\ifnum\latticeA=1 \ifnum\latticeB=4 \drawsq\drawsqlabel{5} \fi\fi
\ifnum\latticeA=2 \ifnum\latticeB=4 \drawsq\drawsqlabel{11} \fi\fi
\ifnum\latticeA=3 \ifnum\latticeB=4 \drawsq\drawsqlabel{8} \fi\fi
\ifnum\latticeA=4 \ifnum\latticeB=4 \drawsq\drawsqlabel{10} \fi\fi
\ifnum\latticeA=1 \ifnum\latticeB=3 \drawsq\drawsqlabel{2} \fi\fi
\ifnum\latticeA=2 \ifnum\latticeB=3 \drawsq\drawsqlabel{3} \fi\fi
\ifnum\latticeA=3 \ifnum\latticeB=3 \drawsq\drawsqlabel{7} \fi\fi
\ifnum\latticeA=4 \ifnum\latticeB=3 \drawsq\drawsqlabel{6} \fi\fi
\ifnum\latticeA=1 \ifnum\latticeB=2 \drawsq\drawsqlabel{9} \fi\fi
\ifnum\latticeA=2 \ifnum\latticeB=2 \drawsq\drawsqlabel{4} \fi\fi
\ifnum\latticeA=1 \ifnum\latticeB=1 \drawsq\drawsqlabel{1} \fi\fi
}
\drawsqlat
\end{renewcommand}
\end{center}

Each Young tableau with shape $\lambda\vdash n$ corresponds to a \PMlinkname{set partition}{Partition} of $[n]=\{1,\dots,n\}$.  
%Moreover, Young tableaux with a fixed size but arbitrary shape are in bijective %correspondence with $\Pi_n$, the lattice of partitions of $[n]$, and the %lattice structure has meaning for Young tableaux.

A filling is a \emph{semi-standard tableau} if the labels
monotonically increase in each row and strictly increase in each
column.  One semi-standard tableau with shape $\lambda$ is

\begin{center}
\begin{renewcommand}{\latticebody}{%
\ifnum\latticeA=1 \ifnum\latticeB=4 \drawsq\drawsqlabel{1} \fi\fi
\ifnum\latticeA=2 \ifnum\latticeB=4 \drawsq\drawsqlabel{1} \fi\fi
\ifnum\latticeA=3 \ifnum\latticeB=4 \drawsq\drawsqlabel{2} \fi\fi
\ifnum\latticeA=4 \ifnum\latticeB=4 \drawsq\drawsqlabel{3} \fi\fi
\ifnum\latticeA=1 \ifnum\latticeB=3 \drawsq\drawsqlabel{2} \fi\fi
\ifnum\latticeA=2 \ifnum\latticeB=3 \drawsq\drawsqlabel{4} \fi\fi
\ifnum\latticeA=3 \ifnum\latticeB=3 \drawsq\drawsqlabel{4} \fi\fi
\ifnum\latticeA=4 \ifnum\latticeB=3 \drawsq\drawsqlabel{5} \fi\fi
\ifnum\latticeA=1 \ifnum\latticeB=2 \drawsq\drawsqlabel{5} \fi\fi
\ifnum\latticeA=2 \ifnum\latticeB=2 \drawsq\drawsqlabel{6} \fi\fi
\ifnum\latticeA=1 \ifnum\latticeB=1 \drawsq\drawsqlabel{8} \fi\fi
}
\drawsqlat
\end{renewcommand}
\end{center}

Finally, a semi-standard tableau is a \emph{standard Young tableau} if it includes each label from $1$ to $n$ exactly once.  Hence a standard Young tableau is both a semi-standard tableau and a Young tableau.  One standard Young tableau with shape $\lambda$ is

\begin{center}
\begin{renewcommand}{\latticebody}{%
\ifnum\latticeA=1 \ifnum\latticeB=4 \drawsq\drawsqlabel{1} \fi\fi
\ifnum\latticeA=2 \ifnum\latticeB=4 \drawsq\drawsqlabel{3} \fi\fi
\ifnum\latticeA=3 \ifnum\latticeB=4 \drawsq\drawsqlabel{6} \fi\fi
\ifnum\latticeA=4 \ifnum\latticeB=4 \drawsq\drawsqlabel{8} \fi\fi
\ifnum\latticeA=1 \ifnum\latticeB=3 \drawsq\drawsqlabel{2} \fi\fi
\ifnum\latticeA=2 \ifnum\latticeB=3 \drawsq\drawsqlabel{4} \fi\fi
\ifnum\latticeA=3 \ifnum\latticeB=3 \drawsq\drawsqlabel{7} \fi\fi
\ifnum\latticeA=4 \ifnum\latticeB=3 \drawsq\drawsqlabel{9} \fi\fi
\ifnum\latticeA=1 \ifnum\latticeB=2 \drawsq\drawsqlabel{5} \fi\fi
\ifnum\latticeA=2 \ifnum\latticeB=2 \drawsq\drawsqlabel{10} \fi\fi
\ifnum\latticeA=1 \ifnum\latticeB=1 \drawsq\drawsqlabel{11} \fi\fi
}
\drawsqlat
\end{renewcommand}
\end{center}

There is some variation in this terminology.  For example, Fulton uses the terms tableau and Young tableau interchangeably for what we call a semi-standard Young tableau.

\begin{thebibliography}{99}
\bibitem{Fu1997}
William~Fulton. \emph{Young tableaux: with applications to representation theory and geometry}.  Cambridge University Press, 1997.

\bibitem{Sa2001}
Bruce~E.~Sagan. \emph{The symmetric group: representations, combinatorial algorithms, and symmetric functions}, 2nd ed.  Springer, 2001.

\bibitem{St1999}
Richard~P.~Stanley. \emph{Enumerative combinatorics, volume 2}.  Cambridge University Press, 1999.
\end{thebibliography}

\PMlinkescapeword{column}
\PMlinkescapeword{fixed}
\PMlinkescapeword{label}
\PMlinkescapeword{labels}
\PMlinkescapeword{row}
\PMlinkescapeword{partitions}
\PMlinkescapeword{size}
\PMlinkescapeword{structure}
\PMlinkescapeword{terms}
\PMlinkescapeword{variation}
%%%%%
%%%%%
\end{document}
