\documentclass[12pt]{article}
\usepackage{pmmeta}
\pmcanonicalname{UniquenessOfDivisionAlgorithmInEuclideanDomain}
\pmcreated{2013-03-22 17:53:00}
\pmmodified{2013-03-22 17:53:00}
\pmowner{pahio}{2872}
\pmmodifier{pahio}{2872}
\pmtitle{uniqueness of division algorithm in Euclidean domain}
\pmrecord{7}{40366}
\pmprivacy{1}
\pmauthor{pahio}{2872}
\pmtype{Theorem}
\pmcomment{trigger rebuild}
\pmclassification{msc}{13F07}
\pmrelated{KrullValuation}
\pmrelated{Quotient}
\pmdefines{incomplete quotient}

\endmetadata

% this is the default PlanetMath preamble.  as your knowledge
% of TeX increases, you will probably want to edit this, but
% it should be fine as is for beginners.

% almost certainly you want these
\usepackage{amssymb}
\usepackage{amsmath}
\usepackage{amsfonts}

% used for TeXing text within eps files
%\usepackage{psfrag}
% need this for including graphics (\includegraphics)
%\usepackage{graphicx}
% for neatly defining theorems and propositions
 \usepackage{amsthm}
% making logically defined graphics
%%%\usepackage{xypic}

% there are many more packages, add them here as you need them

% define commands here

\theoremstyle{definition}
\newtheorem*{thmplain}{Theorem}

\begin{document}
\textbf{Theorem.}\, Let $a,\,b$ be non-zero elements of a Euclidean domain $D$ with the Euclidean valuation $\nu$.\, The {\em incomplete quotient} $q$ and the remainder $r$ of the division algorithm
$$a = qb+r \quad \mbox{where} \quad r = 0 \;\; \mbox{or}\;\; \nu(r) < \nu(b)$$
are unique if and only if
\begin{align}
\nu(a+b) \leqq \max\{\nu(a),\,\nu(b)\}.
\end{align}

{\em Proof.}\, Assume first (1) for the elements $a,\,b$ of $D$.\, If we had
\begin{align*}
\begin{cases}
a = qb+r \quad \mbox{with} \quad r = 0 \;\;\;\;\lor\;\; \nu(r) < \nu(b),\\
a = q'b+r' \quad \mbox{with} \quad r' = 0 \;\;\lor\;\; \nu(r') < \nu(b)
\end{cases}
\end{align*}
and\, $r' \neq r$,\, $q' \neq q$,\, then the \PMlinkname{properties of the Euclidean valuation}{EuclideanValuation} and the assumption yield the \PMlinkescapetext{chain} of inequalities
$$\nu(b) \leqq \nu((q'-q)b) = \nu(r'-r) \leqq \max\{\nu(r'),\,\nu(-r)\} < \nu(b)$$
which is impossible.\, We must infer that\, $r'-r = 0$\, or\, $q'-q = 0$.\, But these two conditions are \PMlinkname{equivalent}{Equivalent3}.\, Thus the division algorithm is unique.

Conversely, assume that (1) is not true for non-zero elements $a,\,b$ of $D$, i.e.
$$\nu(a+b) > \max\{\nu(a),\,\nu(b)\}.$$
Then we obtain two repsesentations
$$b = 0(a+b)+b = 1(a+b)-a$$
where\, $\nu(b) < \nu(a+b)$\, and\, $\nu(-a) = \nu(a) < \nu(a+b)$.\, Thus the incomplete quotient and the remainder are not unique.

%%%%%
%%%%%
\end{document}
