\documentclass[12pt]{article}
\usepackage{pmmeta}
\pmcanonicalname{UniqueFactorizationAndIdealsInRingOfIntegers}
\pmcreated{2015-05-06 15:32:53}
\pmmodified{2015-05-06 15:32:53}
\pmowner{pahio}{2872}
\pmmodifier{pahio}{2872}
\pmtitle{unique factorization and ideals in ring of integers}
\pmrecord{17}{39171}
\pmprivacy{1}
\pmauthor{pahio}{2872}
\pmtype{Theorem}
\pmcomment{trigger rebuild}
\pmclassification{msc}{13B22}
\pmclassification{msc}{11R27}
\pmsynonym{equivalence of UFD and PID}{UniqueFactorizationAndIdealsInRingOfIntegers}
%\pmkeywords{Dedekind domain}
\pmrelated{ProductOfFinitelyGeneratedIdeals}
\pmrelated{PIDsAreUFDs}
\pmrelated{NumberFieldThatIsNotNormEuclidean}
\pmrelated{DivisorTheory}
\pmrelated{FundamentalTheoremOfIdealTheory}
\pmrelated{EquivalentDefinitionsForUFD}

\endmetadata

% this is the default PlanetMath preamble.  as your knowledge
% of TeX increases, you will probably want to edit this, but
% it should be fine as is for beginners.

% almost certainly you want these
\usepackage{amssymb}
\usepackage{amsmath}
\usepackage{amsfonts}

% used for TeXing text within eps files
%\usepackage{psfrag}
% need this for including graphics (\includegraphics)
%\usepackage{graphicx}
% for neatly defining theorems and propositions
 \usepackage{amsthm}
% making logically defined graphics
%%%\usepackage{xypic}

% there are many more packages, add them here as you need them

% define commands here

\theoremstyle{definition}
\newtheorem*{thmplain}{Theorem}

\begin{document}
\textbf{Theorem.}\; Let $O$ be the maximal order, i.e. the ring of 
integers of an algebraic number field.\, Then $O$ is a unique 
factorization domain if and only if $O$ is a principal ideal domain.

{\em Proof.}\; $1^{\underline{o}}$.  Suppose that $O$ is a PID.

We first state, that any prime number $\pi$ of $O$ generates a 
prime ideal $(\pi)$ of $O$.\, For if\, $(\pi) = \mathfrak{ab}$,\, then we have the principal ideals\, $\mathfrak{a} = (\alpha)$\, and $\mathfrak{b} = (\beta)$.\, It follows that\, $(\pi) = (\alpha\beta)$, i.e.\, $\pi = \lambda\alpha\beta$\, with some $\lambda\in O$,\, and since $\pi$ is prime, one of $\alpha$ and $\beta$ must be a unit of $O$.\, Thus one of $\mathfrak{a}$ and $\mathfrak{b}$ is the unit ideal $O$, and accordingly $(\pi)$ is a maximal ideal of $O$, so also a prime ideal.

Let a non-zero element $\gamma$ of $O$ be split to prime number factors $\pi_i$, $\varrho_j$ in two ways:\, $\gamma = \pi_1\cdots\pi_r = \varrho_1\cdots\varrho_s$.\, Then also the principal ideal $(\gamma)$ splits to principal prime ideals in two ways:\, $(\gamma) = (\pi_1)\cdots(\pi_r) = (\varrho_1)\cdots(\varrho_s)$.\, Since the prime factorization of ideals is unique, the \PMlinkescapetext{sequence}\, $(\pi_1),\,\ldots,\,(\pi_r)$\, must be, up to the \PMlinkescapetext{order}, identical with\, $(\varrho_1),\,\ldots,\,(\varrho_s)$\, (and\, $r = s$).\, Let\, $(\pi_1) = (\varrho_{j_1})$.\, Then $\pi_1$ and $\varrho_{j_1}$ are associates of each other; the same may be said of all pairs\, $(\pi_i,\,\varrho_{j_i})$.\, So we have seen that the factorization in $O$ is unique.

$2^{\underline{o}}$.  Suppose then that $O$ is a UFD.

Consider any prime ideal $\mathfrak{p}$ of $O$.\, Let $\alpha$ be a non-zero element of $\mathfrak{p}$ and let $\alpha$ have the prime factorization $\pi_1\cdots\pi_n$.\, Because $\mathfrak{p}$ is a prime ideal and divides the ideal product $(\pi_1)\cdots(\pi_n)$, $\mathfrak{p}$ must divide one principal ideal\, $(\pi_i) = (\pi)$.\, This means that\, $\pi \in \mathfrak{p}$.\, We write\, $(\pi) = \mathfrak{pa}$, whence\, $\pi\in \mathfrak{p}$\, and\, $\pi\in \mathfrak{a}$.\, Since $O$ is a Dedekind domain, every its ideal can be generated by two elements, one of which may be chosen freely (see the two-generator property).\, Therefore we can write
$$\mathfrak{p} = (\pi,\,\gamma),\,\,\, \mathfrak{a} = (\pi,\,\delta).$$
We multiply these, getting\, 
$\mathfrak{pa} = (\pi^2,\,\pi\gamma,\,\pi\delta,\,\gamma\delta)$,\, and so\, $\gamma\delta\in \mathfrak{pa} = (\pi)$.\, Thus\, $\gamma\delta = \lambda\pi$\, with some $\lambda\in O$.\, According to the unique factorization, we have\, $\pi\,|\,\gamma$\, or\, $\pi\,|\,\delta$.

The latter alternative means that\, $\delta = \delta_1\pi$ (with\, $\delta_1\in O$),\, whence\, $\mathfrak{a} = (\pi,\,\delta_1\pi) = (\pi)(1,\,\delta_1) = (\pi)(1) = (\pi)$;\, thus we had\, $\mathfrak{pa} = (\pi) = \mathfrak{p}(\pi)$\, which would imply the absurdity\, $\mathfrak{p} = (1)$.\, But the former alternative means that\, $\gamma = \gamma_1\pi$ (with\, $\gamma_1\in O$),\, which shows that
$$\mathfrak{p} = (\pi,\,\gamma_1\pi) = (\pi)(1,\,\gamma_1) = (\pi)(1) = (\pi).$$
In other words, an arbitrary prime ideal $\mathfrak{p}$ of $O$ is principal.\, It follows that all ideals of $O$ are principal. Q.E.D.

%%%%%
%%%%%
\end{document}
