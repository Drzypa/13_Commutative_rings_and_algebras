\documentclass[12pt]{article}
\usepackage{pmmeta}
\pmcanonicalname{TensorProduct}
\pmcreated{2013-03-22 12:21:26}
\pmmodified{2013-03-22 12:21:26}
\pmowner{rmilson}{146}
\pmmodifier{rmilson}{146}
\pmtitle{tensor product}
\pmrecord{12}{32043}
\pmprivacy{1}
\pmauthor{rmilson}{146}
\pmtype{Definition}
\pmcomment{trigger rebuild}
\pmclassification{msc}{13-00}
\pmclassification{msc}{18-00}
\pmrelated{Module}
\pmrelated{OuterMultiplication}

\endmetadata

\usepackage{amsmath}
\usepackage{amsfonts}
\usepackage{amssymb}
%%\usepackage{xypic}

\newcommand{\reals}{\mathbb{R}}
\newcommand{\natnums}{\mathbb{N}}
\newcommand{\cnums}{\mathbb{C}}

\newcommand{\lp}{\left(}
\newcommand{\rp}{\right)}
\newcommand{\lb}{\left[}
\newcommand{\rb}{\right]}

\newtheorem{proposition}{Proposition}
\begin{document}
{\bf Summary.}  The tensor product is a formal bilinear multiplication
of two modules or vector spaces.  In essence, it permits us to replace
bilinear maps from two such objects by an equivalent linear map from
the tensor product of the two objects.  The origin of this operation
lies in classic differential geometry and physics, which had need of
multiply indexed geometric objects such as the first and second
fundamental forms, and the stress tensor --- see \PMlinkname{Tensor Product (Classical)}{TensorProductClassical}.

{\bf Definition (Standard).} Let $R$ be a commutative ring, and let $A,
B$ be $R$-modules. There exists an $R$-module $A\otimes B$, called the
\emph{tensor product} of $A$ and $B$ over $R$, together with a canonical
bilinear homomorphism
$$\otimes: A\times B\rightarrow A\otimes B,$$
distinguished, up to isomorphism, by the following universal
property.
Every bilinear $R$-module homomorphism
$$\phi: A\times B\rightarrow C,$$
lifts to a unique $R$-module homomorphism
$$\tilde{\phi}: A\otimes B\rightarrow C,$$
such that
$$\phi(a,b) = \tilde{\phi}(a\otimes b)$$
for all $a\in A,\; b\in B.$  Diagramatically:

$$
\xymatrix{\ar[dr]^(.55)\phi \ar[r]^\otimes A\times B& A\otimes B
  \ar@{-->}[d]^(.4){\exists !\, \tilde{\phi}}\\  &C}
$$

The tensor product $A\otimes B$ can be constructed by taking the free
$R$-module generated by all formal symbols 
$$a\otimes b,\quad a\in A,\;b\in B,$$
and quotienting by the obvious bilinear relations:
\begin{align*}
  (a_1+a_2)\otimes b &= a_1\otimes b + a_2\otimes b,\quad &&a_1,a_2\in
  A,\; b\in B \\
  a\otimes(b_1+b_2) &= a\otimes b_1 +  a\otimes b_2,\quad &&a\in A,\;b_1,b_2\in
  B \\
  r(a\otimes b) &= (ra)\otimes b= a\otimes (rb)\quad &&a\in A,\;b\in
  B,\; r\in R  
\end{align*}

{\bf Note.} \PMlinkescapetext{In order to make the base ring $R$ clear,
the tensor product $A\otimes B$ is sometimes written as $A\otimes_R B$.}

{\bf Basic \PMlinkescapetext{properties}.} Let $R$ be a commutative ring and $L,M,N$ be $R$-modules, then, as modules, we have the following isomorphisms:
\begin{enumerate}
\item $R\otimes M\cong M$,
\item $M\otimes N\cong N\otimes M$,
\item $(L\otimes M) \otimes N\cong L\otimes (M\otimes N)$
\item $(L\oplus M)\otimes N \cong (L\otimes N) \oplus (M\otimes N)$
\end{enumerate}

{\bf Definition (Categorical).} Using the language of categories, all
of the above can be expressed quite simply by stating that for all
$R$-modules $M$, the functor $ (-) \otimes M$ is left-adjoint to the
functor $\mathrm{Hom}(M,-)$.
%%%%%
%%%%%
\end{document}
