\documentclass[12pt]{article}
\usepackage{pmmeta}
\pmcanonicalname{UlrichModule}
\pmcreated{2013-03-22 18:13:38}
\pmmodified{2013-03-22 18:13:38}
\pmowner{yshen}{21076}
\pmmodifier{yshen}{21076}
\pmtitle{Ulrich module}
\pmrecord{8}{40813}
\pmprivacy{1}
\pmauthor{yshen}{21076}
\pmtype{Definition}
\pmcomment{trigger rebuild}
\pmclassification{msc}{13C14}

\endmetadata

% this is the default PlanetMath preamble.  as your knowledge
% of TeX increases, you will probably want to edit this, but
% it should be fine as is for beginners.

% almost certainly you want these
\usepackage{amssymb}
\usepackage{amsmath}
\usepackage{amsfonts}

% used for TeXing text within eps files
%\usepackage{psfrag}
% need this for including graphics (\includegraphics)
%\usepackage{graphicx}
% for neatly defining theorems and propositions
%\usepackage{amsthm}
% making logically defined graphics
%%%\usepackage{xypic}

% there are many more packages, add them here as you need them

% define commands here
\def\rank{\operatorname{rank}}
\begin{document}
A maximal Cohen-Macaulay module $M$ over a Noetherian local ring $(R,\mathfrak{m},k)$ is Ulrich if $e(M)=\mu(M)$, where $e(M)$ is the Hilbert-Samuel multiplicity of $M$ and $\mu(M)$ is the minimal number of generators of $M$. When $M$ is a maximal Cohen-Macaulay module and $\mathfrak{m}$ has a minimal reduction $I$ generated by a system of parameters, $M$ is Ulrich if and only if $\mathfrak{m}M=IM$.
%%%%%
%%%%%
\end{document}
