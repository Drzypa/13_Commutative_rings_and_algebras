\documentclass[12pt]{article}
\usepackage{pmmeta}
\pmcanonicalname{ProofOfEulerFoursquareIdentity}
\pmcreated{2013-03-22 13:18:10}
\pmmodified{2013-03-22 13:18:10}
\pmowner{Thomas Heye}{1234}
\pmmodifier{Thomas Heye}{1234}
\pmtitle{proof of Euler four-square identity}
\pmrecord{7}{33807}
\pmprivacy{1}
\pmauthor{Thomas Heye}{1234}
\pmtype{Proof}
\pmcomment{trigger rebuild}
\pmclassification{msc}{13A99}

\endmetadata

% this is the default PlanetMath preamble.  as your knowledge
% of TeX increases, you will probably want to edit this, but
% it should be fine as is for beginners.

% almost certainly you want these
\usepackage{amssymb}
\usepackage{amsmath}
\usepackage{amsfonts}

% used for TeXing text within eps files
%\usepackage{psfrag}
% need this for including graphics (\includegraphics)
%\usepackage{graphicx}
% for neatly defining theorems and propositions
%\usepackage{amsthm}

%%%\usepackage{xypic}

% there are many more packages, add them here as you need them

% define commands here
\begin{document}
Using Lagrange's identity, we have
\begin{eqnarray}
\label{1}
\left(\sum_{k=1}^4 x_ky_k\right)^2 & = \left(\sum_{k=1}^4
x_k^2\right)\left(\sum_{k=1}^4 y_k^2\right) -\sum_{1 \le k < i \le 4} (x_ky_i
-x_iy_k)^2\mbox{.}
\end{eqnarray}
We group the six squares into 3 groups of two squares and rewrite:
\begin{eqnarray}
\label{2}
&(x_1y_2 -x_2y_1)^2 +(x_3y_4 -x_4y_3)^2  \\ \nonumber
= &((x_1y_2 -x_2y_1) +(x_3y_4 -x_4y_3))^2  -2((x_1y_2 -x_2y_1)(x_3y_4 -x_4y_3)) \\
\label{3}
&(x_1y_3 -x_3y_1)^2 +(x_2y_4 -x_4y_2)^2 \\  \nonumber
 =& ((x_1y_3 -x_3y_1) -(x_2y_4
-x_4y_2))^2 +2(x_1y_3 -x_3y_1)(x_2y_4 -x_4y_2) \\
\label{4}
&(x_1y_4 -x_4y_1)^2 +(x_2y_3 -x_3y_2)^2\\
 =& ((x_1y_4 -x_4y_1) +(x_2y_3
-x_3y_2))^2
-2(x_1y_4 -x_4y_1)(x_2y_3 -x_3y_2)\mbox{.}
\end{eqnarray}
Using
\begin{eqnarray}
\label{5}
-2((x_1y_2 -x_2y_1)(x_3y_4 -x_4y_3))
&+2(x_1y_3 -x_3y_1)(x_2y_4 -x_4y_2) \\
\nonumber
-2(x_1y_4 -x_4y_1)(x_2y_3 -x_3y_2)& =0
\end{eqnarray}
we get
\begin{eqnarray}
\label{6}
\sum_{1 \le k < i \le 4} (x_ky_i -x_iy_k)^2 & = ((x_1y_2 -x_2y_1) & +(x_3y_4
-x_4y_3))^2 \\
&+((x_1y_3 -x_3y_1) -(x_2y_4
-x_4y_2))^2 \\ \nonumber
&+((x_1y_4 -x_4y_1) +(x_2y_3
-x_3y_2))^2
\end{eqnarray}
by adding equations \ref{2}-\ref{4}. We put the result of equation \ref{6} into
\ref{1} and get
\begin{eqnarray}
\left(\sum_{k=1}^4 x_ky_k\right)^2 \\
\nonumber  =\left(\sum_{k=1}^4
x_k^2\right)\left(\sum_{k=1}^4 y_k^2\right) &-( (x_1y_2 -x_2y_1 +x_3y_4
-x_4y_3)^2 \\ \nonumber
&-(x_1y_3 -x_3y_1 +x_4y_2 -x_2y_4
)^2 & -(x_1y_4 -x_4y_1 +x_2y_3
-x_3y_2)^2
\end{eqnarray}
which is equivalent to the claimed identity.
%%%%%
%%%%%
\end{document}
