\documentclass[12pt]{article}
\usepackage{pmmeta}
\pmcanonicalname{PartiallyOrderedRing}
\pmcreated{2013-03-22 16:55:04}
\pmmodified{2013-03-22 16:55:04}
\pmowner{CWoo}{3771}
\pmmodifier{CWoo}{3771}
\pmtitle{partially ordered ring}
\pmrecord{8}{39179}
\pmprivacy{1}
\pmauthor{CWoo}{3771}
\pmtype{Definition}
\pmcomment{trigger rebuild}
\pmclassification{msc}{13J25}
\pmclassification{msc}{16W80}
\pmclassification{msc}{06F25}
\pmsynonym{po-ring}{PartiallyOrderedRing}
\pmsynonym{l-ring}{PartiallyOrderedRing}
\pmsynonym{lattice-ordered ring}{PartiallyOrderedRing}
\pmdefines{lattice ordered ring}
\pmdefines{positive cone}

\endmetadata

\usepackage{amssymb,amscd}
\usepackage{amsmath}
\usepackage{amsfonts}

% used for TeXing text within eps files
%\usepackage{psfrag}
% need this for including graphics (\includegraphics)
%\usepackage{graphicx}
% for neatly defining theorems and propositions
\usepackage{amsthm}
% making logically defined graphics
%%\usepackage{xypic}
\usepackage{pst-plot}
\usepackage{psfrag}

% define commands here
\newtheorem{prop}{Proposition}
\newtheorem{thm}{Theorem}
\newtheorem{ex}{Example}
\newcommand{\real}{\mathbb{R}}
\begin{document}
A ring $R$ that is a poset at the same time is called a \emph{partially ordered ring}, or a \emph{po-ring}, if, for $a,b,c\in R$,
\begin{itemize}
\item $a\le b$ implies $a+c\le b+c$, and
\item $0\le a$ and $0\le b$ implies $0\le ab$.
\end{itemize}

Note that $R$ does not have to be associative.

If the underlying poset of a po-ring $R$ is in fact a lattice, then $R$ is called a \emph{lattice-ordered ring}, or an \emph{l-ring} for short.

\textbf{Remark}.  The underlying abelian group of a po-ring (with addition being the binary operation) is a po-group.  The same is true for l-rings.

Below are some examples of po-rings:
\begin{itemize}
\item Clearly, any (totally) ordered ring is a po-ring.
\item The ring of continuous functions over a topological space is an l-ring.
\item Any matrix ring over an ordered field is an l-ring if we define $(a_{ij})\le (b_{ij})$ whenever $a_{ij}\le b_{ij}$ for all $i,j$.
\end{itemize}

\textbf{Remark}.  Let $R$ be a po-ring.  The set $R^+:=\lbrace r\in R\mid 0\le r\rbrace$ is called the \emph{positive cone} of $R$.

\begin{thebibliography}{8}
\bibitem{gb} G. Birkhoff {\em Lattice Theory}, 3rd Edition, AMS Volume XXV, (1967).
\end{thebibliography}
%%%%%
%%%%%
\end{document}
