\documentclass[12pt]{article}
\usepackage{pmmeta}
\pmcanonicalname{ProofThatAGcdDomainIsIntegrallyClosed}
\pmcreated{2013-03-22 18:19:27}
\pmmodified{2013-03-22 18:19:27}
\pmowner{CWoo}{3771}
\pmmodifier{CWoo}{3771}
\pmtitle{proof that a gcd domain is integrally closed}
\pmrecord{6}{40955}
\pmprivacy{1}
\pmauthor{CWoo}{3771}
\pmtype{Derivation}
\pmcomment{trigger rebuild}
\pmclassification{msc}{13G05}

\usepackage{amssymb,amscd}
\usepackage{amsmath}
\usepackage{amsfonts}
\usepackage{mathrsfs}

% used for TeXing text within eps files
%\usepackage{psfrag}
% need this for including graphics (\includegraphics)
%\usepackage{graphicx}
% for neatly defining theorems and propositions
\usepackage{amsthm}
% making logically defined graphics
%%\usepackage{xypic}
\usepackage{pst-plot}

% define commands here
\newcommand*{\abs}[1]{\left\lvert #1\right\rvert}
\newtheorem{prop}{Proposition}
\newtheorem{thm}{Theorem}
\newtheorem{ex}{Example}
\newcommand{\real}{\mathbb{R}}
\newcommand{\pdiff}[2]{\frac{\partial #1}{\partial #2}}
\newcommand{\mpdiff}[3]{\frac{\partial^#1 #2}{\partial #3^#1}}

\newcommand{\GCD}{\operatorname{GCD}}
\begin{document}
\begin{prop} Every gcd domain is integrally closed. \end{prop}

\begin{proof}  Let $D$ be a gcd domain.  For any $a,b\in D$, let $\GCD(a,b)$ be the collection of all gcd's of $a$ and $b$.  For this proof, we need two facts:
\begin{enumerate}
\item $\GCD(ma,mb)=m\GCD(a,b)$.
\item If $\GCD(a,b)=[1]$ and $\GCD(a,c)=[1]$, then $\GCD(a,bc)=[1]$.
\end{enumerate}

The proof of the two properties above can be found \PMlinkname{here}{PropertiesOfAGcdDomain}.  For convenience, we let $\gcd(a,b)$ be any one of the representatives in $\GCD(a,b)$.

Let $K$ be the field of fraction of $D$, and $a/b \in K$ ($a,b\in D$ and $b\ne 0$) is a root of a monic polynomial $p(x)\in D[x]$.  We may, from property (1) above, assume that $\gcd(a,b)=1$.  

Write $$f(x)=x^n+c_{n-1}x^{n-1}+\cdots+c_0.$$ So we have
$$0=(a/b)^n+c_{n-1}(a/b)^{n-1}+\cdots+c_0.$$  Multiply the equation by $b^n$ then rearrange, and we get
$$-a^n=c_{n-1}ba^{n-1}+\cdots+c_0b^n=b(c_{n-1}a^{n-1}+\cdots+c_0b^{n-1}).$$
Therefore, $b\mid a^n$.  Since $\gcd(a,b)=1$, $1=\gcd(a^n,b)=b$, by repeated applications of property (2), and one application of property (1) above.  Therefore $b$ is an associate of 1, hence a unit and we have $a/b\in D$.

\end{proof}

Together with the additional property (call it property 3)
\begin{quote} if $\GCD(a,b)=[1]$ and $a\mid bc$, then $a\mid c$ (proof found \PMlinkname{here}{PropertiesOfAGcdDomain}), \end{quote}
we have the following

\begin{prop} Every gcd domain is a Schreier domain. \end{prop}

\begin{proof}  
That a gcd domain is integrally closed is clear from the previous paragraph.  We need to show that $D$ is pre-Schreier, that is, every non-zero element is primal.  Suppose $c$ is non-zero in $D$, and $c\mid ab$ with $a,b\in D$.
Let $r=\gcd(a,c)$ and $rt=a$, $rs=c$.  Then $1=\gcd(s,t)$ by property (1) above.  Next, since $c\mid ab$, write $cd=ab$ so that $rsd=rtb$.  This implies that $sd=tb$.  So $s\mid tb$ together with $\gcd(s,t)=1$ show that $s\mid b$ by property (3).  So we have just shown the existence of $r,s\in D$ with $c=rs$, $r\mid a$ and $s\mid b$.  Therefore, $c$ is primal and $D$ is a Schreier domain.  

\end{proof}

%%%%%
%%%%%
\end{document}
