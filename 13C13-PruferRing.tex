\documentclass[12pt]{article}
\usepackage{pmmeta}
\pmcanonicalname{PruferRing}
\pmcreated{2015-05-05 15:21:07}
\pmmodified{2015-05-05 15:21:07}
\pmowner{pahio}{2872}
\pmmodifier{pahio}{2872}
\pmtitle{Pr\"ufer ring}
\pmrecord{89}{35533}
\pmprivacy{1}
\pmauthor{pahio}{2872}
\pmtype{Theorem}
\pmcomment{trigger rebuild}
\pmclassification{msc}{13C13}
\pmclassification{msc}{13F05}
%\pmkeywords{fractional ideal}
%\pmkeywords{invertible ideal}
%\pmkeywords{inverse ideal}
\pmrelated{LeastCommonMultiple}
\pmrelated{GeneratorsOfInverseIdeal}
\pmrelated{ProductOfIdeals}
\pmrelated{MultiplicationRing}
\pmrelated{PruferDomain}
\pmrelated{InvertibilityOfRegularlyGeneratedIdeal}
\pmrelated{MultiplicationRuleGivesInverseIdeal}
\pmrelated{ContentOfPolynomial}
\pmrelated{ProductOfFinitelyGeneratedIdeals}
\pmdefines{Pr\"ufer ring}
\pmdefines{coefficient module}
\pmdefines{Gaussian ring}

\endmetadata

% this is the default PlanetMath preamble.  as your knowledge
% of TeX increases, you will probably want to edit this, but
% it should be fine as is for beginners.

% almost certainly you want these
\usepackage{amssymb}
\usepackage{amsmath}
\usepackage{amsfonts}

% used for TeXing text within eps files
%\usepackage{psfrag}
% need this for including graphics (\includegraphics)
%\usepackage{graphicx}
% for neatly defining theorems and propositions
 \usepackage{amsthm}
% making logically defined graphics
%%%\usepackage{xypic}

% there are many more packages, add them here as you need them

% define commands here
\theoremstyle{definition}
\newtheorem{thmplain}{Theorem}
\begin{document}
\textbf{Definition.}\,  A commutative ring $R$ with non-zero unity is a \emph{Pr\"ufer ring} (cf. Pr\"ufer domain) if every finitely generated regular ideal of $R$ is invertible. (It can be proved that if every \PMlinkescapetext{regular} ideal of $R$ generated by two elements is invertible, then all finitely generated \PMlinkescapetext{regular} ideals are invertible; cf. invertibility of regularly generated ideal.)

Denote generally by\, $\mathfrak{m}_p$\, the $R$-module generated by the coefficients of a polynomial $p$ in $T[x]$, where $T$ is the total ring of fractions of $R$.\, Such {\em coefficient modules} are, of course, fractional ideals of $R$.\\

\textbf{Theorem 1 (Pahikkala 1982).}\, Let $R$ be a commutative ring with non-zero unity and let $T$ be the total ring of fractions of $R$.\, Then, $R$ is a Pr\"ufer ring iff the equation
\begin{align}
        \mathfrak{m}_f\mathfrak{m}_g = \mathfrak{m}_{fg}
\end{align}
holds whenever $f$ and $g$ belong to the polynomial ring $T[x]$ and at least one of the fractional ideals $\mathfrak{m}_f$ and $\mathfrak{m}_g$ is \PMlinkescapetext{regular}. (See also product of finitely generated ideals.)\\


\textbf{Theorem 2 (Pahikkala 1982).} \, The commutative ring $R$ with non-zero unity is Pr\"ufer ring iff the multiplication rule
                $$(a,\,b)(c,\,d) = (ac,\,ad+bc,\,bd)$$
for the integral ideals of $R$ holds whenever at least one of the generators $a$, $b$, $c$ and $d$ is not zero divisor.\\


The proofs are found in the paper

J. Pahikkala 1982: ``Some formulae for multiplying and inverting ideals''.\, -- {\em Annales universitatis turkuensis} 183. Turun yliopisto (University of Turku).\\

Cf. the entries ``\PMlinkname{multiplication rule gives inverse ideal}{MultiplicationRuleGivesInverseIdeal}'' and ``\PMlinkname{two-generator property}{TwoGeneratorProperty}''.

An additional characterization of Pr\"ufer ring is found here in the entry ``\PMlinkname{least common multiple}{LeastCommonMultiple}'', several other characterizations in [1] (p. 238--239).\\

\textbf{Note.}\, A commutative ring $R$ satisfying the equation (1) for all polynomials $f,\,g$ is called a {\em Gaussian ring.}\, Thus any \PMlinkname{Pr\"ufer domain}{PruferDomain} is always a Gaussian ring, and \PMlinkname{conversely}{Converse}, an integral domain, which is a Gaussian ring, is a Pr\"ufer domain.\, Cf. [2].

\begin{thebibliography}{9}
\bibitem{LM}{\sc M. Larsen \& P. McCarthy:} {\em Multiplicative theory of ideals}.\, Academic Press. New York (1971).
\bibitem{SG}{\sc Sarah Glaz:} ``The weak dimensions of Gaussian rings''. -- {\em Proc. Amer. Math. Soc.} (2005).
\end{thebibliography}
%%%%%
%%%%%
\end{document}
