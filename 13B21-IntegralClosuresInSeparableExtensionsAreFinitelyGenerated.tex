\documentclass[12pt]{article}
\usepackage{pmmeta}
\pmcanonicalname{IntegralClosuresInSeparableExtensionsAreFinitelyGenerated}
\pmcreated{2013-03-22 17:02:12}
\pmmodified{2013-03-22 17:02:12}
\pmowner{rm50}{10146}
\pmmodifier{rm50}{10146}
\pmtitle{integral closures in separable extensions are finitely generated}
\pmrecord{5}{39323}
\pmprivacy{1}
\pmauthor{rm50}{10146}
\pmtype{Theorem}
\pmcomment{trigger rebuild}
\pmclassification{msc}{13B21}
\pmclassification{msc}{12F05}
\pmrelated{IntegralClosureIsRing}

% this is the default PlanetMath preamble.  as your knowledge
% of TeX increases, you will probably want to edit this, but
% it should be fine as is for beginners.

% almost certainly you want these
\usepackage{amssymb}
\usepackage{amsmath}
\usepackage{amsfonts}

% used for TeXing text within eps files
%\usepackage{psfrag}
% need this for including graphics (\includegraphics)
%\usepackage{graphicx}
% for neatly defining theorems and propositions
\usepackage{amsthm}
% making logically defined graphics
%%%\usepackage{xypic}

% there are many more packages, add them here as you need them

% define commands here
\newcommand{\Nats}{\mathbb{N}}
\newcommand{\Ints}{\mathbb{Z}}
\newcommand{\Reals}{\mathbb{R}}
\newcommand{\Complex}{\mathbb{C}}
\newcommand{\Proj}{\mathbb{P}}
\newcommand{\Rats}{\mathbb{Q}}
\newcommand{\Gal}{\operatorname{Gal}}
\newcommand{\Cl}{\operatorname{Cl}}
\newcommand{\Alg}{\mathcal{O}}
\newcommand{\ol}{\overline}
\newcommand{\Leg}[2]{\left(\frac{#1}{#2}\right)}
\newcommand{\Spec}[1]{\text{Spec }#1}
\newcommand{\Pic}[1]{\text{Pic }#1}
\newcommand{\kx}{k[x_1,\ldots,x_n]}
\newcommand{\Order}[1]{\left\lvert #1 \right\rvert}
\renewcommand{\frak}[1]{\mathfrak{#1}}
\newcommand{\ip}[2]{(#1,#2)}
\newcommand{\conv}[2]{(#1*#2)}
\newcommand{\Hom}{\mathrm{Hom}}

%% \theoremstyle{plain} %% This is the default
\newtheorem{thm}{Theorem}

\begin{document}
The theorem below generalizes to arbitrary integral ring extensions (under certain conditions) the fact that the ring of integers of a number field is finitely generated over $\Ints$. The proof parallels the proof of the number field result.

\begin{thm} Let $B$ be an integrally closed Noetherian domain with field of fractions $K$. Let $L$ be a finite separable extension of $K$, and let $A$ be the integral closure of $B$ in $L$. Then $A$ is a finitely generated $B$-module.
\end{thm}

\begin{proof}
We first show that the \PMlinkname{trace}{Trace2} $Tr^L_K$ maps $A$ to $B$. Choose $u\in A$ and let $f=Irr(u,K)\in K[x]$ be the minimal polynomial for $u$ over $K$; assume $f$ is of degree $d$. Let the conjugates of $u$ in some splitting field be $u=a_1,\ldots,a_d$. Then the $a_i$ are all integral over $B$ since they satisfy $u$'s monic polynomial in $B[x]$. Since the coefficients of $F$ are polynomials in the $a_i$, they too are integral over $B$. But the coefficients are in $K$, and $B$ is integrally closed (in $K$), so the coefficients are in $B$. But $Tr^L_K(u)$ is just the coefficient of $x^{d-1}$ in $f$, and thus $Tr^L_K(u)\in B$. This proves the claim.

Now, choose a basis $\omega_1,\ldots,\omega_d$ of $L/K$. We may assume $\omega_i\in A$ by multiplying each by an appropriate element of $B$. (To see this, let $Irr(\omega_i,K)\in K[x] = x^d+k_1x^{d-1}+\ldots+k_d$. Choose $b\in B$ such that $bk_i\in B\ \forall i$. Then $(b\omega)^d+bk_1(b\omega)^{d-1}+\ldots+b^dk_d=0$ and thus $b\omega\in A$). Define a linear map $\varphi:L\rightarrow K^d:a\mapsto(Tr^L_K (a\omega_1),\ldots,Tr^L_K (a\omega_d))$. 

$\varphi$ is 1-1, since if $u\in\ker\varphi, u\neq 0$, then $Tr(uL)=0$. But $uL=L$, so $Tr^L_K$ is identically zero, which cannot be since $L$ is separable over $K$ (it is a standard result that separability is equivalent to nonvanishing of the trace map; see for example \cite{bib:morandi}, Chapter 8). 

But $Tr^L_K:A\rightarrow B$ by the above, so $\varphi:A\hookrightarrow B^d$. Since $B$ is Noetherian, any submodule of a finitely generated module is also finitely generated, so $A$ is finitely generated as a $B$-module.
\end{proof}

\begin{thebibliography}{10}
\bibitem{bib:morandi}
P.~Morandi, \emph{Field and Galois Theory}, Springer, 2006.
\end{thebibliography}
%%%%%
%%%%%
\end{document}
