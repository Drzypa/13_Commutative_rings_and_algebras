\documentclass[12pt]{article}
\usepackage{pmmeta}
\pmcanonicalname{DivisorTheoryInFiniteExtension}
\pmcreated{2013-03-22 17:59:59}
\pmmodified{2013-03-22 17:59:59}
\pmowner{pahio}{2872}
\pmmodifier{pahio}{2872}
\pmtitle{divisor theory in finite extension}
\pmrecord{7}{40513}
\pmprivacy{1}
\pmauthor{pahio}{2872}
\pmtype{Theorem}
\pmcomment{trigger rebuild}
\pmclassification{msc}{13A18}
\pmclassification{msc}{13F05}
\pmclassification{msc}{12J20}
\pmclassification{msc}{13A05}
\pmclassification{msc}{11A51}
\pmrelated{FiniteExtensionsOfDedekindDomainsAreDedekind}

% this is the default PlanetMath preamble.  as your knowledge
% of TeX increases, you will probably want to edit this, but
% it should be fine as is for beginners.

% almost certainly you want these
\usepackage{amssymb}
\usepackage{amsmath}
\usepackage{amsfonts}

% used for TeXing text within eps files
%\usepackage{psfrag}
% need this for including graphics (\includegraphics)
%\usepackage{graphicx}
% for neatly defining theorems and propositions
 \usepackage{amsthm}
% making logically defined graphics
%%%\usepackage{xypic}

% there are many more packages, add them here as you need them

% define commands here

\theoremstyle{definition}
\newtheorem*{thmplain}{Theorem}

\begin{document}
\PMlinkescapeword{exponent} \PMlinkescapeword{exponents}
\textbf{Theorem.}\, Let the integral domain $\mathcal{O}$, with the quotient field $k$, have the divisor theory\, $\mathcal{O}^* \to \mathfrak{D}$, determined (see divisors and exponents) by the \PMlinkname{exponent}{ExponentValuation2} system $N_0$ of $k$.\, If $K/k$ is a finite extension, then the exponent system $N$, consisting of the \PMlinkname{continuations}{ContinuationOfExponent} of all exponents in $N_0$ to the field $K$, determines the divisor theory of the integral closure of $\mathcal{O}$ in $K$.\\


\textbf{Corollary.}\, In the ring of integers $\mathcal{O}$ of any algebraic number field $\mathbb{Q}(\vartheta)$, there is a divisor theory $\mathcal{O}^* \to \mathfrak{D}$, determined by the set of all exponent valuations of $\mathbb{Q}(\vartheta)$.

\begin{thebibliography}{9}
\bibitem{BS}{\sc S. Borewicz \& I. Safarevic}: {\em Zahlentheorie}.\, Birkh\"auser Verlag. Basel und Stuttgart (1966).
\end{thebibliography}
%%%%%
%%%%%
\end{document}
