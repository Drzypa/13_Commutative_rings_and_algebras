\documentclass[12pt]{article}
\usepackage{pmmeta}
\pmcanonicalname{SymmetricMultilinearFunction}
\pmcreated{2013-03-22 16:10:53}
\pmmodified{2013-03-22 16:10:53}
\pmowner{Mathprof}{13753}
\pmmodifier{Mathprof}{13753}
\pmtitle{symmetric multilinear function}
\pmrecord{11}{38269}
\pmprivacy{1}
\pmauthor{Mathprof}{13753}
\pmtype{Definition}
\pmcomment{trigger rebuild}
\pmclassification{msc}{13A99}
\pmdefines{skew-symmetric multilinear function}

\endmetadata

% this is the default PlanetMath preamble.  as your knowledge
% of TeX increases, you will probably want to edit this, but
% it should be fine as is for beginners.

% almost certainly you want these
\usepackage{amssymb}
\usepackage{amsmath}
\usepackage{amsfonts}

% used for TeXing text within eps files
%\usepackage{psfrag}
% need this for including graphics (\includegraphics)
%\usepackage{graphicx}
% for neatly defining theorems and propositions
%\usepackage{amsthm}
% making logically defined graphics
%%%\usepackage{xypic}

% there are many more packages, add them here as you need them

% define commands here

\begin{document}
Let $R$ be a commutative ring with identity and $M,N$ be unital $R$-modules.

Suppose that $\phi : M \times \cdots \times M \to N$ is a multilinear map, where there
are $n$ copies of $M$. 

Let $H$ be a subgroup of $S_n$, the symmetric group on $\{1, \ldots ,n\}$, and
$\chi : H \to R$ satisfy 
\begin{enumerate}
\item
$\chi(e) = 1$ 
\item
$\chi(g_1g_2) = \chi(g_1)\chi(g_2)$ for all $g_1, g_2 \in H $
\end{enumerate}
\PMlinkescapeword{symmetric}
We say that $\phi$ is \emph{symmetric with respect to $H$ and $\chi$}
if 
$$ \phi(m_{\sigma(1)} , \ldots, m_{\sigma(n)}) = \chi(\sigma)\phi(m_1,\ldots,m_n)$$
holds for all $\sigma\in H$ and all $m_i\in M$. 

Now suppose that $H = S_n$.

If $\chi=1$ then we say that $\phi$ is a \emph{symmetric multilinear function}.
\PMlinkescapeword{skew-symmetric}
If $\chi = \epsilon$, the sign of the permutation $\sigma$, we say that 
$\phi$ is a \emph{skew-symmetric multilinear function}.

For example, the permanent is a symmetric multilinear function of its rows (columns).

The determinant is a skew-symmetric multilinear function of its rows (columns).
%%%%%
%%%%%
\end{document}
