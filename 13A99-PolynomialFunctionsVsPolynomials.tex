\documentclass[12pt]{article}
\usepackage{pmmeta}
\pmcanonicalname{PolynomialFunctionsVsPolynomials}
\pmcreated{2013-03-22 19:18:03}
\pmmodified{2013-03-22 19:18:03}
\pmowner{joking}{16130}
\pmmodifier{joking}{16130}
\pmtitle{polynomial functions vs polynomials}
\pmrecord{4}{42236}
\pmprivacy{1}
\pmauthor{joking}{16130}
\pmtype{Theorem}
\pmcomment{trigger rebuild}
\pmclassification{msc}{13A99}

% this is the default PlanetMath preamble.  as your knowledge
% of TeX increases, you will probably want to edit this, but
% it should be fine as is for beginners.

% almost certainly you want these
\usepackage{amssymb}
\usepackage{amsmath}
\usepackage{amsfonts}

% used for TeXing text within eps files
%\usepackage{psfrag}
% need this for including graphics (\includegraphics)
%\usepackage{graphicx}
% for neatly defining theorems and propositions
%\usepackage{amsthm}
% making logically defined graphics
%%%\usepackage{xypic}

% there are many more packages, add them here as you need them

% define commands here

\begin{document}
Let $k$ be a field. Recall that a function
$$f:k\to k$$
is called polynomial function, iff there are $a_0,\ldots,a_n\in k$ such that
$$f(x)=a_0+a_1x+a_2x^2+\cdots +a_nx^n$$
for any $x\in k$.

The ring of all polynomial functions (together with obvious addition and multiplication) we denote by $k\{x\}$. Also denote by $k[x]$ the ring of polynomials (see \PMlinkname{this entry}{PolynomialRing} for details).

There is a canonical function $T:k[x]\to k\{x\}$ such that for any polynomial
$$W=\sum_{i=1}^n a_i\cdot x^i$$
we have that $T(W)$ is a polynomial function given by
$$T(W)(x)=\sum_{i=1}^n a_i\cdot x^i.$$
(Although we use the same notation for polynomials and polynomial functions these concepts are not the same). This function is called the \textbf{evaluation map}. As a simple exercise we leave the following to the reader:

\textbf{Proposition 1.} The evaluation map $T$ is a ring homomorphisms which is ,,onto''.

The question is: when $T$ is ,,1-1''?

\textbf{Proposition 2.} $T$ is ,,1-1'' if and only if $k$ is an infinite field.

\textit{Proof.} ,,$\Rightarrow$'' Assume that $k=\{a_1,\ldots,a_n\}$ is a finite field. Put
$$W=(x-a_1)\cdots (x-a_n).$$
Then for any $x\in k$ we have that $x=a_i$ for some $i$ and
$$T(W)(x)=(x-a_1)\cdots (x-a_n)=(a_i-a_1)\cdots (a_i-a_i)\cdots (a_i-a_n)=0$$
which shows that $W\in\mathrm{Ker}T$ although $W$ is nonzero. Thus $T$ is not ,,1-1''.

,,$\Leftarrow$'' Assume, that
$$W=\sum_{i=1}^n a_i\cdot x^i$$
is a polynomial with positive degree, i.e. $n\geqslant 1$ and $a_n\neq 0$ such that $T(W)$ is a zero function. It follows from the Bezout's theorem that $W$ has at most $n$ roots (in fact this is true over any integral domain). Thus since $k$ is an infinite field, then there exists $a\in k$ which is not a root of $W$. In particular
$$T(W)(a)\neq 0.$$
Contradiction, since $T(W)$ is a zero function. Thus $T$ is ,,1-1'', which completes the proof. $\square$

This shows that the evaluation map $T$ is an isomorphism only when $k$ is infinite. So the interesting question is what is a kernel of $T$, when $k$ is a finite field?

\textbf{Proposition 3.} Assume that $k=\{a_1,\ldots,a_n\}$ is a finite field and
$$W=(x-a_1)\cdots (x-a_n).$$
Then $T(W)=0$ and if $T(U)=0$ for some polynomial $U$, then $W$ divides $U$. In particular 
$$\mathrm{Ker}T=(W).$$

\textit{Proof.} In the proof of proposition 2 we've shown that $T(W)=0$. Now if $T(U)=0$, then every $a_i$ is a root of $U$. It follows from the Bezout's theorem that $(x-a_i)$ must divide $U$ for any $i$. In particular $W$ divides $U$. This (together with the fact that $T(W)=0$) shows that the ideal $\mathrm{Ker}T$ is generated by $W$. $\square$.

\textbf{Corollary 4.} If $k$ is a finite field of order $q>1$, then $k\{x\}$ has exactly $q^q$ elements.

\textit{Proof.} Let $k=\{a_1,\ldots,a_q\}$ and 
$$W=(x-a_1)\cdots (x-a_q).$$
By propositions 1 and 3 (and due to First Isomorphism Theorem for rings) we have that
$$k\{x\}\simeq k[x]/(W).$$
But the degree of $W$ is equal to $q$. It follows that dimension of $k[x]/(W)$ (as a vector space over $k$) is equal to
$$\mathrm{dim}_{k}k[x]/(W)=q.$$
Thus $k\{x\}$ is isomorphic to $q$ copies of $k$ as a vector space
$$k\{x\}\simeq k\times\cdots\times k.$$
This completes the proof, since $k$ has $q$ elements. $\square$

\textbf{Remark.} Also all of this hold, if we replace $k$ with an integral domain (we can always pass to its field of fractions). However this is not really interesting, since finite integral domains are exactly fields (Wedderburn's little theorem).
%%%%%
%%%%%
\end{document}
