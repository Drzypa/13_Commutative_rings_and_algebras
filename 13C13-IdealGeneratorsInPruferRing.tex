\documentclass[12pt]{article}
\usepackage{pmmeta}
\pmcanonicalname{IdealGeneratorsInPruferRing}
\pmcreated{2013-03-22 14:33:04}
\pmmodified{2013-03-22 14:33:04}
\pmowner{pahio}{2872}
\pmmodifier{pahio}{2872}
\pmtitle{ideal generators in  Pr\"ufer ring}
\pmrecord{20}{36102}
\pmprivacy{1}
\pmauthor{pahio}{2872}
\pmtype{Result}
\pmcomment{trigger rebuild}
\pmclassification{msc}{13C13}
\pmrelated{FractionalIdeal}
\pmrelated{ProductOfFinitelyGeneratedIdeals}

% this is the default PlanetMath preamble.  as your knowledge
% of TeX increases, you will probably want to edit this, but
% it should be fine as is for beginners.

% almost certainly you want these
\usepackage{amssymb}
\usepackage{amsmath}
\usepackage{amsfonts}

% used for TeXing text within eps files
%\usepackage{psfrag}
% need this for including graphics (\includegraphics)
%\usepackage{graphicx}
% for neatly defining theorems and propositions
%\usepackage{amsthm}
% making logically defined graphics
%%%\usepackage{xypic}

% there are many more packages, add them here as you need them

% define commands here
\begin{document}
Let $R$ be a Pr\"ufer ring with total ring of fractions $T$.
\,Let $\mathfrak{a}$ and $\mathfrak{b}$ be fractional ideals of $R$, \PMlinkname{generated by}{IdealGeneratedByASet} $m$ and $n$ elements of $T$, respectively. 
\begin{itemize}
 \item\,Then the  sum ideal  $\mathfrak{a+b}$ may, of course, be generated by $m+n$ elements. 
 \item\,If $\mathfrak{a}$ or $\mathfrak{b}$ is \PMlinkname{regular}{FractionalIdealOfCommutativeRing}, then the   \PMlinkname{product}{ProductOfIdeals} ideal $\mathfrak{ab}$ may be generated by $m+n-1$ elements, since in Pr\"ufer rings the \PMlinkescapetext{formula}
$$(a_1, \,...,\,a_m)(b_1,\,...,\,b_n) =
  (a_1b_1,\,a_1b_2+a_2b_1,\,a_1b_3+a_2b_2+a_3b_1,\, ...,\,a_mb_n)$$
holds.
 \item\,If both $\mathfrak{a}$ and $\mathfrak{b}$ are regular ideals, then the intersection $\mathfrak{a}\cap\mathfrak{b}$ and the quotient ideal \,$\mathfrak{a\colon\!b} = \{r \in R| \quad r\mathfrak{b} \subseteq \mathfrak{a}\}$ \,both may be generated by $m+n$ elements.
 \item\,If $\mathfrak{a}$ is regular, \,then it is also \PMlinkname{invertible}{InvertibleIdeal}.\, Its \PMlinkescapetext{inverse} ideal has the \PMlinkname{expression}{QuotientOfIdeals}
$$\mathfrak{a}^{-1} = [R:\mathfrak{a}] = \{t\in T|\quad t\mathfrak{a} \subseteq R\}$$
and may be generated by $m$ elements of \,$T$ (see the generators of inverse ideal).
\end{itemize}

Cf. also the two-generator property.

\begin{thebibliography}{9}
J. Pahikkala: \,``Some formulae for multiplying and inverting ideals''. $-$ \emph{Annales universitatis turkuensis} 183. \,Turun yliopisto (University of Turku) 1982.
\end{thebibliography}
%%%%%
%%%%%
\end{document}
