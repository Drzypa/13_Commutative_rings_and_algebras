\documentclass[12pt]{article}
\usepackage{pmmeta}
\pmcanonicalname{RegularElementsOfFiniteRing}
\pmcreated{2013-03-22 15:11:09}
\pmmodified{2013-03-22 15:11:09}
\pmowner{pahio}{2872}
\pmmodifier{pahio}{2872}
\pmtitle{regular elements of finite ring}
\pmrecord{19}{36941}
\pmprivacy{1}
\pmauthor{pahio}{2872}
\pmtype{Theorem}
\pmcomment{trigger rebuild}
\pmclassification{msc}{13G05}
\pmclassification{msc}{16U60}
\pmrelated{GroupOfUnits}
\pmrelated{PrimeResidueClass}
\pmrelated{WedderburnsTheorem}
\pmrelated{Unity}

% this is the default PlanetMath preamble.  as your knowledge
% of TeX increases, you will probably want to edit this, but
% it should be fine as is for beginners.

% almost certainly you want these
\usepackage{amssymb}
\usepackage{amsmath}
\usepackage{amsfonts}

% used for TeXing text within eps files
%\usepackage{psfrag}
% need this for including graphics (\includegraphics)
%\usepackage{graphicx}
% for neatly defining theorems and propositions
 \usepackage{amsthm}
% making logically defined graphics
%%%\usepackage{xypic}

% there are many more packages, add them here as you need them

% define commands here
\theoremstyle{definition}
\newtheorem*{thmplain}{Theorem}
\begin{document}
\begin{thmplain}
\, If the finite ring $R$ has regular elements, then it has a unity.\, All regular elements of $R$ form a group under the ring multiplication and with identity element the unity of $R$.\, Thus the regular elements are exactly the units of the ring; the rest of the elements are the zero and the zero divisors.
\end{thmplain}

{\em Proof.}\, Obviously, the set of the regular elements is non-empty and closed under the multiplication.\, Let's think the multiplication table of this set.\, It is a finite \PMlinkescapetext{square where every row only contains} distinct elements (any equation\, $ax = ay$\, reduces to\, $x = y$).\, Hence, for every regular element $a$, the square $a^2$ determines another $a'$ such that\, $a^2a' = a$.\, This implies\, $a'(a^2a')(aa') = a'a(aa')$,\, i.e.\, $(a'a)(aa')^2 = (a'a)(aa')$,\, and since $a'a$ is \PMlinkname{regular}{ZeroDivisor},\, we obtain that\, $(aa')^2 = aa'$.\, So $aa'$ is idempotent, and because it also is \PMlinkescapetext{regular}, it must be the \PMlinkname{unity of the ring}{Unity}:\, $aa' = 1$.\, Thus we see that $R$ has a unity which is a regular element and that $a$ has a multiplicative inverse $a'$, also regular.\, Consequently the regular elements form a group.
%%%%%
%%%%%
\end{document}
