\documentclass[12pt]{article}
\usepackage{pmmeta}
\pmcanonicalname{FreeObjectsInTheCategoryOfCommutativeAlgebras}
\pmcreated{2013-03-22 19:18:13}
\pmmodified{2013-03-22 19:18:13}
\pmowner{joking}{16130}
\pmmodifier{joking}{16130}
\pmtitle{free objects in the category of commutative algebras}
\pmrecord{6}{42239}
\pmprivacy{1}
\pmauthor{joking}{16130}
\pmtype{Theorem}
\pmcomment{trigger rebuild}
\pmclassification{msc}{13P05}
\pmclassification{msc}{11C08}
\pmclassification{msc}{12E05}

% this is the default PlanetMath preamble.  as your knowledge
% of TeX increases, you will probably want to edit this, but
% it should be fine as is for beginners.

% almost certainly you want these
\usepackage{amssymb}
\usepackage{amsmath}
\usepackage{amsfonts}

% used for TeXing text within eps files
%\usepackage{psfrag}
% need this for including graphics (\includegraphics)
%\usepackage{graphicx}
% for neatly defining theorems and propositions
%\usepackage{amsthm}
% making logically defined graphics
%%%\usepackage{xypic}

% there are many more packages, add them here as you need them

% define commands here
\newcommand{\alg}{\mathcal{ALG}_{c}(R)}
\newcommand{\X}{\mathbb{X}}
\newcommand{\Y}{\mathbb{Y}}
\begin{document}
Let $R$ be a commutative ring and let $\alg$ be the category of all commutative algebras over $R$ and algebra homomorphisms. This category together with the forgetful functor is a construct (i.e. it is a concrete category over the category of sets $\mathcal{SET}$). Therefore we can talk about free objects in $\alg$ (see \PMlinkname{this entry}{FreeObjectsInConcreteCategories2} for definitions).

\textbf{Theorem.} For any set $\X$ the polynomial algebra $R[\X]$ (see parent object) is a free object in $\alg$ with $\X$ being a basis. This means that for any commutative algebra $A$ and any function
$$f:\X\to A$$
there exists a unique algebra homomorphism $F:R[\X]\to A$ such that
$$F(x)=f(x)$$
for any $x\in \X$.

\textit{Proof.} Assume that $f:\X\to A$ is a function. If $W\in R[\X]$, then there are finite subsets $A_1,\ldots,A_n\subseteq \X$ (not necessarily disjoint) and natural numbers $n(x,i)$, $i=1,\ldots,n$ such that $W$ can be uniquely expressed as
$$W=\sum_{i=1}^n \bigg(\lambda_i\cdot\prod_{x\in A_i}x^{n(x,i)}\bigg)$$
with $\lambda_i\in R$. Define $F(W)$ by putting
$$F(W)=\sum_{i=1}^n \bigg(\lambda_i\cdot\prod_{x\in A_i}f(x)^{n(x,i)}\bigg).$$
Of course $F$ is well defined and obviously $F(x)=f(x)$. We leave as a simple exercise that $F$ is an algebra homomorphism.
The uniqueness of $F$ again follows from the explicit form of $W$. It is easily seen that $F(W)$ depends only on $F(x)$ for $x\in\X$. This completes the proof. $\square$

\textbf{Corollary 1.} If $\X$ is a set and $\Y\subseteq\X$, then the inclusion $i:\Y\to\X$ induces an algebra monomorphism
$$I:R[\Y]\to R[\X].$$
In particular we can treat $R[\Y]$ as a subalgebra of $R[\X]$.

\textit{Proof.} We have a well-defined function $i:\Y\to R[\X]$, $i(y)=y$. By the theorem we have an extension
$$I:R[\Y]\to R[\X]$$
such that $I(y)=y$. It remains to show, that $I$ is ,,1-1''. Indeed, assume that $I(W)=0$ for some polynomial $W\in R[\Y]$. But if we recall the expression of $W$ as in proof of the theorem and remember that $I$ is an algebra homomorphism, then it is easy to see that $I(y)=y$ implies that
$$I(W)=W.$$
In particular $W=0$, which completes the proof. $\square$

\textbf{Corollary 2.} If $A$ is an $R$-algebra, then there exists a set $\X$ such that
$$A\simeq R[\X]/I$$
for some ideal $I$.

\textit{Proof.} Let $\X=A$ as a set. Define
$$f:\X\to A$$
by $f(x)=x$. By the theorem we have an algebra homomorphism
$$F:R[\X]\to A$$
such that $F(x)=x$ for $x\in \X$. In particular $F$ is ,,onto'' and thus by the First Isomorphism Theorem for algebras we have
$$A\simeq R[\X]/\mathrm{Ker}F$$
which completes the proof. $\square$
%%%%%
%%%%%
\end{document}
