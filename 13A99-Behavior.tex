\documentclass[12pt]{article}
\usepackage{pmmeta}
\pmcanonicalname{Behavior}
\pmcreated{2013-03-22 16:02:29}
\pmmodified{2013-03-22 16:02:29}
\pmowner{Wkbj79}{1863}
\pmmodifier{Wkbj79}{1863}
\pmtitle{behavior}
\pmrecord{15}{38091}
\pmprivacy{1}
\pmauthor{Wkbj79}{1863}
\pmtype{Definition}
\pmcomment{trigger rebuild}
\pmclassification{msc}{13A99}
\pmclassification{msc}{16U99}

% this is the default PlanetMath preamble.  as your knowledge
% of TeX increases, you will probably want to edit this, but
% it should be fine as is for beginners.

% almost certainly you want these
\usepackage{amssymb}
\usepackage{amsmath}
\usepackage{amsfonts}

% used for TeXing text within eps files
%\usepackage{psfrag}
% need this for including graphics (\includegraphics)
%\usepackage{graphicx}
% for neatly defining theorems and propositions
%\usepackage{amsthm}
% making logically defined graphics
%%%\usepackage{xypic}

% there are many more packages, add them here as you need them

% define commands here

\begin{document}
\PMlinkescapeword{generator}
If $R$ is an infinite \PMlinkname{cyclic ring}{CyclicRing3}, the {\sl behavior\/} of $R$ is a nonnegative integer $k$ such that there exists a \PMlinkname{generator}{Generator} $r$ of the additive group of $R$ with $r^2=kr$.

If $R$ is a finite cyclic ring of order $n$, the {\sl behavior\/} of $R$ is a positive divisor $k$ of $n$ such that there exists a generator $r$ of the additive group of $R$ with $r^2=kr$.

For any cyclic ring, behavior exists uniquely.  Moreover, the behavior of a cyclic ring determines many of its \PMlinkescapetext{properties}. 

To the best of my knowledge, this definition first appeared in my master's thesis:

Buck, Warren.  \emph{\PMlinkexternal{Cyclic Rings}{http://planetmath.org/?op=getobj&from=papers&id=336}}.  Charleston, IL: Eastern Illinois University, 2004.
%%%%%
%%%%%
\end{document}
