\documentclass[12pt]{article}
\usepackage{pmmeta}
\pmcanonicalname{QuotientRingModuloPrimeIdeal}
\pmcreated{2013-03-22 17:37:09}
\pmmodified{2013-03-22 17:37:09}
\pmowner{pahio}{2872}
\pmmodifier{pahio}{2872}
\pmtitle{quotient ring modulo prime ideal}
\pmrecord{7}{40038}
\pmprivacy{1}
\pmauthor{pahio}{2872}
\pmtype{Theorem}
\pmcomment{trigger rebuild}
\pmclassification{msc}{13C99}
\pmrelated{CharacterisationOfPrimeIdeals}
\pmrelated{QuotientRing}

% this is the default PlanetMath preamble.  as your knowledge
% of TeX increases, you will probably want to edit this, but
% it should be fine as is for beginners.

% almost certainly you want these
\usepackage{amssymb}
\usepackage{amsmath}
\usepackage{amsfonts}

% used for TeXing text within eps files
%\usepackage{psfrag}
% need this for including graphics (\includegraphics)
%\usepackage{graphicx}
% for neatly defining theorems and propositions
 \usepackage{amsthm}
% making logically defined graphics
%%%\usepackage{xypic}

% there are many more packages, add them here as you need them

% define commands here

\theoremstyle{definition}
\newtheorem*{thmplain}{Theorem}

\begin{document}
\textbf{Theorem.}  Let $R$ be a commutative ring with non-zero unity 1 and $\mathfrak{p}$ an ideal of $R$.  The quotient ring $R/\mathfrak{p}$ is an integral domain if and only if $\mathfrak{p}$ is a prime ideal.\\

{\em Proof.} $1^{\underline{o}}$.  First, let $\mathfrak{p}$ be a prime ideal of $R$.  Then $R/\mathfrak{p}$ is of course a commutative ring and has the unity $1+\mathfrak{p}$.  If the product\, $(r+\mathfrak{p})(s+\mathfrak{p})$ of two residue classes vanishes, i.e. equals $\mathfrak{p}$, then we have\, $rs+\mathfrak{p}= \mathfrak{p}$,\, and therefore $rs$ must belong to $\mathfrak{p}$.  Since $\mathfrak{p}$ is \PMlinkescapetext{prime}, either $r$ or $s$ belongs to $\mathfrak{p}$, i.e.\, $r+\mathfrak{p}= \mathfrak{p}$\, or\, $s+\mathfrak{p}= \mathfrak{p}$.\, Accordingly, 
$R/\mathfrak{p}$ has no zero divisors and is an integral domain. 

$2^{\underline{o}}$.  Conversely, let $R/\mathfrak{p}$ be an integral domain and let the product $rs$ of two elements of $R$ belong to $\mathfrak{p}$.  It follows that\, $(r+\mathfrak{p})(s+\mathfrak{p}) = rs+\mathfrak{p} = \mathfrak{p}$.  Since $R/\mathfrak{p}$ has no zero divisors,\, $r+\mathfrak{p} = \mathfrak{p}$\, or\, $s+\mathfrak{p} =\mathfrak{p} $.  Thus, $r$ or $s$ belongs to $\mathfrak{p}$, i.e. $\mathfrak{p}$ is a prime ideal. 

%%%%%
%%%%%
\end{document}
