\documentclass[12pt]{article}
\usepackage{pmmeta}
\pmcanonicalname{UniversalDerivation}
\pmcreated{2013-03-22 15:27:57}
\pmmodified{2013-03-22 15:27:57}
\pmowner{pbruin}{1001}
\pmmodifier{pbruin}{1001}
\pmtitle{universal derivation}
\pmrecord{9}{37318}
\pmprivacy{1}
\pmauthor{pbruin}{1001}
\pmtype{Definition}
\pmcomment{trigger rebuild}
\pmclassification{msc}{13N15}
\pmclassification{msc}{13N05}
\pmsynonym{K\"ahler differentials}{UniversalDerivation}
%\pmkeywords{derivation}
\pmrelated{Derivation}

% this is the default PlanetMath preamble.  as your knowledge
% of TeX increases, you will probably want to edit this, but
% it should be fine as is for beginners.

% almost certainly you want these
%\usepackage{amssymb}
%\usepackage{amsmath}
%\usepackage{amsfonts}

% used for TeXing text within eps files
%\usepackage{psfrag}
% need this for including graphics (\includegraphics)
%\usepackage{graphicx}
% for neatly defining theorems and propositions
%\usepackage{amsthm}
% making logically defined graphics
%%\usepackage{xypic}

% there are many more packages, add them here as you need them

% define commands here
\begin{document}
Let $R$ be a commutative ring, and let $A$ be a commutative $R$-algebra.  A
\emph{universal derivation} of $A$ over $R$ is defined to be an
$A$-module $\Omega_{A/R}$ together with an $R$-linear derivation $d\colon                              
A\to\Omega_{A/R}$, such that the following universal property holds:
for every $A$-module $M$ and every $R$-linear derivation
$\delta\colon A\to M$ there exists a unique $A$-linear map $f\colon\Omega_{A/R}\to M$ such that $\delta=f\circ d$.

The universal property can be illustrated by a commutative diagram:
$$
\xymatrix{
A \ar[r]^{d\ \ } \ar[dr]_\delta & \Omega_{A/R} \ar@![d]^f \cr
& M
}
$$
An $A$-module with this property can be constructed explicitly, so
$\Omega_{A/R}$ always exists.  It is generated as an $A$-module by
the set $\{dx:x\in A\}$, with the relations
\begin{eqnarray*}
d(ax+by)&=&a\,dx+b\,dy \\
d(xy)&=&x\cdot dy+y\,dx
\end{eqnarray*}
for all $a,b\in R$ and $x,y\in A$.

The universal property implies that $\Omega_{A/R}$ is unique up to
a unique isomorphism.  The $A$-module $\Omega_{A/R}$ is often called
the \emph{module of K\"ahler differentials}.
%%%%%
%%%%%
\end{document}
