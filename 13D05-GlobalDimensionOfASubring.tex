\documentclass[12pt]{article}
\usepackage{pmmeta}
\pmcanonicalname{GlobalDimensionOfASubring}
\pmcreated{2013-03-22 19:05:09}
\pmmodified{2013-03-22 19:05:09}
\pmowner{joking}{16130}
\pmmodifier{joking}{16130}
\pmtitle{global dimension of a subring}
\pmrecord{4}{41974}
\pmprivacy{1}
\pmauthor{joking}{16130}
\pmtype{Theorem}
\pmcomment{trigger rebuild}
\pmclassification{msc}{13D05}
\pmclassification{msc}{16E10}
\pmclassification{msc}{18G20}

\endmetadata

% this is the default PlanetMath preamble.  as your knowledge
% of TeX increases, you will probably want to edit this, but
% it should be fine as is for beginners.

% almost certainly you want these
\usepackage{amssymb}
\usepackage{amsmath}
\usepackage{amsfonts}

% used for TeXing text within eps files
%\usepackage{psfrag}
% need this for including graphics (\includegraphics)
%\usepackage{graphicx}
% for neatly defining theorems and propositions
%\usepackage{amsthm}
% making logically defined graphics
%%%\usepackage{xypic}

% there are many more packages, add them here as you need them

% define commands here

\begin{document}
Let $S$ be a ring with identity and $R\subset S$ a subring, such that $R$ is contained in the center of $S$. In this case $S$ is a (left) $R$-module via multiplication. Throughout by modules we will understand left modules and by global dimension we will understand left global dimension (we will denote it by $\mbox{gl dim}(S)$).

\textbf{Proposition.} Assume that $\mbox{gl dim}(S)=n<\infty$. If $S$ is free as a $R$-module, then $\mbox{gl dim}(R)\leq n+1$.

\textit{Proof.} Let $M$ be a $R$-module. Then, there exists exact sequence
$$0\rightarrow K\rightarrow P_n\rightarrow\cdots\rightarrow P_0\rightarrow M\rightarrow 0,$$
of $R$-modules, where each $P_i$ is projective (module $K$ is just a kernel of a map $P_n\to P_{n-1}$). We will show, that $K$ is also projective (and since $M$ is arbitrary, it will show that $\mbox{gl dim}(R)\leq n+1$).

Since $S$ is free as a $R$-module, then the extension of scalars $(-\otimes_{R} S)$ is an exact functor from the cateogry of $R$-modules to the category of $S$-modules. Furthermore for any projective $R$-module $M$, the $S$-module $M\otimes_{R} S$ is projective (in the category of $S$-modules). Thus we have following exact sequence of $S$-modules
$$0\rightarrow K\otimes_{R} S\rightarrow P_n\otimes_{R} S\rightarrow\cdots\rightarrow P_0\otimes_{R} S\rightarrow M\otimes_{R} S\rightarrow 0,$$
where each $P_i\otimes_{R} S$ is a projective $S$-module. But projective dimension of $M\otimes_{R} S$ is at most $n$ (since $\mbox{gl dim}(S)=n$). Thus $K\otimes_{R} S$ is a projective $S$-module (please, see \PMlinkname{this entry}{ExactSequencesForModulesWithFiniteProjectiveDimension} for more details).

Note that the restriction of scalars functor also maps projective $S$-modules into projective $R$-modules. Thus $K\otimes_{R} S$ is a projective $R$-module. But $S$ is free $R$-module, so 
$$S\simeq \bigoplus_{i\in I} R,$$
for some index set $I$. Finally we have
$$K\otimes_{R} S\simeq K\otimes_{R} \big(\bigoplus_{i\in I} R\big)\simeq \bigoplus_{i\in I} \big(K\otimes_{R}R\big)\simeq \bigoplus_{i\in I} K.$$
This shows, that $K$ is a direct summand of a projective $R$-module $K\otimes_{R} R$ and therefore $K$ is projective, which completes the proof. $\square$
%%%%%
%%%%%
\end{document}
