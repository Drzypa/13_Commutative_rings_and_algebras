\documentclass[12pt]{article}
\usepackage{pmmeta}
\pmcanonicalname{PolynomialRingOverAField}
\pmcreated{2013-03-22 17:42:55}
\pmmodified{2013-03-22 17:42:55}
\pmowner{pahio}{2872}
\pmmodifier{pahio}{2872}
\pmtitle{polynomial ring over a field}
\pmrecord{14}{40160}
\pmprivacy{1}
\pmauthor{pahio}{2872}
\pmtype{Theorem}
\pmcomment{trigger rebuild}
\pmclassification{msc}{13F07}
%\pmkeywords{unique factorization}
%\pmkeywords{Euclid's algorithm}
\pmrelated{FieldAdjunction}
\pmrelated{PolynomialRingOverIntegralDomain}
\pmrelated{PolynomialRingWhichIsPID}
\pmdefines{coprime}

\endmetadata

% this is the default PlanetMath preamble.  as your knowledge
% of TeX increases, you will probably want to edit this, but
% it should be fine as is for beginners.

% almost certainly you want these
\usepackage{amssymb}
\usepackage{amsmath}
\usepackage{amsfonts}

% used for TeXing text within eps files
%\usepackage{psfrag}
% need this for including graphics (\includegraphics)
%\usepackage{graphicx}
% for neatly defining theorems and propositions
 \usepackage{amsthm}
% making logically defined graphics
%%%\usepackage{xypic}

% there are many more packages, add them here as you need them

% define commands here

\theoremstyle{definition}
\newtheorem*{thmplain}{Theorem}

\begin{document}
\textbf{Theorem.}\, The polynomial ring over a field is a Euclidean domain.\\

{\em Proof.}\, Let $K[X]$ be the polynomial ring over a field $K$ in the indeterminate $X$.\, Since $K$ is an integral domain and any polynomial ring over integral domain is an integral domain, the ring $K[X]$ is an integral domain.

The degree $\nu(f)$, defined for every $f$ in $K[X]$ except the zero polynomial, satisfies the requirements of a Euclidean valuation in $K[X]$.\, In fact, the degrees of polynomials are non-negative integers.\, If $f$ and $g$ belong to $K[X]$ and the latter of them is not the zero polynomial, then, as is well known, the long division\, $f/g$\, gives two unique polynomials $q$ and $r$ in $K[X]$ such that 
$$f \;=\; qg+r,$$
where\, $\nu(r) < \nu(g)$\, or\, $r$ is the zero polynomial.\, The second property usually required for the Euclidean valuation, is justified by
$$\nu(fg) \;=\; \nu(f)+\nu(g) \;\geqq\; \nu(f).$$


The theorem implies, similarly as in the ring $\mathbb{Z}$ of the integers, that one can perform in $K[X]$ a Euclid's algorithm which yields a greatest common divisor of two polynomials.\, Performing several \PMlinkescapetext{consecutive} Euclid's algorithms one obtains a gcd of many polynomials; such a gcd is always in the same polynomial ring $K[X]$.\\

Let $d$ be a greatest common divisor of certain polynomials.\, Then apparently also $kd$, where $k$ is any non-zero element of $K$, is a gcd of the same polynomials.\, They do not have other gcd's than $kd$, for if $d'$ is an arbitrary gcd of them, then
$$d' \mid d \quad \mbox{and} \quad d \mid d',$$
i.e. $d$ and $d'$ are associates in the ring $K[X]$ and thus $d'$ is gotten from $d$ by multiplication by an element of the field $K$.\, So we can write the

\textbf{Corollary 1.}\, The greatest common divisor of polynomials in the ring $K[X]$ is unique up to multiplication by a non-zero element of the field $K$.  The \PMlinkname{monic}{Monic2} gcd of polynomials is unique.\\

If the monic gcd of two polynomials is 1, they may be called {\em coprime}.\\

Using the Euclid's algorithm as in $\mathbb{Z}$, one can prove the

\textbf{Corollary 2.}\, If $f$ and $g$ are two non-zero polynomials in $K[X]$, this ring contains such polynomials $u$ and $v$ that
$$\gcd(f,\,g) \;=\; uf+vg$$
and especially, if $f$ and $g$ are coprime, then $u$ and $v$ may be chosen such that\, $uf+vg = 1$.\\

\textbf{Corollary 3.}\, If a product of polynomials in $K[X]$ is divisible by an irreducible polynomial of $K[X]$, then at least one \PMlinkname{factor}{Product} of the product is divisible by the irreducible polynomial.\\

\textbf{Corollary 4.}\, A polynomial ring over a field is always a principal ideal domain.\\

\textbf{Corollary 5.}\, The factorisation of a non-zero polynomial, i.e. the \PMlinkescapetext{presentation} of the polynomial as product of irreducible polynomials, is unique up to constant factors in each polynomial ring $K[X]$ over a field $K$ containing the polynomial.\, Especially, $K[X]$ is a UFD.\\

\textbf{Example.}\, The factorisations of the trinomial \,$X^4-X^2-2$\, into monic irreducible prime factors are\\
$(X^2-2)(X^2+1)$\; in\; $\mathbb{Q}[X]$,\\
$(X^2-2)(X+i)(X-i)$\; in\; $\mathbb{Q}(i)[X]$,\\
$(X+\sqrt{2})(X-\sqrt{2})(X^2+1)$\; in\; $\mathbb{Q}(\sqrt{2})[X]$,\\
$(X+\sqrt{2})(X-\sqrt{2})(X+i)(X-i)$\; in\; $\mathbb{Q}(\sqrt{2},\,i)[X]$.

%%%%%
%%%%%
\end{document}
