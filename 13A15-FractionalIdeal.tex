\documentclass[12pt]{article}
\usepackage{pmmeta}
\pmcanonicalname{FractionalIdeal}
\pmcreated{2013-03-22 12:42:38}
\pmmodified{2013-03-22 12:42:38}
\pmowner{djao}{24}
\pmmodifier{djao}{24}
\pmtitle{fractional ideal}
\pmrecord{5}{32995}
\pmprivacy{1}
\pmauthor{djao}{24}
\pmtype{Definition}
\pmcomment{trigger rebuild}
\pmclassification{msc}{13A15}
\pmclassification{msc}{13F05}
\pmrelated{IdealClassGroup}
\pmdefines{ideal group}

% this is the default PlanetMath preamble.  as your knowledge
% of TeX increases, you will probably want to edit this, but
% it should be fine as is for beginners.

% almost certainly you want these
\usepackage{amssymb}
\usepackage{amsmath}
\usepackage{amsfonts}

% used for TeXing text within eps files
%\usepackage{psfrag}
% need this for including graphics (\includegraphics)
%\usepackage{graphicx}
% for neatly defining theorems and propositions
%\usepackage{amsthm}
% making logically defined graphics
%%%\usepackage{xypic} 

% there are many more packages, add them here as you need them

% define commands here
\renewcommand{\a}{{\mathfrak{a}}}
\renewcommand{\b}{{\mathfrak{b}}}
\begin{document}
\section{Basics}

Let $A$ be an integral domain with field of fractions $K$. Then $K$ is
an $A$--module, and we define a {\em fractional ideal} of $A$ to be a
submodule of $K$ which is finitely generated as an $A$--module.

The product of two fractional ideals $\a$ and $\b$ of $A$ is defined
to be the submodule of $K$ generated by all the products $x \cdot y
\in K$, for $x \in \a$ and $y \in \b$. This product is denoted $\a
\cdot \b$, and it is always a fractional ideal of $A$ as well. Note
that, if $A$ itself is considered as a fractional ideal of $A$, then
$\a \cdot A = \a$. Accordingly, the set of fractional ideals is always
a monoid under this product operation, with identity element $A$.

We say that a fractional ideal $\a$ is {\em invertible} if there
exists a fractional ideal $\a'$ such that $\a \cdot \a' = A$. It can
be shown that if $\a$ is invertible, then its inverse must be $\a' =
(A:\a)$, the annihilator\footnote{In general, for any fractional
ideals $\a$ and $\b$, the annihilator of $\b$ in $\a$ is the
fractional ideal $(\a:\b)$ consisting of all $x \in K$ such that
$x\cdot\b \subset \a$.} of $\a$ in $A$.

\section{Fractional ideals in Dedekind domains}

We now suppose that $A$ is a Dedekind domain. In this case, every
nonzero fractional ideal is invertible, and consequently the nonzero
fractional ideals in $A$ form a group under ideal multiplication,
called the {\em ideal group} of $A$.

The {\em unique factorization of ideals} theorem states that every
fractional ideal in $A$ factors uniquely into a finite product of
prime ideals of $A$ and their (fractional ideal) inverses. It follows
that the ideal group of $A$ is freely generated as an abelian group by
the nonzero prime ideals of $A$.

A fractional ideal of $A$ is said to be {\em principal} if it is
generated as an $A$--module by a single element. The set of nonzero
principal fractional ideals is a subgroup of the ideal group of $A$,
and the quotient group of the ideal group of $A$ by the subgroup of
principal fractional ideals is nothing other than the ideal class
group of $A$.
%%%%%
%%%%%
\end{document}
