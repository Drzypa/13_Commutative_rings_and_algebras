\documentclass[12pt]{article}
\usepackage{pmmeta}
\pmcanonicalname{PlaceAsExtensionOfHomomorphism}
\pmcreated{2013-03-22 14:57:21}
\pmmodified{2013-03-22 14:57:21}
\pmowner{pahio}{2872}
\pmmodifier{pahio}{2872}
\pmtitle{place as extension of homomorphism}
\pmrecord{9}{36651}
\pmprivacy{1}
\pmauthor{pahio}{2872}
\pmtype{Theorem}
\pmcomment{trigger rebuild}
\pmclassification{msc}{13A18}
\pmclassification{msc}{12E99}
\pmclassification{msc}{13F30}
\pmsynonym{extension theorem}{PlaceAsExtensionOfHomomorphism}
\pmrelated{RamificationOfArchimedeanPlaces}

% this is the default PlanetMath preamble.  as your knowledge
% of TeX increases, you will probably want to edit this, but
% it should be fine as is for beginners.

% almost certainly you want these
\usepackage{amssymb}
\usepackage{amsmath}
\usepackage{amsfonts}

% used for TeXing text within eps files
%\usepackage{psfrag}
% need this for including graphics (\includegraphics)
%\usepackage{graphicx}
% for neatly defining theorems and propositions
 \usepackage{amsthm}
% making logically defined graphics
%%%\usepackage{xypic}

% there are many more packages, add them here as you need them

% define commands here
\theoremstyle{definition}
\newtheorem*{thmplain}{Theorem}
\begin{document}
\begin{thmplain}
\,If $f$ is a ring homomorphism from a subring $\mathfrak{o}$ of a field $k$ to an algebraically closed field $F$ such that \,$f(1) = 1$, \,then there exists a \PMlinkname{place}{PlaceOfField} 
              $$\varphi: \,k\to F\cup\{\infty\}$$
of the field $k$ such that
                  $$\varphi|_\mathfrak{o} = f.$$
\end{thmplain}

\textbf{Note.} \,That $F$ should be algebraically closed, does not \PMlinkescapetext{mean any restriction}, since every field is extendable to such one.

\begin{thebibliography}{8}
\bibitem{artin}Emil Artin: {\em \PMlinkescapetext{Theory of Algebraic Numbers}}. \,Lecture notes. \,Mathematisches Institut, G\"ottingen (1959).
\end{thebibliography}
%%%%%
%%%%%
\end{document}
