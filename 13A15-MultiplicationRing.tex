\documentclass[12pt]{article}
\usepackage{pmmeta}
\pmcanonicalname{MultiplicationRing}
\pmcreated{2013-03-22 14:27:02}
\pmmodified{2013-03-22 14:27:02}
\pmowner{PrimeFan}{13766}
\pmmodifier{PrimeFan}{13766}
\pmtitle{multiplication ring}
\pmrecord{17}{35967}
\pmprivacy{1}
\pmauthor{PrimeFan}{13766}
\pmtype{Definition}
\pmcomment{trigger rebuild}
\pmclassification{msc}{13A15}
%\pmkeywords{ideal multiplication}
\pmrelated{PruferRing}
\pmrelated{DedekindDomain}
\pmrelated{DivisibilityInRings}

\endmetadata

% this is the default PlanetMath preamble.  as your knowledge
% of TeX increases, you will probably want to edit this, but
% it should be fine as is for beginners.

% almost certainly you want these
\usepackage{amssymb}
\usepackage{amsmath}
\usepackage{amsfonts}

% used for TeXing text within eps files
%\usepackage{psfrag}
% need this for including graphics (\includegraphics)
%\usepackage{graphicx}
% for neatly defining theorems and propositions
 \usepackage{amsthm}
% making logically defined graphics
%%%\usepackage{xypic}

% there are many more packages, add them here as you need them

% define commands here
\theoremstyle{definition}
\newtheorem*{thmplain}{Theorem}
\begin{document}
Let $R$ be a commutative ring with non-zero unity.\, If $\mathfrak{a}$ and $\mathfrak{b}$ are two \PMlinkname{\em fractional ideals}{FractionalIdealOfCommutativeRing} of $R$ with\, $\mathfrak{a} \subseteq \mathfrak{b}$\, and if $\mathfrak{b}$ is \PMlinkname{invertible}{FractionalIdealOfCommutativeRing}, then there is a \PMlinkescapetext{fractional ideal} $\mathfrak{c}$ of $R$ such that\, $\mathfrak{a} = \mathfrak{bc}$\, (one can choose\, $\mathfrak{c} = \mathfrak{b}^{-1}\mathfrak{a}$).

\textbf{Definition.}\, Let $R$ be a commutative ring with non-zero unity and let $\mathfrak{a}$ and $\mathfrak{b}$ be ideals of $R$.\, The ring $R$ is a {\em multiplication ring} if\, $\mathfrak{a} \subseteq \mathfrak{b}$\, always implies that there exists a \PMlinkescapetext{fractional ideal} $\mathfrak{c}$ of $R$ such that\, $\mathfrak{a} = \mathfrak{bc}$.

\begin{thmplain}
 \, Every Dedekind domain is a multiplication ring.\, If a multiplication ring has no zero divisors, it is a Dedekind domain.
\end{thmplain}

\begin{thebibliography}{9}
\bibitem{LM}{\sc M. Larsen \& P. McCarthy:} {\em Multiplicative theory of ideals}.\, Academic Press. New York (1971).
\end{thebibliography}
%%%%%
%%%%%
\end{document}
