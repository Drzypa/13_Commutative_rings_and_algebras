\documentclass[12pt]{article}
\usepackage{pmmeta}
\pmcanonicalname{IdealsWithMaximalRadicalsArePrimary}
\pmcreated{2013-03-22 19:04:31}
\pmmodified{2013-03-22 19:04:31}
\pmowner{joking}{16130}
\pmmodifier{joking}{16130}
\pmtitle{ideals with maximal radicals are primary}
\pmrecord{4}{41961}
\pmprivacy{1}
\pmauthor{joking}{16130}
\pmtype{Theorem}
\pmcomment{trigger rebuild}
\pmclassification{msc}{13C99}

% this is the default PlanetMath preamble.  as your knowledge
% of TeX increases, you will probably want to edit this, but
% it should be fine as is for beginners.

% almost certainly you want these
\usepackage{amssymb}
\usepackage{amsmath}
\usepackage{amsfonts}

% used for TeXing text within eps files
%\usepackage{psfrag}
% need this for including graphics (\includegraphics)
%\usepackage{graphicx}
% for neatly defining theorems and propositions
%\usepackage{amsthm}
% making logically defined graphics
%%%\usepackage{xypic}

% there are many more packages, add them here as you need them

% define commands here

\begin{document}
\textbf{Proposition.} Assume that $R$ is a commutative ring and $I\subseteq R$ is an ideal, such that the radical $r(I)$ of $I$ is a maximal ideal. Then $I$ is a primary ideal.

\textit{Proof.} We will show, that every zero divisor in $R/I$ is nilpotent (please, see parent object for details).

First of all, recall that $r(I)$ is an intersection of all prime ideals containing $I$ (please, see \PMlinkname{this entry}{ACharacterizationOfTheRadicalOfAnIdeal} for more details). Since $r(I)$ is maximal, it follows that there is exactly one prime ideal $P=r(I)$ such that $I\subseteq P$. In particular the ring $R/I$ has only one prime ideal (because there is one-to-one correspondence between prime ideals in $R/I$ and prime ideals in $R$ containing $I$). Thus, in $R/I$ an ideal $r(0)$ is prime.

Now assume that $\alpha\in R/I$ is a zero divisor. In particular $\alpha\neq 0+I$ and for some $\beta\neq 0+I \in R/I$ we have
$$\alpha\beta=0+I.$$
But $0+I\in r(0)$ and $r(0)$ is prime. This shows, that either $\alpha\in r(0)$ or $\beta\in r(0)$.

Obviously $\alpha\in r(0)$ (and $\beta\in r(0)$), because $r(0)$ is the only maximal ideal in $R/I$ (the ring $R/I$ is local). Therefore elements not belonging to $r(0)$ are invertible, but $\alpha$ cannot be invertible, because it is a zero divisor.

On the other hand $r(0)=\{x+I\in R/I\ |\ (x+I)^n=0\mbox{ for some }n\in\mathbb{N}\}$. Therefore $\alpha$ is nilpotent and this completes the proof. $\square$

\textbf{Corollary.} Let $p\in\mathbb{N}$ be a prime number and $n\in\mathbb{N}$. Then the ideal $(p^n)\subseteq\mathbb{Z}$ is primary.

\textit{Proof.} Of course the ideal $(p)$ is maximal and we have 
$$r\big((p^n)\big)=r\big((p)^n\big)=(p),$$
since for any prime ideal $P$ (in arbitrary ring $R$) we have $r(P^n)=P$. The result follows from the proposition. $\square$
%%%%%
%%%%%
\end{document}
