\documentclass[12pt]{article}
\usepackage{pmmeta}
\pmcanonicalname{ProofOfHenselsLemma}
\pmcreated{2013-03-22 15:32:16}
\pmmodified{2013-03-22 15:32:16}
\pmowner{rm50}{10146}
\pmmodifier{rm50}{10146}
\pmtitle{proof of Hensel's lemma}
\pmrecord{5}{37432}
\pmprivacy{1}
\pmauthor{rm50}{10146}
\pmtype{Proof}
\pmcomment{trigger rebuild}
\pmclassification{msc}{13H99}
\pmclassification{msc}{12J99}
\pmclassification{msc}{11S99}

\endmetadata

% this is the default PlanetMath preamble.  as your knowledge
% of TeX increases, you will probably want to edit this, but
% it should be fine as is for beginners.

% almost certainly you want these
\usepackage{amssymb}
\usepackage{amsmath}
\usepackage{amsfonts}

% used for TeXing text within eps files
%\usepackage{psfrag}
% need this for including graphics (\includegraphics)
%\usepackage{graphicx}
% for neatly defining theorems and propositions
%\usepackage{amsthm}
% making logically defined graphics
%%%\usepackage{xypic}

% there are many more packages, add them here as you need them

% define commands here
\begin{document}
\textbf{Lemma:}\, Using the setup and terminology of the statement of Hensel's Lemma, for $i\geq 0$,
\begin{eqnarray*}
& \mbox{i) } & |f'(\alpha_i)| = |f'(\alpha_0)| \\
& \mbox{ii) } & \left|\frac{f(\alpha_i)}{f'(\alpha_i)^2}\right| \leq D^{2^i} \\
& \mbox{iii) } & | \alpha_i - \alpha_0 | \leq D \\
& \mbox{iv) } & \alpha_i \in \mathcal{O}_K
\end{eqnarray*}
where $ D=\left|\frac{f(\alpha_0)}{f'(\alpha_0)^2}\right|$.

\textbf{Proof:}
All four statements clearly hold when $i=0$. Suppose they are true for $i$. The proof for $i+1$ essentially uses Taylor's formula. Let $\delta = \left|\frac{-f(\alpha_i)}{f'(\alpha_i)}\right|$. Then
$$f'(\alpha_{i+1}) = f'(\alpha_i+\delta) = f'(\alpha_i) + {\delta}u$$
$$f(\alpha_{i+1}) = f(\alpha_i+\delta) = f(\alpha_i) + f'(\alpha_i)\delta + {\delta^2}v$$
for $u, v \in \mathcal{O}_K$. $|\delta| \leq D^{2^i}|f'(\alpha_i)|$ by induction, and since $D < 1$, it follows that $|\delta| < |f'(\alpha_i)|$. Since the norm is non-Archimedean, we see that $$f'(\alpha_{i+1}) = f'(\alpha_i)$$ proving i).

$f(\alpha_i)+f'(\alpha_i)\delta = 0$ by definition of $\delta$, so $f(\alpha_{i+1}) = {\delta^2}v$ and hence $|f(\alpha_{i+1})| \leq |\delta^2|$. Hence
$$\left|\frac{f(\alpha_{i+1})}{f'(\alpha_{i+1})^2}\right| \leq \frac{|\delta|^2}{|f'(\alpha_{i+1})|^2} = \frac{|\delta|^2}{|f'(\alpha_i)|^2}=\left(\frac{|\delta|}{|f'(\alpha_i)|}\right)^2 = \left(\frac{|f(\alpha_i)|}{|f'(\alpha_i)|^2}\right)^2 \leq D^{2^{i+1}}$$
where the last equality follows by induction. This proves ii).

To prove iii), note that $|\alpha_{i+1}-\alpha_i| = |\delta|$ by the definitions of $\delta$ and $\alpha_{i+1}$, so $|\alpha_{i+1}-\alpha_i| \leq D^{2^i}|f'(\alpha_i)| = D^{2^i}|f'(\alpha_0) < D$ when $i>0$ since $D^2 < D = \left|\frac{f(\alpha_0)}{f'(\alpha_0)^2}\right|$. So by induction, $|\alpha_{i+1}-\alpha_0| \leq D$.

Finally, to prove iv) and \PMlinkescapetext{complete} the proof of the lemma, $\delta \in \mathcal{O}_K$ since $|\delta| < \left|\frac{f(\alpha_0)}{f'(\alpha_0)}\right| \leq 1$ and hence is in the valuation ring of $K$. So by induction, $\alpha_{i+1} = \alpha_i + \delta \in \mathcal{O}_K$.

\textbf{Proof of Hensel's Lemma:}

To prove Hensel's lemma from the above lemma, note that $\delta = \delta_i \to 0$ since $|\delta| \leq D^{2^i}|f'(\alpha_0)|$, so $\{\alpha_i\}$ converges to $\alpha \in \mathcal{O}_K$ since $K$ is complete. Thus $f(\alpha_i)\to f(\alpha)$ by continuity. But $|f(\alpha_i)| \leq |\delta^2| = D^{2^{i+1}}|f'(\alpha_0)|$, so $|f(\alpha_i)| \to 0$, so $f(\alpha)=0$ and the proof is complete.
%%%%%
%%%%%
\end{document}
