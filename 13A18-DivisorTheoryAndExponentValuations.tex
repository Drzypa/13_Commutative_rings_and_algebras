\documentclass[12pt]{article}
\usepackage{pmmeta}
\pmcanonicalname{DivisorTheoryAndExponentValuations}
\pmcreated{2013-03-22 17:59:34}
\pmmodified{2013-03-22 17:59:34}
\pmowner{pahio}{2872}
\pmmodifier{pahio}{2872}
\pmtitle{divisor theory and exponent valuations}
\pmrecord{7}{40504}
\pmprivacy{1}
\pmauthor{pahio}{2872}
\pmtype{Topic}
\pmcomment{trigger rebuild}
\pmclassification{msc}{13A18}
\pmclassification{msc}{12J20}
\pmclassification{msc}{13A05}
\pmsynonym{divisors and exponents}{DivisorTheoryAndExponentValuations}
\pmrelated{ExponentValuation2}
\pmrelated{ImplicationsOfHavingDivisorTheory}

\endmetadata

% this is the default PlanetMath preamble.  as your knowledge
% of TeX increases, you will probably want to edit this, but
% it should be fine as is for beginners.

% almost certainly you want these
\usepackage{amssymb}
\usepackage{amsmath}
\usepackage{amsfonts}

% used for TeXing text within eps files
%\usepackage{psfrag}
% need this for including graphics (\includegraphics)
%\usepackage{graphicx}
% for neatly defining theorems and propositions
 \usepackage{amsthm}
% making logically defined graphics
%%%\usepackage{xypic}

% there are many more packages, add them here as you need them

% define commands here

\theoremstyle{definition}
\newtheorem*{thmplain}{Theorem}

\begin{document}
\PMlinkescapeword{exponents} \PMlinkescapeword{exponent}

A divisor theory \,$\mathcal{O}^* \to \mathfrak{D}$\, of an integral domain $\mathcal{O}$ determines via its prime divisors a certain set $N$ of exponent valuations on the quotient field of $\mathcal{O}$.\, Assume to be known this set of \PMlinkname{exponents}{ExponentValuation2} $\nu_{\mathfrak{p}}$ corresponding the prime divisors $\mathfrak{p}$.\, There is a bijective correspondence between the elements of $N$ and of the set of all prime divisors.\, The set of the prime divisors determines completely the \PMlinkescapetext{structure} of the free monoid $\mathfrak{D}$ of all divisors in question.  The homomorphism \,$\mathcal{O}^* \to \mathfrak{D}$\, is then defined by the condition
\begin{align}
\alpha\; \mapsto\; \prod_i\mathfrak{p_i}^{\nu_{\mathfrak{p}_i}(\alpha)} = (\alpha),´
\end{align}
since for any element $\alpha$ of $\mathcal{O}^*$ there exists only a finite number of exponents $\nu_{\mathfrak{p}_i}$ which do not vanish on $\alpha$ (corresponding the different prime divisor \PMlinkname{factors}{DivisibilityInRings} of the principal divisor $(\alpha)$).\\

One can take the concept of exponent as foundation for divisor theory:\\

\textbf{Theorem.}\, Let $\mathcal{O}$ be an integral domain with quotient field $K$ and $N$ a given set of \PMlinkname{exponents}{ExponentValuation2} of $K$.\, The exponents in $N$ determine, as in (1), a divisor theory of $\mathcal{O}$ iff the following three conditions are in \PMlinkescapetext{force}:
\begin{itemize}
\item For every\, $\alpha \in \mathcal{O}$\, there is at most a finite number of exponents\, $\nu \in N$\, such that\, $\nu(\alpha) \neq 0$.
\item An element\, $\alpha \in K$\, belongs to $\mathcal{O}$ if and only if\, $\nu(\alpha) \geqq 0$\, for each\, 
$\nu \in N$.
\item For any finite set \,$\nu_1,\,\ldots,\,\nu_n$\, of distinct exponents in $N$ and for the arbitrary set\, $k_1,\,\ldots,\,k_n$ of non-negative integers, there exists an element $\alpha$ of $\mathcal{O}$ such that
$$\nu_1(\alpha) = k_1,\,\;\ldots,\,\;\nu_n(\alpha) = k_n.$$
\end{itemize}

For the proof of the theorem, we mention only how to construct the divisors when we have the exponent set $N$ fulfilling the three conditions of the theorem.\, We choose a commutative monoid $\mathfrak{D}$ that allows unique prime factorisation and that may be mapped bijectively onto $N$.\, The exponent in $N$ which corresponds to arbitrary prime element $\mathfrak{p}$ is denoted by $\nu_\mathfrak{p}$.\, Then we obtain the homomorphism
$$\alpha \mapsto \prod_\nu \mathfrak{p}^{\nu_\mathfrak{p}(\alpha)} := (\alpha)$$
which can be seen to satisfy all required properties for a divisor theory\, $\mathcal{O}^* \to \mathfrak{D}$.

\begin{thebibliography}{9}
\bibitem{BS}{\sc S. Borewicz \& I. Safarevic}: {\em Zahlentheorie}.\, Birkh\"auser Verlag. Basel und Stuttgart (1966).
\end{thebibliography}
%%%%%
%%%%%
\end{document}
