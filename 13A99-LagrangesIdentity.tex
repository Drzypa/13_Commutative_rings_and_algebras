\documentclass[12pt]{article}
\usepackage{pmmeta}
\pmcanonicalname{LagrangesIdentity}
\pmcreated{2013-03-22 13:18:01}
\pmmodified{2013-03-22 13:18:01}
\pmowner{mathcam}{2727}
\pmmodifier{mathcam}{2727}
\pmtitle{Lagrange's identity}
\pmrecord{21}{33802}
\pmprivacy{1}
\pmauthor{mathcam}{2727}
\pmtype{Theorem}
\pmcomment{trigger rebuild}
\pmclassification{msc}{13A99}

\endmetadata

% this is the default PlanetMath preamble.  as your knowledge
% of TeX increases, you will probably want to edit this, but
% it should be fine as is for beginners.

% almost certainly you want these
\usepackage{amssymb}
\usepackage{amsmath}
\usepackage{amsfonts}

% used for TeXing text within eps files
%\usepackage{psfrag}
% need this for including graphics (\includegraphics)
%\usepackage{graphicx}
% for neatly defining theorems and propositions
\usepackage{amsthm}
% making logically defined graphics
%%%\usepackage{xypic}

% there are many more packages, add them here as you need them

% define commands here
\begin{document}
Let $R$ be a commutative ring, and let
$x_1, \ldots, x_n, y_1, \ldots, y_n$ be arbitrary elements in $R$. Then
\[\left(\sum_{k=1}^n x_ky_k\right)^2 =\left(\sum_{k=1}^n x_k^2\right)\left(\sum_{k=1}^n
y_k^2\right)
- \sum_{1 \le k < i \le n} (x_ky_i -x_iy_k)^2\mbox{.}\]

\begin{proof}
Since $R$ is commutative, we can apply the binomial formula.We start out with
\begin{equation}
\left(\sum_{i=1}^n x_iy_i\right)^2 =\sum_{i=1}^n (x_i^2y_i^2) +\sum_{1
\leq i< j\leq n} 2 x_iy_jx_jy_i
\end{equation}
Using the binomial formula, we see that
\[(x_iy_j -x_jy_i)^2 =x_i^2y_j^2 -2x_ix_jy_iy_j +x_j^2y_i^2.\]
So we get

\begin{eqnarray}
\left(\sum\limits_{i=1}^n x_iy_i\right)^2 +\sum\limits_{1
\leq i< j\leq n}^n(x_iy_j -x_jy_i)^2
&=&\sum\limits_{i=1}^n \left(x_i^2y_i^2\right) +\sum\limits_{1
\leq i< j\leq n}^n \left(x_i^2y_j^2
+x_j^2y_i^2\right)  \\
&=&\left(\sum\limits_{i=1}^n x_i^2\right)\left(\sum\limits_{i=1}^n y_i^2\right)
\end{eqnarray}
Note that changing the roles of $i$ and $j$ in $x_iy_j -x_jy_i$, we get
\[x_jy_i -x_iy_j =-(x_iy_j -x_jy_i),\]
but the negative sign will disappear when we square. So we can rewrite the last equation to
\begin{equation}
\left(\sum\limits_{i=1}^n x_iy_i\right)^2 +\sum\limits_{1 \le i <j \le n} (x_iy_j -x_jy_i)^2
=\left(\sum_{i=1}^n x_i^2\right)\left(\sum_{i=1}^n y_i^2\right).
\end{equation}
This is equivalent to the stated identity.
\end{proof}
%%%%%
%%%%%
\end{document}
