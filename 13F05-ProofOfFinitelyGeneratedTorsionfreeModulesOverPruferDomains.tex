\documentclass[12pt]{article}
\usepackage{pmmeta}
\pmcanonicalname{ProofOfFinitelyGeneratedTorsionfreeModulesOverPruferDomains}
\pmcreated{2013-03-22 18:36:14}
\pmmodified{2013-03-22 18:36:14}
\pmowner{gel}{22282}
\pmmodifier{gel}{22282}
\pmtitle{proof of finitely generated torsion-free modules over Pr\"ufer domains}
\pmrecord{4}{41335}
\pmprivacy{1}
\pmauthor{gel}{22282}
\pmtype{Proof}
\pmcomment{trigger rebuild}
\pmclassification{msc}{13F05}
\pmclassification{msc}{13C10}

% this is the default PlanetMath preamble.  as your knowledge
% of TeX increases, you will probably want to edit this, but
% it should be fine as is for beginners.

% almost certainly you want these
\usepackage{amssymb}
\usepackage{amsmath}
\usepackage{amsfonts}

% used for TeXing text within eps files
%\usepackage{psfrag}
% need this for including graphics (\includegraphics)
%\usepackage{graphicx}
% for neatly defining theorems and propositions
\usepackage{amsthm}
% making logically defined graphics
%%%\usepackage{xypic}

% there are many more packages, add them here as you need them

% define commands here
\newtheorem*{theorem*}{Theorem}
\newtheorem*{lemma*}{Lemma}
\newtheorem*{corollary*}{Corollary}
\newtheorem{theorem}{Theorem}
\newtheorem{lemma}{Lemma}
\newtheorem{corollary}{Corollary}


\begin{document}
\PMlinkescapeword{direct sum}
\PMlinkescapeword{image}
\PMlinkescapeword{component}
\PMlinkescapeword{projection}
Let $M$ be a finitely generated torsion-free module over a Pr\"ufer domain $R$ with field of fractions $k$. We show that $M$ is isomorphic to a \PMlinkname{direct sum}{DirectSum} of finitely generated ideals in $R$.

We shall write $k\otimes M$ for the vector space over $k$ generated by $M$. This is just the \PMlinkname{localization}{LocalizationOfAModule} of $M$ at $R\setminus\{0\}$ and, as $M$ is torsion-free, the natural map $M\rightarrow k\otimes M$ is one-to-one and we can regard $M$ as a subset of $k\otimes M$.

As $M$ is finitely generated, the vector space $k\otimes M$ will \PMlinkname{finite dimensional}{Dimension2}, and we use induction on its dimension $n$.
Supposing that $n>0$, choose any basis $e_1,\ldots,e_n$ and define the linear map $f\colon k\otimes M\rightarrow k$ by \PMlinkname{projection}{Projection} onto the first component,
\begin{equation*}
f(x_1e_1+\cdots+x_ne_n)=x_1.
\end{equation*}
Restricting to $M$, this gives a nonzero map $M\rightarrow k$. Furthermore, as $M$ is finitely generated, $f(M)$ will be a finitely generated fractional ideal in $k$. Choosing any nonzero $c\in R$ such that $\mathfrak{a}\equiv cf(M)\subseteq R$,
\begin{equation*}
g\colon M\rightarrow\mathfrak{a},\ g(u)=cf(u)
\end{equation*}
defines a homorphism from $M$ onto the nonzero and finitely generated ideal $\mathfrak{a}$. As $R$ is Pr\"ufer and invertible ideals are projective, $g$ has a right-inverse $h\colon\mathfrak{a}\rightarrow M$.
Then $h$ has the left-inverse $g$ and is one-to-one, so defines an isomorphism between $\mathfrak{a}$ and its \PMlinkname{image}{ImageOfALinearTransformation}. We decompose $M$ as the direct sum of the kernel of $g$ and the image of $h$,
\begin{equation*}
M = \operatorname{ker}(g)\oplus \operatorname{Im}(h)\cong\operatorname{ker}(g)\oplus \mathfrak{a}.
\end{equation*}
Projection from the finitely generated module $M$ onto $\operatorname{ker}(g)$ shows that it is finitely generated and,
\begin{equation*}
\operatorname{dim}(k\otimes\operatorname{ker}(g))=\operatorname{dim}(k\otimes M)-\operatorname{dim}(k\otimes\mathfrak{a})=n-1.
\end{equation*}
So, the result follows from applying the induction hypothesis to $\operatorname{ker}(g)$.

%%%%%
%%%%%
\end{document}
