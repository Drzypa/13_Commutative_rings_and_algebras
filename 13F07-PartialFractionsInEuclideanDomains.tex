\documentclass[12pt]{article}
\usepackage{pmmeta}
\pmcanonicalname{PartialFractionsInEuclideanDomains}
\pmcreated{2013-03-22 15:40:18}
\pmmodified{2013-03-22 15:40:18}
\pmowner{stevecheng}{10074}
\pmmodifier{stevecheng}{10074}
\pmtitle{partial fractions in Euclidean domains}
\pmrecord{4}{37612}
\pmprivacy{1}
\pmauthor{stevecheng}{10074}
\pmtype{Result}
\pmcomment{trigger rebuild}
\pmclassification{msc}{13F07}
\pmsynonym{partial fraction decomposition in Euclidean domains}{PartialFractionsInEuclideanDomains}

\endmetadata

\usepackage{amssymb}
\usepackage{amsmath}
\usepackage{amsfonts}
\usepackage{amsthm}
\usepackage{enumerate}

% used for TeXing text within eps files
%\usepackage{psfrag}
% need this for including graphics (\includegraphics)
%\usepackage{graphicx}
% making logically defined graphics
%%%\usepackage{xypic}

\providecommand{\abs}[1]{\lvert#1\rvert}
\providecommand{\defnterm}[1]{\emph{#1}}

\newtheorem{thm}{Theorem}
\begin{document}
This entry states and proves the existence of partial fraction decompositions
on an Euclidean domain.

In the following, we use $\nu$ to denote the Euclidean valuation function
of an Euclidean domain $E$, with the convention that $\nu(0) = -\infty$.

For a gentle introduction:
\begin{enumerate}
\item
See \PMlinkname{partial fractions of fractional numbers}{PartialFractions} for the case when $E$ consists of the
integers and $\nu(k) = \abs{k}$
for $k \neq 0$.
\item
See partial fractions of expressions for the case when $E$ 
consists of polynomials over the complex field,
with $\nu(p)$ being the degree of the polynomial $p$.
\item
See partial fractions for polynomials for the case when $E$ is
the ring of polynomials over any field, and $\nu$ is the degree
of polynomials.
\end{enumerate}

\begin{thm}
\label{thm:rel-prime}
Let $p$, $q_1 \neq 0$ and $q_2 \neq 0$ be elements of an Euclidean domain $E$,
with $q_1$ and $q_2$ be relatively prime.
Then there exist $\alpha_1$ and $\alpha_2$ in $E$
such
that
\[
\frac{p}{q_1 \, q_2} = \frac{\alpha_1}{q_1} + \frac{\alpha_2}{q_2}\,.
\]
\begin{proof}
By the Euclidean algorithm, we can obtain elements $s_1$ and $s_2$ in $E$
such that
\[
1 = s_1 \, q_1 + s_2 \, q_2\,.
\]
Then
\[
\frac{p}{q_1 \, q_2} = \frac{p \, s_2 }{q_1} + \frac{p \, s_1 }{q_2}\,,
\]
so we can take $\alpha_1 = p \, s_2$ and $\alpha_2 = p \, s_1$.
\end{proof}
\end{thm}

\begin{thm}
\label{thm:powers}
Let $p$ and $q \neq 0$ be elements of an Euclidean domain $E$,
and $n$ be any positive integer.
Then there exist elements
$\alpha_1, \dotsc, \alpha_n, \beta$ in $E$ such that
\begin{equation*}
\frac{p}{q^n} = \beta + \frac{\alpha_1}{q} + \frac{\alpha_2}{q^2} + \dotsb + \frac{\alpha_n}{q^n}\,, \quad
\nu (\alpha_j) < \nu (q)\,.
\end{equation*}

\begin{proof}
Let $r_0 = p$.
Iterating through $k = 1, \dotsc, n$ in order, 
using the division algorithm, 
we can find elements $r_k$ and $s_k$
such that
\[
r_{k-1} = r_k \, q + s_k\,, \quad \nu (s_k) < \nu (q)\,.
\]
Then
\begin{align*}
p = r_0 &= r_1 \, q + s_1 \\
&= (r_2 q + s_2) \, q + s_1 \\
&= \hdots \\
&= r_n \, q^n + s_n \, q^{n-1} + s_{n-1} \, q^{n-2} + \dotsb + s_2 \, q + s_1 \\
\frac{p}{q^n} &= r_n + \frac{s_n}{q} + \frac{s_{n-1}}{q^2} + \dotsb + \frac{s_2}{q^{n-1}} + \frac{s_1}{q^n}\,.
\end{align*}
So set $\beta = r_n$ and $\alpha_j = s_{n-j + 1}$.
\end{proof}
\end{thm}

\begin{thm}
\label{thm:partial-fractions}
Let $p$ and $q \neq 0$ be elements of an Euclidean domain $E$.
Let $q = \phi_1^{n_1} \, \phi_2^{n_2} \, \dotsb \, \phi_k^{n_k}$
be a factorization of $q$ to prime factors $\phi_i$.
Then there exist elements $\alpha_{ij}, \beta$ in $E$
such
that
\begin{equation*}
\frac{p}{q} = \beta + \sum_{i=1}^k \sum_{j=1}^{n_i} \frac{\alpha_{ij}}{\phi_i^j}\,, \quad 
\nu (\alpha_{ij}) < \nu (\phi_i)\,.
\end{equation*}

\begin{proof}
Apply Theorem \ref{thm:rel-prime} inductively to obtain
elements $s_i$ in $E$ such that
\[
\frac{p}{q} = \sum_{i=1}^k \frac{s_i}{\phi_i^{n_i}}
\]
(the factors $\phi_i$ are relatively prime).
Then apply Theorem \ref{thm:powers} to obtain elements
$\alpha_{ij}$ and $\beta_i$ in $E$ such that
\[
\frac{s_i}{\phi_i^{n_i}} = \beta_i + \sum_{j=1}^{n_i} \frac{\alpha_{ij}}{\phi_i^j}
\]
with $\nu (\alpha_{ij}) < \nu (\phi_i)$.
Take $\beta = \beta_1 + \dotsb + \beta_k$.
\end{proof}
\end{thm}
%%%%%
%%%%%
\end{document}
