\documentclass[12pt]{article}
\usepackage{pmmeta}
\pmcanonicalname{EvaluationHomomorphism}
\pmcreated{2013-03-22 14:13:51}
\pmmodified{2013-03-22 14:13:51}
\pmowner{mathcam}{2727}
\pmmodifier{mathcam}{2727}
\pmtitle{evaluation homomorphism}
\pmrecord{6}{35671}
\pmprivacy{1}
\pmauthor{mathcam}{2727}
\pmtype{Theorem}
\pmcomment{trigger rebuild}
\pmclassification{msc}{13P05}
\pmclassification{msc}{11C08}
\pmclassification{msc}{12E05}
\pmsynonym{substitution homomorphism}{EvaluationHomomorphism}
\pmrelated{LectureNotesOnPolynomialInterpolation}
\pmdefines{evaluation homomorphism}

% this is the default PlanetMath preamble.  as your knowledge
% of TeX increases, you will probably want to edit this, but
% it should be fine as is for beginners.

% almost certainly you want these
\usepackage{amssymb}
\usepackage{amsmath}
\usepackage{amsfonts}

% used for TeXing text within eps files
%\usepackage{psfrag}
% need this for including graphics (\includegraphics)
%\usepackage{graphicx}
% for neatly defining theorems and propositions
\usepackage{amsthm}
% making logically defined graphics
%%%\usepackage{xypic}

% there are many more packages, add them here as you need them

% define commands here

\newtheorem{theorem}{Theorem}
\newtheorem{defn}{Definition}
\newtheorem{prop}{Proposition}
\newtheorem{lemma}{Lemma}
\newtheorem{cor}{Corollary}
\begin{document}
Let $R$ be a commutative ring and let $R[X]$ be the ring of polynomials with coefficients in $R$. 

\begin{theorem}
Let $S$ be a commutative ring, and let $\psi\colon R\to S$ be a homomorphism.  Further, let $s\in S$.  Then there is a unique homomorphism $\phi\colon R[X]\to S$ taking $X$ to $s$ and taking every $r\in R$ to $\psi(r)$. 
\end{theorem}

This amounts to saying that polynomial rings are free objects in the category of $R$-algebras; the theorem then states that they are projective.  This is true in much greater generality; in fact, the property of being projective is intended to extract the essential property of being free.

\begin{proof}
We first prove existence.  Let $f\in R[X]$.  Then by definition there is some finite list of $a_i$ such that $f = \sum_i a_i X^i$.  Then define $\phi(f)$ to be $\sum_i \psi(a_i) s^i$.  It is clear from the definition of addition and multiplication on polynomials that $\phi$ is a homomorphism; the definition makes it clear that $\phi(X)=s$ and $\phi(r)=\psi(r)$. 

Now, to show uniqueness, suppose $\gamma$ is any homomorphism satisfying the conditions of the theorem, and let $f\in R[X]$.  Write $f = \sum_i a_i X^i$ as before.  Then $\gamma(a_i) = \psi(a_i)$ and $\gamma(s)$ by assumption.  But then since $\gamma$ is a homomorphism, $\gamma(a_iX^i) = \psi(a_i)s^i$ and $\gamma(f) = \sum_i \psi(a_i) s^i = \phi(f)$.
\end{proof}
%%%%%
%%%%%
\end{document}
