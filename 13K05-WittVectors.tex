\documentclass[12pt]{article}
\usepackage{pmmeta}
\pmcanonicalname{WittVectors}
\pmcreated{2013-03-22 15:14:31}
\pmmodified{2013-03-22 15:14:31}
\pmowner{alozano}{2414}
\pmmodifier{alozano}{2414}
\pmtitle{Witt vectors}
\pmrecord{5}{37017}
\pmprivacy{1}
\pmauthor{alozano}{2414}
\pmtype{Definition}
\pmcomment{trigger rebuild}
\pmclassification{msc}{13K05}
\pmclassification{msc}{13J10}
\pmdefines{Witt polynomials}

% this is the default PlanetMath preamble.  as your knowledge
% of TeX increases, you will probably want to edit this, but
% it should be fine as is for beginners.

% almost certainly you want these
\usepackage{amssymb}
\usepackage{amsmath}
\usepackage{amsthm}
\usepackage{amsfonts}

% used for TeXing text within eps files
%\usepackage{psfrag}
% need this for including graphics (\includegraphics)
%\usepackage{graphicx}
% for neatly defining theorems and propositions
%\usepackage{amsthm}
% making logically defined graphics
%%%\usepackage{xypic}

% there are many more packages, add them here as you need them

% define commands here

\newtheorem{thm}{Theorem}
\newtheorem{defn}{Definition}
\newtheorem{prop}{Proposition}
\newtheorem{lemma}{Lemma}
\newtheorem{cor}{Corollary}

\theoremstyle{definition}
\newtheorem{exa}{Example}

% Some sets
\newcommand{\Nats}{\mathbb{N}}
\newcommand{\Ints}{\mathbb{Z}}
\newcommand{\Reals}{\mathbb{R}}
\newcommand{\Complex}{\mathbb{C}}
\newcommand{\Rats}{\mathbb{Q}}
\newcommand{\Gal}{\operatorname{Gal}}
\newcommand{\Cl}{\operatorname{Cl}}
\begin{document}
In this entry we define a commutative ring, the Witt vectors, which is particularly useful in number theory, algebraic geometry and other areas of commutative algebra. The Witt vectors are named after Ernst Witt.

\begin{thm}
Let $p$ be a prime and let $\mathbb{K}$ be a perfect ring of characteristic $p$. There exists a unique \PMlinkname{strict $p$-ring}{PRing} $W(\mathbb{K})$ with residue ring $\mathbb{K}$. 
\end{thm}

\begin{defn}
Let $\mathbb{K}$ be a perfect ring of characteristic $p$. The unique \PMlinkname{strict $p$-ring}{PRing} $W(\mathbb{K})$ with residue ring $\mathbb{K}$ is called the ring of Witt vectors with coefficients in $\mathbb{K}$.
\end{defn}

Next, we give an explicit construction of the Witt vectors.

\begin{defn}
Let $p$ be a prime number and let $\{ X_i\}_{i=0}^\infty$ be a sequence of indeterminates. The polynomials $W_n\in\Ints[X_1,\ldots,X_n]$ given by:
\begin{eqnarray*}
W_0 &=& X_0,\\
W_1 &=& X_0^p+pX_1,\\
W_n &=& X_0^{p^n}+pX_1^{p^{n-1}}+\ldots+p^nX_n=\sum_{i=0}^n p^iX_i^{p^{n-i}}.
\end{eqnarray*}
are called the Witt polynomials.
\end{defn}

\begin{prop}
Let $\{X_i\},\ \{Y_i\}$ be two sequences of indeterminates. For every polynomial in two variables $Q(U,V)\in \Ints[U,V]$ there exist polynomials $\{t_i\}_{i=0}^\infty$ in the variables $\{X_i\}$ and $\{Y_i\}$
$$t_i \in \Ints[\{X_i\},\{Y_i\}]$$
such that
$$W_n(t_0,t_1,t_2,\ldots,t_n)=Q(W_n(X_0,X_1,\ldots),W_n(Y_0,Y_1,\ldots))$$
for all $n\geq 0$.
\end{prop}
\begin{proof}
See \cite{serre}, p. 40.
\end{proof}

Let $S_0,\ S_1,\ S_2,\ldots$ (resp. $P_0,\ P_1,\ P_2,\ldots$) be the polynomials 
$t_0,\ t_1,\ t_2,\ldots$ associated with $Q(U,V)=U+V$ (resp. $Q(U,V)=U\cdot V$) given by the previous proposition. We will use the polynomials $S_i$, $P_i$ to define the addition and multiplication in a new ring. In the following proposition, the notation $R^\infty$ stands for the set of all sequences $(r_1,r_2,\ldots)$ of elements in $R$.

\begin{thm}
Let $\mathbb{K}$ be a perfect ring of characteristic $p$. We define a ring $W=(\mathbb{K}^\infty,+,\cdot)$ where the addition and multiplication, for $k,h \in \mathbb{K}^\infty$, are defined by:
$$k+h=(S_0(k,h),S_1(k,h),\ldots),\quad k\cdot h =(P_0(k,h),P_1(k,h),\ldots).$$
Then the ring $W$ concides with $W(\mathbb{K})$, the ring of Witt vectors with coefficients in $\mathbb{K}$.
\end{thm}

\begin{defn}
Let $\mathbb{K}$ be a perfect ring of characteristic $p$. We define the ring of Witt vectors of length $n$ (over $\mathbb{K}$) to be the ring $W_n(\mathbb{K})=(\mathbb{K}^{n-1},+,\cdot)$, where, for $k,h \in \mathbb{K}^{n-1}$:
$$ k+h=(S_0(k,h),\ldots,S_{n-1}(k,h)),\quad k\cdot h=(P_0(k,h),\ldots,P_{n-1}(k,h)).$$
\end{defn}

It is clear from the definitions that:
$$W(\mathbb{K})=\varprojlim W_n(\mathbb{K})$$
In words, $W(\mathbb{K})$ is the projective limit of the Witt vectors of finite length.

\begin{exa}
Let $\mathbb{K}=\mathbb{F}_p$. Then $W_n(\mathbb{F}_p)=\Ints/p^n\Ints$. Thus:
$$W(\mathbb{F}_p)=\Ints_p,$$
the ring of \PMlinkname{$p$-adic integers}{PAdicIntegers}.
\end{exa}
\begin{thebibliography}{9}
\bibitem{serre} J. P. Serre, {\em Local Fields},
Springer-Verlag, New York.
\end{thebibliography}
%%%%%
%%%%%
\end{document}
