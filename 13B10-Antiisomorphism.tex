\documentclass[12pt]{article}
\usepackage{pmmeta}
\pmcanonicalname{Antiisomorphism}
\pmcreated{2013-03-22 16:01:08}
\pmmodified{2013-03-22 16:01:08}
\pmowner{Mathprof}{13753}
\pmmodifier{Mathprof}{13753}
\pmtitle{anti-isomorphism}
\pmrecord{15}{38057}
\pmprivacy{1}
\pmauthor{Mathprof}{13753}
\pmtype{Definition}
\pmcomment{trigger rebuild}
\pmclassification{msc}{13B10}
\pmclassification{msc}{16B99}
\pmdefines{anti-endomorphism}
\pmdefines{anti-homomorphism}
\pmdefines{anti-isomorphic}
\pmdefines{anti-automorphism}

% this is the default PlanetMath preamble.  as your knowledge
% of TeX increases, you will probably want to edit this, but
% it should be fine as is for beginners.

% almost certainly you want these
\usepackage{amssymb}
\usepackage{amsmath}
\usepackage{amsfonts}

% used for TeXing text within eps files
%\usepackage{psfrag}
% need this for including graphics (\includegraphics)
%\usepackage{graphicx}
% for neatly defining theorems and propositions
%\usepackage{amsthm}
% making logically defined graphics
%%%\usepackage{xypic}

% there are many more packages, add them here as you need them

% define commands here

\begin{document}
Let $R$ and $S$ be rings and $f: R\longrightarrow S$ be a function such 
that $f(r_{1}r_{2}) = f(r_{2})f(r_{1})$ for all $r_{1}, r_{2} \in R$.


If $f$ is a homomorphism of the additive groups of $R$ and $S$,
then $f$ is called an {\it anti-homomorphsim}.

If $f$ is a bijection and anti-homomorphism,
then $f$ is called an {\it anti-isomorphism}.

If $f$ is an anti-homomorphism and $R=S$
then $f$ is called an {\it anti-endomorphism}.

If $f$ is an anti-isomorphism and $R=S$
then $f$ is called an {\it anti-automorphism}.


As an example, when $m \neq n$, the mapping that sends a matrix to its transpose
(or to its conjugate transpose if the matrix is complex) is an anti-isomorphism
of $M_{m,n} \to M_{n,m}$. 


$R$ and $S$ are \emph{anti-isomorphic} if there is an anti-isomorphism $R \to S$.

All of the things defined in this entry are also defined for groups.



%%%%%
%%%%%
\end{document}
