\documentclass[12pt]{article}
\usepackage{pmmeta}
\pmcanonicalname{EisensteinCriterion}
\pmcreated{2013-03-22 12:16:32}
\pmmodified{2013-03-22 12:16:32}
\pmowner{Daume}{40}
\pmmodifier{Daume}{40}
\pmtitle{Eisenstein criterion}
\pmrecord{13}{31724}
\pmprivacy{1}
\pmauthor{Daume}{40}
\pmtype{Theorem}
\pmcomment{trigger rebuild}
\pmclassification{msc}{13A05}
\pmsynonym{Eisenstein irreducibility criterion}{EisensteinCriterion}
\pmrelated{GausssLemmaII}
\pmrelated{IrreduciblePolynomial2}
\pmrelated{Monic2}
\pmrelated{AlternativeProofThatSqrt2IsIrrational}

\endmetadata

% this is the default PlanetMath preamble.  as your knowledge
% of TeX increases, you will probably want to edit this, but
% it should be fine as is for beginners.

% almost certainly you want these
\usepackage{amssymb}
\usepackage{amsmath}
\usepackage{amsfonts}

% used for TeXing text within eps files
%\usepackage{psfrag}
% need this for including graphics (\includegraphics)
%\usepackage{graphicx}
% for neatly defining theorems and propositions
\usepackage{amsthm}
% making logically defined graphics
%%%\usepackage{xypic} 

% there are many more packages, add them here as you need them

% define commands here

% The below lines should work as the command
% \renewcommand{\bibname}{References}
% without creating havoc when rendering an entry in
% the page-image mode.
\makeatletter
\@ifundefined{bibname}{}{\renewcommand{\bibname}{References}}
\makeatother
\begin{document}
\PMlinkescapeword{index}
\PMlinkescapeword{divide}

\newtheorem*{thm}{Theorem}
\begin{thm}[Eisenstein criterion]
Let $f$ be a primitive polynomial over a commutative unique factorization domain $R$, say
$$f(x)=a_0 + a_1x + a_2x^2 + \ldots + a_nx^n \;.$$
If $R$ has an irreducible element $p$ such that
$$p\mid a_m \qquad 0\le m\le n-1$$
$$p^2 \nmid a_0$$
$$p \nmid a_n$$
then $f$ is irreducible.
\end{thm}

\begin{proof}
Suppose
$$f=(b_0 + \ldots + b_s x^s)(c_0 + \ldots + c_t x^t)$$
where $s>0$ and $t>0$. Since $a_0 = b_0 c_0$, we know that $p$ divides one but not both of $b_0$ and $c_0$; suppose $p \mid c_0$. By hypothesis, not all the $c_m$ are divisible by $p$; let $k$ be the smallest index such that $p\nmid c_k$. We have $a_k = b_0 c_k + b_1 c_{k-1} + \ldots + b_k c_0$.
We also have $p\mid a_k$, and $p$ divides every summand except one on the right side, which yields a contradiction. QED
\end{proof}
%%%%%
%%%%%
\end{document}
