\documentclass[12pt]{article}
\usepackage{pmmeta}
\pmcanonicalname{OrdinaryGeneratingFunctionForLinearRecursiveRelations}
\pmcreated{2013-03-22 19:20:07}
\pmmodified{2013-03-22 19:20:07}
\pmowner{joking}{16130}
\pmmodifier{joking}{16130}
\pmtitle{ordinary generating function for linear recursive relations}
\pmrecord{7}{42282}
\pmprivacy{1}
\pmauthor{joking}{16130}
\pmtype{Theorem}
\pmcomment{trigger rebuild}
\pmclassification{msc}{13H05}
\pmclassification{msc}{13B35}
\pmclassification{msc}{13F25}
\pmclassification{msc}{13J05}

% this is the default PlanetMath preamble.  as your knowledge
% of TeX increases, you will probably want to edit this, but
% it should be fine as is for beginners.

% almost certainly you want these
\usepackage{amssymb}
\usepackage{amsmath}
\usepackage{amsfonts}

% used for TeXing text within eps files
%\usepackage{psfrag}
% need this for including graphics (\includegraphics)
%\usepackage{graphicx}
% for neatly defining theorems and propositions
\usepackage{amsthm}
% making logically defined graphics
%%%\usepackage{xypic}

% there are many more packages, add them here as you need them

% define commands here

\begin{document}
Let $(a_n)$ be a linear recursive sequence with values in a (commutative) ring $R$, i.e. there exist constants $\beta_1,\ldots,\beta_k\in R$ such that for any $n>k$ we have
$$a_n=\beta_1\cdot a_{n-1} + \cdots + \beta_i\cdot a_{n-i} + \cdots + \beta_{k}\cdot a_{n-k}.$$
Now consider the ordinary generating function for this sequence
$$f(t)=\sum_{n\geqslant 0}a_n\cdot t^n$$
which is a formal power series in the ring of formal power series $R[[t]]$. We will try to find the closed form of $f(t)$.

First write down first $k$ elements of this power series:
$$f(t)=\sum_{n=0}^{k}a_n\cdot t^n + \sum_{n>k}a_n\cdot t^n$$
and note that for $n>k$ we can use our recursive relation:
$$f(t)=\sum_{n=0}^{k}a_n\cdot t^n + \sum_{n>k}(\beta_1\cdot a_{n-1}+\cdots +\beta_k\cdot a_{n-k})\cdot t^n$$
which gives us
\begin{align}
f(t)=\sum_{n=0}^{k}a_n\cdot t^n + \left(\beta_1\cdot t\cdot \sum_{n>k}a_{n-1}\cdot t^{n-1}\right) +\cdots +\left(\beta_k\cdot t^k\cdot \sum_{n>k}a_{n-k}\cdot t^{n-k}\right).
\end{align}
Now focus on those sums on the right. For any $j$ we have
$$\sum_{n>k}a_{n-j}\cdot t^{n-j} = \left(\sum_{n\geqslant 0}a_n\cdot t^n \right) - \left(\sum_{n=0}^{k-j}a_n\cdot t^n\right) = f(t) - \left(\sum_{n=0}^{k-j}a_n\cdot t^n\right).$$
Inserting this into (1) gives us
$$f(t) = \sum_{n=0}^{k}a_n\cdot t^n + \left(\beta_1\cdot t\cdot(f(t)-\sum_{n=0}^{k-1}a_n\cdot t^n)\right) + \cdots + \left(\beta_k\cdot t^k\cdot(f(t)-\sum_{n=0}^{0}a_n\cdot t^n)\right)$$
which can be simplified as follows:
$$f(t) = \sum_{n=0}^{k}a_n\cdot t^n  + \sum_{j=1}^{k}\left(\beta_j\cdot t^j\cdot(f(t)-\sum_{n=0}^{k-j}a_n\cdot t^n)\right).$$
Taking the components with $f(t)$ to the left side gives us
$$f(t)-\sum_{j=1}^{k}\beta_j\cdot t^j\cdot f(t) = \sum_{n=0}^{k}a_n\cdot t^n  - \sum_{j=1}^{k}\left(\beta_j\cdot t^j\cdot\sum_{n=0}^{k-j}a_n\cdot t^n\right),$$
which finally gives us the closed formula for $f(t)$ (note that all sums are finite):
$$f(t)=\frac{\sum\limits_{n=0}^{k}a_n\cdot t^n - \sum\limits_{j=1}^{k}\left(\beta_j\cdot t^j\cdot\sum\limits_{n=0}^{k-j}a_n\cdot t^n\right)}{1-\sum\limits_{j=1}^{k}\beta_j\cdot t^j}.$$
Note that in the denominator we have a power series with the constant term $1$, thus (by general properties of formal power series) it is invertible in $R[[t]]$. Thus we proved the following theorem:

\textbf{Theorem.} If $(a_n)$ is a recursive sequnce given by 
$$a_n=\beta_1\cdot a_{n-1}+\cdots+\beta_k\cdot a_{n-k}$$
for all $n>k$ then the ordinary generating function
$$f(t)=\sum_{n\geqslant 0}a_n\cdot t^n$$
has its closed form given by
$$f(t)=\frac{\sum\limits_{n=0}^{k}a_n\cdot t^n  - \sum\limits_{j=1}^{k}\left(\beta_j\cdot t^j\cdot\sum\limits_{n=0}^{k-j}a_n\cdot t^n\right)}{1-\sum\limits_{j=1}^{k}\beta_j\cdot t^j}.$$

\textbf{Remark.} Note that if we replace $R$ with (for example) the reals $\mathbb{R}$ then the theorem is still valid if we treat those power series as functions. Of course such equalites hold only there where those functions are defined.
%%%%%
%%%%%
\end{document}
