\documentclass[12pt]{article}
\usepackage{pmmeta}
\pmcanonicalname{ProofOfHilbertsNullstellensatz}
\pmcreated{2013-03-22 15:27:46}
\pmmodified{2013-03-22 15:27:46}
\pmowner{pbruin}{1001}
\pmmodifier{pbruin}{1001}
\pmtitle{proof of Hilbert's Nullstellensatz}
\pmrecord{4}{37314}
\pmprivacy{1}
\pmauthor{pbruin}{1001}
\pmtype{Proof}
\pmcomment{trigger rebuild}
\pmclassification{msc}{13A10}

\endmetadata

% this is the default PlanetMath preamble.  as your knowledge
% of TeX increases, you will probably want to edit this, but
% it should be fine as is for beginners.

% almost certainly you want these
\usepackage{amssymb}
\usepackage{amsmath}
\usepackage{amsfonts}

% used for TeXing text within eps files
%\usepackage{psfrag}
% need this for including graphics (\includegraphics)
%\usepackage{graphicx}
% for neatly defining theorems and propositions
%\usepackage{amsthm}
% making logically defined graphics
%%%\usepackage{xypic}

% there are many more packages, add them here as you need them

% define commands here
\begin{document}
Let $K$ be an algebraically closed field, let $n\ge 0$, and let $I$ be an ideal of the polynomial ring $K[x_1,\ldots,x_n]$.  Let $f\in K[x_1,\ldots,x_n]$ be
a polynomial with the property that
$$
f(a_1,\ldots,a_n)=0\hbox{ for all }(a_1,\ldots,a_n)\in V(I).
$$
Suppose that $f^r\not\in I$ for all $r>0$; in particular, $I$ is strictly smaller than $K[x_1,\ldots,x_n]$ and $f\ne 0$.  Consider the ring
$$
R=K[x_1,\ldots,x_n,1/f]\subset K(x_1,\ldots,x_n).
$$
The $R$-ideal $RI$ is strictly smaller than $R$, since
$$
RI=\bigcup_{r=0}^\infty f^{-r}I
$$
does not contain the unit element.  Let $y$ be an indeterminate over
$K[x_1,\ldots,x_n]$, and let $J$ be the inverse image of $RI$ under
the homomorphism
$$
\phi\colon K[x_1,\ldots,x_n,y]\to R
$$
acting as the identity on $K[x_1,\ldots,x_n]$ and sending $y$ to
$1/f$.  Then $J$ is strictly smaller than $K[x_1,\ldots,x_n,y]$, so
the weak Nullstellensatz gives us an element $(a_1,\ldots,a_n,b)\in                       
K^{n+1}$ such that $g(a_1,\ldots,a_n,b)=0$ for all $g\in J$.  In
particular, we see that $g(a_1,\ldots,a_n)=0$ for all $g\in I$.  Our
assumption on $f$ therefore implies $f(a_1,\ldots,a_n)=0$.  However,
$J$ also contains the element $1-yf$ since $\phi$ sends this element
to zero.  This leads to the following contradiction:
$$
0=(1-yf)(a_1,\ldots,a_n,b)=1-bf(a_1,\ldots,a_n)=1.
$$
The assumption that $f^r\not\in I$ for all $r>0$ is therefore false,
i.e.~there is an $r>0$ with $f^r\in I$.
%%%%%
%%%%%
\end{document}
