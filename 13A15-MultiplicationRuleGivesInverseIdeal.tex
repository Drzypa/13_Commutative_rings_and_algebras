\documentclass[12pt]{article}
\usepackage{pmmeta}
\pmcanonicalname{MultiplicationRuleGivesInverseIdeal}
\pmcreated{2013-03-22 15:24:16}
\pmmodified{2013-03-22 15:24:16}
\pmowner{pahio}{2872}
\pmmodifier{pahio}{2872}
\pmtitle{multiplication rule gives inverse ideal}
\pmrecord{5}{37243}
\pmprivacy{1}
\pmauthor{pahio}{2872}
\pmtype{Theorem}
\pmcomment{trigger rebuild}
\pmclassification{msc}{13A15}
\pmclassification{msc}{16D25}
\pmrelated{PruferRing}
\pmrelated{Characterization}

\endmetadata

% this is the default PlanetMath preamble.  as your knowledge
% of TeX increases, you will probably want to edit this, but
% it should be fine as is for beginners.

% almost certainly you want these
\usepackage{amssymb}
\usepackage{amsmath}
\usepackage{amsfonts}

% used for TeXing text within eps files
%\usepackage{psfrag}
% need this for including graphics (\includegraphics)
%\usepackage{graphicx}
% for neatly defining theorems and propositions
 \usepackage{amsthm}
% making logically defined graphics
%%%\usepackage{xypic}

% there are many more packages, add them here as you need them

% define commands here

\theoremstyle{definition}
\newtheorem*{thmplain}{Theorem}
\begin{document}
\begin{thmplain}
Let $R$ be a commutative ring with non-zero unity.\, If an ideal\, $(a,\,b)$\, of $R$, with $a$ or $b$ \PMlinkname{regular}{RegularElement}, obeys the multiplication rule
\begin{align}
      (a,\,b)(c,\,d) = (ac,\,ad\!+\!bc,\,bd)
\end{align}
with all ideals $(c,\,d)$\, of $R$, then\, $(a,\,b)$ is an invertible ideal.
\end{thmplain}

{\em Proof.}\, The rule gives
$$(a,\,b)^2 = (a,\,-b)(a,\,b) = (a^2,\,ab\!-\!ba,\,b^2) = (a^2,\,b^2).$$
Thus the product $ab$ may be written in the form
$$ab = ua^2\!+\!vb^2,$$
where $u$ and $v$ are elements of $R$.\, Let's assume that e.g. $a$ is regular.\, Then $a$ has the multiplicative inverse $a^{-1}$ in the total ring of fractions $R$.\, Again applying the rule yields
$$(a,\,b)(va,\,a-vb)(a^{-2}) = (va^2,\,a^2-vab+vab,\,ab-vb^2)(a^{-2}) =
 (va^2,\,a^2,\,ua^2)(a^{-2}) = (v,\,1,\,u) = R.$$
Consequently the ideal\, $(a,\,b)$\, has an inverse ideal (which may be a \PMlinkname{fractional ideal}{FractionalIdealOfCommutativeRing}); this settles the proof.

\textbf{Remark.}\, The rule (1) in the theorem may be replaced with the rule
\begin{align}
(a,\,b)(c,\,d) = (ac,\,(a\!+\!b)(c\!+\!d),\,bd)
\end{align}
as is seen from the identical equation\, $(a\!+\!b)(c\!+\!d)\!-\!ac\!-\!bd = ad+bc$.
%%%%%
%%%%%
\end{document}
