\documentclass[12pt]{article}
\usepackage{pmmeta}
\pmcanonicalname{APolynomialOfDegreeNOverAFieldHasAtMostNRoots}
\pmcreated{2013-03-22 15:09:01}
\pmmodified{2013-03-22 15:09:01}
\pmowner{alozano}{2414}
\pmmodifier{alozano}{2414}
\pmtitle{a polynomial of degree $n$ over a field has at most $n$ roots}
\pmrecord{5}{36897}
\pmprivacy{1}
\pmauthor{alozano}{2414}
\pmtype{Theorem}
\pmcomment{trigger rebuild}
\pmclassification{msc}{13P05}
\pmclassification{msc}{11C08}
\pmclassification{msc}{12E05}
%\pmkeywords{roots}
%\pmkeywords{polynomial}
%\pmkeywords{field}
\pmrelated{Root}
\pmrelated{FactorTheorem}
\pmrelated{PolynomialCongruence}
\pmrelated{EveryPrimeHasAPrimitiveRoot}
\pmrelated{CongruenceOfArbitraryDegree}

\endmetadata

% this is the default PlanetMath preamble.  as your knowledge
% of TeX increases, you will probably want to edit this, but
% it should be fine as is for beginners.

% almost certainly you want these
\usepackage{amssymb}
\usepackage{amsmath}
\usepackage{amsthm}
\usepackage{amsfonts}

% used for TeXing text within eps files
%\usepackage{psfrag}
% need this for including graphics (\includegraphics)
%\usepackage{graphicx}
% for neatly defining theorems and propositions
%\usepackage{amsthm}
% making logically defined graphics
%%%\usepackage{xypic}

% there are many more packages, add them here as you need them

% define commands here

\newtheorem*{thm}{Theorem}
\newtheorem{defn}{Definition}
\newtheorem{prop}{Proposition}
\newtheorem*{lemma}{Lemma}
\newtheorem{cor}{Corollary}

\theoremstyle{definition}
\newtheorem{exa}{Example}

% Some sets
\newcommand{\Nats}{\mathbb{N}}
\newcommand{\Ints}{\mathbb{Z}}
\newcommand{\Reals}{\mathbb{R}}
\newcommand{\Complex}{\mathbb{C}}
\newcommand{\Rats}{\mathbb{Q}}
\newcommand{\Gal}{\operatorname{Gal}}
\newcommand{\Cl}{\operatorname{Cl}}
\begin{document}
\begin{lemma}[cf. factor theorem]
Let $R$ be a commutative ring with identity and let $p(x)\in R[x]$ be a polynomial with coefficients in $R$. The element $a\in R$ is a root of $p(x)$ if and only if $(x-a)$ divides $p(x)$.
\end{lemma}
\begin{proof}
See proof of factor theorem using division.
\end{proof}
\ \\
\begin{thm}
Let $F$ be a field and let $p(x)$ be a non-zero polynomial in $F[x]$ of degree $n\geq 0$. Then $p(x)$ has at most $n$ roots in $F$ (counted with multiplicity).
\end{thm}

\begin{proof}
We proceed by induction. The case $n=0$ is trivial since $p(x)$ is a non-zero constant, thus $p(x)$ cannot have any roots.

Suppose that any polynomial in $F[x]$ of degree $n$ has at most $n$ roots and let $p(x)\in F[x]$ be a polynomial of degree $n+1$. If $p(x)$ has no roots then the result is trivial, so let us assume that $p(x)$ has at least one root $a\in F$. Then, by the lemma above, there exist a polynomial $q(x)$ such that:
$$p(x)=(x-a)\cdot q(x).$$
Hence, $q(x)\in F[x]$ is a polynomial of degree $n$. By the induction hypothesis, the polynomial $q(x)$ has at most $n$ roots. It is clear that any root of $q(x)$ is a root of $p(x)$ and if $b\neq a$ is a root of $p(x)$ then $b$ is also a root of $q(x)$. Thus, $p(x)$ has at most $n+1$ roots, which concludes the proof of the theorem.
\end{proof}

Note: The fundamental theorem of algebra states that if $F$ is algebraically closed then any polynomial of degree $n$ has exactly $n$ roots (counted with multiplicity).
%%%%%
%%%%%
\end{document}
