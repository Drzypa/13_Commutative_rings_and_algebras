\documentclass[12pt]{article}
\usepackage{pmmeta}
\pmcanonicalname{BilinearityAndCommutativeRings}
\pmcreated{2013-03-22 17:24:19}
\pmmodified{2013-03-22 17:24:19}
\pmowner{Algeboy}{12884}
\pmmodifier{Algeboy}{12884}
\pmtitle{bilinearity and commutative rings}
\pmrecord{5}{39777}
\pmprivacy{1}
\pmauthor{Algeboy}{12884}
\pmtype{Theorem}
\pmcomment{trigger rebuild}
\pmclassification{msc}{13C99}

\usepackage{latexsym}
\usepackage{amssymb}
\usepackage{amsmath}
\usepackage{amsfonts}
\usepackage{amsthm}

%%\usepackage{xypic}

%-----------------------------------------------------

%       Standard theoremlike environments.

%       Stolen directly from AMSLaTeX sample

%-----------------------------------------------------

%% \theoremstyle{plain} %% This is the default

\newtheorem{thm}{Theorem}

\newtheorem{coro}[thm]{Corollary}

\newtheorem{lem}[thm]{Lemma}

\newtheorem{lemma}[thm]{Lemma}

\newtheorem{prop}[thm]{Proposition}

\newtheorem{conjecture}[thm]{Conjecture}

\newtheorem{conj}[thm]{Conjecture}

\newtheorem{defn}[thm]{Definition}

\newtheorem{remark}[thm]{Remark}

\newtheorem{ex}[thm]{Example}



%\countstyle[equation]{thm}



%--------------------------------------------------

%       Item references.

%--------------------------------------------------


\newcommand{\exref}[1]{Example-\ref{#1}}

\newcommand{\thmref}[1]{Theorem-\ref{#1}}

\newcommand{\defref}[1]{Definition-\ref{#1}}

\newcommand{\eqnref}[1]{(\ref{#1})}

\newcommand{\secref}[1]{Section-\ref{#1}}

\newcommand{\lemref}[1]{Lemma-\ref{#1}}

\newcommand{\propref}[1]{Prop\-o\-si\-tion-\ref{#1}}

\newcommand{\corref}[1]{Cor\-ol\-lary-\ref{#1}}

\newcommand{\figref}[1]{Fig\-ure-\ref{#1}}

\newcommand{\conjref}[1]{Conjecture-\ref{#1}}


% Normal subgroup or equal.

\providecommand{\normaleq}{\unlhd}

% Normal subgroup.

\providecommand{\normal}{\lhd}

\providecommand{\rnormal}{\rhd}
% Divides, does not divide.

\providecommand{\divides}{\mid}

\providecommand{\ndivides}{\nmid}


\providecommand{\union}{\cup}

\providecommand{\bigunion}{\bigcup}

\providecommand{\intersect}{\cap}

\providecommand{\bigintersect}{\bigcap}










\begin{document}
We show that a bilinear map $b:U\times V\to W$ is almost always definable only for commutative rings.
The exceptions lie only where non-trivial commutators act trivially on one of the
three modules.

\begin{lemma}
Let $R$ be a ring and $U,V$ and $W$ be $R$-modules.
If $b:U\times V\to W$ is $R$-bilinear then $b$ is also $R$-middle linear.
\end{lemma}
\begin{proof}
Given $r\in R$, $u\in U$ and $v\in V$ then 
$b(ru,v)=rb(u,v)$ and $b(u,rv)=rb(u,v)$ so $b(ru,v)=b(u,rv)$.
\end{proof}

\begin{thm}
Let $R$ be a ring and $U,V$ and $W$ be faithful $R$-modules.
If $b:U\times V\to W$ is $R$-bilinear and (left or right) non-degenerate, 
then $R$ must be commutative.
\end{thm}
\begin{proof}
We may assume that $b$ is left non-degenerate.
Let $r,s\in R$.  Then for all $u\in U$ and $v\in V$ it follows that
\begin{multline*}
b((sr)u,v)=sb(ru,v)=sb(u,rv)=b(su,rv)=b((rs)u,v).
\end{multline*}
Therefore $b([s,r]u,v)=0$, where $[s,r]=sr-rs$.  This makes 
$[s,r]u$ an element of the left radical of $b$ as it is true for all $v\in V$.
However $b$ is non-degenerate so the radical is trivial and so $[s,r]u=0$ for
all $u\in U$.  Since $U$ is a faithful $R$-module this makes $[s,r]=0$ for all 
$s,r\in R$.  That is, $R$ is commutative.
\end{proof}

Alternatively we can interpret the result in a weaker fashion as:
\begin{coro}
Let $R$ be a ring and $U,V$ and $W$ be $R$-modules.
If $b:U\times V\to W$ is $R$-bilinear with $W=\langle b(U,V)\rangle$ then 
every element $[R,R]$ acts trivially
on one of the three modules $U$, $V$ or $W$.
\end{coro}
\begin{proof}
Suppose $[r,s]\in [R,R]$, $[r,s]U\neq 0$ and $[r,s]V\neq 0$.  Then we have shown
$0=b([r,s]u,v)=[r,s]b(u,v)$ for all $u\in U$ and $v\in V$.  
As $W=\langle b(U,V)\rangle$ it follows that $[r,s]W=0$.
\end{proof}

Whenever a non-commutative ring is required for a biadditive map $U\times V\to W$ 
it is therefore often preferable to use a scalar map instead.
%%%%%
%%%%%
\end{document}
