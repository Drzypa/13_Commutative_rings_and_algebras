\documentclass[12pt]{article}
\usepackage{pmmeta}
\pmcanonicalname{TopicEntryOnRationalNumbers}
\pmcreated{2013-03-22 19:08:15}
\pmmodified{2013-03-22 19:08:15}
\pmowner{pahio}{2872}
\pmmodifier{pahio}{2872}
\pmtitle{topic entry on rational numbers}
\pmrecord{12}{42036}
\pmprivacy{1}
\pmauthor{pahio}{2872}
\pmtype{Topic}
\pmcomment{trigger rebuild}
\pmclassification{msc}{13B30}
\pmclassification{msc}{11A99}
\pmclassification{msc}{03E99}
\pmsynonym{entries on rational numbers}{TopicEntryOnRationalNumbers}
%\pmkeywords{rational number}
\pmrelated{TopicEntryOnRealNumbers}

% this is the default PlanetMath preamble.  as your knowledge
% of TeX increases, you will probably want to edit this, but
% it should be fine as is for beginners.

% almost certainly you want these
\usepackage{amssymb}
\usepackage{amsmath}
\usepackage{amsfonts}

% used for TeXing text within eps files
%\usepackage{psfrag}
% need this for including graphics (\includegraphics)
%\usepackage{graphicx}
% for neatly defining theorems and propositions
 \usepackage{amsthm}
% making logically defined graphics
%%%\usepackage{xypic}

% there are many more packages, add them here as you need them

% define commands here

\theoremstyle{definition}
\newtheorem*{thmplain}{Theorem}

\begin{document}
\section*{PlanetMath articles concerning the rational numbers}


 

    \subsection*{Definition}
\begin{itemize}

\item rational number

\item fraction

\end{itemize}


    \subsection*{Properties}
\begin{itemize}

\item commensurable numbers

\item rational algebraic integers

\end{itemize}

    \subsection*{Forms}
\begin{itemize}

\item partial fractions

\item any rational number is a sum of unit fractions

\item Egyptian fractions

\item mixed fraction

\item decimal fraction

\item decimal expansion

\item order valuation

\item continued fraction

\end{itemize}


    \subsection*{Where are rational numbers?}
\begin{itemize}

\item rational points on one dimensional sphere

\item rational sine and cosine

\item rational Briggsian logarithms of integers

\item irrational to an irrational power can be rational

\item square roots of rationals

\item Roth's theorem

\item rational root theorem

\item canonical form of element of number field

\item \PMlinkid{solutions of the equation}{12127}\, $x^y = y^x$

\end{itemize}


    \subsection*{Sets}
\begin{itemize}

\item rings of rational numbers

\item prime subfield

\item proof that the rationals are countable

\item another proof of cardinality of the rationals

\item dense set

\end{itemize}


    \subsection*{Polynomials and functions}
\begin{itemize}

\item Bernoulli polynomials and numbers

\item Euler polynomials

\item Dirichlet's function

\item a pathological function of Riemann

\end{itemize}


   \subsection*{Rational vs. irrational}
\begin{itemize}

\item theory of rational and irrational numbers

\end{itemize}



%%%%%
%%%%%
\end{document}
