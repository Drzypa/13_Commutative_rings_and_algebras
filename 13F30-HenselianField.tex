\documentclass[12pt]{article}
\usepackage{pmmeta}
\pmcanonicalname{HenselianField}
\pmcreated{2013-03-22 13:28:37}
\pmmodified{2013-03-22 13:28:37}
\pmowner{mps}{409}
\pmmodifier{mps}{409}
\pmtitle{henselian field}
\pmrecord{9}{34047}
\pmprivacy{1}
\pmauthor{mps}{409}
\pmtype{Definition}
\pmcomment{trigger rebuild}
\pmclassification{msc}{13F30}
\pmclassification{msc}{13A18}
\pmclassification{msc}{11R99}
\pmclassification{msc}{12J20}
%\pmkeywords{hensel}
%\pmkeywords{valuation}
%\pmkeywords{non archimedean}
\pmrelated{Valuation}
\pmrelated{ValuationDomainIsLocal}
\pmrelated{ValuationRingOfAField}
\pmdefines{valuation ring}
\pmdefines{residue field}
\pmdefines{residue class field}
\pmdefines{Hensel property}
\pmdefines{henselian}
\pmdefines{henselisation}

\endmetadata

% this is the default PlanetMath preamble.  as your knowledge
% of TeX increases, you will probably want to edit this, but
% it should be fine as is for beginners.

% almost certainly you want these
\usepackage{amssymb}
\usepackage{amsmath}
\usepackage{amsfonts}

% used for TeXing text within eps files
%\usepackage{psfrag}
% need this for including graphics (\includegraphics)
%\usepackage{graphicx}
% for neatly defining theorems and propositions
%\usepackage{amsthm}
% making logically defined graphics
%%%\usepackage{xypic}

% there are many more packages, add them here as you need them

% define commands here

\usepackage{amssymb}
\usepackage{eufrak}
\usepackage[dvips]{epsfig,graphics}
\usepackage{graphicx,psfrag}

% Theorems

\newtheorem{df}{Definition}[section] 
\newtheorem{tm}[df]{Theorem} 
\newtheorem{lm}[df]{Lemma} 
\newtheorem{crl}[df]{Corollary} 
\newtheorem{pp}[df]{Proposition}

% Spelling issues

\def\centre{\center}

% Textrm commands
% \def\bugger{{\rm bugger}}

\def\Int{{\rm Int}_{k,C}}
\def\Intn{{\rm Int}_{k,C}^{n}}
\def\acl{{\rm acl}}
\def\ch{{\rm char}}
\def\dcl{{\rm dcl}}
\def\dom{{\rm dom}}
\def\elt{{\rm elt}}
\def\eut{{\rm eut}}
\def\fiber{{\rm fib}}
\def\germ{{\rm germ}}
\def\graph{{\rm graph}}
\def\gr{{\rm grd}}
\def\ilt{{\rm ilt}}
\def\lit{{\rm ilt}}
\def\iut{{\rm iut}}
\def\lex{{\rm lex}}
\def\mo{{\rm m.o.}}
\def\rad{{\rm rad}}

\def\red{{\rm red}}
\def\rex{{\rm rex}}
\def\rcl{{\rm rcl}}
\def\tp{{\rm tp}}

\def\Aff{{\rm Aff}}
\def\Hom{{\rm Hom}}
\def\Res{{\rm Res}}
\def\Gl{{\rm Gl}}

% Math frac commands

\newcommand{\ma}{\mathfrak{A}}
\newcommand{\mb}{\mathfrak{B}}
\newcommand{\mc}{\mathfrak{C}}
\newcommand{\md}{\mathfrak{D}}

% Other commands

\newcommand{\acf}{ACF_{val}}
\newcommand{\bm}{\begin{displaymath}}
\newcommand{\cl}{\bf{d}}
\newcommand{\fp}{f^{\prime}}
\newcommand{\gda}{G_{D}^{A}}
\newcommand{\gind}{\downarrow^{g}}
\newcommand{\op}{\bf{o}}
\newcommand{\half}{{\tiny \begin{array}{l}1 \\ \overline{2}\end{array}}}
\def\isom{\simeq}
\newcommand{\mgb}{\mathbf{M}}
\newcommand{\nin}{\not\in}
\def\nom{\vartriangleleft}
\newcommand{\ra}{\rightarrow}
\newcommand{\rcf}{RCVF_{G}}
\newcommand{\real}{\textrm{real}}
\newcommand{\rk}{{\bf Remark:}}
\newcommand{\sequ}{\left< x_{n}:x<\omega \right>}
\newcommand{\sq}{ $\square$}
\newcommand{\xb}{\overline{x}}

\newcommand{\Aut}{\textrm{Aut}}
\newcommand{\PR}{^{\prime}} 
\newcommand{\RS}{\mathbf{R}^{\star}} 
\newcommand{\LG}{L_{4}}
\newcommand{\RR}{\mathbf{R}}  
\newcommand{\RA}{\rightarrow}

\newcommand{\cla}[1]{\lceil #1 \rceil}

%%
\DeclareMathOperator{\res}{res}
\newcommand{\val}{|\!\cdot\!|}
\begin{document}
Let $\val$ be a non-archimedean valuation on a field $K$.  Let
$V=\{x:|x|\le 1\}$.  Since $\val$ is ultrametric, $V$ is closed under
addition and in fact an additive group.  The other valuation axioms
ensure that $V$ is a ring.  We call $V$ the \emph{valuation ring} of
$K$ with respect to the valuation $\val$.  Note that the field of
fractions of $V$ is $K$.

The set $\mu=\{x:|x|<1\}$ is a maximal ideal of $V$.  The factor
$R:=V/\mu$ is called the \emph{residue field} or the \emph{residue
class field}.

The map $\res:V \to V/\mu$ given by $x \mapsto x+\mu$ is called the
\emph{residue map}. We extend the definition of the residue map to
sequences of elements from $V$, and hence to $V[X]$ so that if $f(X)
\in V[X]$ is given by $\sum_{i \leq n} a_{i}X^{i}$ then $\res(f) \in
R[X]$ is given by $\sum_{i \leq n} \res(a{i})X^{i}$.

\bigskip

\par\noindent{\bf Hensel property:} Let $f(x) \in V[x]$. Suppose
$\res(f)(x)$ has a simple root $e \in k$. Then $f(x)$ has a root $e\PR
\in V$ and $\res(e\PR)=e$.

\medskip

Any valued field satisfying the Hensel property is called
\emph{henselian}. The completion of a non-archimedean valued field $K$
with respect to the valuation (cf. constructing the reals from the
rationals as the completion with respect to the standard metric) is a
henselian field.

Every non-archimedean valued field $K$ has a unique (up to
isomorphism) smallest henselian field $K^h$ containing it. We call
$K^h$ the \emph{henselisation} of $K$.

%%%%%
%%%%%
\end{document}
