\documentclass[12pt]{article}
\usepackage{pmmeta}
\pmcanonicalname{HomogeneousElementsOfAGradedRing}
\pmcreated{2013-03-22 14:14:52}
\pmmodified{2013-03-22 14:14:52}
\pmowner{mathcam}{2727}
\pmmodifier{mathcam}{2727}
\pmtitle{homogeneous elements of a graded ring}
\pmrecord{6}{35694}
\pmprivacy{1}
\pmauthor{mathcam}{2727}
\pmtype{Definition}
\pmcomment{trigger rebuild}
\pmclassification{msc}{13A02}
\pmrelated{HomogeneousIdeal}
\pmdefines{homogeneous element}
\pmdefines{homogeneous degree}
\pmdefines{irrelevant ideal}
\pmdefines{homogeneous union}

\endmetadata

% this is the default PlanetMath preamble.  as your knowledge
% of TeX increases, you will probably want to edit this, but
% it should be fine as is for beginners.

% almost certainly you want these
\usepackage{amssymb}
\usepackage{amsmath}
\usepackage{amsfonts}
\usepackage{amsthm}

% used for TeXing text within eps files
%\usepackage{psfrag}
% need this for including graphics (\includegraphics)
%\usepackage{graphicx}
% for neatly defining theorems and propositions
%\usepackage{amsthm}
% making logically defined graphics
%%%\usepackage{xypic}

% there are many more packages, add them here as you need them

% define commands here

\newcommand{\mc}{\mathcal}
\newcommand{\mb}{\mathbb}
\newcommand{\mf}{\mathfrak}
\newcommand{\ol}{\overline}
\newcommand{\ra}{\rightarrow}
\newcommand{\la}{\leftarrow}
\newcommand{\La}{\Leftarrow}
\newcommand{\Ra}{\Rightarrow}
\newcommand{\nor}{\vartriangleleft}
\newcommand{\Gal}{\text{Gal}}
\newcommand{\GL}{\text{GL}}
\newcommand{\Z}{\mb{Z}}
\newcommand{\R}{\mb{R}}
\newcommand{\Q}{\mb{Q}}
\newcommand{\C}{\mb{C}}
\newcommand{\<}{\langle}
\renewcommand{\>}{\rangle}
\begin{document}
Let $k$ be a field, and let $R$ be a connected commutative $k$-algebra \PMlinkname{graded}{GradedAlgebra} by $\mb{N}^m$.  Then via the grading, we can decompose $R$ into a direct sum of vector spaces:  $R=\coprod_{\omega\in\mb{N}^m} R_\omega$, where $R_0=k$.

For an arbitrary ring element $x\in R$, we define the \emph{homogeneous degree} of $x$ to be the value $\omega$ such that $x\in R_\omega$, and we denote this by $\deg(x)=\omega$.  (See also homogeneous ideal)

A set of some importance (ironically), is the \emph{irrelevant ideal} of $R$, denoted by $R^+$, and given by

\begin{align*}
R_+=\coprod_{\omega\neq 0}R_\omega.
\end{align*}

Finally, we often need to consider the elements of such a ring $R$ without using the grading, and we do this by looking at the \emph{homogeneous union} of $R$:

\begin{align*}
\mathcal{H}(R)=\bigcup_\omega R_\omega.
\end{align*}

In particular, in defining a homogeneous system of parameters, we are looking at elements of $\mathcal{H}(R_+)$.
\begin{thebibliography}{9}
\bibitem{Stan} Richard P. Stanley, {\em Combinatorics and Commutative Algebra}, Second edition, Birkhauser Press.  Boston, MA.  1986.
\end{thebibliography}
%%%%%
%%%%%
\end{document}
