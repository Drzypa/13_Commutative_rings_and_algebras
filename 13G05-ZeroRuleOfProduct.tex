\documentclass[12pt]{article}
\usepackage{pmmeta}
\pmcanonicalname{ZeroRuleOfProduct}
\pmcreated{2013-03-22 15:06:46}
\pmmodified{2013-03-22 15:06:46}
\pmowner{pahio}{2872}
\pmmodifier{pahio}{2872}
\pmtitle{zero rule of product}
\pmrecord{11}{36848}
\pmprivacy{1}
\pmauthor{pahio}{2872}
\pmtype{Result}
\pmcomment{trigger rebuild}
\pmclassification{msc}{13G05}
\pmsynonym{product to zero rule}{ZeroRuleOfProduct}
\pmrelated{CancellationRing}
\pmrelated{EulersDerivationOfTheQuarticFormula}
\pmrelated{GroupingMethodForFactorizingPolynomials}
\pmrelated{HyperbolasOrthogonalToEllipses}

% this is the default PlanetMath preamble.  as your knowledge
% of TeX increases, you will probably want to edit this, but
% it should be fine as is for beginners.

% almost certainly you want these
\usepackage{amssymb}
\usepackage{amsmath}
\usepackage{amsfonts}

% used for TeXing text within eps files
%\usepackage{psfrag}
% need this for including graphics (\includegraphics)
%\usepackage{graphicx}
% for neatly defining theorems and propositions
%\usepackage{amsthm}
% making logically defined graphics
%%%\usepackage{xypic}

% there are many more packages, add them here as you need them

% define commands here
\begin{document}
For real and complex numbers, and more generally for elements of an integral domain, a product equals to zero if and only if at least one of the \PMlinkescapetext{factors} equals to zero.\, For two elements $a$ and $b$, we have
    $$ab \;=\; 0 \quad \Longleftrightarrow \quad a \,=\, 0\; \lor \;b \,=\, 0. $$

For example, this rule can be used in solving polynomial equations:
$$x^3\!-\!x^2\!-\!2x\!+\!2 \;=\; 0$$
$$(x^3\!-\!x^2)\!+\!(-2x\!+\!2) \;=\; 0$$
$$x^2(x\!-\!1)\!-\!2(x\!-\!1) \;=\; 0$$
$$(x\!-\!1)(x^2\!-\!2) \;=\; 0$$
$$x\!-\!1 \;=\; 0 \;\lor\; x^2\!-\!2 \;=\; 0$$
$$x \;=\; 1 \;\lor\; x \;=\; \pm\sqrt{2}$$

The used sign ``$\lor$'' is the logical or.
%%%%%
%%%%%
\end{document}
