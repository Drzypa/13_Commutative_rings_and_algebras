\documentclass[12pt]{article}
\usepackage{pmmeta}
\pmcanonicalname{ExamplesOfIntegrallyClosedExtensions}
\pmcreated{2013-03-22 17:01:32}
\pmmodified{2013-03-22 17:01:32}
\pmowner{rm50}{10146}
\pmmodifier{rm50}{10146}
\pmtitle{examples of integrally closed extensions}
\pmrecord{9}{39311}
\pmprivacy{1}
\pmauthor{rm50}{10146}
\pmtype{Example}
\pmcomment{trigger rebuild}
\pmclassification{msc}{13B22}
\pmclassification{msc}{11R04}

% this is the default PlanetMath preamble.  as your knowledge
% of TeX increases, you will probably want to edit this, but
% it should be fine as is for beginners.

% almost certainly you want these
\usepackage{amssymb}
\usepackage{amsmath}
\usepackage{amsfonts}

% used for TeXing text within eps files
%\usepackage{psfrag}
% need this for including graphics (\includegraphics)
%\usepackage{graphicx}
% for neatly defining theorems and propositions
%\usepackage{amsthm}
% making logically defined graphics
%%%\usepackage{xypic}

% there are many more packages, add them here as you need them

% define commands here
\newcommand{\Nats}{\mathbb{N}}
\newcommand{\Ints}{\mathbb{Z}}
\newcommand{\Reals}{\mathbb{R}}
\newcommand{\Complex}{\mathbb{C}}
\newcommand{\Rats}{\mathbb{Q}}
\newcommand{\Gal}{\operatorname{Gal}}
\newcommand{\Cl}{\operatorname{Cl}}
\newcommand{\Alg}{\mathcal{O}}
\newcommand{\ol}{\overline}
\renewcommand{\frak}[1]{\mathfrak{#1}}
\newcommand{\ip}[2]{(#1,#2)}
\newcommand{\conv}[2]{(#1*#2)}

\begin{document}
\textbf{Example. } $\Ints[\sqrt{5}]$ is not integrally closed, for $u=\frac{1+\sqrt{5}}{2}\in\Rats[\sqrt{5}]$ is integral over $\Ints[\sqrt{5}]$ since $u^2-u-1=0$, but $u\notin\Ints[\sqrt{5}]$.

\textbf{Example. } $R=\Ints[\sqrt{2},\sqrt{3}]$ is not integrally closed. Note that $(\sqrt{6}+\sqrt{2})/2\notin R$, but that
\[\left(\frac{\sqrt{6}+\sqrt{2}}{2}\right)^2=2+\sqrt{3}\]
and so $(\sqrt{6}+\sqrt{2})/2$ is integral over $\Ints$ since it satisfies the polynomial $(z^2-2)^2-3=0$.

\textbf{Example. } $\Alg_K$ is integrally closed when $[K:\Rats]<\infty$. For if $u\in K$ is integral over $\Alg_K$, then $\Ints\subset\Alg_K\subset \Alg_K[u]$ are all integral extensions, so $u$ is integral over $\Ints$, so $u\in\Alg_K$ by definition. In fact, $\Alg_K$ can be defined as the integral closure of $\Ints$ in $K$.

\textbf{Example. } $\Complex[x,y]/(y^2-x^3)$. This is a domain because $y^2-x^3$ is irreducible hence a prime ideal. But this quotient ring is not integrally closed. To see this, parameterize $\Complex[x,y]\rightarrow\Complex[t]$ by
\begin{eqnarray*}
&x&\mapsto t^2\\
&y&\mapsto t^3
\end{eqnarray*}
The kernel of this map is $(y^2-x^3)$, and its image is $\Complex[t^2,t^3]$. Hence
\[\Complex[x,y]/(y^2-x^3)\cong\Complex[t^2,t^3]\]
and the field of fractions of the latter ring is obviously $\Complex(t)$. Now, $t$ is integral over $\Complex[t^2,t^3]$ ($z^2-t^2$ is its polynomial), but is not in $\Complex[t^2,t^3]$. $t$ corresponds to $\frac{y}{x}$ in the original ring $\Complex[x,y]/(y^2-x^3)$, which is thus not integrally closed (the minimal polynomial of $\frac{y}{x}$ is $z^2-x$ since $(\frac{y}{x})^2-x=\frac{y^2}{x^2}-x=\frac{x^3}{x^2}-x=0$). The failure of integral closure in this coordinate ring is due to a codimension 1 singularity of $y^2-x^3$ at $0$.

\textbf{Example. } $A=\Complex[x,y,z]/(z^2-xy)$ is integrally closed. For again, parameterize $A\rightarrow \Complex[u,v]$ by
\begin{eqnarray*}
&x&\mapsto u^2\\
&y&\mapsto v^2\\
&z&\mapsto uv
\end{eqnarray*}
The kernel of this map is $z^2-xy$ and its image is $B=\Complex[u^2,v^2,uv]$. Claim $B$ is integrally closed. We prove this by showing that the integral closure of $\Complex[x,y]$ in $\Complex(x,y,\sqrt{xy})$ is $\Complex[x,y,\sqrt{xy}]$. Choose $r+s\sqrt{xy}\in\Complex(x,y,\sqrt{xy}), r,s\in\Complex(x,y)$ such that $r+s\sqrt{xy}$ is integral over $\Complex[x,y]$. Then $r-s\sqrt{xy}$ is also integral over $\Complex[x,y]$, so their sum is. Hence $2r$ is integral over $\Complex[x,y]$. But $\Complex[x,y]$ is a UFD, hence integrally closed, so $2r\in\Complex[x,y]$ and thus $r\in\Complex[x,y]$. Similarly, $s\sqrt{xy}$ is integral over $\Complex[x,y]$, hence $s^2xy\in\Complex[x,y], s\in\Complex(x,y)$. Clearly, then, $s$ can have no denominator, so $s\in\Complex[x,y]$. Hence $r+s\sqrt{xy}\in\Complex[x,y,\sqrt{xy}]$.

%%%%%
%%%%%
\end{document}
