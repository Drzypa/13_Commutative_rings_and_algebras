\documentclass[12pt]{article}
\usepackage{pmmeta}
\pmcanonicalname{RingHomomorphism}
\pmcreated{2013-03-22 11:48:50}
\pmmodified{2013-03-22 11:48:50}
\pmowner{djao}{24}
\pmmodifier{djao}{24}
\pmtitle{ring homomorphism}
\pmrecord{12}{30357}
\pmprivacy{1}
\pmauthor{djao}{24}
\pmtype{Definition}
\pmcomment{trigger rebuild}
\pmclassification{msc}{13B10}
\pmclassification{msc}{16B99}
\pmclassification{msc}{81P05}
\pmrelated{Ring}
\pmdefines{unital}
\pmdefines{ring isomorphism}
\pmdefines{ring epimorphism}
\pmdefines{ring monomorphism}
\pmdefines{homomorphism}
\pmdefines{isomorphism}
\pmdefines{epimorphism}
\pmdefines{monomprhism}

\usepackage{amssymb}
\usepackage{amsmath}
\usepackage{amsfonts}
\usepackage{graphicx}
%%%%\usepackage{xypic}
\begin{document}
Let $R$ and $S$ be rings. A \emph{ring homomorphism} is a function $f: R \longrightarrow S$ such that:
\begin{itemize}
\item $f(a+b) = f(a)+f(b)$ for all $a,b \in R$
\item $f(a\cdot b) = f(a) \cdot f(b)$ for all $a,b \in R$
\end{itemize}

A \emph{ring isomorphism} is a ring homomorphism which is a bijection. A \emph{ring monomorphism} (respectively, \emph{ring epimorphism}) is a ring homomorphism which is an injection (respectively, surjection).

When working in a context in which all rings have a multiplicative identity, one also requires that $f(1_R) = 1_S$. Ring homomorphisms which satisfy this property are called \emph{unital} ring homomorphisms.
%%%%%
%%%%%
%%%%%
%%%%%
\end{document}
