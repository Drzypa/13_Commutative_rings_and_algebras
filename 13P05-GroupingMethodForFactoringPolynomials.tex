\documentclass[12pt]{article}
\usepackage{pmmeta}
\pmcanonicalname{GroupingMethodForFactoringPolynomials}
\pmcreated{2013-03-22 15:06:49}
\pmmodified{2013-03-22 15:06:49}
\pmowner{pahio}{2872}
\pmmodifier{pahio}{2872}
\pmtitle{grouping method for factoring polynomials}
\pmrecord{12}{36850}
\pmprivacy{1}
\pmauthor{pahio}{2872}
\pmtype{Algorithm}
\pmcomment{trigger rebuild}
\pmclassification{msc}{13P05}
\pmrelated{DifferenceOfSquares}
\pmrelated{ExampleOfGcd}
\pmrelated{ZeroRuleOfProduct}
\pmdefines{grouping method}

\endmetadata

% this is the default PlanetMath preamble.  as your knowledge
% of TeX increases, you will probably want to edit this, but
% it should be fine as is for beginners.

% almost certainly you want these
\usepackage{amssymb}
\usepackage{amsmath}
\usepackage{amsfonts}

% used for TeXing text within eps files
%\usepackage{psfrag}
% need this for including graphics (\includegraphics)
%\usepackage{graphicx}
% for neatly defining theorems and propositions
%\usepackage{amsthm}
% making logically defined graphics
%%%\usepackage{xypic}

% there are many more packages, add them here as you need them

% define commands here
\begin{document}
Factoring a given polynomial may in certain special cases \PMlinkescapetext{succeed} by using the following {\em grouping method}:
\begin{enumerate}
\item \PMlinkescapetext{Group the terms of the polynomial in two (or sometimes more) suitable groups}.
\item Factorize the \PMlinkescapetext{groups} separately.
\item The whole polynomial may then possibly be written in form of a product.
\end{enumerate}

\textbf{Examples}

a) \,\,$x^3-x^2-x+1 = \{x^3-x^2\}+\{-x+1\} = x^2(x-1)-1(x-1) = (x-1)(x^2-1)\\ = (x-1)^2(x+1)$

b) \,\,$x^4+3x^3-3x-1 = \{x^4-1\}+\{3x^3-3x\} = (x^2+1)(x^2-1)+3x(x^2-1)\\ = 
(x^2-1)(x^2+1+3x) = (x-1)(x+1)(x^2+3x+1)$

c) \,\,$x^4+4 = \{x^4+4x^2+4\}-4x^2 = (x^2+2)^2-(2x)^2 = (x^2+2+2x)(x^2+2-2x)\\ = (x^2+2x+2)(x^2-2x+2)$ 

d) \,\,$x^4+x^2+1 = \{x^4+2x^2+1\}-x^2 = (x^2+1)^2-x^2 = (x^2+1+x)(x^2+1-x)\\
= (x^2+x+1)(x^2-x+1)$

The trinomials $x^2\!+\!3x\!+\!1$, $x^2\!\pm\!2x\!+\!2$ and $x^2\!\pm\!x\!+\!1$ are irreducible polynomials.
%%%%%
%%%%%
\end{document}
