\documentclass[12pt]{article}
\usepackage{pmmeta}
\pmcanonicalname{ExampleOfUsingEisensteinCriterion}
\pmcreated{2013-03-22 19:10:14}
\pmmodified{2013-03-22 19:10:14}
\pmowner{pahio}{2872}
\pmmodifier{pahio}{2872}
\pmtitle{example of using Eisenstein criterion}
\pmrecord{5}{42077}
\pmprivacy{1}
\pmauthor{pahio}{2872}
\pmtype{Example}
\pmcomment{trigger rebuild}
\pmclassification{msc}{13A05}
\pmclassification{msc}{11C08}

% this is the default PlanetMath preamble.  as your knowledge
% of TeX increases, you will probably want to edit this, but
% it should be fine as is for beginners.

% almost certainly you want these
\usepackage{amssymb}
\usepackage{amsmath}
\usepackage{amsfonts}

% used for TeXing text within eps files
%\usepackage{psfrag}
% need this for including graphics (\includegraphics)
%\usepackage{graphicx}
% for neatly defining theorems and propositions
 \usepackage{amsthm}
% making logically defined graphics
%%%\usepackage{xypic}

% there are many more packages, add them here as you need them

% define commands here

\theoremstyle{definition}
\newtheorem*{thmplain}{Theorem}

\begin{document}
For showing the \PMlinkname{irreducibility}{IrreduciblePolynomial2} of the polynomial
$$P(x) \;:=\; x^5\!+\!5x\!+\!11$$
one would need a prime number dividing its other coefficients except the first one, but there is no such prime.\, However, a suitable \PMlinkescapetext{substitution} \,$x := y\!+\!a$\, may change the situation.\, Since\, the binomial coefficients of 
$(y\!-\!1)^5$ except the first and the last one are divisible by 5 and $11 \equiv 1 \pmod{5}$,\, we try
$$x \;:=\; y-1.$$
Then
$$P(y\!-\!1) \;=\; y^5\!-\!5y^4\!+\!10y^3\!-\!10y^2\!+\!10y\!+\!5.$$
Thus the prime 5 divides other coefficients except the first one and the square of 5 does not divide the constant term of this polynomial in $y$, whence the Eisenstein criterion says that $P(y\!-\!1)$ is irreducible (in the field $\mathbb{Q}$ of its coefficients).\, Apparently, also $P(x)$ must be irreducible.\\

It would be easy also to see that $P(x)$ does not have \PMlinkname{rational zeroes}{RationalRootTheorem}.
%%%%%
%%%%%
\end{document}
