\documentclass[12pt]{article}
\usepackage{pmmeta}
\pmcanonicalname{CancellationIdeal}
\pmcreated{2015-05-06 14:49:08}
\pmmodified{2015-05-06 14:49:08}
\pmowner{pahio}{2872}
\pmmodifier{pahio}{2872}
\pmtitle{cancellation ideal}
\pmrecord{11}{37236}
\pmprivacy{1}
\pmauthor{pahio}{2872}
\pmtype{Definition}
\pmcomment{trigger rebuild}
\pmclassification{msc}{13B30}
\pmsynonym{cancellative ideal}{CancellationIdeal}
\pmrelated{CancellativeSemigroup}
\pmrelated{IdealDecompositionInDedekindDomain}
\pmdefines{cancellative}

\endmetadata

% this is the default PlanetMath preamble.  as your knowledge
% of TeX increases, you will probably want to edit this, but
% it should be fine as is for beginners.

% almost certainly you want these
\usepackage{amssymb}
\usepackage{amsmath}
\usepackage{amsfonts}

% used for TeXing text within eps files
%\usepackage{psfrag}
% need this for including graphics (\includegraphics)
%\usepackage{graphicx}
% for neatly defining theorems and propositions
 \usepackage{amsthm}
% making logically defined graphics
%%%\usepackage{xypic}

% there are many more packages, add them here as you need them

% define commands here

\theoremstyle{definition}
\newtheorem*{thmplain}{Theorem}
\begin{document}
Let $R$ be a commutative ring containing regular elements and $\mathfrak{S}$ be the multiplicative semigroup of the non-zero fractional ideals of $R$.\, A fractional ideal $\mathfrak{a}$ of $R$ is called a {\em cancellation ideal} or simply {\em cancellative}, if it is a cancellative element of $\mathfrak{S}$, i.e. if 
$$\mathfrak{ab = ac}\, \Rightarrow\, \mathfrak{b = c}
\quad\forall\,\,\mathfrak{b,\,c}\in\mathfrak{S}.$$

\begin{itemize}
 \item Each invertible ideal is cancellative.
 \item A finite product $\mathfrak{a}_1\mathfrak{a}_2...\mathfrak{a}_m$ of fractional ideals is cancellative iff every $\mathfrak{a}_i$ is such.
 \item The fractional ideal\,  
$\mathfrak{a}/r := \{ar^{-1}\!:\,\,\,a\in\mathfrak{a}\}$,\, where $\mathfrak{a}$ is an integral ideal of $R$ and $r$ a regular element of $R$, is cancellative if and only if $\mathfrak{a}$ is cancellative in the multiplicative semigroup of the non-zero integral ideals of $R$.
 \item If\, $r\in R$,\, then the principal ideal $(r)$ of $R$ is cancellative if  and only if $r$ is a regular element of the total ring of fractions of $R$.
 \item If\, $\mathfrak{a}_1\!+\!\mathfrak{a}_2\!+\!...\!+\!\mathfrak{a}_m$\, is a cancellation ideal and $n$ a positive integer, then
     $$(\mathfrak{a}_1\!+\!\mathfrak{a}_2\!+\!...\!+\!\mathfrak{a}_m)^n =
      \mathfrak{a}_1^n\!+\!\mathfrak{a}_2^n\!+\!...\!+\!\mathfrak{a}_m^n.$$
In particular, if the ideal\, $(a_1,\,a_2,\,...,\,a_m)$\, of $R$ is cancellative, then
     $$(a_1,\,a_2,\,...,\,a_m)^n = (a_1^n,\,a_2^n,\,...,\,a_m^n).$$
\end{itemize}

\begin{thebibliography}{9}
 \bibitem{RG}{\sc R. Gilmer:} {\em Multiplicative ideal theory}.\, Queens University Press. Kingston, Ontario (1968).
 \bibitem{LM}{\sc M. Larsen \& P. McCarthy:} {\em Multiplicative theory of ideals}.\, Academic Press. New York (1971).
\end{thebibliography}
%%%%%
%%%%%
\end{document}
