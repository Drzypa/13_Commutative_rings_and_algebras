\documentclass[12pt]{article}
\usepackage{pmmeta}
\pmcanonicalname{HowToMultiplyPolynomials}
\pmcreated{2013-03-22 18:21:10}
\pmmodified{2013-03-22 18:21:10}
\pmowner{Algeboy}{12884}
\pmmodifier{Algeboy}{12884}
\pmtitle{how to multiply polynomials}
\pmrecord{16}{40989}
\pmprivacy{1}
\pmauthor{Algeboy}{12884}
\pmtype{Definition}
\pmcomment{trigger rebuild}
\pmclassification{msc}{13P05}
\pmclassification{msc}{11C08}
\pmclassification{msc}{12E05}
\pmsynonym{FOIL}{HowToMultiplyPolynomials}

\endmetadata

\usepackage{latexsym}
\usepackage{amssymb}
\usepackage{amsmath}
\usepackage{amsfonts}
\usepackage{amsthm}


\usepackage{tabls}
%%\usepackage{xypic}

%-----------------------------------------------------

%       Standard theoremlike environments.

%       Stolen directly from AMSLaTeX sample

%-----------------------------------------------------

%% \theoremstyle{plain} %% This is the default

\newtheorem{thm}{Theorem}

\newtheorem{coro}[thm]{Corollary}

\newtheorem{lem}[thm]{Lemma}

\newtheorem{lemma}[thm]{Lemma}

\newtheorem{prop}[thm]{Proposition}

\newtheorem{conjecture}[thm]{Conjecture}

\newtheorem{conj}[thm]{Conjecture}

\newtheorem{defn}[thm]{Definition}

\newtheorem{remark}[thm]{Remark}

\newtheorem{ex}[thm]{Example}



%\countstyle[equation]{thm}



%--------------------------------------------------

%       Item references.

%--------------------------------------------------


\newcommand{\exref}[1]{Example-\ref{#1}}

\newcommand{\thmref}[1]{Theorem-\ref{#1}}

\newcommand{\defref}[1]{Definition-\ref{#1}}

\newcommand{\eqnref}[1]{(\ref{#1})}

\newcommand{\secref}[1]{Section-\ref{#1}}

\newcommand{\lemref}[1]{Lemma-\ref{#1}}

\newcommand{\propref}[1]{Prop\-o\-si\-tion-\ref{#1}}

\newcommand{\corref}[1]{Cor\-ol\-lary-\ref{#1}}

\newcommand{\figref}[1]{Fig\-ure-\ref{#1}}

\newcommand{\conjref}[1]{Conjecture-\ref{#1}}


% Normal subgroup or equal.

\providecommand{\normaleq}{\unlhd}

% Normal subgroup.

\providecommand{\normal}{\lhd}

\providecommand{\rnormal}{\rhd}
% Divides, does not divide.

\providecommand{\divides}{\mid}

\providecommand{\ndivides}{\nmid}


\providecommand{\union}{\cup}

\providecommand{\bigunion}{\bigcup}

\providecommand{\intersect}{\cap}

\providecommand{\bigintersect}{\bigcap}










\begin{document}
We show how to multiply polynomials using a visual tool: \emph{length times width}.  

To multiply the polynomials $x^2+x+1$ with $3x-9$:
\begin{enumerate}
\item Create a grid of $3$ rows and $2$ columns, and
label the rows by $x^2$, $x$, and $1$; label the columns by $3x$ and $-9$,
\begin{equation}
\begin{array}{c|c|c|}
 & 3x & -9 \\
\hline
x^2 & & \\
\hline 
x & & \\
\hline 
1 & & \\
\hline
\end{array}
\end{equation}
\item In each box in the grid, fill in the corresponding
product of the row label times the column label.  Make sure to
change the exponent of $x$ to the sum of the exponents of $x$
in the row and columns.
\begin{equation}
\begin{array}{c|c|c|}
 & 3x & -9 \\
\hline
x^2 & x^2\cdot 3x & \\
\hline 
x & & \\
\hline 
1 & & \\
\hline
\end{array}
\quad\rightarrow\quad
\begin{array}{c|c|c|}
 & 3x & -9 \\
\hline
x^2 & 3x^{3} & x^2\cdot (-9)\\
\hline 
x &  & \\
\hline 
1 &  &  \\
\hline
\end{array}
\quad\rightarrow\quad
\begin{array}{c|c|c|}
 & 3x & -9 \\
\hline
x^2 & 3x^{3} & -9x^2\\
\hline 
x & 3x^2 & -9x\\
\hline 
1 & 3x & -9 \\
\hline
\end{array}
\end{equation}
\item Add up all the entries in the grid, making sure to
combine any entries which have the same power of $x$:
\begin{equation}
(x^2+x+1)(3x-9)=3x^3-9x^2+3x^2-9x+3x-9=3x^3-6x^2-6x-9.
\end{equation}
\end{enumerate}


\section{How do this method work?}

This is a visual way to use the \textbf{distributive property}, that is, $a(b+c)=ab+ac$.
This is usually explained for numbers, but it also works for multiplication and addition
of polynomials.  We will use lengths to start with.

Suppose we are given two lengths, the first length is 2 units the second 3 units.  We can 
make a picture to represent $2\cdot 3$ by connecting length with area, ``length times width''.

\begin{equation}
\begin{xy}<10mm,0mm>:<0mm,10mm>::
(0,0) = "xy"; (1,0) = "x1";  (2,0) = "x2"; (3,0) = "x3";
(0,-1) = "y1"; 
(0,-2) = "y2";
"xy"; "x1" **@{-};  "x1"; "x2" **@{-}; "x2"; "x3" **@{-};
"xy"; "y1" **@{-};  "y1"; "y2" **@{-};
"xy" *+{\bullet}; "x1" *+{\bullet}; "x2" *+{\bullet}; "x3" *+{\bullet};
"y1" *+{\bullet}; "y2" *+{\bullet};
\end{xy}
\qquad\rightarrow\qquad
\begin{xy}<10mm,0mm>:<0mm,10mm>::
(0,0) = "xy"; (1,0) = "x1";  (2,0) = "x2"; (3,0) = "x3";
(0,-1) = "y1"; 
(0,-2) = "y2";
"xy"; "x3" **@{-}; "y1"; (3,-1) **@{--}; "y2"; (3,-2) **@{--};
"xy"; "y2" **@{-}; "x1"; (1,-2) **@{--}; "x2"; (2,-2) **@{--}; "x3"; (3,-2) **@{--};
"xy" *+{\bullet}; "x1" *+{\bullet}; "x2" *+{\bullet}; "x3" *+{\bullet};
"y1" *+{\bullet}; 
"y2" *+{\bullet}; 
\end{xy}
\end{equation}
Notice that we can work out the area by working out the area of each small box and then
add them up.  That should look similar the approach we suggest above.

Now with algebra we start to replace actual numbers with variables which allow us to
fluctuate the values.  But with the variables in place we can draw conclusions which will
be true regardless of the specific values of the variables.  So visually,
what we can think of when we write $x^2+x+1$ is a line segment divided into three parts.  The
first part will represent the length $x^2$, the middle segment represents the length $x$,
and the final segment will represent $1$ unit.  Similarly, we can make a line segment for
$3x-9$ which will have two parts.  However, because we have negative numbers now, we need to
keep track of directions.


\begin{equation}
\begin{xy}<5mm,0mm>:<0mm,5mm>::
(0,0) = "xy"; (6,0) = "x1";  (-9,0) = "x2";
(0,1) = "y1"; 
(0,3) = "y2";
(0,7) = "y3";
"x1"; "x2" **@{-};
"xy"; "y3" **@{-};
"x1"; (6,7) **@{--};  "x2"; (-9,7) **@{--};
(-9,1); (6,1) **@{--};
(-9,3); (6,3) **@{--};
(-9,7); (6,7) **@{--};
"xy" *+{\bullet}; "x1" *+{\bullet}; "x2" *+{\bullet};
"y1" *+{\bullet}; "y2" *+{\bullet}; "y3" *+{\bullet};
(-4.5,-0.5) *+{-9};
(3,-0.5) *+{3x};
(0.5,5) *+{x^2};
(0.5,2) *+{x};
(0.5,0.5) *+{1};
\end{xy}
\end{equation}


\section{Frequently asked questions.}

\begin{enumerate}
\item Does it matter which polynomial I write in the row or in the column?\\
\begin{quote} No.  Mathematicians often refer to this property by saying: ``multiplication 
of polynomials is commutative.''  That simply means that, for example, 
$(x^2+x+1)\cdot(3x-9)=(3x-9)\cdot (x^2+x+1)$ and so the rows and columns can be swapped.\\
\textbf{CAUTION:} you must swap the entire polynomial, not just one or two of the terms.
For example, the example below has swapped the final row and column.  Thus, this represents
the product of $(3x+x+1)(x^2-9)$ and \emph{not} $(x^2+x+1)(3x-9)$.
\begin{equation}
\begin{array}{c|c|c|}
 & x^2 & -9 \\
\hline
3x & & \\
\hline 
x & & \\
\hline 
1 & & \\
\hline
\end{array}
\end{equation}
Try it!
\end{quote}

\item What if I want to multiply longer or shorter polynomials?
\begin{quote}
This method still works, you simply need to have as sufficient rows and columns 
to fit the polynomials.  The following tables are set up for the 
following multiplications: $(x-2)(x+2)$ and $(\sqrt{91}x^{1001}-7x)(\pi x^6+x^5-7x^2)$.
\begin{equation}
\begin{array}{c|c|c|}
 & x & 2\\
\hline
x & & \\
\hline
-2 & & \\
\hline 
\end{array}
\qquad
\begin{array}{c|c|c|c|}
 & \pi x^6 & x^5 & -7x^2 \\
\hline
\sqrt{91}x^{1001} & & & \\
\hline
-7x & & & \\
\hline
\end{array}
\end{equation}
\end{quote}

\item Do I need to draw in the lines of the grid?
\begin{quote}
No.  This just helps explain the grid, but it can be helpful to draw lines to separate the labels from the product.
For example:
\begin{equation}
\begin{array}{c|cc}
 & 3x & -9 \\
\hline
x^2 & 3x^{3} & -9x^2\\
x & 3x^3 & -9x\\
1 & 3x & -9 \\
\end{array}
\end{equation}
\end{quote}

\item Do I need to have my polynomial ``sorted'' from highest power to lowest power?
\begin{quote}
No.  In fact, you can swap the rows or the columns as much as you like.  This is because
addition of polynomials is commutative.  For example, find the product of
$(1+x^2+x)(3x-9)$ and compare it to the first example. \textbf{CAUTION:} do not swap
individual terms from the rows with individual terms from the column; see 1.

You can also have
the same power of $x$ appear multiple times in a row or column.  That is you, can 
multiply $(7x+3-5x)(2x^2+11)$ without combining the $7x$ and $-5x$ to begin with. 
\end{quote}

\item Is this the same as F.O.I.L?
\begin{quote}
No.  F.O.I.L. stands for ``First, Outers, Inners, Lasts''.  That is a method which shows
how to multiply ``binomials'', that is, $(x-2)(x+3)$.  Unlike FOIL, the method here
allows you to multiply polynomials of any size.  Furthermore, the method works in any
language, whereas FOIL is meaningless when translated.
\end{quote}
\end{enumerate}

\section{For teachers}

This approach can be used in elementary algebra classes efficiently and without
the draw backs of methods such as FOIL, and also with out the abstract feel of
subscripts used or $\Sigma$ notation.

FOIL methods have two fundamental limitations: they require the inputs be binomials
and that the student feel comfortable with the English language.

Using a visual algorithm is considerably less language specific.  Furthermore,
it helps to motivate the reason multiplication of polynomials works.


%%%%%
%%%%%
\end{document}
