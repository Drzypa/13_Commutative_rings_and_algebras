\documentclass[12pt]{article}
\usepackage{pmmeta}
\pmcanonicalname{ExtendedIdeal}
\pmcreated{2013-03-22 12:55:34}
\pmmodified{2013-03-22 12:55:34}
\pmowner{drini}{3}
\pmmodifier{drini}{3}
\pmtitle{extended ideal}
\pmrecord{6}{33280}
\pmprivacy{1}
\pmauthor{drini}{3}
\pmtype{Definition}
\pmcomment{trigger rebuild}
\pmclassification{msc}{13A15}
\pmclassification{msc}{14K99}
\pmrelated{ContractedIdeal}

\endmetadata

% this is the default PlanetMath preamble.  as your knowledge
% of TeX increases, you will probably want to edit this, but
% it should be fine as is for beginners.

% almost certainly you want these
\usepackage{amssymb}
\usepackage{amsmath}
\usepackage{amsfonts}

% used for TeXing text within eps files
%\usepackage{psfrag}
% need this for including graphics (\includegraphics)
%\usepackage{graphicx}
% for neatly defining theorems and propositions
%\usepackage{amsthm}
% making logically defined graphics
%%%\usepackage{xypic} 

% there are many more packages, add them here as you need them

% define commands here
\begin{document}
Let $f: A \to B$ be a ring map. We can look at the ideal generated by the image of $\mathfrak{a}$, which is called an {\em extended ideal} and is denoted by $\mathfrak{a}^e$. \\
\\
It is not true in general that if $\mathfrak{a}$ is an ideal in $A$,  the image of $\mathfrak{a}$ under $f$ will be an ideal in $B$. (For example, consider the embedding $f: \mathbb{Z} \to \mathbb{Q}$. The image of the ideal $(2) \subset \mathbb{Z}$ is {\em not} an ideal in $\mathbb{Q}$, since the only ideals in $\mathbb{Q}$ are $\{0\}$ and all of $\mathbb{Q}$.)
%%%%%
%%%%%
\end{document}
