\documentclass[12pt]{article}
\usepackage{pmmeta}
\pmcanonicalname{Localization}
\pmcreated{2013-03-22 11:50:21}
\pmmodified{2013-03-22 11:50:21}
\pmowner{djao}{24}
\pmmodifier{djao}{24}
\pmtitle{localization}
\pmrecord{11}{30391}
\pmprivacy{1}
\pmauthor{djao}{24}
\pmtype{Definition}
\pmcomment{trigger rebuild}
\pmclassification{msc}{13B30}
\pmsynonym{ring of fractions}{Localization}
\pmrelated{FractionField}

\usepackage{amssymb}
\usepackage{amsmath}
\usepackage{amsfonts}
\usepackage{graphicx}
%%%%\usepackage{xypic}
\begin{document}
Let $R$ be a commutative ring and let $S$ be a nonempty multiplicative subset of $R$. The {\em localization} of $R$ at $S$ is the ring $S^{-1} R$ whose elements are equivalence classes of $R \times S$ under the equivalence relation $(a,s) \sim (b,t)$ if $r(at - bs) = 0$ for some $r \in S$. Addition and multiplication in $S^{-1}R$ are defined by:
\begin{itemize}
\item $(a,s) + (b,t) = (at+bs,st)$
\item $(a,s) \cdot (b,t) = (a \cdot b,s \cdot t)$
\end{itemize}
The equivalence class of $(a,s)$ in $S^{-1}R$ is usually denoted $a/s$. For $a \in R$, the localization of $R$ at the minimal multiplicative set containing $a$ is written as $R_a$. When $S$ is the complement of a prime ideal $\mathfrak{p}$ in $R$, the localization of $R$ at $S$ is written $R_{\mathfrak{p}}$.
%%%%%
%%%%%
%%%%%
%%%%%
\end{document}
