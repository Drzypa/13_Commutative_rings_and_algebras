\documentclass[12pt]{article}
\usepackage{pmmeta}
\pmcanonicalname{AlternativeDefinitionOfKrullValuation}
\pmcreated{2013-03-22 17:02:08}
\pmmodified{2013-03-22 17:02:08}
\pmowner{polarbear}{3475}
\pmmodifier{polarbear}{3475}
\pmtitle{alternative definition of Krull valuation}
\pmrecord{10}{39322}
\pmprivacy{1}
\pmauthor{polarbear}{3475}
\pmtype{Definition}
\pmcomment{trigger rebuild}
\pmclassification{msc}{13F30}
\pmclassification{msc}{13A18}
\pmclassification{msc}{12J20}
\pmclassification{msc}{11R99}
\pmrelated{OrderValuation}
\pmrelated{Valuation}
\pmrelated{Krullvaluation}
\pmrelated{KrullValuation}

% this is the default PlanetMath preamble.  as your knowledge
% of TeX increases, you will probably want to edit this, but
% it should be fine as is for beginners.

% almost certainly you want these
\usepackage{amssymb}
\usepackage{amsmath}
\usepackage{amsfonts}

% used for TeXing text within eps files
%\usepackage{psfrag}
% need this for including graphics (\includegraphics)
%\usepackage{graphicx}
% for neatly defining theorems and propositions
\usepackage{amsthm}
% making logically defined graphics
%%%\usepackage{xypic}

% there are many more packages, add them here as you need them

% define commands here
\newtheorem{defn}{Definition}
\begin{document}
 Let $G$ be an abelian totally ordered group, denoted additively. We adjoin to $G$ a new element $\infty$ such that $g < +\infty$, for all $g\in G$ and we extend the addition on $G_{\infty} = G \cup \{+\infty\}$ by declaring $g + (+\infty) = (+\infty) + (+\infty) = +\infty$.
\begin{defn} Let $R$ be an unital ring, a \emph{valuation} of $R$ with values in $G$ is a function from $R$ to $G_{\infty}$ such that , for all $x, y \in R$:

1) $v(xy)= v(x)+v(y)$,

2) $v(x+y) \geq \min \{v(x),v(y)\}$,

3) $v(x)=+\infty$ iff $v(x)=0$.

\end{defn}
\textbf{Remarks} a) The condition 1) means that $v$ is a homomorhism of $R\smallsetminus\{0\}$ with multiplication in the group $G$. In particular, $v(1)=0$ and $v(-x)=v(x)$, for all $x\in G$. If $x$ is invertible then $0=v(1)=v(xx^{-1})=v(x)+v(x^{-1})$, so $v(x^{-1}) = -v(x)$.\newline
b) If 3) is replaced by the \PMlinkescapetext{weaker} condition $v(0)=+\infty$ then the set $P = v^{-1}\{+\infty\}$ is a prime ideal of $R$ and $v$ is \PMlinkescapetext{induced} on the integral domain $R/P$.\newline
c) In particular, conditions 1) and 3) \PMlinkescapetext{imply} that $R$ is an integral domain and let $K$ be its quotient field. There is a unique valuation of $K$ with values in $G$ that extends $v$, namely $v(x/y)=v(x)-v(y)$, for all $x\in R$ and $y\in R\smallsetminus\{0\}$.\newline
d) The element $v(x)$ is sometimes denoted by $vx$.

%%%%%
%%%%%
\end{document}
