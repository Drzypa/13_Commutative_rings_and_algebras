\documentclass[12pt]{article}
\usepackage{pmmeta}
\pmcanonicalname{ExtensionOfValuationFromCompleteBaseField}
\pmcreated{2013-03-22 15:01:01}
\pmmodified{2013-03-22 15:01:01}
\pmowner{pahio}{2872}
\pmmodifier{pahio}{2872}
\pmtitle{extension of valuation from complete base field}
\pmrecord{9}{36724}
\pmprivacy{1}
\pmauthor{pahio}{2872}
\pmtype{Theorem}
\pmcomment{trigger rebuild}
\pmclassification{msc}{13F30}
\pmclassification{msc}{13A18}
\pmclassification{msc}{12J20}
\pmclassification{msc}{11R99}
%\pmkeywords{algebraic field extension}
\pmrelated{CompleteUltrametricField}
\pmrelated{ValueGroupOfCompletion}
\pmrelated{NthRoot}

\endmetadata

% this is the default PlanetMath preamble.  as your knowledge
% of TeX increases, you will probably want to edit this, but
% it should be fine as is for beginners.

% almost certainly you want these
\usepackage{amssymb}
\usepackage{amsmath}
\usepackage{amsfonts}

% used for TeXing text within eps files
%\usepackage{psfrag}
% need this for including graphics (\includegraphics)
%\usepackage{graphicx}
% for neatly defining theorems and propositions
%\usepackage{amsthm}
% making logically defined graphics
%%%\usepackage{xypic}

% there are many more packages, add them here as you need them

% define commands here
\begin{document}
Here the valuations are of rank one, and it may be supposed that the values are real numbers. 

\begin{itemize}
 \item Assume a finite field extension $K/k$ and a valuation of $K$. \,If the base field is \PMlinkname{complete}{Complete} with regard to this valuation, so is also the extension field.
 \item If $K/k$ is an algebraic field extension and if the base field $k$ is \PMlinkname{complete}{Complete} with regard to its valuation \,$|\cdot|$, \, then this valuation has one and only one extension to the field $K$. \,This extension is determined by
  $$|\alpha| = \sqrt[n]{|N(\alpha)|}\quad (\alpha \in K),$$
where $N(\alpha)$ is the norm of the element $\alpha$ in the simple field extension $k(\alpha)/k$ and $n$ is the degree of this field extension.
\end{itemize}

These theorems concern also Archimedean valuations.
%%%%%
%%%%%
\end{document}
