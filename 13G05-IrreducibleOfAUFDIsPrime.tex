\documentclass[12pt]{article}
\usepackage{pmmeta}
\pmcanonicalname{IrreducibleOfAUFDIsPrime}
\pmcreated{2013-03-22 18:04:35}
\pmmodified{2013-03-22 18:04:35}
\pmowner{pahio}{2872}
\pmmodifier{pahio}{2872}
\pmtitle{irreducible of a UFD is prime}
\pmrecord{7}{40611}
\pmprivacy{1}
\pmauthor{pahio}{2872}
\pmtype{Theorem}
\pmcomment{trigger rebuild}
\pmclassification{msc}{13G05}
\pmclassification{msc}{13F15}
\pmrelated{PrimeElementIsIrreducibleInIntegralDomain}

\endmetadata

% this is the default PlanetMath preamble.  as your knowledge
% of TeX increases, you will probably want to edit this, but
% it should be fine as is for beginners.

% almost certainly you want these
\usepackage{amssymb}
\usepackage{amsmath}
\usepackage{amsfonts}

% used for TeXing text within eps files
%\usepackage{psfrag}
% need this for including graphics (\includegraphics)
%\usepackage{graphicx}
% for neatly defining theorems and propositions
 \usepackage{amsthm}
% making logically defined graphics
%%%\usepackage{xypic}

% there are many more packages, add them here as you need them

% define commands here

\theoremstyle{definition}
\newtheorem*{thmplain}{Theorem}

\begin{document}
\PMlinkescapeword{generator}

Any irreducible element of a factorial ring $D$ is a prime element of $D$.\\

{\em Proof.}\, Let $p$ be an arbitrary irreducible element of $D$.\, Thus $p$ is a non-unit.\,  If\, 
$ab \in (p)\smallsetminus\{0\}$,\, then\, $ab = cp$\, with\, 
$c \in D$.\, We write $a,\,b,\,c$ as products of irreducibles:
$$a \;=\; p_1\cdots p_l, \quad b \;=\; q_1\cdots q_m, \quad c \;=\; r_1\cdots r_n$$
Here, one of those first two products may me empty, i.e. it may be a unit.\, We have
\begin{align}
p_1\cdots p_l\,q_1\cdots q_m \;=\; r_1\cdots r_n\,p.
\end{align}
Due to the uniqueness of prime factorization, every factor $r_k$ is an associate of certain of the $l\!+\!m$ irreducibles on the left hand side of (1).\, Accordingly, $p$ has to be an associate of one of the $p_i$'s or $q_j$'s.\, It means that either\, $a \in (p)$\, or\, $b \in (p)$.\, Thus, $(p)$ is a prime ideal of $D$, and its generator must be a prime element.



%%%%%
%%%%%
\end{document}
