\documentclass[12pt]{article}
\usepackage{pmmeta}
\pmcanonicalname{DedekindDomainsWithFinitelyManyPrimesArePIDs}
\pmcreated{2013-03-22 18:35:18}
\pmmodified{2013-03-22 18:35:18}
\pmowner{gel}{22282}
\pmmodifier{gel}{22282}
\pmtitle{Dedekind domains with finitely many primes are PIDs}
\pmrecord{6}{41315}
\pmprivacy{1}
\pmauthor{gel}{22282}
\pmtype{Theorem}
\pmcomment{trigger rebuild}
\pmclassification{msc}{13F05}
\pmclassification{msc}{11R04}
%\pmkeywords{Dedekind domain}
%\pmkeywords{prime ideal}
%\pmkeywords{principal ideal domain}
\pmrelated{DivisorTheory}

\endmetadata

% this is the default PlanetMath preamble.  as your knowledge
% of TeX increases, you will probably want to edit this, but
% it should be fine as is for beginners.

% almost certainly you want these
\usepackage{amssymb}
\usepackage{amsmath}
\usepackage{amsfonts}

% used for TeXing text within eps files
%\usepackage{psfrag}
% need this for including graphics (\includegraphics)
%\usepackage{graphicx}
% for neatly defining theorems and propositions
\usepackage{amsthm}
% making logically defined graphics
%%%\usepackage{xypic}

% there are many more packages, add them here as you need them

% define commands here
\newtheorem*{theorem*}{Theorem}
\newtheorem*{lemma*}{Lemma}
\newtheorem*{corollary*}{Corollary}
\newtheorem{theorem}{Theorem}
\newtheorem{lemma}{Lemma}
\newtheorem{corollary}{Corollary}


\begin{document}
A commutative ring in which there are only finitely many maximal ideals is known as a semi-local ring. For such rings, the property of being a Dedekind domain and of being a principal ideal domain coincide.

\begin{theorem*}
A Dedekind domain in which there are only finitely many prime ideals is a principal ideal domain.
\end{theorem*}

This result is sometimes proven using the chinese remainder theorem or, alternatively, it follows directly from the fact that invertible ideals in semi-local rings are principal.

Suppose that $R$ is a Dedekind domain such as the ring of algebraic integers in a number field. Although there are infinitely many prime ideals in such a ring, we can use the result that localizations of Dedekind domains are Dedekind and apply the above theorem to localizations of $R$.

In particular, if $\mathfrak{p}$ is a nonzero prime ideal, then $R_\mathfrak{p}\equiv(R\setminus\mathfrak{p})^{-1}R$ is a Dedekind domain with a unique nonzero prime ideal, so the theorem shows that it is a principal ideal domain.

%%%%%
%%%%%
\end{document}
