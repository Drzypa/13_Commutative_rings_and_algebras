\documentclass[12pt]{article}
\usepackage{pmmeta}
\pmcanonicalname{IntegralDomain}
\pmcreated{2013-03-22 11:50:24}
\pmmodified{2013-03-22 11:50:24}
\pmowner{djao}{24}
\pmmodifier{djao}{24}
\pmtitle{integral domain}
\pmrecord{16}{30393}
\pmprivacy{1}
\pmauthor{djao}{24}
\pmtype{Definition}
\pmcomment{trigger rebuild}
\pmclassification{msc}{13G05}
\pmsynonym{domain}{IntegralDomain}
\pmrelated{CancellationRing}
\pmrelated{ZeroDivisor}
\pmrelated{WhyEuclideanDomains}

\usepackage{amssymb}
\usepackage{amsmath}
\usepackage{amsfonts}
%\usepackage{graphicx}
%%%%%\usepackage{xypic}
\begin{document}
An \emph{integral domain}, or \emph{domain}, is a commutative cancellation ring with an identity element $1 \neq 0$.

Integral domains are sometimes allowed to be noncommutative, but we adopt the convention that an integral domain is commutative unless otherwise specified.

This notion has essentially nothing to do with the \PMlinkname{domain of a function}{Domain}.  It is also not very closely related to the notion of integral, which is applied to ring elements, or that of integral closure, which is applied to extensions of rings, although these concepts are normally applied to integral domains.  An integral domain shares some of the properties of the integers (more than other kinds of rings, but by no means all those of interest).  Integral domains have fraction fields, which play the role of the rational numbers, and they each have a characteristic (which is either a prime number or zero).
%%%%%
%%%%%
%%%%%
%%%%%
\end{document}
