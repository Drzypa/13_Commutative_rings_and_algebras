\documentclass[12pt]{article}
\usepackage{pmmeta}
\pmcanonicalname{ExtensionByLocalization}
\pmcreated{2013-03-22 14:24:42}
\pmmodified{2013-03-22 14:24:42}
\pmowner{pahio}{2872}
\pmmodifier{pahio}{2872}
\pmtitle{extension by localization}
\pmrecord{15}{35916}
\pmprivacy{1}
\pmauthor{pahio}{2872}
\pmtype{Definition}
\pmcomment{trigger rebuild}
\pmclassification{msc}{13B30}
\pmsynonym{ring extension by localization}{ExtensionByLocalization}
\pmrelated{TotalRingOfFractions}
\pmrelated{ClassicalRingOfQuotients}
\pmrelated{FiniteRingHasNoProperOverrings}
\pmdefines{ring of fractions}
\pmdefines{ring of quotients}

% this is the default PlanetMath preamble.  as your knowledge
% of TeX increases, you will probably want to edit this, but
% it should be fine as is for beginners.

% almost certainly you want these
\usepackage{amssymb}
\usepackage{amsmath}
\usepackage{amsfonts}

% used for TeXing text within eps files
%\usepackage{psfrag}
% need this for including graphics (\includegraphics)
%\usepackage{graphicx}
% for neatly defining theorems and propositions
%\usepackage{amsthm}
% making logically defined graphics
%%%\usepackage{xypic}

% there are many more packages, add them here as you need them

% define commands here
\begin{document}
\PMlinkescapeword{extension}
Let $R$ be a commutative ring and let $S$ be a non-empty multiplicative subset of $R$.\, Then the \PMlinkname{localisation}{Localization} of $R$ at $S$ gives the commutative ring \,$S^{-1}R$\, but, generally, it has no subring isomorphic to $R$.\, Formally, $S^{-1}R$ consists of all elements $\frac{a}{s}$ ($a \in R$, $s \in S$).\, Therefore, $S^{-1}R$ is called also a {\em ring of quotients} of $R$.\, If\, $0 \in S$, then\, 
$S^{-1}R = \{0\}$;\, we assume now that\, $0 \notin S$.

\begin{itemize}
\item The mapping \,$a \mapsto \frac{as}{s}$, where $s$ is any element of $S$, is well-defined and a homomorphism from $R$ to $S^{-1}R$. \,All elements of $S$ are mapped to units of $S^{-1}R$.
\item If, especially, $S$ contains no zero divisors of the ring $R$, then the above mapping is an isomorphism from $R$ to a certain subring of $S^{-1}R$, and we may think that\, $S^{-1}R \supseteq R$.\, In this case, the ring of fractions of $R$ is an extension ring of $R$; this concerns of course the case that $R$ is an integral domain.\, But if $R$ is a finite ring, then\, $S^{-1}R = R$,\, and no proper extension is obtained.
\end{itemize}
%%%%%
%%%%%
\end{document}
