\documentclass[12pt]{article}
\usepackage{pmmeta}
\pmcanonicalname{RingWithoutIrreducibles}
\pmcreated{2014-05-29 11:39:19}
\pmmodified{2014-05-29 11:39:19}
\pmowner{pahio}{2872}
\pmmodifier{pahio}{2872}
\pmtitle{ring without irreducibles}
\pmrecord{15}{36990}
\pmprivacy{1}
\pmauthor{pahio}{2872}
\pmtype{Example}
\pmcomment{trigger rebuild}
\pmclassification{msc}{13G05}
\pmrelated{FieldOfAlgebraicNumbers}

\endmetadata

% this is the default PlanetMath preamble.  as your knowledge
% of TeX increases, you will probably want to edit this, but
% it should be fine as is for beginners.

% almost certainly you want these
\usepackage{amssymb}
\usepackage{amsmath}
\usepackage{amsfonts}

% used for TeXing text within eps files
%\usepackage{psfrag}
% need this for including graphics (\includegraphics)
%\usepackage{graphicx}
% for neatly defining theorems and propositions
 \usepackage{amsthm}
% making logically defined graphics
%%%\usepackage{xypic}

% there are many more packages, add them here as you need them

% define commands here

\theoremstyle{definition}
\newtheorem*{thmplain}{Theorem}
\begin{document}
An integral domain may not \PMlinkescapetext{contain} any 
irreducible elements.\, One such example is the ring of all 
algebraic integers.\, Any nonzero non-unit $\vartheta$ of this ring satisfies an equation
    $$x^n\!+\!a_1x^{n-1}\!+\!\cdots\!+\!a_{n-1}x\!+\!a_n = 0$$
with integer coefficients $a_j$, since it is an algebraic 
integer; moreover,\, we can assume that\, 
$a_n = \mbox{N}(\vartheta) \neq \pm 1$\, (see norm and trace of 
algebraic number: \PMlinkescapetext{theorem} 2).\, The element 
$\vartheta$ has the \PMlinkescapetext{decomposition}
$$\vartheta = \sqrt{\vartheta}\!\cdot\!\sqrt{\vartheta}.$$
Here, $\sqrt{\vartheta}$ belongs to the ring because it
satisfies the equation
$$x^{2n}\!+\!a_1x^{2n-2}\!+\!\cdots\!+\!a_{n-1}x^2\!+\!a_n = 0,$$
and it is no unit.\, Thus the element $\vartheta$ is not irreducible.
%%%%%
%%%%%
\end{document}
