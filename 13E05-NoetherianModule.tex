\documentclass[12pt]{article}
\usepackage{pmmeta}
\pmcanonicalname{NoetherianModule}
\pmcreated{2013-03-22 11:44:57}
\pmmodified{2013-03-22 11:44:57}
\pmowner{yark}{2760}
\pmmodifier{yark}{2760}
\pmtitle{Noetherian module}
\pmrecord{24}{30189}
\pmprivacy{1}
\pmauthor{yark}{2760}
\pmtype{Definition}
\pmcomment{trigger rebuild}
\pmclassification{msc}{13E05}
\pmclassification{msc}{33C75}
\pmclassification{msc}{33E05}
\pmclassification{msc}{14J27}
\pmclassification{msc}{86A30}
\pmclassification{msc}{14H52}
%\pmkeywords{commutative algebra algebraic geometry}
\pmrelated{Noetherian}
\pmdefines{Noetherian}
\pmdefines{Noetherian left module}
\pmdefines{Noetherian right module}
\pmdefines{left Noetherian module}
\pmdefines{right Noetherian module}

\usepackage{amssymb}
\usepackage{amsmath}
\usepackage{amsfonts}
\begin{document}
\PMlinkescapeword{equivalent}
\PMlinkescapephrase{generated by}
\PMlinkescapeword{left}
\PMlinkescapephrase{left noetherian}
\PMlinkescapeword{property}
\PMlinkescapeword{right}
\PMlinkescapephrase{right noetherian}
\PMlinkescapeword{similar}
\PMlinkescapeword{simple}

A (left or right) module $M$ over a ring $R$ is said to be \emph{Noetherian}
if the following equivalent conditions hold:
\begin{enumerate}
\item Every submodule of $M$ is finitely generated over $R$.
\item The ascending chain condition holds on submodules.
\item Every nonempty family of submodules has a maximal element.
\end{enumerate}

For example, the $\mathbb{Z}$-module $\mathbb{Q}$ is not Noetherian,
as it is not finitely generated,
but the $\mathbb{Z}$-module $\mathbb{Z}$ is Noetherian,
as every submodule is generated by a single element.

Observe that changing the ring can change whether a module is Noetherian or not:
for example, the $\mathbb{Q}$-module $\mathbb{Q}$ is Noetherian,
since it is \PMlinkname{simple}{SimpleModule}
(has no nontrivial submodules). 

There is also a notion of \PMlinkname{Noetherian for rings}{Noetherian}:
a ring is left Noetherian if it is Noetherian as a left module over itself,
and right Noetherian if it is Noetherian as a right module over itself.
For non-commutative rings, these two notions can differ.

The corresponding property for groups is usually called the maximal condition.

Finally, there is the somewhat related notion of a
\PMlinkname{Noetherian topological space}{NoetherianTopologicalSpace}.

%%%%%
%%%%%
%%%%%
%%%%%
\end{document}
