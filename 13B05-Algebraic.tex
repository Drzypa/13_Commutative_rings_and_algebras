\documentclass[12pt]{article}
\usepackage{pmmeta}
\pmcanonicalname{Algebraic}
\pmcreated{2013-11-05 18:32:06}
\pmmodified{2013-11-05 18:32:06}
\pmowner{drini}{3}
\pmmodifier{pahio}{2872}
\pmtitle{algebraic}
\pmrecord{8}{30705}
\pmprivacy{1}
\pmauthor{drini}{2872}
\pmtype{Definition}
\pmcomment{trigger rebuild}
\pmclassification{msc}{13B05}
\pmclassification{msc}{11R04}
\pmclassification{msc}{11R32}
\pmrelated{AlgebraicNumber}
\pmrelated{FiniteExtension}
\pmrelated{ProofOfTranscendentalRootTheorem}

\usepackage{amssymb}
\usepackage{amsmath}
\usepackage{amsfonts}
\usepackage{graphicx}
%%%\usepackage{xypic}
\begin{document}
Let $K$ be an extension field of $F$ and let $a\in K$. 

If there is  a nonzero polynomial $f\in F[x]$ such that 
$f(a)=0$ (in $K$) we say that $a$ is \emph{algebraic over $F$}.

For example, $\sqrt{2}\in\mathbb{R}$ is algebraic over 
$\mathbb{Q}$ since there is a nonzero polynomial with rational 
coefficients, namely $f(x)=x^2-2$, such that $f(\sqrt{2})=0$.

If all elements of $K$ are algebraic over $F$, one says that 
the {\it field extension} $K/F$ is {\it algebraic}.
%%%%%
%%%%%
%%%%%
\end{document}
