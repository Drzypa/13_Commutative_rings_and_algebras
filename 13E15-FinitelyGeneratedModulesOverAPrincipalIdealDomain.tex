\documentclass[12pt]{article}
\usepackage{pmmeta}
\pmcanonicalname{FinitelyGeneratedModulesOverAPrincipalIdealDomain}
\pmcreated{2013-03-22 13:55:22}
\pmmodified{2013-03-22 13:55:22}
\pmowner{yark}{2760}
\pmmodifier{yark}{2760}
\pmtitle{finitely generated modules over a principal ideal domain}
\pmrecord{21}{34680}
\pmprivacy{1}
\pmauthor{yark}{2760}
\pmtype{Topic}
\pmcomment{trigger rebuild}
\pmclassification{msc}{13E15}

\endmetadata

\usepackage{amssymb}
\usepackage{amsmath}
\usepackage{amsfonts}
\usepackage{amsthm}

\newtheorem*{cor*}{Corollary}
\newtheorem*{lem*}{Lemma}
\newtheorem*{thm*}{Theorem}

\def\tor{\operatorname{tor}}
\begin{document}
\PMlinkescapephrase{generated by}
\PMlinkescapephrase{generating set}

Let $R$ be a principal ideal domain and let $M$ be a finitely generated $R$-
module.

\begin{lem*}
\label{l1}
Let $M$ be a submodule of the $R$-module $R^n$. Then $M$ is 
free and finitely generated by $s\le n$ elements.
\end{lem*}
\begin{proof}
For $n=1$ this is clear,
since $M$ is an ideal of $R$ and is generated by some element $a \in R$.
Now suppose that the statement is true
for all submodules of $R^m, 1 \le m \le n-1$.

For a submodule $M$ of $R^n$
we define $f\colon M\to R$ by $(k_1, \ldots, k_n) \mapsto k_1$.
The image of $f$ is an ideal $\mathfrak{I}$ in $R$.
If $\mathfrak{I}=\{0\}$, then $M \subseteq \ker(f)=(0) \times R^{n-1}$.
Otherwise, $\mathfrak{I}=(g), g \ne 0$.
In the first case, elements of $\ker(f)$
can be bijectively mapped to $R^{n-1}$ by the function 
$\ker(f) \to R^{n-1}$ given by
$(0, k_1, \ldots, k_{n-1}) \mapsto (k_1,\ldots, k_{n-1})$;
so the image of $M$ under this mapping is a submodule of $R^{n-1}$,
which by the induction hypothesis is finitely generated and free.

Now let $x \in M$ such that $f(x) =gh$ and $y \in M$ with $f(y)=g$.
Then $f(x-hy)=f(x) -f(hy)=0$,
which is equivalent to $x-hy \in \ker(f) \cap R^n:=N$
which is isomorphic to a submodule of $R^{n-1}$.
This shows that $Rx+N=M$.

Let $\{g_1, \ldots, g_s\}$ be a basis of $N$.
By assumption, $s \le n-1$.
We'll show that $\{x, g_1, \ldots, g_s\}$ is linearly independent.
So let $rx +\sum_{i=1}^s r_ig_i=0$.
The first component of the $g_i$ are 0,
so the first component of $rx$ must also be 0.
Since $f(x)$ is a multiple of $g \ne 0$ and $0=r\cdot f(x)$, then $r=0$.
Since $\{g_1, \ldots, g_s\}$ are linearly independent,
$\{x, g_1, \ldots, g_s\}$ is a generating set of $M$ with $s+1\le n$ elements.
\end{proof}

\begin{cor*}
\label{c1}
If $M$ is a finitely generated $R$-module
over a PID generated by $s$ elements
and $N$ is a submodule of $M$,
then $N$ can be generated by $s$ or fewer elements.
\end{cor*}
\begin{proof}
Let $\{g_1, \ldots, g_s\}$ be a generating set of $M$
and $f\colon R^s \to M$, $(r_1, \ldots, r_s) \mapsto \sum_{i=1}^s r_ig_i$.
Then the inverse image $N^{'}$ of $N$ is a submodule of $R^s$,
and according to lemma \ref{l1} can be generated by $s$ or fewer elements.
Let $n_1,\ldots, n_t$ be a generating set of $n^{'}$;
then $t \le s$, and since $f$ is surjective,
$f(n_1), \ldots, f(n_t)$ is a generating set of $N$.
\end{proof}

\begin{thm*}
Let $M$ be a finitely generated module over a principal ideal domain $R$.
\begin{description}
\item[(I)]
Note that $M/\tor(M)$ is torsion-free, that is, $\tor(M/\tor(M))=\{0\}$.
In particular, if $M$ is torsion-free, then $M$ is free.
\item[(II)]
Let $\tor(M)$ be a proper submodule of $M$.
Then there exists a finitely generated free submodule $F$ of $M$
such that $M=F \oplus \tor(M)$.
\end{description}
\end{thm*}
Proof of (I): Let $T=\tor(M)$.
For $m\in M$, $\overline{m}$ denotes the coset modulo $T$ generated by $m$.
Let $m$ be a torsion element of $M/T$,
so there exists $\alpha \in R \setminus \{0\}$
such that $\alpha\cdot \overline{m}=0$,
which means $\alpha\cdot \overline{m} \subseteq T$.
But then $\alpha \cdot m$ is a member of $T$,
and this implies that $M/T$ has no non-zero torsion elements
(which is obvious if $M=\tor(M)$).

Now let $M$ be a finitely generated torsion-free $R$-module.
Choose a maximal linearly independent subset $S$ of $M$,
and let $F$ be the submodule of $M$ generated by $S$.
Let $\{m_1,\dots,m_n\}$ be a set of generators of $M$.
For each $i=1,\dots,n$ there is a non-zero $r_i\in R$
such that $r_i\cdot m_i\in F$.
Put $r=\prod_{i=1}^n r_i$.
Then $r$ is non-zero,
and we have $r\cdot m_i\in F$ for each $i=1,\dots,n$.
As $M$ is torsion-free, the multiplication by $r$ is injective,
so $M \cong r \cdot M \subseteq F$.
So $M$ is isomorphic to a submodule of a free module, and is therefore free.

Proof of (II): Let $\pi\colon M \to M/T$ be defined by $a \mapsto a + T$.
Then $\pi$ is surjective,
so $m_1, \ldots, m_t \in M$ can be chosen such that $\pi(m_i)=n_i$,
where the $n_i$'s are a basis of $M/T$.
If $0_M=\sum_{i=1}^t a_im_i$, then $0_n=\sum_{i=1}^t a_in_i$.
Since $n_1,\ldots,n_t$ are linearly independent in $N$
it follows $0=a_1=\ldots =a_t$.
So the submodule spanned by $m_1,\ldots,m_t$ of $M$ is free.

Now let $m$ be some element of $M$ and $\pi(m)=\sum_{i=1}^t a_in_i$.
This is equivalent to $m-\left(\sum_{i=1}^t a_in_i\right) \in \ker(\pi)=T$.
Hence, any $m\in M$ is a sum of the form $f+t$,
for some $f \in F$ and $t \in T$. 
Since $F$ is torsion-free, $F \cap T=\{0\}$,
and it follows that $M=F \oplus T$.
%%%%%
%%%%%
\end{document}
