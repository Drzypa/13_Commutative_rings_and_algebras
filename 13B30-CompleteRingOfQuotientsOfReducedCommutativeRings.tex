\documentclass[12pt]{article}
\usepackage{pmmeta}
\pmcanonicalname{CompleteRingOfQuotientsOfReducedCommutativeRings}
\pmcreated{2013-03-22 18:27:33}
\pmmodified{2013-03-22 18:27:33}
\pmowner{jocaps}{12118}
\pmmodifier{jocaps}{12118}
\pmtitle{complete ring of quotients of reduced commutative rings}
\pmrecord{6}{41125}
\pmprivacy{1}
\pmauthor{jocaps}{12118}
\pmtype{Theorem}
\pmcomment{trigger rebuild}
\pmclassification{msc}{13B30}
\pmrelated{CompleteRingOfQuotients}
\pmrelated{essentialmonomorphism}
\pmdefines{rational extension}

% this is the default PlanetMath preamble.  as your knowledge
% of TeX increases, you will probably want to edit this, but
% it should be fine as is for beginners.

% almost certainly you want these
\usepackage{amssymb}
\usepackage{amsmath}
\usepackage{amsfonts}

% used for TeXing text within eps files
%\usepackage{psfrag}
% need this for including graphics (\includegraphics)
%\usepackage{graphicx}
% for neatly defining theorems and propositions
%\usepackage{amsthm}
% making logically defined graphics
%%%\usepackage{xypic}

% there are many more packages, add them here as you need them

% define commands here

\begin{document}
There is a characterization of complete ring of quotients of reduced commutative rings. Let $A$ be a \PMlinkname{reduced}{ReducedRing} commutative ring, then if $B$ is an overring of $A$ and if for any element $b\in B\backslash\{0\}$ there is an $a\in A$ such that $ab\in A\backslash\{0\}$, then $B$ is said to be a \emph{rational extension} of $A$. See how similar this is with the definition of essential extension in the category of rings, obviously all rational extensions of reduced commutative rings are also essential extensions. Furthermore there is a maximum (upto $A$-isomorphism) rational extension of $A$ and this is in fact the complete ring of quotients of $A$.
%%%%%
%%%%%
\end{document}
