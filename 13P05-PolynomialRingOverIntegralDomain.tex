\documentclass[12pt]{article}
\usepackage{pmmeta}
\pmcanonicalname{PolynomialRingOverIntegralDomain}
\pmcreated{2013-03-22 15:10:06}
\pmmodified{2013-03-22 15:10:06}
\pmowner{pahio}{2872}
\pmmodifier{pahio}{2872}
\pmtitle{polynomial ring over integral domain}
\pmrecord{10}{36918}
\pmprivacy{1}
\pmauthor{pahio}{2872}
\pmtype{Theorem}
\pmcomment{trigger rebuild}
\pmclassification{msc}{13P05}
\pmrelated{RingAdjunction}
\pmrelated{FormalPowerSeries}
\pmrelated{ZeroPolynomial2}
\pmrelated{PolynomialRingOverFieldIsEuclideanDomain}
\pmdefines{coefficient ring}

% this is the default PlanetMath preamble.  as your knowledge
% of TeX increases, you will probably want to edit this, but
% it should be fine as is for beginners.

% almost certainly you want these
\usepackage{amssymb}
\usepackage{amsmath}
\usepackage{amsfonts}

% used for TeXing text within eps files
%\usepackage{psfrag}
% need this for including graphics (\includegraphics)
%\usepackage{graphicx}
% for neatly defining theorems and propositions
 \usepackage{amsthm}
% making logically defined graphics
%%%\usepackage{xypic}

% there are many more packages, add them here as you need them

% define commands here

\theoremstyle{definition}
\newtheorem*{thmplain}{Theorem}
\begin{document}
\begin{thmplain}
\, If the {\em coefficient ring} $R$  is an integral domain, then so is also its polynomial ring $R[X]$.
\end{thmplain}

{\em Proof.}\, Let $f(X)$ and $g(X)$ be two non-zero polynomials in $R[X]$ and let $a_f$ and $b_g$ be their leading coefficients, respectively.\, Thus\, $a_f \neq 0$,\, $b_g \neq 0$,\, and because $R$ has no zero divisors,\, $a_fb_g \neq 0$.\, But the product $a_fb_g$ is the leading coefficient of $f(X)g(X)$ and so $f(X)g(X)$ cannot be the zero polynomial.\, Consequently, $R[X]$ has no zero divisors, Q.E.D.\\

\textbf{Remark.}\, The theorem may by induction be generalized for the polynomial ring\, $R[X_1,\,X_2,\,\ldots,\,X_n]$.
%%%%%
%%%%%
\end{document}
