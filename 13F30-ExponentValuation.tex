\documentclass[12pt]{article}
\usepackage{pmmeta}
\pmcanonicalname{ExponentValuation}
\pmcreated{2013-03-22 17:59:31}
\pmmodified{2013-03-22 17:59:31}
\pmowner{pahio}{2872}
\pmmodifier{pahio}{2872}
\pmtitle{exponent valuation}
\pmrecord{12}{40503}
\pmprivacy{1}
\pmauthor{pahio}{2872}
\pmtype{Definition}
\pmcomment{trigger rebuild}
\pmclassification{msc}{13F30}
\pmclassification{msc}{13A18}
\pmclassification{msc}{12J20}
\pmclassification{msc}{11R99}
\pmsynonym{exponent of field}{ExponentValuation}
%\pmkeywords{exponent}
\pmrelated{DiscreteValuation}
\pmrelated{OrderValuation}
\pmrelated{UltrametricTriangleInequality}
\pmrelated{DivisorTheoryAndExponentValuations}
\pmrelated{DivisorTheory}
\pmdefines{exponent of a field}
\pmdefines{exponent of the field}

% this is the default PlanetMath preamble.  as your knowledge
% of TeX increases, you will probably want to edit this, but
% it should be fine as is for beginners.

% almost certainly you want these
\usepackage{amssymb}
\usepackage{amsmath}
\usepackage{amsfonts}

% used for TeXing text within eps files
%\usepackage{psfrag}
% need this for including graphics (\includegraphics)
%\usepackage{graphicx}
% for neatly defining theorems and propositions
 \usepackage{amsthm}
% making logically defined graphics
%%%\usepackage{xypic}

% there are many more packages, add them here as you need them

% define commands here

\theoremstyle{definition}
\newtheorem*{thmplain}{Theorem}

\begin{document}
\textbf{Definition}.\, A function $\nu$ defined in a field $K$ is called an {\em exponent valuation} or shortly an {\em exponent} of the field, if it satisfies the following conditions:
\begin{enumerate}
\item \,$\nu(0) \,=\, \infty$\, and $\nu(\alpha)$ runs all rational integers when $\alpha$ runs the nonzero elements of $K$.
\item \,$\nu(\alpha\beta) \;=\; \nu(\alpha)\!+\!\nu(\beta)$.
\item \,$\nu(\alpha\!+\!\beta) \;\geqq\; \min\{\nu(\alpha),\,\nu(\beta)\}$.\\
\end{enumerate}
Note that because of the discrete value set $\mathbb{Z}$, an exponent valuation belongs to the discrete valuations, and 
because of notational causes, to the order valuations.\\

\textbf{Properties.}\\
$\nu(1) = 0$\\
$\nu(-\alpha) = \nu(\alpha)$\\
$\displaystyle\nu\left(\frac{\alpha}{\beta}\right) = \nu(\alpha)\!-\!\nu(\beta)$\\
$\nu(\alpha^n) = n\,\nu(\alpha)$\\
$\nu(\alpha_1+\ldots+\alpha_n) \geqq \min\{\nu(\alpha),\,\ldots,\,\nu(\alpha_n)\}$\\
$\nu(\alpha\!+\!\beta) = \min\{\nu(\alpha),\,\nu(\beta)\} \quad \mbox{if}\;\;\;\nu(\alpha) \neq \nu(\beta)$\\


\textbf{Example.}\, If an integral domain $\mathcal{O}$ has a divisor theory \,$\mathcal{O}^* \to \mathfrak{D}$, then for each prime divisor $\mathfrak{p}$ there is an exponent valuation $\nu_{\mathfrak{p}}$ of the quotient field $K$ of $\mathcal{O}$.\, It is given by
\[
\nu_{\mathfrak{p}}(\alpha) \;=:\;  
\begin{cases}
& \infty \quad \mbox{when  } \alpha = 0,\\
& \max\;\{k \in \mathbb{Z}\,\vdots\;\; \mathfrak{p}^k \mid (\alpha)\} \;\; \mbox{   when  } \,\alpha \neq 0;
\end{cases} 
\]
$$\nu_{\mathfrak{p}}(\xi) \;=:\; 
\nu_{\mathfrak{p}}(\alpha)-\nu_{\mathfrak{p}}(\beta) \; \mbox{ when }\; 
\xi = \frac{\alpha}{\beta}\mbox{ with }\,\alpha,\,\beta \in \mathcal{O}^*.$$
Hence, $\mathfrak{p}^{\nu_{\mathfrak{p}}(\alpha)}$ exactly divides $\alpha$.\, Apparently, $\nu_{\mathfrak{p}}(\xi)$ does not depend on the quotient form $\frac{\alpha}{\beta}$ for $\xi$.\, It is not hard to show that $\nu_{\mathfrak{p}}$ defined above is an exponent of the field $K$.

Different prime divisors $\mathfrak{p}$ and $\mathfrak{q}$ determine different exponents $\nu_{\mathfrak{p}}$ and $\nu_{\mathfrak{q}}$, since the condition 3 of the \PMlinkname{definition of divisor theory}{DivisorTheory} guarantees such an element $\gamma$ of $\mathcal{O}$ which in divisible by $\mathfrak{p}$ but not by $\mathfrak{q}$; then\, $\nu_{\mathfrak{p}}(\gamma) \geqq 1$,\, $\nu_{\mathfrak{q}}(\gamma) = 0$.\\


\textbf{Theorem.}\, Let\, $\nu_1,\,\ldots,\,\nu_r$\, be different exponents of a field $K$.\, Then for arbitrary set \,$n_1,\,\ldots,\,n_r$\, of integers, there exists in $K$ an element $\xi$ such that
$$\nu_1(\xi) \;=\; n_1,\;\;\ldots,\;\;\nu_r(\xi) \;=\; n_r.$$

The proof of this theorem is found in [1].

\begin{thebibliography}{9}
\bibitem{BS}{\sc S. Borewicz \& I. Safarevic}: {\em Zahlentheorie}.\, Birkh\"auser Verlag. Basel und Stuttgart (1966).
\end{thebibliography}
%%%%%
%%%%%
\end{document}
