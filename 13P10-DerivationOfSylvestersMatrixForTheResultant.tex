\documentclass[12pt]{article}
\usepackage{pmmeta}
\pmcanonicalname{DerivationOfSylvestersMatrixForTheResultant}
\pmcreated{2013-03-22 14:36:47}
\pmmodified{2013-03-22 14:36:47}
\pmowner{rspuzio}{6075}
\pmmodifier{rspuzio}{6075}
\pmtitle{derivation of Sylvester's matrix for the resultant}
\pmrecord{4}{36189}
\pmprivacy{1}
\pmauthor{rspuzio}{6075}
\pmtype{Derivation}
\pmcomment{trigger rebuild}
\pmclassification{msc}{13P10}

% this is the default PlanetMath preamble.  as your knowledge
% of TeX increases, you will probably want to edit this, but
% it should be fine as is for beginners.

% almost certainly you want these
\usepackage{amssymb}
\usepackage{amsmath}
\usepackage{amsfonts}

% used for TeXing text within eps files
%\usepackage{psfrag}
% need this for including graphics (\includegraphics)
%\usepackage{graphicx}
% for neatly defining theorems and propositions
%\usepackage{amsthm}
% making logically defined graphics
%%%\usepackage{xypic}

% there are many more packages, add them here as you need them

% define commands here
\begin{document}
Sylvester's matrix representation of the resultant can be easily derived by thinking of polynomials as linear equations in powers of $x$.  For ease of exposition, we shall consider the case where $p$ is of second order and $q$ is of third order but, once the basic idea has been grasped, it is trivial to extend it to polynomials of any orders whatsoever.

Let us start with the polynomial equations $p(x) = 0$ and $q(x) = 0$.  Written out in full, they look like
 $$a_0 x^2 + a_1 x + a_2 = 0$$
 $$b_0 x^3 + b_1 x^2 + b_2 x + b_3 = 0$$
These two equations can be combined into a single matrix equation:
 $$\begin{pmatrix} 0 & a_0 & a_1 & a_2 \cr
 b_0 & b_1 & b_2 & b_3 \cr
 \end {pmatrix} \begin {pmatrix}
 x^3 \cr x^2 \cr x \cr 1 \cr
 \end {pmatrix} = \begin {pmatrix}
 0 \cr 0 \cr 0 \cr 0 \cr \end{pmatrix}$$

Since square matrices enjoy properties which matrices of arbitrary size do not, we will add more equations so as to come up with a new matrix equation involving a square matrices.  There is no harm in adding an equation of the forn $x^k p(x) = 0$ or $x^k q(x) = 0$ to the system $p(x) = 0, q(x) = 0$ because the enlarged system will have exactly the same solutions as the original system of two equations.  Consider the system
 $$p(x) = 0$$
 $$x p(x) = 0$$
 $$x^2 p(x) = 0$$
 $$q(x) = 0$$
 $$x q(x) = 0$$
This system may be written as a matrix equation
 $$\begin{pmatrix} 0 & 0 & a_0 & a_1 & a_2 \cr
 0 & a_0 & a_1 & a_2 & 0 \cr
 a_0 & a_1 & a_2 & 0 & 0 \cr
 0 & b_0 & b_1 & b_2 & b_3 \cr
 b_0 & b_1 & b_2 & b_3 & 0 \cr
 \end {pmatrix} \begin {pmatrix}
 x^4 \cr x^3 \cr x^2 \cr x \cr 1 \cr
 \end {pmatrix} = \begin {pmatrix}
 0 \cr 0 \cr 0 \cr 0 \cr 0 \cr \end{pmatrix}$$
Now we have a square matrix.  One important property of matrix equations involving square matrices is that they only have non-trivial solutions when the determinant of the matrix vanishes.  The system $p(x) = 0, q(x) = 0$ only has a solution when $p$ and $q$ have a common root.  Hence the determinant will vanish whenever $p$ and $q$ have a common root.

Note that, at this stage, we cannot jump to the converse conclusion that $p$ and $q$ always have a common root when the determinant vanishes.  All we can say is that, if the determinant vanishes, there will be some non-zero vector in the kernel of the matrix, but we cannot say that the vector will be of the special form $(x^4, x^3, x^2, x, 1)$ that appears in the system.  To assert the converse conclusion, we need to first prove that the determinant indeed equals the resultant.  For this proof, please see the entry proof that Sylvester's determinant equals the resultant.
%%%%%
%%%%%
\end{document}
