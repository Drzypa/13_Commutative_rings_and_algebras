\documentclass[12pt]{article}
\usepackage{pmmeta}
\pmcanonicalname{EuclideanDomain}
\pmcreated{2013-03-22 12:40:42}
\pmmodified{2013-03-22 12:40:42}
\pmowner{yark}{2760}
\pmmodifier{yark}{2760}
\pmtitle{Euclidean domain}
\pmrecord{13}{32955}
\pmprivacy{1}
\pmauthor{yark}{2760}
\pmtype{Definition}
\pmcomment{trigger rebuild}
\pmclassification{msc}{13F07}
\pmsynonym{Euclidean ring}{EuclideanDomain}
\pmrelated{PID}
\pmrelated{UFD}
\pmrelated{EuclidsAlgorithm}
\pmrelated{Ring}
\pmrelated{IntegralDomain}
\pmrelated{EuclideanValuation}
\pmrelated{WhyEuclideanDomains}

\usepackage{amssymb}

\newcommand{\Z}{\mathbb{Z}}

\begin{document}
\PMlinkescapeword{even}

A \emph{Euclidean domain} is an integral domain
on which a Euclidean valuation can be defined.

Every Euclidean domain is a principal ideal domain,
and therefore also a unique factorization domain.

Any two elements of a Euclidean domain have a greatest common divisor,
which can be computed using the Euclidean algorithm.

An example of a Euclidean domain is the ring $\Z$.
Another example is the polynomial ring $F[x]$, where $F$ is any field.
Every field is also a Euclidean domain.
%%%%%
%%%%%
\end{document}
