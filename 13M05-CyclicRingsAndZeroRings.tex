\documentclass[12pt]{article}
\usepackage{pmmeta}
\pmcanonicalname{CyclicRingsAndZeroRings}
\pmcreated{2013-03-22 17:14:43}
\pmmodified{2013-03-22 17:14:43}
\pmowner{Wkbj79}{1863}
\pmmodifier{Wkbj79}{1863}
\pmtitle{cyclic rings and zero rings}
\pmrecord{9}{39576}
\pmprivacy{1}
\pmauthor{Wkbj79}{1863}
\pmtype{Result}
\pmcomment{trigger rebuild}
\pmclassification{msc}{13M05}
\pmclassification{msc}{13A99}
\pmclassification{msc}{16U99}

\endmetadata

\usepackage{amssymb}
\usepackage{amsmath}
\usepackage{amsfonts}
\usepackage{pstricks}
\usepackage{psfrag}
\usepackage{graphicx}
\usepackage{amsthm}
%%\usepackage{xypic}
\newtheorem{lemma}{Lemma}
\begin{document}
\PMlinkescapeword{generator}
\PMlinkescapeword{order}

\begin{lemma}
Let $n$ be a positive integer and $R$ be a cyclic ring of \PMlinkname{order}{OrderRing} $R$.  Then the following are equivalent:

\begin{enumerate}
\item $R$ is a zero ring;
\item $R$ has behavior $n$;
\item $R \cong n\mathbb{Z}_{n^2}$.
\end{enumerate}
\end{lemma}

\begin{proof}
To show that 1 implies 2, let $R$ have behavior $k$.  Then there exists a \PMlinkname{generator}{Generator} $r$ of the additive group of $R$ such that $r^2=kr$.  Since $R$ is a zero ring, $r^2=0_R$.  Since $kr=r^2=0_R=nr$, it must be the case that $k \equiv n \mod n$.  By definition of behavior, $k$ \PMlinkname{divides}{Divides} $n$.  Hence, $k=n$.

The fact that 2 implies 3 follows immediately from the theorem that is stated and proven at \PMlinkname{cyclic rings that are isomorphic to $k\mathbb{Z}_{kn}$}{CyclicRingsThatAreIsomorphicToKmathbbZ_kn}.

The fact that 3 implies 1 follows immediately since $n\mathbb{Z}_{n^2}$ is a zero ring.
\end{proof}

\begin{lemma}
Let $R$ be an infinite \PMlinkescapetext{cyclic ring}.  Then the following are equivalent:

\begin{enumerate}
\item $R$ is a zero ring;
\item $R$ has behavior $0$;
\item $R$ is \PMlinkname{isomorphic}{Isomorphism7} to the subring $\mathbf{B}$ of $\mathbf{M}_{2\operatorname{x}2}(\mathbb{Z})$:

$$\mathbf{B}=\left\{ \left. \left( \begin{array}{cc}
n & -n \\
n & -n \end{array} \right) \right\vert n \in \mathbb{Z} \right\}.$$
\end{enumerate}
\end{lemma}

\begin{proof}
To show that 1 implies 2, the contrapositive of the theorem that is stated and proven at \PMlinkname{cyclic rings that are isomorphic to $k\mathbb{Z}$}{CyclicRingsThatAreIsomorphicToKmathbbZ} can be used.  If $R$ does not have behavior $0$, then its behavior $k$ must be positive by definition, in which case $R \cong k\mathbb{Z}$.  It is clear that $k\mathbb{Z}$ is not a zero ring.

To show that 2 implies 3, let $r$ be a generator of the additive group of $R$.  It can be easily verified that $\varphi \colon R \to \mathbf{B}$ defined by $\displaystyle \varphi(nr)=\left( \begin{array}{cc}
n & -n \\
n & -n \end{array} \right)$ is a ring isomorphism.

The fact that 3 implies 1 follows immediately since $\mathbf{B}$ is a zero ring.
\end{proof}
%%%%%
%%%%%
\end{document}
