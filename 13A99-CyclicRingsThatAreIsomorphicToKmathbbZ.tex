\documentclass[12pt]{article}
\usepackage{pmmeta}
\pmcanonicalname{CyclicRingsThatAreIsomorphicToKmathbbZ}
\pmcreated{2013-03-22 16:02:42}
\pmmodified{2013-03-22 16:02:42}
\pmowner{Wkbj79}{1863}
\pmmodifier{Wkbj79}{1863}
\pmtitle{cyclic rings that are isomorphic to $k\mathbb{Z}$}
\pmrecord{10}{38095}
\pmprivacy{1}
\pmauthor{Wkbj79}{1863}
\pmtype{Corollary}
\pmcomment{trigger rebuild}
\pmclassification{msc}{13A99}
\pmclassification{msc}{16U99}
\pmrelated{MathbbZ}

\endmetadata

\usepackage{amssymb}
\usepackage{amsmath}
\usepackage{amsfonts}

\usepackage{psfrag}
\usepackage{graphicx}
\usepackage{amsthm}
%%\usepackage{xypic}

\newtheorem*{cor*}{Corollary}
\begin{document}
\begin{cor*}
An infinite \PMlinkname{cyclic ring}{CyclicRing3} with positive behavior $k$ is isomorphic to $k\mathbb{Z}$.
\end{cor*}

\begin{proof}
Note that $k\mathbb{Z}$ is an \PMlinkescapetext{infinite cyclic ring} and that $k$ is a \PMlinkname{generator}{Generator} of its additive group.  Since $k^2=k(k)$, then $k\mathbb{Z}$ has behavior $k$.
\end{proof}
%%%%%
%%%%%
\end{document}
