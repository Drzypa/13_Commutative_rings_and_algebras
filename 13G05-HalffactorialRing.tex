\documentclass[12pt]{article}
\usepackage{pmmeta}
\pmcanonicalname{HalffactorialRing}
\pmcreated{2013-03-22 18:31:14}
\pmmodified{2013-03-22 18:31:14}
\pmowner{pahio}{2872}
\pmmodifier{pahio}{2872}
\pmtitle{half-factorial ring}
\pmrecord{7}{41213}
\pmprivacy{1}
\pmauthor{pahio}{2872}
\pmtype{Definition}
\pmcomment{trigger rebuild}
\pmclassification{msc}{13G05}
\pmsynonym{half-factorial domain}{HalffactorialRing}
\pmdefines{HFD}

% this is the default PlanetMath preamble.  as your knowledge
% of TeX increases, you will probably want to edit this, but
% it should be fine as is for beginners.

% almost certainly you want these
\usepackage{amssymb}
\usepackage{amsmath}
\usepackage{amsfonts}

% used for TeXing text within eps files
%\usepackage{psfrag}
% need this for including graphics (\includegraphics)
%\usepackage{graphicx}
% for neatly defining theorems and propositions
 \usepackage{amsthm}
% making logically defined graphics
%%%\usepackage{xypic}

% there are many more packages, add them here as you need them

% define commands here

\theoremstyle{definition}
\newtheorem*{thmplain}{Theorem}

\begin{document}
An integral domain $D$ is called a {\em half-factorial ring} (HFD) if it satisfies the following conditions:
\begin{itemize}
\item Every nonzero element of $D$ that is not a unit can be factored into a product of a finite number of irreducibles.
\item If\, $p_1p_2\cdots p_m$\, and\, $q_1q_2\cdots q_n$\, are two factorizations of the same element $a$ into irreducibles, then\, $m = n$.
\end{itemize}

If, in \PMlinkescapetext{addition}, the irreducibles $p_i$ and $q_j$ are always pairwise associates, then $D$ is a factorial ring (UFD).\\

For example, many \PMlinkname{orders}{OrderInAnAlgebra} in the maximal order of an algebraic number field are half-factorial rings, e.g. $\mathbb{Z}[3\sqrt{2}]$ is a HFD but not a UFD (see \PMlinkexternal{this paper}{http://www.math.ndsu.nodak.edu/faculty/coykenda/paper6b.pdf}).

%%%%%
%%%%%
\end{document}
