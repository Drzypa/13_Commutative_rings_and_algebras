\documentclass[12pt]{article}
\usepackage{pmmeta}
\pmcanonicalname{SmithNormalForm}
\pmcreated{2013-03-22 13:52:08}
\pmmodified{2013-03-22 13:52:08}
\pmowner{Mathprof}{13753}
\pmmodifier{Mathprof}{13753}
\pmtitle{Smith normal form}
\pmrecord{21}{34611}
\pmprivacy{1}
\pmauthor{Mathprof}{13753}
\pmtype{Topic}
\pmcomment{trigger rebuild}
\pmclassification{msc}{13F10}
\pmdefines{elementary divisor}

\endmetadata

% this is the default PlanetMath preamble.  as your knowledge
% of TeX increases, you will probably want to edit this, but
% it should be fine as is for beginners.

% almost certainly you want these
\usepackage{amssymb}
\usepackage{amsmath}
\usepackage{amsfonts}

% used for TeXing text within eps files
%\usepackage{psfrag}
% need this for including graphics (\includegraphics)
%\usepackage{graphicx}
% for neatly defining theorems and propositions
%\usepackage{amsthm}
% making logically defined graphics
%%%\usepackage{xypic}

% there are many more packages, add them here as you need them

% define commands here
\begin{document}
This topic gives a version of the Gauss elimination algorithm for a commutative principal ideal domain which is usually described only for a field.

Let $A \ne 0$ be a $m \times n$-matrix with entries from a commutative principal ideal
domain $R$. For $a \in R \setminus \{0\}$ $\delta(a)$ denotes the number of
prime factors of $a$. Start with $t=1$ and choose $j_t$ to be the smallest
column index of $A$ with a non-zero entry.
\begin{description}
\item[(I)]
If $a_{t, j_t}=0$ and $a_{k,j_t} \ne 0$, exchange rows 1 and $k$.
\item[(II)]
If there is an entry at position $(k,j_t)$ such that $a_{t,j_t} \nmid
a_{k,j_t}$, then set $\beta =\gcd\left(a_{t,j_t}, a_{k,j_t}\right)$ and choose
$\sigma, \tau \in R$ such that
\begin{equation*}
\sigma\cdot a_{t,j_t}-\tau \cdot a_{k,j_t}=\beta.
\end{equation*}
By left-multiplication with an appropriate matrix it can be achieved that row 1
of the matrix product is the sum of row 1 multiplied by $\sigma$ and row $k$
multiplied by $(-\tau)$. Then we get $\beta$ at position $(t,j_t)$, where
$\delta(\beta) < \delta(a_{t,j_t})$.
Repeating these steps one obtains a matrix having an entry at position
$(t,j_t)$ that divides all entries in column $j_t$.
\item[(III)]
Finally, adding appropriate multiples of row $t$, it can be achieved that all
entries in column $j_t$ except for that at position $(t,j_t)$ are zero. This can
be achieved by left-multiplication with an appropriate matrix.
\end{description}
Applying the steps described above to the remaining non-zero columns of the
resulting matrix (if any), we get an $m \times n$-matrix with column indices
$j_1, \ldots, j_r$ where $r \le \min(m,n)$, each of which satisfies the
following:
\begin{enumerate}
\item
the entry at position $(l,j_l)$ is non-zero;
\item
all entries below and above position $(l,j_l)$ as well as entries left of
$(l,j_l)$ are zero.
\end{enumerate}
Furthermore, all rows below the $r$-th row are zero.

Now we can re-order the columns of this matrix so that elements on positions
$(i,i)$ for $1 \le i\le r$ are nonzero and $\delta(a_{i,i}) \le
\delta(a_{i+1,i+1})$ for $1 \le i < r$; and all columns right of the $r$-th
column (if present) are zero. For short set $\alpha_i$ for the element at
position $(i,i)$. $\delta$ has non-negative integer values; so
$\delta(\alpha_1)=0$ is equivalent to $\alpha_1$ being a unit of $R$.
$\delta(\alpha_i) =\delta(\alpha_{i+1})$ can either happen if $\alpha_i$ and
$\alpha_{i+1}$ differ by a unit factor, or if they are relatively prime. In the
latter case one can add column $i+1$ to column $i$ (which doesn't change
$\alpha_i)$ and then apply appropriate row manipulations to get $\alpha_i=1$.
And for $\delta(\alpha_i)<\delta(\alpha_{i+1})$ and $\alpha_i \nmid
\alpha_{i+1}$ one can apply step (II) after adding column $i+1$ to column $i$.
This diminishes the minimum $\delta$-values for non-zero entries of the matrix,
and by reordering columns etc. we end up with a matrix whose diagonal elements
$\alpha_i$ satisfy $\alpha_i \mid \alpha_{i+1}\;\forall\;1 \le i < r$.

Since all row and column manipulations involved in the process are \PMlinkescapetext{invertible},
this shows that there exist invertible $m \times m$ and $n \times n$-matrices
$S,T$ so that $SAT$ is

$$\left[\begin{matrix}
\alpha_1 & 0 & \ldots & 0 \\
0 & \alpha_2 &  \ldots & 0 \\
\vdots &\vdots & \ddots & \vdots\\
0 & 0 & \ldots & \alpha_r  \\
0 & 0 & \ldots & 0 \\
\vdots & \vdots & \ldots & \vdots \\
0 & 0 & \ldots & 0 
\end{matrix}\right].$$
This is the \textbf{Smith normal form} of the matrix. The elements $\alpha_i$ are unique up to associates and are called
\textbf{elementary divisors}.
%%%%%
%%%%%
\end{document}
