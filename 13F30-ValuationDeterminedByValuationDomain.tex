\documentclass[12pt]{article}
\usepackage{pmmeta}
\pmcanonicalname{ValuationDeterminedByValuationDomain}
\pmcreated{2013-03-22 14:54:58}
\pmmodified{2013-03-22 14:54:58}
\pmowner{pahio}{2872}
\pmmodifier{pahio}{2872}
\pmtitle{valuation determined by valuation domain}
\pmrecord{10}{36602}
\pmprivacy{1}
\pmauthor{pahio}{2872}
\pmtype{Theorem}
\pmcomment{trigger rebuild}
\pmclassification{msc}{13F30}
\pmclassification{msc}{13A18}
\pmclassification{msc}{12J20}
\pmclassification{msc}{11R99}
\pmrelated{ValuationDomainIsLocal}
\pmrelated{KrullValuationDomain}
\pmrelated{PlaceOfField}

% this is the default PlanetMath preamble.  as your knowledge
% of TeX increases, you will probably want to edit this, but
% it should be fine as is for beginners.

% almost certainly you want these
\usepackage{amssymb}
\usepackage{amsmath}
\usepackage{amsfonts}

% used for TeXing text within eps files
%\usepackage{psfrag}
% need this for including graphics (\includegraphics)
%\usepackage{graphicx}
% for neatly defining theorems and propositions
 \usepackage{amsthm}
% making logically defined graphics
%%%\usepackage{xypic}

% there are many more packages, add them here as you need them

% define commands here
\theoremstyle{definition}
\newtheorem*{thmplain}{Theorem}
\begin{document}
\begin{thmplain}
\, Every valuation domain determines a Krull valuation of the field of fractions.
\end{thmplain}

{\em Proof.} \,Let $R$ be a valuation domain, $K$ its field of fractions and $E$ the group of units of $R$.  Then $E$ is a normal subgroup of the multiplicative group\, $K^* = K\!\smallsetminus\!\{0\}$.\, So we can form the factor group\, $K^*/E$, consisting of all cosets $aE$ where\, $a\in K^*$,\, and attach to it the additional ``coset'' $0E$ getting thus a multiplicative group\, $K/E$\, equipped with zero.\, If\, $\mathfrak{m} = R\!\smallsetminus\!E$\, is the maximal ideal of $R$ (any valuation domain has a unique maximal ideal 
--- cf. valuation domain is local), then we denote\, $\mathfrak{m}^* = \mathfrak{m}\!\smallsetminus\!\{0\}$\, and\, $S = \mathfrak{m}^*/E = \{aE:\,\,a\in \mathfrak{m}^*\}$.\, Then the subsemigroup $S$ of $K/E$ makes $K/E$ an ordered group equipped with zero.\, It is not hard to check that the mapping
                         $$x\mapsto |x| := xE$$
from $K$ to\, $K/E$\, is a Krull valuation of the field $K$.
%%%%%
%%%%%
\end{document}
