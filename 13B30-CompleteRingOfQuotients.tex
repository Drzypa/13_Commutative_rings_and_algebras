\documentclass[12pt]{article}
\usepackage{pmmeta}
\pmcanonicalname{CompleteRingOfQuotients}
\pmcreated{2013-03-22 16:20:29}
\pmmodified{2013-03-22 16:20:29}
\pmowner{jocaps}{12118}
\pmmodifier{jocaps}{12118}
\pmtitle{complete ring of quotients}
\pmrecord{17}{38473}
\pmprivacy{1}
\pmauthor{jocaps}{12118}
\pmtype{Definition}
\pmcomment{trigger rebuild}
\pmclassification{msc}{13B30}
\pmrelated{CompleteRingOfQuotientsOfReducedCommutativeRings}
\pmrelated{EpimorphicHull}
\pmdefines{fraction of rings}
\pmdefines{complete ring of quotients}

\endmetadata

% this is the default PlanetMath preamble.  as your knowledge
% of TeX increases, you will probably want to edit this, but
% it should be fine as is for beginners.

% almost certainly you want these
\usepackage{amssymb}
\usepackage{amsmath}
\usepackage{amsfonts}

% used for TeXing text within eps files
%\usepackage{psfrag}
% need this for including graphics (\includegraphics)
%\usepackage{graphicx}
% for neatly defining theorems and propositions
\usepackage{amsthm}
\newtheorem*{rem}{Remark}
% making logically defined graphics
%%%\usepackage{xypic}

% there are many more packages, add them here as you need them

% define commands here

\begin{document}
Consider a commutative unitary ring $R$ and set 
$$\mathcal S:=\{ \mathrm{Hom}_R(I,R) : I \textrm{ is dense in } R \}$$
(here $\mathrm{Hom}_R(I,R)$ is the set of $R$-module morphisms from $I$ to $R$) and define $A:=\bigcup_{B\in\mathcal S} B$.

Now we shall assign a ring structure to $A$ by defining its addition and multiplication. Given two dense ideals $I_1,I_2\subset R$ and two elements 
$f_i\in\mathrm{Hom}_R(I_i,R)$ for $i\in\{1,2\}$, one can easily check that $I_1\cap I_2$ and $f_2^{-1} (I_1)$ are nontrivial (i.e. they aren't $\{0\}$) and in fact also dense ideals so we define

$f_1+f_2\in\mathrm{Hom}_R(I_1\cap I_2,R)$ by $(f_1+f_2)(x)=f_1(x)+f_2(x)$

$f_1*f_2\in\mathrm{Hom}_R(f_2^{-1}(I_1),R)$ by $(f_1*f_2)(x)=f_1(f_2(x))$

It is easy to check that $A$ is in fact a commutative ring with unity.
The elements of $A$ are called \PMlinkescapetext{\emph{fractions}}.

There is also an equivalence relation that one can define on $A$.
Given $f_i\in \mathrm{Hom}_R(I_i,R)$ for $i\in\{1,2\}$, we write
$$f_1\sim f_2 \Leftrightarrow f_1|I_1\cap I_2 = f_2|I_1\cap I_2$$
(i.e. $f_1$ and $f_2$ belong to the same equivalence class iff they agree on the intersection of the dense ideal where they are defined). 

The factor ring $Q(R):=A/\sim$ is then called the \emph{complete ring of quotients}.

\begin{rem}
$R\subset T(R)\subset Q(R)$, where $T(R)$ is the total quotient ring. One can also in general define complete ring of quotients on noncommutative rings.
\end{rem}

\begin{thebibliography}{99}

\bibitem[Huckaba]{Huckaba}
\textbf{J.A. Huckaba},
"Commutative rings with zero divisors",
Marcel Dekker 1988

\end{thebibliography}
%%%%%
%%%%%
\end{document}
