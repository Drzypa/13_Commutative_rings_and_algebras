\documentclass[12pt]{article}
\usepackage{pmmeta}
\pmcanonicalname{ValueGroupOfCompletion}
\pmcreated{2013-03-22 14:58:14}
\pmmodified{2013-03-22 14:58:14}
\pmowner{pahio}{2872}
\pmmodifier{pahio}{2872}
\pmtitle{value group of completion}
\pmrecord{10}{36670}
\pmprivacy{1}
\pmauthor{pahio}{2872}
\pmtype{Theorem}
\pmcomment{trigger rebuild}
\pmclassification{msc}{13F30}
\pmclassification{msc}{13J10}
\pmclassification{msc}{13A18}
\pmclassification{msc}{12J20}
%\pmkeywords{rank one valuation}
\pmrelated{KrullValuation}
\pmrelated{ExtensionOfValuationFromCompleteBaseField}
\pmdefines{value group of the completion}

% this is the default PlanetMath preamble.  as your knowledge
% of TeX increases, you will probably want to edit this, but
% it should be fine as is for beginners.

% almost certainly you want these
\usepackage{amssymb}
\usepackage{amsmath}
\usepackage{amsfonts}

% used for TeXing text within eps files
%\usepackage{psfrag}
% need this for including graphics (\includegraphics)
%\usepackage{graphicx}
% for neatly defining theorems and propositions
 \usepackage{amsthm}
% making logically defined graphics
%%%\usepackage{xypic}

% there are many more packages, add them here as you need them

% define commands here
\theoremstyle{definition}
\newtheorem*{thmplain}{Theorem}
\begin{document}
Let $k$ be a field and\, $|\cdot|$\, its non-archimedean valuation of \PMlinkname{rank one}{KrullValuation}.\, Then its value group\, $|k\!\smallsetminus\!\{0\}|$\, may be considered to be a subgroup of the multiplicative group of $\mathbb{R}$.\, In the completion $K$ of the valued field $k$, the extension of the valuation is defined by
             $$|x| \;=:\; \lim_{n\to\infty}|x_n|,$$
when the Cauchy sequence \,$x_1,\,x_2,\,\ldots,\,x_n,\,\ldots$\, of elements of $k$ determines the element $x$ of $K$.

\begin{thmplain}
\, \,The non-archimedean field $k$ and its completion $K$ have the same value group.
\end{thmplain}

{\em Proof.} \,Of course,\, $|k| \subseteq |K|$.\, Let\, 
$x = \lim_{n\to\infty}x_n$\, be any non-zero element of $K$, where $x_j$'s form a Cauchy sequence in $k$.\, Then there exists a positive number $n_0$ such that
           $$|x_n\!-\!x| \;<\; |x|$$
for all\, $n > n_0$.\, For all these values of $n$ we have
       $$|x_n| \;=\; |x\!+\!(x_n\!-\!x)| \;=\; |x|$$
according to the ultrametric triangle inequality.\, Thus we see that\, 
$|K|\subseteq |k|$.
%%%%%
%%%%%
\end{document}
