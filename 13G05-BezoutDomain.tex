\documentclass[12pt]{article}
\usepackage{pmmeta}
\pmcanonicalname{BezoutDomain}
\pmcreated{2013-03-22 14:19:53}
\pmmodified{2013-03-22 14:19:53}
\pmowner{CWoo}{3771}
\pmmodifier{CWoo}{3771}
\pmtitle{Bezout domain}
\pmrecord{10}{35801}
\pmprivacy{1}
\pmauthor{CWoo}{3771}
\pmtype{Definition}
\pmcomment{trigger rebuild}
\pmclassification{msc}{13G05}
\pmsynonym{B\'ezout domain}{BezoutDomain}
\pmrelated{GcdDomain}
\pmrelated{DivisibilityByProduct}
\pmdefines{Bezout identity}

% this is the default PlanetMath preamble.  as your knowledge
% of TeX increases, you will probably want to edit this, but
% it should be fine as is for beginners.

% almost certainly you want these
\usepackage{amssymb}
\usepackage{amsmath}
\usepackage{amsfonts}

% used for TeXing text within eps files
%\usepackage{psfrag}
% need this for including graphics (\includegraphics)
%\usepackage{graphicx}
% for neatly defining theorems and propositions
%\usepackage{amsthm}
% making logically defined graphics
%%%\usepackage{xypic}

% there are many more packages, add them here as you need them

% define commands here
\begin{document}
A \emph{Bezout domain} $D$ is an integral domain such that every finitely generated ideal of $D$ is \PMlinkname{principal}{PID}.

\textbf{Remarks}.  
\begin{itemize}
\item A PID is obviously a Bezout domain.  
\item Furthermore, a Bezout domain is a gcd domain.  To see this, suppose $D$ is a Bezout domain with $a,b\in D$.  By definition, there is a $d\in D$ such that $(d)=(a,b)$, the ideal generated by $a$ and $b$.  So $a\in (d)$ and $b\in (d)$ and therefore, $d\mid a$ and $d\mid b$.  Next, suppose $c\in D$ and that $c\mid a$ and $c\mid b$.  Then both $a,b\in (c)$ and so $d\in (c)$.  This means that $c\mid d$ and we are done.
\item From the discussion above, we see in a Bezout domain $D$, a greatest common divisor exists for every pair of elements.  Furthermore, if $\operatorname{gcd}(a,b)$ denotes one such greatest common divisor between $a,b\in D$, then for some $r,s\in D$:
$$\operatorname{gcd}(a,b)=ra+sb.$$
The above equation is known as the \emph{Bezout identity}, or Bezout's Lemma.
\end{itemize}
%%%%%
%%%%%
\end{document}
