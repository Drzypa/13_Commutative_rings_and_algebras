\documentclass[12pt]{article}
\usepackage{pmmeta}
\pmcanonicalname{MaximalIdealsOfRingOfFormalPowerSeries}
\pmcreated{2013-03-22 19:10:49}
\pmmodified{2013-03-22 19:10:49}
\pmowner{pahio}{2872}
\pmmodifier{pahio}{2872}
\pmtitle{maximal ideals of ring of formal power series}
\pmrecord{7}{42088}
\pmprivacy{1}
\pmauthor{pahio}{2872}
\pmtype{Result}
\pmcomment{trigger rebuild}
\pmclassification{msc}{13H05}
\pmclassification{msc}{13J05}
\pmclassification{msc}{13C13}
\pmclassification{msc}{13F25}

% this is the default PlanetMath preamble.  as your knowledge
% of TeX increases, you will probably want to edit this, but
% it should be fine as is for beginners.

% almost certainly you want these
\usepackage{amssymb}
\usepackage{amsmath}
\usepackage{amsfonts}

% used for TeXing text within eps files
%\usepackage{psfrag}
% need this for including graphics (\includegraphics)
%\usepackage{graphicx}
% for neatly defining theorems and propositions
 \usepackage{amsthm}
% making logically defined graphics
%%%\usepackage{xypic}

% there are many more packages, add them here as you need them

% define commands here

\theoremstyle{definition}
\newtheorem*{thmplain}{Theorem}

\begin{document}
Suppose that $R$ is a commutative ring with non-zero unity.\\

If $\mathfrak{m}$ is a maximal ideal of $R$, then\, $\mathfrak{M} \,:=\, \mathfrak{m}\!+\!(X)$\, is a maximal ideal of the ring $R[[X]]$ of formal power series.

Also the converse is true, i.e. if $\mathfrak{M}$ is a maximal ideal of $R[[X]]$, then there is a maximal ideal 
$\mathfrak{m}$ of $R$ such that\, $\mathfrak{M} \,=\, \mathfrak{m}\!+\!(X)$.\\


\textbf{Note.}\, In the special case that $R$ is a field, the only maximal ideal of which is the zero ideal $(0)$, this corresponds to the only maximal ideal $(X)$ of $R[[X]]$ (see \PMlinkid{formal power series over field}{12087}).\\


We here prove the first assertion.\, So, $\mathfrak{m}$ is assumed to be maximal.\, Let 
$$f(x) \;:=\; a_0\!+\!a_1X\!+\!a_2X^2\!+\ldots$$
be any formal power series in $R[[X]]\!\smallsetminus\!\mathfrak{M}$.\, Hence, the constant term $a_0$ cannot lie in $\mathfrak{m}$.\, According to the criterion for maximal ideal, there is an element $r$ of $R$ such that\, $1\!+\!ra_0 \in \mathfrak{m}$.\, Therefore
$$1\!+\!rf(X) \;=\; (1\!+\!ra_0)\!+\!r(a_1\!+\!a_2X\!+\!a_3X^2\!+\ldots)X \;\in\;\mathfrak{m}\!+\!(X)
 \;=\; \mathfrak{M},$$
whence the same criterion says that $\mathfrak{M}$ is a maximal ideal of $R[[X]]$.

%%%%%
%%%%%
\end{document}
