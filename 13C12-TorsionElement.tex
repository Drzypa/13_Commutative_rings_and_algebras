\documentclass[12pt]{article}
\usepackage{pmmeta}
\pmcanonicalname{TorsionElement}
\pmcreated{2013-03-22 13:54:41}
\pmmodified{2013-03-22 13:54:41}
\pmowner{mathcam}{2727}
\pmmodifier{mathcam}{2727}
\pmtitle{torsion element}
\pmrecord{7}{34665}
\pmprivacy{1}
\pmauthor{mathcam}{2727}
\pmtype{Definition}
\pmcomment{trigger rebuild}
\pmclassification{msc}{13C12}
\pmdefines{torsion submodule}
\pmdefines{torsion module}

\endmetadata

% this is the default PlanetMath preamble.  as your knowledge
% of TeX increases, you will probably want to edit this, but
% it should be fine as is for beginners.

% almost certainly you want these
\usepackage{amssymb}
\usepackage{amsmath}
\usepackage{amsfonts}

% used for TeXing text within eps files
%\usepackage{psfrag}
% need this for including graphics (\includegraphics)
%\usepackage{graphicx}
% for neatly defining theorems and propositions
%\usepackage{amsthm}
% making logically defined graphics
%%%\usepackage{xypic}

% there are many more packages, add them here as you need them

% define commands here
\begin{document}
\PMlinkescapeword{clearly}
Let $R$ be a commutative ring, and $M$ an $R$-module. We call an element $m\in M$ a \emph{torsion element} if there exists a non-zero-divisor $\alpha \in R$ such that $\alpha\cdot m=0$. The set is denoted by $tor(M)$.

$tor(M)$ is not empty since $0 \in tor(M)$. Let $m, n \in tor(M)$, so there exist $\alpha, \beta \ne 0 \in R$ such that $0=\alpha\cdot m=\beta\cdot n$. Since $\alpha\beta \cdot (m-n)=\beta\cdot \alpha\cdot m -\alpha\cdot \beta\cdot n=0, \alpha\beta\ne 0$, this implies that $m-n \in tor(M)$. So $tor(M)$ is a subgroup of $M$. Clearly $\tau\cdot m \in tor(M)$ for any non-zero $\tau \in R$. This shows that $tor(M)$ is a submodule of $M$, the \textbf{torsion submodule} of $M$.  In particular, a module that equals its own torsion submodule is said to be a \emph{torsion module}.
%%%%%
%%%%%
\end{document}
