\documentclass[12pt]{article}
\usepackage{pmmeta}
\pmcanonicalname{DefinitionOfPrimeIdealByArtin}
\pmcreated{2013-03-22 18:44:31}
\pmmodified{2013-03-22 18:44:31}
\pmowner{pahio}{2872}
\pmmodifier{pahio}{2872}
\pmtitle{definition of prime ideal by Artin}
\pmrecord{9}{41515}
\pmprivacy{1}
\pmauthor{pahio}{2872}
\pmtype{Definition}
\pmcomment{trigger rebuild}
\pmclassification{msc}{13C99}
\pmclassification{msc}{06A06}
\pmrelated{EveryRingHasAMaximalIdeal}
\pmdefines{prime ideal}

\endmetadata

% this is the default PlanetMath preamble.  as your knowledge
% of TeX increases, you will probably want to edit this, but
% it should be fine as is for beginners.

% almost certainly you want these
\usepackage{amssymb}
\usepackage{amsmath}
\usepackage{amsfonts}

% used for TeXing text within eps files
%\usepackage{psfrag}
% need this for including graphics (\includegraphics)
%\usepackage{graphicx}
% for neatly defining theorems and propositions
 \usepackage{amsthm}
% making logically defined graphics
%%%\usepackage{xypic}

% there are many more packages, add them here as you need them

% define commands here

\theoremstyle{definition}
\newtheorem*{thmplain}{Theorem}

\begin{document}
\PMlinkescapeword{ideals} \PMlinkescapeword{ideal}

\textbf{Lemma.}\, Let $R$ be a commutative ring and $S$ a multiplicative semigroup consisting of a subset of $R$.\, If there exist \PMlinkid{ideals}{371} of $R$ which are disjoint with $S$, then the set $\mathfrak{S}$ of all such ideals has a maximal element with respect to the set inclusion.

{\em Proof.}\, Let $C$ be an arbitrary chain in $\mathfrak{S}$.\, Then the union
$$\mathfrak{b} \;:=\; \bigcup_{\mathfrak{a} \in C}\mathfrak{a},$$
which belongs to $\mathfrak{S}$, may be taken for the upper bound of $C$, since it clearly is an ideal of $R$ and disjoint with $S$.\, Because $\mathfrak{S}$ thus is inductively ordered with respect to ``$\subseteq$'', our assertion follows from Zorn's lemma.\\

\textbf{Definition.}\, The maximal elements in the Lemma are {\em prime ideals} of the commutative ring.\\

The ring $R$ itself is always a prime ideal ($S = \varnothing$).\, If $R$ has no zero divisors, the zero ideal $(0)$ is a prime ideal ($S = R\!\smallsetminus\!\{0\}$).

If the ring $R$ has a non-zero unity element 1, the prime ideals corresponding the semigroup \,$S = \{1\}$\, are the maximal ideals of $R$.


\begin{thebibliography}{9}
\bibitem{Artin} {\sc Emil Artin}: {\em Theory of Algebraic Numbers}.\, Lecture notes.\, Mathematisches Institut, G\"ottingen (1959).
\end{thebibliography}

%%%%%
%%%%%
\end{document}
