\documentclass[12pt]{article}
\usepackage{pmmeta}
\pmcanonicalname{ProofOfTheWeakNullstellensatz}
\pmcreated{2013-03-22 15:27:43}
\pmmodified{2013-03-22 15:27:43}
\pmowner{pbruin}{1001}
\pmmodifier{pbruin}{1001}
\pmtitle{proof of the weak Nullstellensatz}
\pmrecord{4}{37313}
\pmprivacy{1}
\pmauthor{pbruin}{1001}
\pmtype{Proof}
\pmcomment{trigger rebuild}
\pmclassification{msc}{13A10}
\pmclassification{msc}{13A15}

\endmetadata

% this is the default PlanetMath preamble.  as your knowledge
% of TeX increases, you will probably want to edit this, but
% it should be fine as is for beginners.

% almost certainly you want these
\usepackage{amssymb}
\usepackage{amsmath}
\usepackage{amsfonts}

% used for TeXing text within eps files
%\usepackage{psfrag}
% need this for including graphics (\includegraphics)
%\usepackage{graphicx}
% for neatly defining theorems and propositions
%\usepackage{amsthm}
% making logically defined graphics
%%%\usepackage{xypic}

% there are many more packages, add them here as you need them

% define commands here
\begin{document}
Let $K$ be an algebraically closed field, let $n\ge 0$, and let $I$ be
an ideal in the polynomial ring $K[x_1,\ldots,x_n]$.  Suppose $I$ is
strictly smaller than $K[x_1,\ldots,x_n]$.  Then $I$ is contained in a
maximal ideal $M$ of $K[x_1,\ldots,x_n]$ (note that we don't have to
accept Zorn's lemma to find such an $M$, since $K[x_1,\ldots,x_n]$ is
Noetherian by Hilbert's basis theorem), and the quotient ring
$$
L=K[x_1,\ldots,x_n]/M
$$
is a field.  We view $K$ as a subfield of $L$ via the natural
homomorphism $K\hookrightarrow L$, and we denote the images of
$x_1,\ldots,x_n$ in $L$ by $\bar x_1,\ldots,\bar x_n$.  Let
$\{t_1,\ldots,t_m\}$ be a transcendence basis of $L$ over $K$; it is
finite since $L$ is finitely generated as a $K$-algebra.  Now $L$ is
an algebraic extension of $K(t_1,\ldots,t_m)$.  By multiplying the
minimal polynomial of $\bar x_i$ over $K(t_1,\ldots,t_m)$ by a
suitable element of $K[t_1,\ldots,t_m]$ for each $i$, we obtain
non-zero polynomials $f_i\in K[t_1,\ldots,t_m][X]$ with the
property that $f_i(\bar x_i)=0$ in $L$:
$$
f_i=c_{i,0}+c_{i,1}X+\cdots+c_{i,d_i}X^{d_i}
\qquad(1\le i\le n)
$$
for certain integers $d_i>0$ and polynomials $c_{i,j}\in                                  
K[t_1,\ldots,t_m]$ with $c_{i,d_i}\ne 0$.  Since $K$ is algebraically
closed (hence infinite), we can choose $u_1,\ldots,u_n\in K$ such that
$c_{i,d_i}(u_1,\ldots,u_m)\ne 0$ for all $i$.  We define a homomorphism
$$
\phi\colon K[t_1,\ldots,t_m] \longrightarrow K
$$
by taking $\phi$ to be the identity on $K$ and sending $t_j$ to $u_j$.
Let $N$ be the kernel of this homomorphism.  Then $\phi$ can be
extended to the localization $K[t_1,\ldots,t_m]_N$ of
$K[t_1,\ldots,t_m]$.  Since $c_{i,d_i}\not\in N$ for all $i$, the
$\bar x_i$ are integral over this ring.  Since $K$ is algebraically
closed, the extension theorem for ring homomorphisms implies that
$\phi$ can be extended to a homomorphism
$$
\phi\colon(K[t_1,\ldots,t_m]_N)[\bar x_1,\ldots,\bar x_n]=L
\longrightarrow K.
$$
Because $L$ is an extension field of $K$ and $\phi$ is the identity on
$K$, we see that $\phi$ is actually an isomorphism, that $m=0$, and
that $N$ is the zero ideal of $K$.  Now let $a_1=\phi(\bar                                
x_1),\ldots,a_n=\phi(\bar x_n)$.  Then for all polynomials $f$ in the
ideal $I$ we started with, the fact that $f\in M$ implies
$$
f(a_1,\ldots,a_n)=\phi(f(x_1,\ldots,x_n)+M)=0.
$$
We conclude that the zero set $V(I)$ of $I$ is not empty.
%%%%%
%%%%%
\end{document}
