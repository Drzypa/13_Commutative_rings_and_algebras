\documentclass[12pt]{article}
\usepackage{pmmeta}
\pmcanonicalname{MaximalIdeal}
\pmcreated{2013-03-22 11:50:57}
\pmmodified{2013-03-22 11:50:57}
\pmowner{djao}{24}
\pmmodifier{djao}{24}
\pmtitle{maximal ideal}
\pmrecord{8}{30410}
\pmprivacy{1}
\pmauthor{djao}{24}
\pmtype{Definition}
\pmcomment{trigger rebuild}
\pmclassification{msc}{13A15}
\pmclassification{msc}{16D25}
\pmclassification{msc}{81R50}
\pmclassification{msc}{46M20}
\pmclassification{msc}{18B40}
\pmclassification{msc}{22A22}
\pmclassification{msc}{46L05}
\pmrelated{ProperIdeal}
\pmrelated{Module}
\pmrelated{Comaximal}
\pmrelated{PrimeIdeal}
\pmrelated{EveryRingHasAMaximalIdeal}

\usepackage{amssymb}
\usepackage{amsmath}
\usepackage{amsfonts}
\usepackage{graphicx}
%%%%\usepackage{xypic}
\begin{document}
Let $R$ be a ring with identity. A proper left (right, two-sided) ideal $\mathfrak{m} \subsetneq R$ is said to be {\em maximal} if $\mathfrak{m}$ is not a proper subset of any other proper left (right, two-sided) ideal of $R$.

One can prove:
\begin{itemize}
\item A left ideal $\mathfrak{m}$ is maximal if and only if $R/\mathfrak{m}$ is a simple left $R$-module.
\item A right ideal $\mathfrak{m}$ is maximal if and only if $R/\mathfrak{m}$ is a simple right $R$-module.

\item A two-sided ideal $\mathfrak{m}$ is maximal if and only if $R/\mathfrak{m}$ is a simple ring.
\end{itemize}

All maximal ideals are prime ideals. If $R$ is commutative, an ideal $\mathfrak{m} \subset R$ is maximal if and only if the quotient ring $R/\mathfrak{m}$ is a field.
%%%%%
%%%%%
%%%%%
%%%%%
\end{document}
