\documentclass[12pt]{article}
\usepackage{pmmeta}
\pmcanonicalname{FinitelyGeneratedTorsionfreeModulesOverPruferDomains}
\pmcreated{2013-03-22 18:36:11}
\pmmodified{2013-03-22 18:36:11}
\pmowner{gel}{22282}
\pmmodifier{gel}{22282}
\pmtitle{finitely generated torsion-free modules over Pr\"ufer domains}
\pmrecord{4}{41334}
\pmprivacy{1}
\pmauthor{gel}{22282}
\pmtype{Theorem}
\pmcomment{trigger rebuild}
\pmclassification{msc}{13F05}
\pmclassification{msc}{13C10}
%\pmkeywords{Pr\"ufer domain}
%\pmkeywords{projective module}
\pmrelated{EquivalentCharacterizationsOfDedekindDomains}

% this is the default PlanetMath preamble.  as your knowledge
% of TeX increases, you will probably want to edit this, but
% it should be fine as is for beginners.

% almost certainly you want these
\usepackage{amssymb}
\usepackage{amsmath}
\usepackage{amsfonts}

% used for TeXing text within eps files
%\usepackage{psfrag}
% need this for including graphics (\includegraphics)
%\usepackage{graphicx}
% for neatly defining theorems and propositions
\usepackage{amsthm}
% making logically defined graphics
%%%\usepackage{xypic}

% there are many more packages, add them here as you need them

% define commands here
\newtheorem*{theorem*}{Theorem}
\newtheorem*{lemma*}{Lemma}
\newtheorem*{corollary*}{Corollary}
\newtheorem{theorem}{Theorem}
\newtheorem{lemma}{Lemma}
\newtheorem{corollary}{Corollary}


\begin{document}
\begin{theorem*}
Let $M$ be a finitely generated torsion-free module over a Pr\"ufer domain $R$. Then, $M$ is isomorphic to a \PMlinkname{direct sum}{DirectSum}
\begin{equation*}
M\cong \mathfrak{a}_1\oplus\cdots\oplus\mathfrak{a}_n
\end{equation*}
of finitely generated ideals $\mathfrak{a}_1,\ldots,\mathfrak{a}_n$.
\end{theorem*}

As invertible ideals are projective and direct sums of projective modules are themselves projective, this theorem shows that $M$ is also a projective module. Conversely, if every finitely generated torsion-free module over an integral domain $R$ is projective then, in particular, every finitely generated nonzero ideal of $R$ will be projective and hence invertible. So, we get the following characterization of Pr\"ufer domains.

\begin{corollary*}
An integral domain $R$ is Pr\"ufer if and only if every finitely generated torsion-free $R$-module is \PMlinkname{projective}{ProjectiveModule}.
\end{corollary*}

%%%%%
%%%%%
\end{document}
