\documentclass[12pt]{article}
\usepackage{pmmeta}
\pmcanonicalname{DerivativeOfPolynomial}
\pmcreated{2013-03-22 18:20:02}
\pmmodified{2013-03-22 18:20:02}
\pmowner{pahio}{2872}
\pmmodifier{pahio}{2872}
\pmtitle{derivative of polynomial}
\pmrecord{12}{40966}
\pmprivacy{1}
\pmauthor{pahio}{2872}
\pmtype{Definition}
\pmcomment{trigger rebuild}
\pmclassification{msc}{13P05}
\pmclassification{msc}{11C08}
\pmclassification{msc}{12E05}
\pmrelated{DerivativesByPureAlgebra}
\pmrelated{PolynomialFunction}
\pmrelated{Multiplicity}
\pmrelated{DiscriminantOfAlgebraicNumber}
\pmdefines{derivative of the polynomial}

% this is the default PlanetMath preamble.  as your knowledge
% of TeX increases, you will probably want to edit this, but
% it should be fine as is for beginners.

% almost certainly you want these
\usepackage{amssymb}
\usepackage{amsmath}
\usepackage{amsfonts}

% used for TeXing text within eps files
%\usepackage{psfrag}
% need this for including graphics (\includegraphics)
%\usepackage{graphicx}
% for neatly defining theorems and propositions
 \usepackage{amsthm}
% making logically defined graphics
%%%\usepackage{xypic}

% there are many more packages, add them here as you need them

% define commands here

\theoremstyle{definition}
\newtheorem*{thmplain}{Theorem}

\begin{document}
Let $R$ be an arbitrary commutative ring.\, If
$$f(X) \,:=\, \sum_{i=1}^na_iX^i$$
is a polynomial in the ring $R[X]$, one can form in a polynomial ring \,$R[X,\,Y]$\, the polynomial
$$f(X\!+\!Y) \,=\, \sum_{i=1}^na_i(X\!+\!Y)^i.$$
Expanding this by the \PMlinkname{powers}{GeneralAssociativity} of $Y$ yields uniquely the form
\begin{align}
f(X\!+\!Y) \,:=\, f(X)+f_1(X)\,Y+f_2(X,\,Y)\,Y^2,
\end{align}
where\, $f_1(X) \in R[X]$\, and\, $f_2(X,\,Y) \in R[X,\,Y]$.

We define the polynomial $f_1(X)$ in (1) the {\em derivative of the polynomial} $f(X)$ and denote it by $f'(X)$ or 
$\displaystyle\frac{df}{dX}$.\\

It is apparent that this algebraic definition of derivative of polynomial is in harmony with the definition of \PMlinkname{derivative}{Derivative2} of analysis when $R$ is $\mathbb{R}$ or $\mathbb{C}$; then we identify substitution homomorphism and polynomial function.\\

It is easily shown the linearity of the derivative of polynomial and the product rule
$$(fg)' = f'g+g'f$$
with its generalisations.\, Especially:
$$(X^n)' = nX^{n-1} \quad\mbox{for}\;\; n = 1,\,2,\,3,\,\ldots$$\\

\textbf{Remark.}\; The polynomial ring $R[X]$ may be thought to be a subring of $R[[X]]$, the ring of formal power series in $X$.\, The \PMlinkname{derivatives defined in}{FormalPowerSeries} $R[[X]]$ extend the concept of derivative of polynomial and obey \PMlinkescapetext{similar} laws.\\

If we have a polynomial\, $f \in R\,[X_1,\,X_2,\,\ldots,\,X_m]$,\, we can analogically define the {\em partial derivatives} of $f$, denoting them by $\displaystyle\frac{\partial f}{\partial X_i}$.\, Then, e.g. the ``\PMlinkname{Euler's theorem on homogeneous functions}{EulersTheoremOnHomogeneousFunctions}'' 
$$X_1\frac{\partial f}{\partial X_1}+X_2\frac{\partial f}{\partial X_2}+\ldots+X_m\frac{\partial f}{\partial X_m}
\;=\; nf$$
is true for a homogeneous polynomial $f$ of degree $n$.




%%%%%
%%%%%
\end{document}
