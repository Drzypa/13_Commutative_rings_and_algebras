\documentclass[12pt]{article}
\usepackage{pmmeta}
\pmcanonicalname{Algebramodule}
\pmcreated{2013-03-22 13:20:50}
\pmmodified{2013-03-22 13:20:50}
\pmowner{mathcam}{2727}
\pmmodifier{mathcam}{2727}
\pmtitle{algebra (module)}
\pmrecord{5}{33865}
\pmprivacy{1}
\pmauthor{mathcam}{2727}
\pmtype{Definition}
\pmcomment{trigger rebuild}
\pmclassification{msc}{13B99}
\pmclassification{msc}{20C99}
\pmclassification{msc}{16S99}
\pmdefines{Jacobi identity}

% this is the default PlanetMath preamble.  as your knowledge
% of TeX increases, you will probably want to edit this, but
% it should be fine as is for beginners.

% almost certainly you want these
\usepackage{amssymb}
\usepackage{amsmath}
\usepackage{amsfonts}

% used for TeXing text within eps files
%\usepackage{psfrag}
% need this for including graphics (\includegraphics)
%\usepackage{graphicx}
% for neatly defining theorems and propositions
%\usepackage{amsthm}
% making logically defined graphics
%%%\usepackage{xypic}

% there are many more packages, add them here as you need them

% define commands here
\begin{document}
Given a commutative ring $R$, an algebra over $R$ is a
module $M$ over $R$, endowed with a law of composition
$$f:M\times M\to M$$
which is $R$-bilinear.

Most of the important algebras in mathematics belong to
one or the other of two classes: the unital associative
algebras, and the Lie algebras.

\section{Unital associative algebras}
In these cases, the ``product'' (as it is called) of two elements
$v$ and $w$ of the module, is denoted simply by $vw$ or $v\centerdot w$ or the
like.

Any unital associative algebra is an algebra in the sense of djao (a
sense which is also used by Lang in his book \emph{Algebra}
(Springer-Verlag)).

Examples of unital associative algebras:

-- tensor algebras and quotients of them

-- Cayley algebras, such as the ring of quaternions

-- polynomial rings

-- the ring of endomorphisms of a vector space, in which the bilinear
product of two mappings is simply the composite mapping.

\section{Lie algebras}
In these cases the bilinear product is denoted by $[v,w]$,
and satisfies
$$[v,v]=0\textrm{ for all }v\in M$$
$$[v,[w,x]]+[w,[x,v]]+[x,[v,w]]=0\textrm{ for all }v,w,x\in M$$
The second of these formulas is called the Jacobi identity. One proves
easily
$$[v,w]+[w,v]=0\textrm{ for all }v,w\in M$$
for any Lie algebra M.

Lie algebras arise naturally from Lie groups, q.v.
%%%%%
%%%%%
\end{document}
