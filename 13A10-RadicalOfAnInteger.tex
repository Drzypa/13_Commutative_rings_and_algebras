\documentclass[12pt]{article}
\usepackage{pmmeta}
\pmcanonicalname{RadicalOfAnInteger}
\pmcreated{2013-03-22 11:45:21}
\pmmodified{2013-03-22 11:45:21}
\pmowner{KimJ}{5}
\pmmodifier{KimJ}{5}
\pmtitle{radical of an integer}
\pmrecord{12}{30200}
\pmprivacy{1}
\pmauthor{KimJ}{5}
\pmtype{Definition}
\pmcomment{trigger rebuild}
\pmclassification{msc}{13A10}
\pmclassification{msc}{81-00}
\pmclassification{msc}{18-00}
\pmsynonym{square-free part}{RadicalOfAnInteger}
%\pmkeywords{number theory}
\pmrelated{RadicalOfAnIdeal}
\pmrelated{PowerOfAnInteger}
\pmdefines{radical}

\usepackage{amssymb}
\usepackage{amsmath}
\usepackage{amsfonts}
\usepackage{graphicx}
%%%%\usepackage{xypic}
\begin{document}
Given a natural number $n$, let $n = p_1^{{\alpha}_1} \cdots p_k^{{\alpha}_k}$ be its unique factorization as a product of distinct prime powers. Define the \PMlinkescapetext{radical} of $n$, denoted $\mbox{rad}(n)$, to be the product $p_1 \cdots p_k$. The radical of a square-free number is itself.
%%%%%
%%%%%
%%%%%
%%%%%
\end{document}
