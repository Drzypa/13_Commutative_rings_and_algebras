\documentclass[12pt]{article}
\usepackage{pmmeta}
\pmcanonicalname{TeichmullerCharacter}
\pmcreated{2013-03-22 15:09:04}
\pmmodified{2013-03-22 15:09:04}
\pmowner{alozano}{2414}
\pmmodifier{alozano}{2414}
\pmtitle{Teichm\"uller character}
\pmrecord{7}{36898}
\pmprivacy{1}
\pmauthor{alozano}{2414}
\pmtype{Definition}
\pmcomment{trigger rebuild}
\pmclassification{msc}{13H99}
\pmclassification{msc}{11S99}
\pmclassification{msc}{12J99}
\pmsynonym{Teichmuler character}{TeichmullerCharacter}
\pmsynonym{Teichmuller lift}{TeichmullerCharacter}
\pmsynonym{Teichm\"uller lift}{TeichmullerCharacter}
%\pmkeywords{roots of unity}
\pmrelated{PAdicIntegers}

\endmetadata

% this is the default PlanetMath preamble.  as your knowledge
% of TeX increases, you will probably want to edit this, but
% it should be fine as is for beginners.

% almost certainly you want these
\usepackage{amssymb}
\usepackage{amsmath}
\usepackage{amsthm}
\usepackage{amsfonts}

% used for TeXing text within eps files
%\usepackage{psfrag}
% need this for including graphics (\includegraphics)
%\usepackage{graphicx}
% for neatly defining theorems and propositions
%\usepackage{amsthm}
% making logically defined graphics
%%%\usepackage{xypic}

% there are many more packages, add them here as you need them

% define commands here

\newtheorem{thm}{Theorem}
\newtheorem*{defn}{Definition}
\newtheorem{prop}{Proposition}
\newtheorem{lemma}{Lemma}
\newtheorem*{cor}{Corollary}

\theoremstyle{definition}
\newtheorem{exa}{Example}
\newtheorem*{rem}{Remark}

% Some sets
\newcommand{\Nats}{\mathbb{N}}
\newcommand{\Ints}{\mathbb{Z}}
\newcommand{\Reals}{\mathbb{R}}
\newcommand{\Complex}{\mathbb{C}}
\newcommand{\Rats}{\mathbb{Q}}
\newcommand{\Gal}{\operatorname{Gal}}
\newcommand{\Cl}{\operatorname{Cl}}
\begin{document}
Before we define the Teichm\"uller character, we begin with a corollary of Hensel's lemma.

\begin{cor}
Let $p$ be a prime number. The ring of \PMlinkname{$p$-adic integers}{PAdicIntegers} $\Ints_p$ contains exactly $p-1$ distinct $(p-1)$th roots of unity. Furthermore, every $(p-1)$th root of unity is distinct modulo $p$.
\end{cor}
\begin{proof}
Notice that $\Rats_p$, the $p$-adic rationals, is a field. Therefore $f(x)=x^{p-1}-1$ has at most $p-1$ roots in $\Rats_p$ (see \PMlinkname{this entry}{APolynomialOfDegreeNOverAFieldHasAtMostNRoots}). Moreover, if we let $a\in \Ints$ with $1\leq a \leq p-1$ then $f(a)=a^{p-1}-1\equiv 0 \mod p$ by Fermat's little theorem. Since $f'(a)=(p-1)\cdot a^{p-2}$ is non-zero modulo $p$, the trivial case of Hensel's lemma implies that there exist a root of $x^{p-1}-1$ in $\Ints_p$ which is congruent to $a$ modulo $p$. Hence, there are at least $p-1$ roots in $\Ints_p$, and we can conclude that there are exactly $p-1$ roots.  
\end{proof}

\begin{defn}
The Teichm\"uller character is a homomorphism of multiplicative groups:
$$\omega \colon \mathbb{F}_p^\times \to \Ints_p^\times$$
such that $\omega(a)$ is the unique $(p-1)$th root of unity in $\Ints_p$ which is congruent to $a$ modulo $p$ (which exists by the corollary above). The map $\omega$ is sometimes called the Teichm\"uller lift of $\mathbb{F}_p$ to $\Ints_p$ ($0\mod p$ would lift to $0\in \Ints_p$).
\end{defn}

\begin{rem}
Some authors define the Teichm\"uller character to be the homomorphism:
$$\hat{\omega}\colon \Ints_p^\times \to \Ints_p^\times$$
defined by 
$$\hat{\omega}(z)=\lim_{n\to \infty} z^{p^n}.$$
Notice that for any $z\in \Ints_p^\times$, $\hat{\omega}(z)$ is a $(p-1)$th root of unity:
$$(\hat{\omega}(z))^p=\left( \lim_{n\to \infty} z^{p^n} \right)^p= \lim_{n\to \infty} z^{p^{n+1}}=\hat{\omega}(z).$$
Thus, the value $\hat{\omega}(z)$ is the same than $\omega(z \mod p)$.
\end{rem}
%%%%%
%%%%%
\end{document}
