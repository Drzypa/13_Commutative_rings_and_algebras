\documentclass[12pt]{article}
\usepackage{pmmeta}
\pmcanonicalname{EuclideanValuation}
\pmcreated{2013-03-22 12:40:45}
\pmmodified{2013-03-22 12:40:45}
\pmowner{Wkbj79}{1863}
\pmmodifier{Wkbj79}{1863}
\pmtitle{Euclidean valuation}
\pmrecord{15}{32956}
\pmprivacy{1}
\pmauthor{Wkbj79}{1863}
\pmtype{Definition}
\pmcomment{trigger rebuild}
\pmclassification{msc}{13F07}
\pmsynonym{degree function}{EuclideanValuation}
\pmrelated{PID}
\pmrelated{UFD}
\pmrelated{Ring}
\pmrelated{IntegralDomain}
\pmrelated{EuclideanRing}
\pmrelated{ProofThatAnEuclideanDomainIsAPID}
\pmrelated{DedekindHasseValuation}
\pmrelated{EuclideanNumberField}

\usepackage{amssymb}
\usepackage{amsmath}
\usepackage{amsfonts}
\newcommand{\Z}{\mathbb{Z}}
\newcommand{\C}{\mathbb{C}}
\newcommand{\R}{\mathbb{R}}
\newcommand{\Q}{\mathbb{Q}}

\begin{document}
\PMlinkescapeword{acting}
\PMlinkescapeword{inverse}

Let $D$ be an integral domain.  A \emph{Euclidean valuation} is a function from the nonzero elements of $D$ to the nonnegative integers $\nu \colon D \setminus \{0_D\} \to \{ x \in \mathbb{Z} : x \ge 0 \}$ such that the following hold:

\begin{itemize}
\item For any $a,b\in D$ with $b\neq 0_D$, there exist $q,r\in D$ such that $a=bq+r$ with $\nu(r)<\nu(b)$ or $r=0_D$.
\item For any $a,b\in D \setminus \{0_D\}$, we have $\nu(a)\leq\nu(ab)$.
\end{itemize}

Euclidean valuations are important because they let us define greatest common divisors and use Euclid's algorithm. Some facts about Euclidean valuations include:

\begin{itemize}
\item The \PMlinkname{minimal}{MinimalElement} value of $\nu$ is $\nu(1_D)$. That is, $\nu(1_D)\leq\nu(a)$ for any $a\in D \setminus \{0_D\}$.
\item $u\in D$ is a unit if and only if $\nu(u)=\nu(1_D)$.
\item For any $a\in D \setminus \{0_D\}$ and any unit $u$ of $D$, we have $\nu(a)=\nu(au)$.
\end{itemize}

These facts can be proven as follows:

\begin{itemize}
\item If $a\in D \setminus \{0_D\}$, then
\[
\nu(1_D)\leq\nu(1_D\cdot a)=\nu(a).
\]
\item If $u\in D$ is a unit, then let $v\in D$ be its \PMlinkname{inverse}{MultiplicativeInverse}.  Thus,
\[
\nu(1_D)\leq\nu(u)\leq\nu(uv)=\nu(1_D).
\]
Conversely, if $\nu(u)=\nu(1_D)$, then there exist $q,r\in D$ with $\nu(r)<\nu(u)=\nu(1_D)$ or $r=0_D$ such that
\[
1_D=qu+r.
\]
Since $\nu(r)<\nu(1_D)$ is impossible, we must have $r=0_D$.  Hence, $q$ is the inverse of $u$.
\item Let $v\in D$ be the inverse of $u$.  Then
\[
\nu(a)\leq\nu(au)\leq\nu(auv)=\nu(a).
\]
\end{itemize}

Note that an integral domain is a Euclidean domain if and only if it has a Euclidean valuation.

Below are some examples of Euclidean domains and their Euclidean valuations:

\begin{itemize}
\item Any field $F$ is a Euclidean domain under the Euclidean valuation $\nu(a)=0$ for all $a\in F \setminus \{0_F\}$.
\item $\mathbb{Z}$ is a Euclidean domain with absolute value acting as its Euclidean valuation.
\item If $F$ is a field, then $F[x]$, the ring of polynomials over $F$, is a Euclidean domain with degree acting as its Euclidean valuation:  If $n$ is a nonnegative integer and $a_0,\dots,a_n\in F$ with $a_n\neq 0_F$, then
\[
\nu\left(\sum_{j=0}^n a_jx^j\right)=n.
\]
\end{itemize}

Due to the fact that the ring of polynomials over any field is always  a Euclidean domain with degree acting as its Euclidean valuation, some refer to a Euclidean valuation as a \emph{degree function}.  This is done, for example, in Joseph J. Rotman's \emph{\PMlinkescapetext{A First Course in Abstract Algebra}}.
%%%%%
%%%%%
\end{document}
