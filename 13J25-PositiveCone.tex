\documentclass[12pt]{article}
\usepackage{pmmeta}
\pmcanonicalname{PositiveCone}
\pmcreated{2013-03-22 14:46:54}
\pmmodified{2013-03-22 14:46:54}
\pmowner{CWoo}{3771}
\pmmodifier{CWoo}{3771}
\pmtitle{positive cone}
\pmrecord{10}{36430}
\pmprivacy{1}
\pmauthor{CWoo}{3771}
\pmtype{Definition}
\pmcomment{trigger rebuild}
\pmclassification{msc}{13J25}
\pmclassification{msc}{12D15}
\pmrelated{PositivityInOrderedRing}
\pmrelated{FormallyRealField}
\pmdefines{pre-positive cone}

% this is the default PlanetMath preamble.  as your knowledge
% of TeX increases, you will probably want to edit this, but
% it should be fine as is for beginners.

% almost certainly you want these
\usepackage{amssymb,amscd}
\usepackage{amsmath}
\usepackage{amsfonts}

% used for TeXing text within eps files
%\usepackage{psfrag}
% need this for including graphics (\includegraphics)
%\usepackage{graphicx}
% for neatly defining theorems and propositions
\usepackage{amsthm}
% making logically defined graphics
%%%\usepackage{xypic}

% there are many more packages, add them here as you need them

% define commands here

%\newcommand{\qed}{\nobreak \ifvmode \relax \else
%      \ifdim\lastskip<1.5em \hskip-\lastskip
%     \hskip1.5em plus0em minus0.5em \fi \nobreak
%     \vrule height0.75em width0.5em depth0.25em\fi}
\begin{document}
\PMlinkescapeword{closed}

Let $R$ be a commutative ring with 1.  A subset $P$ of $R$ is called a \emph{pre-positive cone} of $R$ provided that 
\begin{enumerate}
\item
$P+P\subseteq P$ ($P$ is additively closed)
\item
$P\cdot P\subseteq P$ ($P$ is multiplicatively closed)
\item
$-1\notin P$
\item
$\operatorname{sqr}(R):=\lbrace r^2\mid r\in R\rbrace \subseteq P.$
\end{enumerate}
As it turns out, a field endowed with a pre-positive cone has an order structure.  The field is called a \PMlinkname{formally real}{FormallyRealField}, orderable, or ordered field.  Before defining what this ``order'' is, let's do some preliminary work.  Let $P_0$ be a pre-positive cone of a field $F$.  By Zorn's Lemma, the set of pre-positive cones extending $P_0$ has a maximal element $P$.  It can be shown that $P$ has two additional properties:
\begin{enumerate}
\item[5.]
$P\cup (-P)=F$
\item[6.]
$P\cap (-P)=(0).$
\end{enumerate}
\begin{proof}  First, suppose there is $a\in F-(P\cup (-P))$.  Let $\overline{P}=P+Pa$.  Then $a\in\overline{P}$ and so $P$ is strictly contained in $\overline{P}$.  Clearly, $\operatorname{sqr}(F)\subseteq \overline{P}$ and $\overline{P}$ is easily seen to be additively closed.  Also, $\overline{P}$ is multiplicatively closed as the equation $(p_1+q_1a)(p_2+q_2a)=(p_1p_2+q_1q_2a^2)+(p_1q_2+q_1p_2)a$ demonstrates.  Since $P$ is a maximal and $\overline{P}$ properly contains $P$, $\overline{P}$ is not a pre-positive cone, which means $-1\in \overline{P}$.  Write $-1=p+qa$.  Then $q(-a)=p+1\in P$.  Since $q\in P$, $1/q=q(1/q)^2\in P$, $-a=(1/q)(p+1)\in P$, contradicting the assumption that $a\notin -P$.  Therefore, $P\cup (-P)=F$.

For the second part, suppose $a\in P\cap (-P)$.  Since $a\in -P$, $-a\in P$.  If $a\neq 0$, then $-1=a(-a)(1/a)^2\in P$, a contradiction. 
\end{proof}

A subset $P$ of a field $F$ satisfying conditions 1, 2, 5 and 6 is called a \emph{positive cone} of $F$.  
A positive cone is a pre-positive cone.  If $a\in F$, then either $a\in P$ or $-a\in P$.  In either case, $a^2\in P$.  
Next, if $-1\in P$, then $1\in -P$.  But $1=1^2\in P$, we have $1\in P\cap (-P)$, contradicting Condition 6 of $P$. 

Now, define a binary relation $\leq$, on $F$ by: $$a\leq b\Longleftrightarrow b-a\in P$$
It is not hard to see that $\leq$ is a total order on $F$.  In addition, with the additive and multiplicative structures on $F$, we also have the 
following two rules:
\begin{enumerate}
\item
$a\leq b \Rightarrow a+c\leq b+c$
\item
$0\leq a$ and $0\leq b\Rightarrow 0\leq ab$.
\end{enumerate}
Thus, $F$ is a field ordered by $\leq$.

\textbf{Remark}.  Positive cones may be defined for more general ordered algebraic structures, such as partially ordered groups, or partially ordered rings.


\begin{thebibliography}{99}
\bibitem{ap} A. Prestel, \emph{Lectures on Formally Real Fields}, Springer, 1984
\end{thebibliography}
%%%%%
%%%%%
\end{document}
