\documentclass[12pt]{article}
\usepackage{pmmeta}
\pmcanonicalname{AFiniteIntegralDomainIsAField}
\pmcreated{2013-03-22 12:50:02}
\pmmodified{2013-03-22 12:50:02}
\pmowner{yark}{2760}
\pmmodifier{yark}{2760}
\pmtitle{a finite integral domain is a field}
\pmrecord{11}{33158}
\pmprivacy{1}
\pmauthor{yark}{2760}
\pmtype{Theorem}
\pmcomment{trigger rebuild}
\pmclassification{msc}{13G05}
\pmrelated{FiniteRingHasNoProperOverrings}

\endmetadata

\usepackage{amssymb}
\usepackage{amsmath}
\usepackage{amsfonts}

\begin{document}
A finite integral domain is a field.

{\bf Proof:\\}
Let $R$ be a finite integral domain.  Let $a$ be nonzero element of $R$.

Define a function $\varphi \colon R \rightarrow R$ by $\varphi(r)=ar$.

Suppose $\varphi(r)=\varphi(s)$ for some $r,s \in R$.  Then $ar=as$, which implies $a(r-s)=0$.  Since $a \neq 0$ and $R$ is a cancellation ring, we have $r-s=0$.  So $r=s$, and hence $\varphi$ is injective.

Since $R$ is finite and $\varphi$ is injective, by the pigeonhole principle we see that $\varphi$ is also surjective.  Thus there exists some $b \in R$ such that $\varphi(b)= ab = 1_R$, and thus $a$ is a unit.

Thus $R$ is a finite division ring.  Since it is commutative, it is also a field.

{\bf Note:\\}
A more general result is that an Artinian integral domain is a field.
%%%%%
%%%%%
\end{document}
