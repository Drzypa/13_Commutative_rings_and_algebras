\documentclass[12pt]{article}
\usepackage{pmmeta}
\pmcanonicalname{FiniteExtensionsOfDedekindDomainsAreDedekind}
\pmcreated{2013-03-22 18:35:30}
\pmmodified{2013-03-22 18:35:30}
\pmowner{gel}{22282}
\pmmodifier{gel}{22282}
\pmtitle{finite extensions of Dedekind domains are Dedekind}
\pmrecord{5}{41319}
\pmprivacy{1}
\pmauthor{gel}{22282}
\pmtype{Theorem}
\pmcomment{trigger rebuild}
\pmclassification{msc}{13A15}
\pmclassification{msc}{13F05}
%\pmkeywords{Dedekind domain}
%\pmkeywords{finite extension}
%\pmkeywords{integral closure}
\pmrelated{FiniteExtension}
\pmrelated{DivisorTheoryInFiniteExtension}

\endmetadata

% this is the default PlanetMath preamble.  as your knowledge
% of TeX increases, you will probably want to edit this, but
% it should be fine as is for beginners.

% almost certainly you want these
\usepackage{amssymb}
\usepackage{amsmath}
\usepackage{amsfonts}

% used for TeXing text within eps files
%\usepackage{psfrag}
% need this for including graphics (\includegraphics)
%\usepackage{graphicx}
% for neatly defining theorems and propositions
\usepackage{amsthm}
% making logically defined graphics
%%%\usepackage{xypic}

% there are many more packages, add them here as you need them

% define commands here
\newtheorem*{theorem*}{Theorem}
\newtheorem*{lemma*}{Lemma}
\newtheorem*{corollary*}{Corollary}
\newtheorem{theorem}{Theorem}
\newtheorem{lemma}{Lemma}
\newtheorem{corollary}{Corollary}


\begin{document}
\begin{theorem*}
Let $R$ be a Dedekind domain with field of fractions $K$. If $L/K$ is a finite extension of fields and $A$ is the integral closure of $R$ in $L$, then $A$ is also a Dedekind domain.
\end{theorem*}

For example, a number field $K$ is a finite extension of $\mathbb{Q}$ and its ring of integers is denoted by $\mathcal{O}_K$. Although such rings can fail to be unique factorization domains, the above theorem shows that they are always Dedekind domains and therefore \PMlinkname{unique factorization of ideals}{IdealDecompositionInDedekindDomain} is satisfied.

%%%%%
%%%%%
\end{document}
