\documentclass[12pt]{article}
\usepackage{pmmeta}
\pmcanonicalname{PrimaryIdeal}
\pmcreated{2013-03-22 14:15:01}
\pmmodified{2013-03-22 14:15:01}
\pmowner{mathcam}{2727}
\pmmodifier{mathcam}{2727}
\pmtitle{primary ideal}
\pmrecord{6}{35697}
\pmprivacy{1}
\pmauthor{mathcam}{2727}
\pmtype{Definition}
\pmcomment{trigger rebuild}
\pmclassification{msc}{13C99}
\pmdefines{primary}
\pmdefines{$P$-primary}

\endmetadata

% this is the default PlanetMath preamble.  as your knowledge
% of TeX increases, you will probably want to edit this, but
% it should be fine as is for beginners.

% almost certainly you want these
\usepackage{amssymb}
\usepackage{amsmath}
\usepackage{amsfonts}
\usepackage{amsthm}

% used for TeXing text within eps files
%\usepackage{psfrag}
% need this for including graphics (\includegraphics)
%\usepackage{graphicx}
% for neatly defining theorems and propositions
%\usepackage{amsthm}
% making logically defined graphics
%%%\usepackage{xypic}

% there are many more packages, add them here as you need them

% define commands here

\newcommand{\mc}{\mathcal}
\newcommand{\mb}{\mathbb}
\newcommand{\mf}{\mathfrak}
\newcommand{\ol}{\overline}
\newcommand{\ra}{\rightarrow}
\newcommand{\la}{\leftarrow}
\newcommand{\La}{\Leftarrow}
\newcommand{\Ra}{\Rightarrow}
\newcommand{\nor}{\vartriangleleft}
\newcommand{\Gal}{\text{Gal}}
\newcommand{\GL}{\text{GL}}
\newcommand{\Z}{\mb{Z}}
\newcommand{\R}{\mb{R}}
\newcommand{\Q}{\mb{Q}}
\newcommand{\C}{\mb{C}}
\newcommand{\<}{\langle}
\renewcommand{\>}{\rangle}
\begin{document}
An ideal $Q$ in a commutative ring $R$ is a \emph{primary ideal} if for all elements $x,y\in R$, we have that if $xy\in Q$, then either $x\in Q$ or $y^n\in Q$ for some $n\in\mb{N}$.

This is clearly a generalization of the notion of a prime ideal, and (very) loosely mirrors the relationship in $\mb{Z}$ between prime numbers and prime powers.

It is clear that every prime ideal is primary.

\textbf{Example.}  Let $Q=(25)$ in $R=\mb{Z}$.  Suppose that $xy\in Q$ but $x\notin Q$.  Then $25|xy$, but 25 does not divide $x$.  Thus 5 must divide $y$, and thus some power of $y$ (namely, $y^2$), must be in $Q$.

The radical of a primary ideal is always a prime ideal.  If $P$ is the radical of the primary ideal $Q$, we say that $Q$ is \emph{$P$-primary}.

%If the radical of the primary ideal $Q$ is the prime ideal $P$, then $Q$ is said to be \emph{$P$-primary}.
%%%%%
%%%%%
\end{document}
