\documentclass[12pt]{article}
\usepackage{pmmeta}
\pmcanonicalname{ProofOfHilbertBasisTheorem}
\pmcreated{2013-03-22 12:59:27}
\pmmodified{2013-03-22 12:59:27}
\pmowner{bwebste}{988}
\pmmodifier{bwebste}{988}
\pmtitle{proof of Hilbert basis theorem}
\pmrecord{6}{33365}
\pmprivacy{1}
\pmauthor{bwebste}{988}
\pmtype{Proof}
\pmcomment{trigger rebuild}
\pmclassification{msc}{13E05}

% this is the default PlanetMath preamble.  as your knowledge
% of TeX increases, you will probably want to edit this, but
% it should be fine as is for beginners.

% almost certainly you want these
\usepackage{amssymb}
\usepackage{amsmath}
\usepackage{amsfonts}

% used for TeXing text within eps files
%\usepackage{psfrag}
% need this for including graphics (\includegraphics)
%\usepackage{graphicx}
% for neatly defining theorems and propositions
%\usepackage{amsthm}
% making logically defined graphics
%%%\usepackage{xypic} 

% there are many more packages, add them here as you need them

\newtheorem{theorem}{Theorem}[section]
\newtheorem{lemma}[theorem]{Lemma}
\newtheorem{proposition}[theorem]{Proposition}
\newtheorem{corollary}[theorem]{Corollary}

\newenvironment{proof}[1][Proof]{\begin{trivlist}
\item[\hskip \labelsep {\bfseries #1}]}{\end{trivlist}}
\newenvironment{definition}[1][Definition]{\begin{trivlist}
\item[\hskip \labelsep {\bfseries #1}]}{\end{trivlist}}
\newenvironment{example}[1][Example]{\begin{trivlist}
\item[\hskip \labelsep {\bfseries #1}]}{\end{trivlist}}
\newenvironment{remark}[1][Remark]{\begin{trivlist}
\item[\hskip \labelsep {\bfseries #1}]}{\end{trivlist}}
\begin{document}
Let $R$ be a noetherian ring and let $f(x) = a_n x^n + a_{n-1}
x^{n-1} + \ldots + a_1 x + a_0 \in R[x]$ with $a_n\neq 0$. Then
call $a_n$ the \emph{initial coefficient} of $f$.

Let $I$ be an ideal in $R[x]$.  We will show $I$ is finitely
generated, so that $R[x]$ is noetherian.  Now let $f_0$ be a
polynomial of least degree in $I$, and if $f_0, f_1, \ldots , f_k$
have been chosen then choose $f_{k+1}$ from $I\smallsetminus (f_0,
f_1, \ldots , f_k)$ of minimal degree.  Continuing inductively
gives a sequence $(f_k)$ of elements of $I$.

Let $a_k$ be the initial coefficient of $f_k$, and
consider the ideal $J=(a_1, a_2, a_3, \ldots )$ of initial
coefficients. Since $R$ is noetherian, $J=(a_0, \ldots , a_N)$ for
some $N$.

Then $I=(f_0, f_1, \ldots , f_N)$.  For if not then
$f_{N+1}\in I\smallsetminus (f_0, f_1, \ldots , f_N)$, and
$a_{N+1} = \sum_{k=0}^N u_k a_k$ for some $u_1, u_2, \ldots ,
u_N\in R$. Let $g(x)=\sum_{k=0}^N u_k f_k x^{\nu_k}$ where $\nu_k
= \operatorname{deg}(f_{N+1})-\operatorname{deg}(f_k)$.

Then $\operatorname{deg}(f_{N+1} - g) <
\operatorname{deg}(f_{N+1})$, and $f_{N+1} - g \in I$ and
$f_{N+1}-g\notin (f_0, f_1, \ldots , f_N)$.  But this contradicts
minimality of $\operatorname{deg}(f_{N+1})$.

Hence, $R[x]$ is noetherian.$\square$
%%%%%
%%%%%
\end{document}
