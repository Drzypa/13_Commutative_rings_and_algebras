\documentclass[12pt]{article}
\usepackage{pmmeta}
\pmcanonicalname{ArithmeticalRing}
\pmcreated{2013-03-22 15:23:58}
\pmmodified{2013-03-22 15:23:58}
\pmowner{PrimeFan}{13766}
\pmmodifier{PrimeFan}{13766}
\pmtitle{arithmetical ring}
\pmrecord{8}{37237}
\pmprivacy{1}
\pmauthor{PrimeFan}{13766}
\pmtype{Theorem}
\pmcomment{trigger rebuild}
\pmclassification{msc}{13A99}
\pmrelated{QuotientOfIdeals}

% this is the default PlanetMath preamble.  as your knowledge
% of TeX increases, you will probably want to edit this, but
% it should be fine as is for beginners.

% almost certainly you want these
\usepackage{amssymb}
\usepackage{amsmath}
\usepackage{amsfonts}

% used for TeXing text within eps files
%\usepackage{psfrag}
% need this for including graphics (\includegraphics)
%\usepackage{graphicx}
% for neatly defining theorems and propositions
 \usepackage{amsthm}
% making logically defined graphics
%%%\usepackage{xypic}

% there are many more packages, add them here as you need them

% define commands here

\theoremstyle{definition}
\newtheorem*{thmplain}{Theorem}
\begin{document}
\begin{thmplain}
If $R$ is a commutative ring, then the following three conditions are equivalent:
\begin{itemize}
 \item For all ideals $\mathfrak{a}$, $\mathfrak{b}$ and $\mathfrak{c}$ of $R$, one has\, $\mathfrak{a\cap(b+c) = (a\cap b)+(a\cap c)}$.
 \item For all ideals $\mathfrak{a}$, $\mathfrak{b}$ and $\mathfrak{c}$ of $R$, one has\, $\mathfrak{a+(b\cap c) = (a+b)\cap(a+c)}$.
 \item For each maximal ideal $\mathfrak{p}$ of $R$ the set of all ideals of $R_{\mathfrak{p}}$, the \PMlinkname{localisation}{Localization} of $R$ at\, $R\!\setminus\!\mathfrak{p}$,\, is totally ordered by set inclusion.
\end{itemize}
\end{thmplain}

The ring $R$ satisfying the conditions of the theorem is called an {\em arithmetical ring}.
%%%%%
%%%%%
\end{document}
