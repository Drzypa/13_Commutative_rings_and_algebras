\documentclass[12pt]{article}
\usepackage{pmmeta}
\pmcanonicalname{PrimeFactorsOfXn1}
\pmcreated{2013-03-22 16:29:51}
\pmmodified{2013-03-22 16:29:51}
\pmowner{pahio}{2872}
\pmmodifier{pahio}{2872}
\pmtitle{prime factors of $x^n-1$}
\pmrecord{12}{38673}
\pmprivacy{1}
\pmauthor{pahio}{2872}
\pmtype{Result}
\pmcomment{trigger rebuild}
\pmclassification{msc}{13G05}
%\pmkeywords{factorization}
\pmrelated{GausssLemmaII}
\pmrelated{IrreducibilityOfBinomialsWithUnityCoefficients}
\pmrelated{FactorsOfNAndXn1}
\pmrelated{ExamplesOfCyclotomicPolynomials}

\endmetadata

% this is the default PlanetMath preamble.  as your knowledge
% of TeX increases, you will probably want to edit this, but
% it should be fine as is for beginners.

% almost certainly you want these
\usepackage{amssymb}
\usepackage{amsmath}
\usepackage{amsfonts}

% used for TeXing text within eps files
%\usepackage{psfrag}
% need this for including graphics (\includegraphics)
%\usepackage{graphicx}
% for neatly defining theorems and propositions
 \usepackage{amsthm}
% making logically defined graphics
%%%\usepackage{xypic}

% there are many more packages, add them here as you need them

% define commands here

\theoremstyle{definition}
\newtheorem*{thmplain}{Theorem}

\begin{document}
We list prime factor \PMlinkescapetext{presentations} of the binomials 
$x^n\!-\!1$ in $\mathbb{Q}$, i.e. in the polynomial ring $\mathbb{Q}[x]$.\, The prime factors can always be chosen to be with integer coefficients and the number of the prime factors equals to \PMlinkname{$\tau(n)$}{TauFunction}; see \PMlinkname{the proof}{FactorsOfNAndXn1}.

$x-1$

$x^2\!-\!1 = (x+1)(x-1)$

$x^3\!-\!1 = (x^2+x+1)(x-1)$

$x^4\!-\!1 = (x^2+1)(x+1)(x-1)$

$x^5\!-\!1 = (x^4+x^3+x^2+x+1)(x-1)$

$x^6\!-\!1 = (x^2+x+1)(x^2-x+1)(x+1)(x-1)$

$x^7\!-\!1 = (x^6+x^5+x^4+x^3+x^2+x+1)(x-1)$

$x^8\!-\!1 = (x^4+1)(x^2+1)(x+1)(x-1)$

$x^9\!-\!1 = (x^6+x^3+1)(x^2+x+1)(x-1)$

$x^{10}\!-\!1 = (x^4+x^3+x^2+1)(x^4-x^3+x^2-x+1)(x+1)(x-1)$

$x^{11}\!-\!1 = (x^{10}\!+\!x^9\!+\!x^8\!+\!x^7\!+\!x^6\!+\!x^5\!+\!x^4\!+\!x^3\!+\!x^2\!+\!x\!+\!1)(x-1)$

$x^{12}\!-\!1 = (x^4-x^2+1)(x^2+x+1)(x^2-x+1)(x^2+1)(x+1)(x-1)$

$x^{13}\!-\!1 =(x^{12}\!+\!x^{11}\!+\!x^{10}\!+\!x^9\!+\!x^8\!+\!x^7\!+
\!x^6\!+\!x^5\!+\!x^4\!+\!x^3\!+\!x^2\!+\!x\!+\!1)(x\!-\!1)$

$x^{14}\!-\!1 = (x^6+x^5+x^4+x^3+x^2+x+1)(x^6-x^5+x^4-x^3+x^2-x+1)(x+1)(x-1)$

$x^{15}\!-\!1 = (x^8-x^7+x^5-x^4+x^3-x+1)(x^4+x^3+x^2+x+1)(x^2+x+1)(x-1)$

$x^{16}\!-\!1 = (x^8+1)(x^4+1)(x^2+1)(x+1)(x-1)$

$x^{17}\!-\!1 = (x^{16}\!+\!x^{15}\!+\!x^{14}\!+\ldots+\!x^2\!+\!x\!+\!1)(x\!-\!1)$

$x^{18}\!-\!1 = (x^6+x^3+1)(x^6-x^3+1)(x^2+x+1)(x^2-x+1)(x+1)(x-1)$

$x^{19}\!-\!1 = (x^{18}\!+\!x^{17}\!+\!x^{16}\!+\ldots+\!x^2\!+\!x\!+\!1)(x-1)$

$x^{20}\!-\!1 = (x^8-x^6+x^4-x^2+1)(x^4+x^3+x^2+x+1)(x^4-x^3+x^2-x+1)(x^2+1)(x+1)(x-1)$

$x^{21}\!-\!1 = (x^{12}\!-\!x^{11}\!+\!x^9\!-\!x^8\!+\!x^6\!-\!x^4\!+\!x^3\!-\!x\!+\!1)(x^6\!+\!x^5\!+\!x^4\!+\!x^3\!+\!x^2\!+\!x\!+\!1)(x^2\!+\!x\!+\!1)
(x\!-\!1)$

$x^{22}\!-\!1 = (x^{10}\!+\!x^9\!+\!x^8\!+\!x^7\!+\!x^6\!+\!x^5\!+\!x^4\!+\!x^3\!+\!x^2\!+\!x\!+\!1)(x^{10}\!-\!x^9\!+\!x^8\!-\!x^7\!+\!x^6\!-\!x^5\!+\!x^4\!-\!x^3\!+\!x^2\!-\!x\!+\!1)(x\!+\!1)(x\!-\!1)$

$x^{23}\!-\!1 = (x^{22}\!+\!x^{21}\!+\!x^{20}\!+\ldots+\!x^2\!+\!x\!+\!1)(x\!-\!1)$

$x^{24}\!-\!1 = (x^8\!-\!x^4\!+\!1)(x^4\!-\!x^2\!+\!1)(x^4\!+\!1)(x^2\!+\!x\!+\!1)(x^2\!-\!x\!+\!1)(x^2\!+\!1)(x\!+\!1)(x\!-\!1)$

\textbf{Note 1.}\, All factors shown above are irreducible polynomials (in the field\, $\mathbb{Q}$\, of their own coefficients), but of course they (except $x\!\pm\!1$) may be split into factors of positive degree in certain extension fields; so e.g.
 $$x^4\!+\!1 = (x^2\!+\!x\sqrt{2}\!+\!1)(x^2\!-\!x\sqrt{2}\!+\!1)\quad 
\mathrm{in\,the\,field}\,\,\,\mathbb{Q}(\sqrt{2}).$$
\textbf{Note 2.}\, The 24 examples of factorizations are true also in the fields of characteristic $\neq 0$, but then many of the factors can be simplified or factored onwards (e.g.\, $x^2\!+\!1 \equiv (x\!+\!1)^2$\, if the \PMlinkname{characteristic}{Characteristic} is 2).
%%%%%
%%%%%
\end{document}
