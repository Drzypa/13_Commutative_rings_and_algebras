\documentclass[12pt]{article}
\usepackage{pmmeta}
\pmcanonicalname{FormalPowerSeriesAsInverseLimits}
\pmcreated{2013-03-22 18:22:41}
\pmmodified{2013-03-22 18:22:41}
\pmowner{rspuzio}{6075}
\pmmodifier{rspuzio}{6075}
\pmtitle{formal power series as inverse limits}
\pmrecord{9}{41020}
\pmprivacy{1}
\pmauthor{rspuzio}{6075}
\pmtype{Result}
\pmcomment{trigger rebuild}
\pmclassification{msc}{13F25}
\pmclassification{msc}{13B35}
\pmclassification{msc}{13J05}
\pmclassification{msc}{13H05}

\endmetadata

% this is the default PlanetMath preamble.  as your knowledge
% of TeX increases, you will probably want to edit this, but
% it should be fine as is for beginners.

% almost certainly you want these
\usepackage{amssymb}
\usepackage{amsmath}
\usepackage{amsfonts}

% used for TeXing text within eps files
%\usepackage{psfrag}
% need this for including graphics (\includegraphics)
%\usepackage{graphicx}
% for neatly defining theorems and propositions
\usepackage{amsthm}
% making logically defined graphics
%%%\usepackage{xypic}

% there are many more packages, add them here as you need them

% define commands here
\newtheorem{thm}{Theorem}
\newtheorem{dfn}{Definition}
\begin{document}
\section{Motivation and Overview}

The ring of formal power series can be described as an inverse
limit.\footnote{It is worth pointing out that, since we are
dealing with formal series, the concept of limit used here has 
nothing to do with convergence but is purely algebraic.}  The
fundamental idea behind this approach is that of truncation ---
given a formal power series $a_0 + a_1 t + a_2 t^2 + a_3 t^3 + 
\cdots$ and an integer $n \ge 0$, we can truncate the series to
order $n$ to obtain $a_0 + a_1 t + \cdots + a_n t^n + 
O(t^{n+1})$.\footnote{Here, the symbol ``$O(t^{n+1})$'' is not
used in the sense of Landau notation but merely as an indicator 
that the power series has been truncated to order $n$.}
(Indeed, we must do this in practical computation since it is
only possible to write down a finite number of terms at a time;
thus the approach taken here has the advantage of being close
to actual practice.)  Furthermore, this procedure of truncation
commutes with ring operations --- given two formal power series,
the truncation of their sum is the sum of their truncations and
the truncation of their product is the product of their 
truncations.  Thus, for every integer $n$ the set of power 
series truncated to order $n$ forms a ring and truncation is a
morphism from the ring of formal power series to this ring.

To obtain our definition, we will proceed in the opposite
direction.  We will begin by defining rings of truncated 
power series and exhibiting truncation morphisms between
different truncations.  Then we will show that these rings 
and morphisms form an inverse system which has a limit which
we will take as the definition of the ring of formal power
series.  Finally, we will complete the circle by demonstrating
that the object so constructed is isomorphic with the usual
definition for ring of formal power series.

\section{Formal Development}

In this section, we will carry out the develpment outlined
above in rigorous detail.  We begin by formalizing this
notion of truncation.

\begin{thm}
Let $A$ be a commutative ring and let $n$ be a positive 
integer.  Then $A[[x]] / \langle x^n \rangle$ is isomorphic
to $A[x] / \langle x^n \rangle$.
\end{thm}

\begin{proof}
We may identify $A[x]$ with the subring of $A[[x]]$ consisting
of series which have all but a finite number of coeficients
equal to zero.  Consider an element $f = \sum_{k=0}^\infty c_n 
x^n$ of $A[[x]]$.  We may write $f = \sum_{k=0}^{n-1} c_n x^k 
+ x^n \sum_{k=0}^\infty c_{k+n} x^k$.  Thus, every element of
$A[[x]]$ is equivalent to an element of the subring $A[x]$
modulo $x^k$.  Hence, if follows rather immediately from the
definition of quotient ring that $A[[x]] \langle x^n \rangle$  
is isomorphic to $A[x] / \langle x^n \rangle$.
\end{proof}

Let us call the isomorphism between 
$A[[x]] / \langle x^n \rangle$ and $A[x] / \langle x^n \rangle$
which is described above $I_n$.  We now define a few more morphisms.

\begin{dfn}
Suppose $m,n$ are integers satifying the inequalities 
$m > n \ge 0$.  Then define the morphisms $t_{mn}, T_{mn},
p_n, P_n$ as follows:
\begin{itemize}
\item Define
$t_{nm} \colon A[x] / \langle x^{m} \rangle \to A[x] / 
\langle x^{n} \rangle$ as the map which sends each
equivalence class $a$ modulo $x^{n}$ to the unique equivalence 
class $b$ modulo $x^{m}$ such that $a \subset b$.
\item Define
$T_{nm} \colon A[[x]] / \langle x^{m} \rangle \to A[[x]] / 
\langle x^{n+1} \rangle$ as the map which sends each
equivalence class $a$ modulo $x^{n}$ to the unique equivalence 
class $b$ modulo $x^{m}$ such that $a \subset b$.
\item For every integer $n > 0$, let $Q_n$ be the quotient
map from $A[[x]]$ to $A[[x]] / \langle x^n \rangle$.
\item For every integer $n > 0$, let $q_n$ be the quotient
map from $A[x]$ to $A[x] / \langle x^n \rangle$.
\end{itemize}
\end{dfn}

These morphisms commute with each other in ways which are
described by the next theorem:

\begin{thm}
Suppose $m,n,k$ are integers satifying the inequalities 
$m > n > k \ge 0$.  Then we have the following relations:
\begin{enumerate}
\item $t_{nk} \circ t_{mn} = t_{mk}$
\item $I_n \circ T_{mn} = t_{mn} \circ I_m$
\item $T_{nk} \circ T_{mn} = T_{mk}$
\item $t_{mn} \circ q_m = q_n$
\item $T_{mn} \circ Q_m = Q_n$
\end{enumerate}
\end{thm}
[More to come]


%%%%%
%%%%%
\end{document}
