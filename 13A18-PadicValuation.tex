\documentclass[12pt]{article}
\usepackage{pmmeta}
\pmcanonicalname{PadicValuation}
\pmcreated{2013-03-22 14:55:50}
\pmmodified{2013-03-22 14:55:50}
\pmowner{pahio}{2872}
\pmmodifier{pahio}{2872}
\pmtitle{p-adic valuation}
\pmrecord{14}{36619}
\pmprivacy{1}
\pmauthor{pahio}{2872}
\pmtype{Definition}
\pmcomment{trigger rebuild}
\pmclassification{msc}{13A18}
\pmsynonym{$p$-adic valuation}{PadicValuation}
\pmrelated{IndependenceOfPAdicValuations}
\pmrelated{IntegralElement}
\pmrelated{OrderValuation}
\pmrelated{StrictDivisibility}
\pmdefines{$p$-integral rational number}
\pmdefines{normed $p$-adic valuation}
\pmdefines{normed archimedean valuation}
\pmdefines{dyadic valuation}
\pmdefines{triadic valuation}
\pmdefines{pentadic valuation}
\pmdefines{heptadic valuation}

\endmetadata

% this is the default PlanetMath preamble.  as your knowledge
% of TeX increases, you will probably want to edit this, but
% it should be fine as is for beginners.

% almost certainly you want these
\usepackage{amssymb}
\usepackage{amsmath}
\usepackage{amsfonts}

% used for TeXing text within eps files
%\usepackage{psfrag}
% need this for including graphics (\includegraphics)
%\usepackage{graphicx}
% for neatly defining theorems and propositions
%\usepackage{amsthm}
% making logically defined graphics
%%%\usepackage{xypic}

% there are many more packages, add them here as you need them

% define commands here
\begin{document}
Let $p$ be a positive prime number.\, For every non-zero rational number $x$ there exists a unique integer $n$ such that 
             $$x = p^n\cdot\frac{u}{v}$$
with some integers $u$ and $v$ indivisible by $p$.\, We define
$$|x|_p :=
 \begin{cases}
   (\frac{1}{p})^n \quad \mathrm{when} \,\, x \neq 0, \\
   0 \quad \mathrm{when} \,\, x=0,
 \end{cases}
$$
obtaining a \PMlinkname{non-trivial}{TrivialValuation} non-archimedean valuation, the so-called $p$-{\em adic valuation}
               $$|\cdot|_p:\,\mathbb{Q} \to \mathbb{R}$$
of the field $\mathbb{Q}$.

The value group of the $p$-adic valuation consists of all integer-powers of the prime number $p$.\, The valuation ring of the valuation is called the ring of the {\em p-integral rational numbers}; their denominators, when \PMlinkname{reduced}{Fraction} to lowest terms, are not divisible by $p$.

The field of rationals has the {\em 2-adic, 3-adic, 5-adic, 7-adic} and so on valuations (which may be called, according to Greek, {\em dyadic, triadic, pentadic, heptadic} and so on).\, They all are \PMlinkname{non-equivalent}{EquivalentValuations} with each other.

If one replaces the \PMlinkescapetext{base} number $\frac{1}{p}$ by any positive \PMlinkescapetext{constant} $\varrho$ less than 1, one obtains an \PMlinkname{equivalent}{EquivalentValuations} $p$-adic valuation; among these the valuation with\, $\varrho = \frac{1}{p}$\, is sometimes called the {\em normed $p$-adic valuation}.\, Analogously we can say that the absolute value is the normed archimedean valuation of $\mathbb{Q}$ which corresponds the infinite prime $\infty$ of $\mathbb{Z}$.

The product of all normed valuations of $\mathbb{Q}$ is the trivial valuation\, $|\cdot|_\mathrm{tr}$,\, i.e.
        $$\prod_{p\,\mathrm{prime}}|x|_p = |x|_\mathrm{tr} \quad 
                            \forall x\in\mathbb{Q}.$$
%%%%%
%%%%%
\end{document}
