\documentclass[12pt]{article}
\usepackage{pmmeta}
\pmcanonicalname{Overring}
\pmcreated{2013-03-22 14:22:33}
\pmmodified{2013-03-22 14:22:33}
\pmowner{pahio}{2872}
\pmmodifier{pahio}{2872}
\pmtitle{overring}
\pmrecord{12}{35867}
\pmprivacy{1}
\pmauthor{pahio}{2872}
\pmtype{Definition}
\pmcomment{trigger rebuild}
\pmclassification{msc}{13B30}
\pmrelated{AConditionOfAlgebraicExtension}

% this is the default PlanetMath preamble.  as your knowledge
% of TeX increases, you will probably want to edit this, but
% it should be fine as is for beginners.

% almost certainly you want these
\usepackage{amssymb}
\usepackage{amsmath}
\usepackage{amsfonts}

% used for TeXing text within eps files
%\usepackage{psfrag}
% need this for including graphics (\includegraphics)
%\usepackage{graphicx}
% for neatly defining theorems and propositions
%\usepackage{amsthm}
% making logically defined graphics
%%%\usepackage{xypic}

% there are many more packages, add them here as you need them

% define commands here
\begin{document}
Let $R$ be a commutative ring having regular elements and let $T$ be the total ring of fractions of $R$. \,Then \,$R \subseteq T$. \,Every subring of $T$ containing $R$ is an {\em overring} of $R$.\\

\textbf{Example.}\, Let $p$ be a rational prime number.\, The \PMlinkname{$p$-integral rational numbers}{PAdicValuation} are the quotients of two integers such that the \PMlinkname{divisor}{Division} is not divisible by $p$.\, The set of all $p$-integral rationals is an overring of $\mathbb{Z}$.
%%%%%
%%%%%
\end{document}
