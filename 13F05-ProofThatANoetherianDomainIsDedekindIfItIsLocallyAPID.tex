\documentclass[12pt]{article}
\usepackage{pmmeta}
\pmcanonicalname{ProofThatANoetherianDomainIsDedekindIfItIsLocallyAPID}
\pmcreated{2013-03-22 18:35:27}
\pmmodified{2013-03-22 18:35:27}
\pmowner{gel}{22282}
\pmmodifier{gel}{22282}
\pmtitle{proof that a Noetherian domain is Dedekind if it is locally a PID}
\pmrecord{4}{41318}
\pmprivacy{1}
\pmauthor{gel}{22282}
\pmtype{Proof}
\pmcomment{trigger rebuild}
\pmclassification{msc}{13F05}
\pmclassification{msc}{13A15}
%\pmkeywords{Dedekind domain}
%\pmkeywords{principal ideal domain}
%\pmkeywords{localization}

% this is the default PlanetMath preamble.  as your knowledge
% of TeX increases, you will probably want to edit this, but
% it should be fine as is for beginners.

% almost certainly you want these
\usepackage{amssymb}
\usepackage{amsmath}
\usepackage{amsfonts}

% used for TeXing text within eps files
%\usepackage{psfrag}
% need this for including graphics (\includegraphics)
%\usepackage{graphicx}
% for neatly defining theorems and propositions
\usepackage{amsthm}
% making logically defined graphics
%%%\usepackage{xypic}

% there are many more packages, add them here as you need them

% define commands here
\newtheorem*{theorem*}{Theorem}
\newtheorem*{lemma*}{Lemma}
\newtheorem*{corollary*}{Corollary}
\newtheorem{theorem}{Theorem}
\newtheorem{lemma}{Lemma}
\newtheorem{corollary}{Corollary}


\begin{document}
\PMlinkescapeword{inverse}
We show that for a \PMlinkname{Noetherian}{Noetherian} domain $R$ with field of fractions $k$, the following are equivalent.
\begin{enumerate}
\item $R$ is Dedekind. That is, it is integrally closed and every nonzero prime ideal is maximal.
\item for every maximal ideal $\mathfrak{m}$,
\begin{equation*}
R_\mathfrak{m}\equiv\left\{s^{-1}x:s\in R\setminus\mathfrak{m},x\in R\right\}
\end{equation*}
is a principal ideal domain.
\end{enumerate}

For a given maximal ideal $\mathfrak{m}$ and ideal $\mathfrak{a}$ of $R$, we shall write $\mathfrak{\bar a}$ for the ideal generated by $\mathfrak{a}$ in $R_\mathfrak{m}$, which consists of the elements of the form $s^{-1}a$ for $a\in\mathfrak{a}$ and $s\in R\setminus\mathfrak{m}$. It is then easily seen that $\mathfrak{p}\mapsto \mathfrak{\bar p}$ gives a bijection between the prime ideals of $R$ contained in $\mathfrak{m}$ and the prime ideals of $R_\mathfrak{m}$, with inverse $\mathfrak{p}\mapsto R\cap\mathfrak{p}$. In particular $\mathfrak{\bar m}$ is the unique maximal ideal of $R_\mathfrak{m}$, which is therefore a local ring.

Now suppose that $R$ is Dedekind, then the localization $R_\mathfrak{m}$ will be a Dedekind domain (localizations of Dedekind domains are Dedekind) with a unique maximal ideal, so it is a principal ideal domain (Dedekind domains with finitely many primes are PIDs).

Only the converse remains to be shown, so suppose that $R$ is a Noetherian domain such that $R_\mathfrak{m}$ is a principal ideal domain for every maximal ideal $\mathfrak{m}$. In particular, $R_\mathfrak{m}$ is integrally closed and every nonzero prime ideal is maximal, so it contains a unique nonzero prime ideal $\mathfrak{\bar m}$.

We start by showing that every nonzero prime ideal $\mathfrak{p}$ of $R$ is maximal. Choose a maximal ideal $\mathfrak{m}$ containing $\mathfrak{p}$. Then, $\mathfrak{\bar p}$ is a nonzero prime ideal, so $\mathfrak{\bar p}=\mathfrak{\bar m}$ and therefore $\mathfrak{p}=\mathfrak{m}$ is maximal.

We finally show that $R$ is integrally closed. So, choose any $x$ integral over $R$ and lying in its field of fractions. Let $\mathfrak{a}$ be the ideal
\begin{equation*}
\mathfrak{a}=\left\{a\in R:ax\in R\right\}.
\end{equation*}
We use proof by contradiction to show that $\mathfrak{a}$ is the whole of $R$. So, supposing that this is not the case, there exists a maximal ideal $\mathfrak{m}$ containing $\mathfrak{a}$. Then $x$ will be integral over the integrally closed ring $R_\mathfrak{m}$ and therefore $x\in R_\mathfrak{m}$. So, $x=s^{-1}y$ for some $s\in R\setminus\mathfrak{m}$ and $y\in R$. Then, $sx=y\in R$ so $s\in\mathfrak{a}\subseteq\mathfrak{m}$, which is the required contradiction. Therefore, $\mathfrak{a}=R$ and, in particular, $1\in\mathfrak{a}$ and $x=1x\in R$, showing that $R$ is integrally closed.

%%%%%
%%%%%
\end{document}
