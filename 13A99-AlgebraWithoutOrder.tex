\documentclass[12pt]{article}
\usepackage{pmmeta}
\pmcanonicalname{AlgebraWithoutOrder}
\pmcreated{2013-03-22 14:46:01}
\pmmodified{2013-03-22 14:46:01}
\pmowner{mathcam}{2727}
\pmmodifier{mathcam}{2727}
\pmtitle{algebra without order}
\pmrecord{11}{36411}
\pmprivacy{1}
\pmauthor{mathcam}{2727}
\pmtype{Definition}
\pmcomment{trigger rebuild}
\pmclassification{msc}{13A99}

\endmetadata

% this is the default PlanetMath preamble.  as your knowledge
% of TeX increases, you will probably want to edit this, but
% it should be fine as is for beginners.

% almost certainly you want these
\usepackage{amssymb}
\usepackage{amsmath}
\usepackage{amsfonts}

% used for TeXing text within eps files
%\usepackage{psfrag}
% need this for including graphics (\includegraphics)
%\usepackage{graphicx}
% for neatly defining theorems and propositions
%\usepackage{amsthm}
% making logically defined graphics
%%%\usepackage{xypic}

% there are many more packages, add them here as you need them

% define commands here
\begin{document}
An \PMlinkname{algebra}{Algebra} $A$ is said to be \PMlinkescapephrase{order} {\em without order} if it is commutative and \\for each $a \in A$, there exists $b \in A$ such that $ab \ne 0$.\\

The phrase algebra without order seems first in the book ``Multipliers of Banach algebras'' by Ronald Larsen. In noncommutative case, the concept is divied into two parts -- without left/right order. However, in the noncommutative case, it is defined in terms of the injectivity of the left (right) regular representation given by $x \in A \mapsto L_x \in L(A)$. 

Note that for an algebra $A$ and an element $x \in A$, $L_x : A \to A$ is the map defined by $L_x(y) = xy$. Then $L_x$ is a linear operator on $A$. It is easy to see that $A$ is without left order if and only if the map $x \in A \mapsto L_x \in L(A)$ is one-one; equivalently, the left ideal $\{x \in A : x \in A\} = \{0\}$. This ideal is is called the left annihilator of $A$. 

Every commutative algebra with identity is without order.\\

Example: $\mathbb{R}^2$ with multiplication defined by $(x_1,x_2) * (y_1,y_2) = (x_1y_1,0)$, ($(x_1,x_2), (y_1,y_2) \in \mathbb{R}^2$) is not an algebra without order as multiplication of (0,1) with any other element gives (0,0).
%%%%%
%%%%%
\end{document}
