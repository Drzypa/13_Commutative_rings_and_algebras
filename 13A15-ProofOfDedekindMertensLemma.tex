\documentclass[12pt]{article}
\usepackage{pmmeta}
\pmcanonicalname{ProofOfDedekindMertensLemma}
\pmcreated{2013-12-15 20:29:20}
\pmmodified{2013-12-15 20:29:20}
\pmowner{pahio}{2872}
\pmmodifier{pahio}{2872}
\pmtitle{proof of Dedekind$-$Mertens lemma}
\pmrecord{5}{87978}
\pmprivacy{1}
\pmauthor{pahio}{2872}
\pmtype{Proof}
\pmclassification{msc}{13A15}
\pmclassification{msc}{13M10}
\pmclassification{msc}{16D10}
\pmclassification{msc}{16D25}

\endmetadata

% this is the default PlanetMath preamble.  as your knowledge
% of TeX increases, you will probably want to edit this, but
% it should be fine as is for beginners.

% almost certainly you want these
\usepackage{amssymb}
\usepackage{amsmath}
\usepackage{amsfonts}

% need this for including graphics (\includegraphics)
\usepackage{graphicx}
% for neatly defining theorems and propositions
\usepackage{amsthm}

% making logically defined graphics
%\usepackage{xypic}
% used for TeXing text within eps files
%\usepackage{psfrag}

% there are many more packages, add them here as you need them

% define commands here

\begin{document}
Let $R$ be subring of the commutative ring $T$ and
$$f(X) = f_0+f_1X+\ldots+f_mX^m \quad \mbox{and} \quad 
  g(X) = g_0+g_1X+\ldots+g_nX^n$$
be arbitrary polynomials in $T[X]$.\, We will prove by induction on $n$ that the $R$-submodules of $T$ generated by the 
coefficients of the polynomials $f$, $g$, and $fg$ satisfy
\begin{align}
M_f^{n+1}M_g \;=\; M_f^nM_{fg}
\end{align}
where the product modules are generated by the products of their generators.

The generators of the right hand side of (1) belong obviously to the left hand side, 
whence only the containment
\begin{align}
M_f^{n+1}M_g \;\subseteq\; M_f^nM_{fg}
\end{align}
has to be proved.\, 

Firstly, (2) is trivial in the case\, $n =0$.\, Let now\, $n
> 0$.\, Define 
$$f_j \;:=\; 0 \quad\mbox{for}\quad j < 0 
\quad\mbox{or}\quad j > m$$
and let $G_n$ be the $R$-submodule of $T$ generated by $g_0,g_1,\ldots,g_{n-1}$.\, We have
$$\sum_{i<n}f_{k-i}g_i \;=\; h_k-f_{k-n}g_n \;\in\; M_{fg}+g_nM_f$$
where $h_k$ is the coefficient of $X^k$ of the polynomial $fg$, and thus by induction we 
can write
$$M_f^nG_n \;\subseteq\; M_f^{n-1}(M_{fg}+g_nM_f) 
\;\subseteq\; M_f^{n-1}M_{fg}+M_f^ng_n.$$
This implies the containment
$$f_iM_f^nG_n \;\subseteq\; M_f^nM_{fg}+M_f^nf_ig_n$$
for every $i$.\, In addition, we have
$$f_ig_n \;\in\; M_{fg}+f_{i+1}G_n+M_{i+2}G_n+\ldots+f_nG_n,$$
whence
$$f_iM_{f}^nG_n \;\subseteq\; 
M_f^nM_{fg}+f_{i+1}M_f^nG_n+\ldots+f_nM_f^nG_n.$$
From this we infer that
$$f_iM_f^nG_n \;\subseteq\; M_f^nM_{fg}$$
is true for each\, $i$ $=$ $m$, $m\!-\!1,\,\ldots,\,0$.\, Thus also (2) is true.\\

\begin{thebibliography}{9}
 \bibitem{PJ}{\sc J. Pahikkala:} ``{Some formulae for multiplying and inverting ideals}''. \,-- {\em Ann. Univ. Turkuensis} \textbf{183} (A) (1982).
 \bibitem{A+G}{\sc J. Arnold \& R. Gilmer:} ``{On the contents of polynomials}''. \,-- {\em Proc. Amer. Math. Soc.} \textbf{24} (1970).
 \bibitem{TC}{\sc T. Coquand}: ``{\it A direct proof of 
 Dedekind--Mertens lemma}''. University of Gothenburg 2006. (Available 
 \PMlinkexternal{here}{http://www.cse.chalmers.se/~coquand/mertens.pdf}.) 
\end{thebibliography}
%%%%%
%%%%%
\end{document}
