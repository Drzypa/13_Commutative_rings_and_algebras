\documentclass[12pt]{article}
\usepackage{pmmeta}
\pmcanonicalname{DecompositionGroup}
\pmcreated{2013-03-22 12:43:25}
\pmmodified{2013-03-22 12:43:25}
\pmowner{djao}{24}
\pmmodifier{djao}{24}
\pmtitle{decomposition group}
\pmrecord{10}{33022}
\pmprivacy{1}
\pmauthor{djao}{24}
\pmtype{Definition}
\pmcomment{trigger rebuild}
\pmclassification{msc}{13B02}
\pmclassification{msc}{11S15}
\pmsynonym{splitting group}{DecompositionGroup}
\pmrelated{Ramification}
\pmrelated{InertialDegree}
\pmdefines{inertia group}

\endmetadata

% this is the default PlanetMath preamble.  as your knowledge
% of TeX increases, you will probably want to edit this, but
% it should be fine as is for beginners.

% almost certainly you want these
\usepackage{amssymb}
\usepackage{amsmath}
\usepackage{amsfonts}

% used for TeXing text within eps files
%\usepackage{psfrag}
% need this for including graphics (\includegraphics)
%\usepackage{graphicx}
% for neatly defining theorems and propositions
%\usepackage{amsthm}
% making logically defined graphics
\usepackage[all]{xypic}

% there are many more packages, add them here as you need them

% define commands here
\newcommand{\p}{{\mathfrak{p}}}
\renewcommand{\a}{{\mathfrak{a}}}
\renewcommand{\b}{{\mathfrak{b}}}
\newcommand{\m}{{\mathfrak{m}}}
\newcommand{\M}{{\mathfrak{M}}}
\renewcommand{\P}{{\mathfrak{P}}}
\newcommand{\C}{\mathbb{C}}
\newcommand{\R}{\mathbb{R}}
\newcommand{\Z}{\mathbb{Z}}
\newcommand{\Q}{\mathbb{Q}}
\newcommand{\N}{\mathbb{N}}
\renewcommand{\H}{\mathcal{H}}
\newcommand{\A}{\mathbb{A}}
\renewcommand{\c}{\mathcal{C}}
\renewcommand{\O}{\mathcal{O}}
\newcommand{\D}{\mathcal{D}}
\newcommand{\lra}{\longrightarrow}
\renewcommand{\div}{\mid}
\newcommand{\res}{\operatorname{res}}
\newcommand{\Spec}{\operatorname{Spec}}
\newcommand{\Gal}{\operatorname{Gal}}
\newcommand{\id}{\operatorname{id}}
\newcommand{\diff}{\operatorname{diff}}
\newcommand{\incl}{\operatorname{incl}}
\newcommand{\Hom}{\operatorname{Hom}}
\renewcommand{\Re}{\operatorname{Re}}
\newcommand{\intersect}{\cap}
\newcommand{\union}{\cup}
\newcommand{\bigintersect}{\bigcap}
\newcommand{\bigunion}{\bigcup}
\newcommand{\ilim}{\,\underset{\longleftarrow}{\lim}\,}
\begin{document}
\section{Decomposition Group}

Let $A$ be a Noetherian integrally closed integral domain with field
of fractions $K$. Let $L$ be a Galois extension of $K$ and denote by
$B$ the integral closure of $A$ in $L$. Then, for any prime ideal $\p
\subset A$, the Galois group $G := \Gal(L/K)$ acts transitively on the
set of all prime ideals $\P \subset B$ containing $\p$. If we fix a
particular prime ideal $\P \subset B$ lying over $\p$, then the
stabilizer of $\P$ under this group action is a subgroup of
$G$, called the {\em decomposition group} at $\P$ and denoted
$D(\P/\p)$. In other words,
$$
D(\P/\p) := \{\sigma \in G \mid \sigma(\P) = (\P)\}.
$$
If $\P' \subset B$ is another prime ideal of $B$ lying over $\p$, then
the decomposition groups $D(\P/\p)$ and $D(\P'/\p)$ are conjugate in
$G$ via any Galois automorphism mapping $\P$ to $\P'$.

\section{Inertia Group}

Write $l$ for the residue field $B/\P$ and $k$ for the residue field
$A/\p$. Assume that the extension $l/k$ is separable (if it is not,
then this development is still possible, but considerably more
complicated; see~\cite[p. 20]{serre}). Any element $\sigma \in
D(\P/\p)$, by definition, fixes $\P$ and hence descends to a well
defined automorphism of the field $l$. Since $\sigma$ also fixes $A$
by virtue of being in $G$, it induces an automorphism of the extension
$l/k$ fixing $k$. We therefore have a group homomorphism
$$
D(\P/\p) \lra \Gal(l/k),
$$
and the \PMlinkname{kernel}{KernelOfAGroupHomomorphism} of this homomorphism is called the {\em inertia group} of
$\P$, and written $T(\P/\p)$. It turns out that this homomorphism is
actually surjective, so there is an exact sequence
\begin{equation}\label{exact}
\xymatrix{
1 \ar[r] & T(\P/\p) \ar[r] & D(\P/\p) \ar[r] & \Gal(l/k) \ar[r] & 1
}
\end{equation}

\section{Decomposition of Extensions}

The decomposition group is so named because it can be used to
decompose the field extension $L/K$ into a series of intermediate
extensions each of which has very simple factorization behavior at
$\p$. If we let $L^D$ denote the fixed field of $D(\P/\p)$ and $L^T$
the fixed field of $T(\P/\p)$, then the exact sequence~\eqref{exact}
corresponds under Galois theory to the lattice of fields
$$
\xymatrix{
L \ar@{-}[d]^e \\
L^T \ar@{-}[d]^f \\
L^D \ar@{-}[d]^g \\
K
}
$$
If we write $e,f,g$ for the degrees of these intermediate extensions
as in the diagram, then we have the following remarkable series of
equalities:
\begin{enumerate}
\item The number $e$ equals the ramification index $e(\P/\p)$ of $\P$
  over $\p$, which is independent of the choice of prime ideal $\P$
  lying over $\p$ since $L/K$ is Galois.
\item The number $f$ equals the inertial degree $f(\P/\p)$ of $\P$
  over $\p$, which is also independent of the choice of prime ideal $\P$
  since $L/K$ is Galois.
\item The number $g$ is equal to the number of prime ideals $\P$ of
  $B$ that lie over $\p \subset A$.
\end{enumerate}

Furthermore, the fields $L^D$ and $L^T$ have the following independent
characterizations:
\begin{itemize}
\item $L^T$ is the smallest intermediate field $F$ such that $\P$ is
  totally ramified over $\P \cap F$, and it is the largest
  intermediate field such that $e(\P \cap F, \p) = 1$.
\item $L^D$ is the smallest intermediate field $F$ such that $\P$ is
  the only prime of $B$ lying over $\P \cap F$, and it is the largest
  intermediate field such that $e(\P \cap F, \p) = f(\P \cap F, \p) =
  1$.
\end{itemize}

Informally, this decomposition of the extension says that the
extension $L^D/K$ encapsulates all of the factorization of $\p$ into
distinct primes, while the extension $L^T/L^D$ is the source of all
the inertial degree in $\P$ over $\p$ and the extension $L/L^T$ is
responsible for all of the ramification that occurs over $\p$.

\section{Localization}

The decomposition groups and inertia groups of $\P$ behave well under
localization. That is, the decomposition and inertia groups of $\P
B_\P \subset B_\P$ over the prime ideal $\p A_\p$ in the localization
$A_\p$ of $A$ are identical to the ones obtained using $A$ and $B$
themselves. In fact, the same holds true even in the completions of
the local rings $A_\p$ and $B_\P$ at $\p$ and $\P$.

\begin{thebibliography}{9}
\bibitem{serre} J.P. Serre, {\em Local Fields}, Springer--Verlag, 1979
  (GTM {\bf 67})
\end{thebibliography}
%%%%%
%%%%%
\end{document}
