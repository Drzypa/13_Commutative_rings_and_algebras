\documentclass[12pt]{article}
\usepackage{pmmeta}
\pmcanonicalname{EquivalentDefinitionsForUFD}
\pmcreated{2013-03-22 19:04:04}
\pmmodified{2013-03-22 19:04:04}
\pmowner{joking}{16130}
\pmmodifier{joking}{16130}
\pmtitle{equivalent definitions for UFD}
\pmrecord{4}{41952}
\pmprivacy{1}
\pmauthor{joking}{16130}
\pmtype{Theorem}
\pmcomment{trigger rebuild}
\pmclassification{msc}{13G05}
\pmrelated{UniqueFactorizationAndIdealsInRingOfIntegers}

% this is the default PlanetMath preamble.  as your knowledge
% of TeX increases, you will probably want to edit this, but
% it should be fine as is for beginners.

% almost certainly you want these
\usepackage{amssymb}
\usepackage{amsmath}
\usepackage{amsfonts}

% used for TeXing text within eps files
%\usepackage{psfrag}
% need this for including graphics (\includegraphics)
%\usepackage{graphicx}
% for neatly defining theorems and propositions
%\usepackage{amsthm}
% making logically defined graphics
%%%\usepackage{xypic}

% there are many more packages, add them here as you need them

% define commands here

\begin{document}
Let $R$ be an integral domain. Define 
$$T=\{u\in R\ |\ u\mbox{ is invertible}\}\cup\{p_1\cdots p_n\in R\ |\ p_i\mbox{ is prime}\}.$$
Of course $0\not\in T$ and $T$ is a multiplicative subset (recall that a prime element multiplied by an invertible element is again prime). Furthermore $R$ is a UFD if and only if $T=R\backslash\{0\}$ (see the parent object for more details).

\textbf{Lemma.} If $a,b\in R$ are such that $ab\in T$, then both $a,b\in T$.

\textit{Proof.} If $ab$ is invertible, then (since $R$ is commutative) both $a,b$ are invertible and thus they belong to $T$. Therefore assume that $ab$ is not invertible. Then
$$ab=p_1\cdots p_k$$
for some prime elements $p_i\in R$. We can group these prime elements in such way that $p_1\cdots p_n$ divides $a$ and $p_{n+1}\cdots p_k$ divides $b$. Thus $a=\alpha p_1\cdots p_n$ and $b=\beta p_{n+1}\cdots p_k$ for some $\alpha,\beta\in R$. Since $R$ is an integral domain we conclude that $\alpha\beta=1$, which means that both $\alpha,\beta$ are invertible in $R$. Therefore (for example) $\alpha p_1$ is prime and thus $a\in T$. Analogously $b\in T$, which completes the proof. $\square$

\textbf{Theorem. (Kaplansky)} An integral domain $R$ is a UFD if and only if every nonzero prime ideal in $R$ contains prime element.

\textit{Proof.} Without loss of generality we may assume that $R$ is not a field, because the thesis trivialy holds for fields. In this case $R$ always contains nonzero prime ideal (just take a maximal ideal).

,,$\Rightarrow$'' Let $P$ be a nonzero prime ideal. In particular $P$ is proper, thus there is nonzero $x\in P$ which is not invertible. By assumption $x\in T$ and since $x$ is not invertible, then there are prime elements $p_1,\ldots, p_k\in R$ such that $x=p_1\cdots p_k\in P$. But $P$ is prime, therefore there is $i\in\{1,\ldots, k\}$ such that $p_i\in P$, which completes this part.

,,$\Leftarrow$'' Assume that $R$ is not a UFD. Thus there is a nonzero $x\in R$ such that $x\not\in T$. Consider an ideal $(x)$. We will show, that $(x)\cap T=\emptyset$. Assume that there is $r\in R$ such that $rx\in T$. It follows that $x\in T$ (by lemma). Contradiction.

Since $(x)\cap T=\emptyset$ and $T$ is a multiplicative subset, then there is a prime ideal $P$ in $R$ such that $(x)\subseteq P$ and $P\cap T=\emptyset$ (please, see \PMlinkname{this entry}{MultiplicativeSetsInRingsAndPrimeIdeals} for more details). But we assumed that every nonzero prime ideal contains prime element (and $P$ is nonzero, since $x\in P$). Obtained contradiction completes the proof. $\square$
%%%%%
%%%%%
\end{document}
