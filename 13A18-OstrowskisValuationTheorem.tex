\documentclass[12pt]{article}
\usepackage{pmmeta}
\pmcanonicalname{OstrowskisValuationTheorem}
\pmcreated{2013-03-22 14:55:30}
\pmmodified{2013-03-22 14:55:30}
\pmowner{pahio}{2872}
\pmmodifier{pahio}{2872}
\pmtitle{Ostrowski's valuation theorem}
\pmrecord{9}{36613}
\pmprivacy{1}
\pmauthor{pahio}{2872}
\pmtype{Theorem}
\pmcomment{trigger rebuild}
\pmclassification{msc}{13A18}

% this is the default PlanetMath preamble.  as your knowledge
% of TeX increases, you will probably want to edit this, but
% it should be fine as is for beginners.

% almost certainly you want these
\usepackage{amssymb}
\usepackage{amsmath}
\usepackage{amsfonts}

% used for TeXing text within eps files
%\usepackage{psfrag}
% need this for including graphics (\includegraphics)
%\usepackage{graphicx}
% for neatly defining theorems and propositions
%\usepackage{amsthm}
% making logically defined graphics
%%%\usepackage{xypic}

% there are many more packages, add them here as you need them

% define commands here
\begin{document}
The field of rational numbers has no other \PMlinkname{non-equivalent}{EquivalentValuations} valuations than
\begin{itemize}
 \item the trivial valuation,
 \item the absolute value, i.e. the complex modulus \,$|\cdot|_\infty$\, and 
 \item the $p$-adic valuations \,$|\cdot|_p$\, when $p$ goes through all positive primes.
\end{itemize}

\textbf{Note.} \,Any valuation $|\cdot|$ of the field $\mathbb{Q}$ defines a metric \,$d(x,\,y) = |x-y|$\, in the field, but $\mathbb{Q}$ is \PMlinkname{complete}{Complete} only with respect to (the ``trivial metric'' defined by) the trivial valuation. \,The field has the proper completions with respect to its other valuations: \,the field of reals $\mathbb{R}$ and the fields $\mathbb{Q}_p$ of \PMlinkname{$p$-adic numbers}{PAdicIntegers}; cf. also \PMlinkname{$p$-adic canonical form}{PAdicCanonicalForm}.
%%%%%
%%%%%
\end{document}
