\documentclass[12pt]{article}
\usepackage{pmmeta}
\pmcanonicalname{SchurPolynomial}
\pmcreated{2013-03-22 16:56:17}
\pmmodified{2013-03-22 16:56:17}
\pmowner{mps}{409}
\pmmodifier{mps}{409}
\pmtitle{Schur polynomial}
\pmrecord{5}{39205}
\pmprivacy{1}
\pmauthor{mps}{409}
\pmtype{Definition}
\pmcomment{trigger rebuild}
\pmclassification{msc}{13B25}
\pmclassification{msc}{05E05}
\pmdefines{Schur function}

\endmetadata

% this is the default PlanetMath preamble.  as your knowledge
% of TeX increases, you will probably want to edit this, but
% it should be fine as is for beginners.

% almost certainly you want these
\usepackage{amssymb}
\usepackage{amsmath}
\usepackage{amsfonts}

% used for TeXing text within eps files
%\usepackage{psfrag}
% need this for including graphics (\includegraphics)
%\usepackage{graphicx}
% for neatly defining theorems and propositions
%\usepackage{amsthm}
% making logically defined graphics
%%%\usepackage{xypic}

% there are many more packages, add them here as you need them

% define commands here
\DeclareMathOperator{\sst}{sst}

\usepackage[all,web]{xypic}

% define commands here
\def\drawsqlat{%
\begin{xy}{
0;<1.7pc,0pc>:<0pc,1.7pc>::
\xylattice{0}{9}{0}{9}}
\end{xy}}
\def\drawsq{\ar@{-}c;c+(1,0)\ar@{-}c;c+(0,1)\ar@{-}c+(1,0);c+(1,1)\ar@{-}c+(0,1);c+(1,1)}
\def\drawsqlabel#1{\save c+(0.5,0.5)*\txt<2pc>{#1} \restore}

\newcommand{\ferrers}[9]{%
\begin{renewcommand}{\latticebody}{%
\ifnum\latticeA<#1 \ifnum\latticeB=9 \drawsq\fi\fi
\ifnum\latticeA<#2 \ifnum\latticeB=8 \drawsq\fi\fi
\ifnum\latticeA<#3 \ifnum\latticeB=7 \drawsq\fi\fi
\ifnum\latticeA<#4 \ifnum\latticeB=6 \drawsq\fi\fi
\ifnum\latticeA<#5 \ifnum\latticeB=5 \drawsq\fi\fi
\ifnum\latticeA<#6 \ifnum\latticeB=4 \drawsq\fi\fi
\ifnum\latticeA<#7 \ifnum\latticeB=3 \drawsq\fi\fi
\ifnum\latticeA<#8 \ifnum\latticeB=2 \drawsq\fi\fi
\ifnum\latticeA<#9 \ifnum\latticeB=1 \drawsq\fi\fi
}
\drawsqlat
\end{renewcommand}
}
\begin{document}
A \emph{Schur polynomial} is a special symmetric polynomial associated
to a partition of an integer, or equivalently to a Young diagram.
Schur polynomials also have a power series generalization, the
\emph{Schur functions}.

First we define some notation.  Let $\lambda$ be a partition of $n$,
and let $T$ be a filling of the Young diagram for $\lambda$.  Then by
$x^T$ we mean the monomial
\[
x^T = \prod_{i=1}^{\infty} x_i^{c_i(T)},
\]
where $c_i(T)$ is the number of times the number $i$ appears in the
filling $T$.  Since $T$ only has finitely many boxes, the product is
finite.  For example, let $\lambda = (3,3,2,2)$, and let $T$ be the
filling
\begin{center}
\begin{renewcommand}{\latticebody}{%
\ifnum\latticeA=1 \ifnum\latticeB=4 \drawsq\drawsqlabel{2} \fi\fi
\ifnum\latticeA=2 \ifnum\latticeB=4 \drawsq\drawsqlabel{4} \fi\fi
\ifnum\latticeA=3 \ifnum\latticeB=4 \drawsq\drawsqlabel{1} \fi\fi
\ifnum\latticeA=1 \ifnum\latticeB=3 \drawsq\drawsqlabel{5} \fi\fi
\ifnum\latticeA=2 \ifnum\latticeB=3 \drawsq\drawsqlabel{2} \fi\fi
\ifnum\latticeA=3 \ifnum\latticeB=3 \drawsq\drawsqlabel{3} \fi\fi
\ifnum\latticeA=1 \ifnum\latticeB=2 \drawsq\drawsqlabel{1} \fi\fi
\ifnum\latticeA=2 \ifnum\latticeB=2 \drawsq\drawsqlabel{4} \fi\fi
\ifnum\latticeA=1 \ifnum\latticeB=1 \drawsq\drawsqlabel{1} \fi\fi
\ifnum\latticeA=2 \ifnum\latticeB=1 \drawsq\drawsqlabel{2} \fi\fi
}
\drawsqlat
\end{renewcommand}
\end{center}
Notice that $1$ and $2$ each appear three times in the filling, while
$3$, $4$, and $5$ each appear only once.  Thus $x^T = x_1^3 x_2^3 x_3
x_4 x_5$.

For convenience let us use $\sst(\lambda, n)$ to denote the collection
of fillings of semi-standard tableaux with shape $\lambda$ by positive
integers from $1$ to $n$.  Then we can define the \emph{Schur
polynomial} $s_{\lambda}(x_1,\dots,x_n)$ to be the polynomial
\[
s_{\lambda}(x_1,\dots,x_n) = \sum_{T\in\sst(\lambda, n)} x^T.
\]

For example, take $n=5$ and consider the partition $\lambda = (1, 1,
1)$ of $3$.  Then the Schur polynomial $s_{\lambda}(x_1,\dots, x_5)$
is
\[
s_{(1,1,1)}(x_1,\dots,x_5) = \sum_{1\le i < j < k\le 5} x_i x_j x_k.
\]
Note that this is the elementary symmetric polynomial of degree 3 in
the variables $x_1,\dots,x_5$.  In fact, $s_{(1^k)}(x_1,\dots,x_n)$ is
always the elementary symmetric polynomial of degree $k$ in the
variables $x_1,\dots,x_n$.  The polynomial $s_{(k)}(x_1,\dots,x_n)$ is 
the complete symmetric polynomial of degree $k$ in $x_1,\dots,x_n$.

To define Schur functions, we consider the set $\sst(\lambda)$ of all
fillings of semi-standard tableaux with shape $\lambda$:
\[
\sst(\lambda) := \bigcup_{n\ge 1} \sst(\lambda, n).
\]
The \emph{Schur function} associated to the partition $\lambda$ is
\[
s_{\lambda}(\mathbf{x}) = \sum_{x\in\sst(\lambda)} x^T
\]
Thus the Schur functions are power series in infinitely many
variables.  For example,
\[
s_{(1,1,1)}(\mathbf{x}) = \sum_{1\le i < j < k} x_i x_j x_k = x_1 x_2 x_3 + \dots + x_{14} x_{42} x_{132} + \dots.
\]
All Schur polynomials and Schur functions are symmetric functions.  In
fact, the Schur polynomials of degree $n$ form a basis for the vector space of
symmetric polynomials of degree $n$.

\begin{thebibliography}{99}
\bibitem{Fu1997}
William~Fulton. \emph{Young tableaux: with applications to
representation theory and geometry}. Cambridge University Press, 1997.

\bibitem{Sa2001}
Bruce~E.~Sagan. \emph{The symmetric group: representations,
combinatorial algorithms, and symmetric functions}, 2nd ed. Springer,
2001.
\end{thebibliography}
%%%%%
%%%%%
\end{document}
