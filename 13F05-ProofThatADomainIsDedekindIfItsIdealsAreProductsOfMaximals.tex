\documentclass[12pt]{article}
\usepackage{pmmeta}
\pmcanonicalname{ProofThatADomainIsDedekindIfItsIdealsAreProductsOfMaximals}
\pmcreated{2013-03-22 18:35:04}
\pmmodified{2013-03-22 18:35:04}
\pmowner{gel}{22282}
\pmmodifier{gel}{22282}
\pmtitle{proof that a domain is Dedekind if its ideals are products of maximals}
\pmrecord{5}{41310}
\pmprivacy{1}
\pmauthor{gel}{22282}
\pmtype{Proof}
\pmcomment{trigger rebuild}
\pmclassification{msc}{13F05}
\pmclassification{msc}{13A15}
%\pmkeywords{Dedekind domain}
%\pmkeywords{maximal ideal}
%\pmkeywords{invertible ideal}
\pmrelated{DedekindDomain}
\pmrelated{MaximalIdeal}
\pmrelated{FractionalIdeal}

\endmetadata

% this is the default PlanetMath preamble.  as your knowledge
% of TeX increases, you will probably want to edit this, but
% it should be fine as is for beginners.

% almost certainly you want these
\usepackage{amssymb}
\usepackage{amsmath}
\usepackage{amsfonts}

% used for TeXing text within eps files
%\usepackage{psfrag}
% need this for including graphics (\includegraphics)
%\usepackage{graphicx}
% for neatly defining theorems and propositions
\usepackage{amsthm}
% making logically defined graphics
%%%\usepackage{xypic}

% there are many more packages, add them here as you need them

% define commands here
\newtheorem*{theorem*}{Theorem}
\newtheorem*{lemma*}{Lemma}
\newtheorem*{corollary*}{Corollary}
\newtheorem{theorem}{Theorem}
\newtheorem{lemma}{Lemma}
\newtheorem{corollary}{Corollary}


\begin{document}
\PMlinkescapeword{Noetherian}
\PMlinkescapeword{invertible}
\PMlinkescapeword{maximal}
\PMlinkescapeword{maximals}
Let $R$ be an integral domain. We show that it is a Dedekind domain if and only if every nonzero proper ideal can be expressed as a product of maximal ideals.
To do this, we make use of the characterization of Dedekind domains as integral domains in which every nonzero integral ideal is invertible (proof that a domain is Dedekind if its ideals are invertible).

First, let us suppose that every nonzero proper ideal in $R$ is a product of maximal ideals.
Let $\mathfrak{m}$ be a maximal ideal and choose a nonzero $x\in\mathfrak{m}$. Then, by assumption,
\begin{equation*}
(x)=\mathfrak{m}_1\cdots\mathfrak{m_n}
\end{equation*}
for some $n\ge 0$ and maximal ideals $\mathfrak{m}_k$. As $(x)$ is a principal ideal, each of the factors $\mathfrak{m}_k$ is invertible.
Also,
\begin{equation*}
\mathfrak{m}_1\cdots\mathfrak{m_n}\subseteq\mathfrak{m}.
\end{equation*}
As $\mathfrak{m}$ is prime, this gives $\mathfrak{m}_k\subseteq\mathfrak{m}$ for some $k$. However, $\mathfrak{m}_k$ is maximal so must equal $\mathfrak{m}$, showing that $\mathfrak{m}$ is indeed invertible.
Then, every nonzero proper ideal is a product of maximal, and hence invertible, ideals and so is invertible, and it follows that $R$ is Dedekind.


We now show the reverse direction, so suppose that $R$ is Dedekind.
Proof by contradiction will be used to show that every nonzero ideal is a product of maximals, so suppose that this is not the case.
Then, as $R$ is defined to be \PMlinkname{Noetherian}{Noetherian}, there is an ideal $\mathfrak{a}$ \PMlinkname{maximal}{MaximalElement} (w.r.t. the partial order of set inclusion) among those proper ideals which are not a product of maximal ideals.
Then $\mathfrak{a}$ cannot be a maximal ideal itself, so is strictly contained in a maximal ideal $\mathfrak{m}$ and, as $\mathfrak{m}$ is invertible, we can write $\mathfrak{a}=\mathfrak{mb}$ for an ideal $\mathfrak{b}$.

Therefore $\mathfrak{a}\subseteq\mathfrak{b}$ and we cannot have equality, otherwise cancelling $\mathfrak{a}$ from $\mathfrak{a}=\mathfrak{ma}$ would give $\mathfrak{m}=R$. So, $\mathfrak{b}$ is strictly larger than $\mathfrak{a}$ and, by the choice of $\mathfrak{a}$, is therefore a product of maximal ideals. Finally, $\mathfrak{a}=\mathfrak{mb}$ is then also a product of maximal ideals.

%%%%%
%%%%%
\end{document}
