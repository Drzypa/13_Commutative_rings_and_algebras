\documentclass[12pt]{article}
\usepackage{pmmeta}
\pmcanonicalname{ModulesOverAlgebarsAndHomomorphismsBetweenThem}
\pmcreated{2013-03-22 19:16:32}
\pmmodified{2013-03-22 19:16:32}
\pmowner{joking}{16130}
\pmmodifier{joking}{16130}
\pmtitle{modules over algebars and homomorphisms between them}
\pmrecord{5}{42207}
\pmprivacy{1}
\pmauthor{joking}{16130}
\pmtype{Definition}
\pmcomment{trigger rebuild}
\pmclassification{msc}{13B99}
\pmclassification{msc}{20C99}
\pmclassification{msc}{16S99}

\endmetadata

% this is the default PlanetMath preamble.  as your knowledge
% of TeX increases, you will probably want to edit this, but
% it should be fine as is for beginners.

% almost certainly you want these
\usepackage{amssymb}
\usepackage{amsmath}
\usepackage{amsfonts}

% used for TeXing text within eps files
%\usepackage{psfrag}
% need this for including graphics (\includegraphics)
%\usepackage{graphicx}
% for neatly defining theorems and propositions
%\usepackage{amsthm}
% making logically defined graphics
%%%\usepackage{xypic}

% there are many more packages, add them here as you need them

% define commands here

\begin{document}
Let $R$ be a ring and let $A$ be an associative algebra (not necessarily unital).

\textbf{Definition.} A (left) $A$-module over $R$ is a pair $(M,\circ)$ where $M$ is a (left) $R$-module and
$$\circ:A\times M\to M$$
is a $R$-bilinear map such that the following conditions hold:
\begin{enumerate}
\item $(a\circ b)\circ x=a\circ(b\circ x)$
\item $r(a\circ x)=(ra)\circ x=a\circ(rx)$
\end{enumerate}
for any $a,b\in A$, $x\in M$ and $r\in R$. We will simply use capital letters to denote modules.

Let $M$ be an $A$-module over $R$. If $M'\subseteq M$ and $A'\subseteq A$ then by $A'M'$ we denote $R$-submodule of $M$ generated by elements of the form $am$ for $a\in A'$ and $m\in M'$. We will call $M$ \textbf{unitary} if $AM=M$. Note, that if $A$ has multiplicative identity $1$, then $M$ is unitary if and only if $1m=m$ for any $m\in M$.

The reason we use name ,,$A$-module over $R$'' instead of ,,$A$-module'' is that these to concepts may differ. The latter means that we treat $A$ simply as a ring and take modules over it. But such module need not be equiped with a ,,good'' $R$-module structure. On the other hand this is always the case, when $M$ is unitary over unital algebra.

If $M$ and $N$ are two $A$-modules over $R$, then a function $f:M\to N$ is called an $A$-homomorphism iff $f$ is an $R$-homomorphism and additionaly $f(am)=af(m)$ for any $a\in A$ and $m\in M$.

It can be easily checked that $A$-modules over $R$ together with $A$-homomorphisms form a category which is abelian. Furthermore, if $A$ is unital, then its full subcategory consisting unitary $R$-modules over $A$ is equivalent to category of unitary $A$-modules.

In most cases it is important to assume that the base ring $R$ is a field, even algebraically closed.
%%%%%
%%%%%
\end{document}
