\documentclass[12pt]{article}
\usepackage{pmmeta}
\pmcanonicalname{ProofOfFiniteInseparableExtensionsOfDedekindDomainsAreDedekind}
\pmcreated{2013-03-22 18:35:42}
\pmmodified{2013-03-22 18:35:42}
\pmowner{gel}{22282}
\pmmodifier{gel}{22282}
\pmtitle{proof of finite inseparable extensions of Dedekind domains are Dedekind}
\pmrecord{5}{41324}
\pmprivacy{1}
\pmauthor{gel}{22282}
\pmtype{Proof}
\pmcomment{trigger rebuild}
\pmclassification{msc}{13A15}
\pmclassification{msc}{13F05}
%\pmkeywords{Dedekind domain}
%\pmkeywords{finite extension}
%\pmkeywords{purely inseparable}
%\pmkeywords{fractional ideal}

% this is the default PlanetMath preamble.  as your knowledge
% of TeX increases, you will probably want to edit this, but
% it should be fine as is for beginners.

% almost certainly you want these
\usepackage{amssymb}
\usepackage{amsmath}
\usepackage{amsfonts}

% used for TeXing text within eps files
%\usepackage{psfrag}
% need this for including graphics (\includegraphics)
%\usepackage{graphicx}
% for neatly defining theorems and propositions
\usepackage{amsthm}
% making logically defined graphics
%%%\usepackage{xypic}

% there are many more packages, add them here as you need them

% define commands here
\newtheorem*{theorem*}{Theorem}
\newtheorem*{lemma*}{Lemma}
\newtheorem*{corollary*}{Corollary}
\newtheorem{theorem}{Theorem}
\newtheorem{lemma}{Lemma}
\newtheorem{corollary}{Corollary}


\begin{document}
\PMlinkescapeword{finite}
\PMlinkescapeword{Noetherian}
\PMlinkescapeword{invertible}
\PMlinkescapeword{inverse}
Let $R$ be a Dedekind domain with field of fractions $K$ and $L/K$ be a field extension. We suppose that $K$ has \PMlinkname{characteristic}{characteristic} $p>0$ and that there is a $q=p^r$ such that $x^q\in K$ for all $x\in L$. In particular, this is satisfied if it is a purely inseparable and finite extension.

We show that the integral closure $A$ of $R$ in $L$ is a Dedekind domain.

We cannot apply the same method of proof as for the proof of finite separable extensions of Dedekind domains are Dedekind, because here $A$ does not have to be finitely generated as an $R$-module.

We use the characterization of Dedekind domains as integral domains in which all nonzero ideals are invertible (see proof that a domain is Dedekind if its ideals are invertible).
Note that for any $x\in A$, $x^q$ is in $K$ and is integral over $R$ so, by integral closure, $x^q\in R$.

So, let $\mathfrak{a}$ be a nonzero ideal in $A$, and let $\mathfrak{b}$ be the ideal of $R$ generated by terms of the form $a^q$ for $a\in\mathfrak{a}$,
\begin{equation*}
\mathfrak{b}=\left(a^q:a\in \mathfrak{a}\right)_R.
\end{equation*}
Then, as $R$ is a Dedekind domain, there is a fractional ideal $\mathfrak{b}^{-1}$ of $R$ such that $\mathfrak{b}\mathfrak{b}^{-1}=R$, and write $\mathfrak{b}^{-1}_A$ for the fractional ideal of $A$ generated by $\mathfrak{b}^{-1}$. Then,
\begin{equation}\label{eq:1}
1\in R=\mathfrak{b}\mathfrak{b}^{-1}\subseteq\mathfrak{a}^q\mathfrak{b}^{-1}_A.
\end{equation}
On the other hand, if $a_1,\ldots,a_q\in\mathfrak{a}$ and $b\in\mathfrak{b}^{-1}$ then
\begin{equation*}
(a_1\cdots a_q b)^q=(a_1^qb)\cdots(a_q^qb)\in R,
\end{equation*}
so $a_1\cdots a_q b$ is integral over $R$ and is in $A$. Therefore, $\mathfrak{a}^q\mathfrak{b}^{-1}_A\subseteq A$. Combining with (\ref{eq:1}) gives $\mathfrak{a}^q\mathfrak{b}^{-1}_A=A$, so $\mathfrak{a}$ is invertible with inverse $\mathfrak{a}^{q-1}\mathfrak{b}^{-1}_A$.
%%%%%
%%%%%
\end{document}
