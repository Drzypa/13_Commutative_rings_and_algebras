\documentclass[12pt]{article}
\usepackage{pmmeta}
\pmcanonicalname{EisensteinCriterionInTermsOfDivisorTheory}
\pmcreated{2013-03-22 18:00:45}
\pmmodified{2013-03-22 18:00:45}
\pmowner{pahio}{2872}
\pmmodifier{pahio}{2872}
\pmtitle{Eisenstein criterion in terms of divisor theory}
\pmrecord{6}{40527}
\pmprivacy{1}
\pmauthor{pahio}{2872}
\pmtype{Theorem}
\pmcomment{trigger rebuild}
\pmclassification{msc}{13A05}
\pmrelated{DivisorTheory}

% this is the default PlanetMath preamble.  as your knowledge
% of TeX increases, you will probably want to edit this, but
% it should be fine as is for beginners.

% almost certainly you want these
\usepackage{amssymb}
\usepackage{amsmath}
\usepackage{amsfonts}

% used for TeXing text within eps files
%\usepackage{psfrag}
% need this for including graphics (\includegraphics)
%\usepackage{graphicx}
% for neatly defining theorems and propositions
 \usepackage{amsthm}
% making logically defined graphics
%%%\usepackage{xypic}

% there are many more packages, add them here as you need them

% define commands here

\theoremstyle{definition}
\newtheorem*{thmplain}{Theorem}

\begin{document}
\PMlinkescapeword{index} \PMlinkescapeword{divide}
The below theorem generalises Eisenstein criterion of irreducibility from UFD's to domains with divisor theory.\\

\newtheorem*{thm}{Theorem}
\begin{thm}
Let\; $f(x) := a_0\!+\!a_1x\!+\ldots+\!a_nx^n$\; be a primitive polynomial over an integral domain $\mathcal{O}$ with \PMlinkname{divisor theory}{DivisorTheory} \,$\mathcal{O}^* \to \mathfrak{D}$.\, If there is a prime divisor\, $\mathfrak{p \in D}$\, such that
\begin{itemize}
\item $\mathfrak{p} \mid a_0,\,a_1,\,\ldots,\,a_{n-1},$
\item $\mathfrak{p} \nmid a_n,$
\item $\mathfrak{p}^2 \nmid a_0,$
\end{itemize}
then the polynomial is irreducible.\\
\end{thm}

{\em Proof.}\,
Suppose that we have in $\mathcal{O}[x]$ the factorisation
$$f(x) = (b_0+b_1x+\ldots+b_sx^s)(c_0+c_1x+\ldots+c_tx^t)$$
with\, $s > 0$\, and\, $t > 0$.\, Because the principal divisor $(a_0)$, i.e. $(b_0)(c_0)$ is divisible by the prime divisor $\mathfrak{p}$ and there is a unique factorisation in the monoid $\mathfrak{D}$, $\mathfrak{p}$ must divide $(b_0)$ or $(c_0)$ but, by $\mathfrak{p}^2 \nmid (a_0)$, not both of $(b_0)$ and $(c_0)$; suppose e.g. that 
$\mathfrak{p} \mid c_0$.\, If $\mathfrak{p}$ would divide all the coefficients $c_j$, then it would divide also the product \,$b_sc_t = a_n$.\, So, there is a certain smallest index $k$ such that\, $p \nmid c_k$.\, Accordingly, in the sum $b_0c_k+b_1c_{k-1}+\ldots+b_kc_0$, the prime divisor $\mathfrak{p}$ \PMlinkname{divides}{DivisibilityInRings} every summand except the first (see the definition of \PMlinkname{divisor theory}{DivisorTheory}); therefore it cannot divide the sum.\, But the value of the sum is $a_k$ which by hypothesis is divisible by the prime divisor.\, This contradiction shows that the polynomial $f(x)$ is irreducible.
%%%%%
%%%%%
\end{document}
