\documentclass[12pt]{article}
\usepackage{pmmeta}
\pmcanonicalname{OrdersOfElementsInIntegralDomain}
\pmcreated{2013-03-22 15:40:28}
\pmmodified{2013-03-22 15:40:28}
\pmowner{pahio}{2872}
\pmmodifier{pahio}{2872}
\pmtitle{orders of elements in integral domain}
\pmrecord{9}{37615}
\pmprivacy{1}
\pmauthor{pahio}{2872}
\pmtype{Theorem}
\pmcomment{trigger rebuild}
\pmclassification{msc}{13G05}
\pmrelated{OrderGroup}
\pmrelated{IdealOfElementsWithFiniteOrder}

% this is the default PlanetMath preamble.  as your knowledge
% of TeX increases, you will probably want to edit this, but
% it should be fine as is for beginners.

% almost certainly you want these
\usepackage{amssymb}
\usepackage{amsmath}
\usepackage{amsfonts}

% used for TeXing text within eps files
%\usepackage{psfrag}
% need this for including graphics (\includegraphics)
%\usepackage{graphicx}
% for neatly defining theorems and propositions
 \usepackage{amsthm}
% making logically defined graphics
%%%\usepackage{xypic}

% there are many more packages, add them here as you need them

% define commands here

\theoremstyle{definition}
\newtheorem*{thmplain}{Theorem}
\begin{document}
\begin{thmplain}
Let\, $(D,\,+,\,\cdot)$\, be an integral domain, i.e. a commutative ring with non-zero unity 1 and no zero divisors.\, All non-zero elements of $D$ have the same \PMlinkname{order}{OrderGroup} in the additive group\, $(D,\,+)$.
\end{thmplain}

{\em Proof.}\, Let $a$ be arbitrary non-zero element.\, Any \PMlinkname{multiple}{GeneralAssociativity} $na$ may be  written as
  $$na = n(1a) = \underbrace{1a+1a+\cdots+1a}_{n} = (\underbrace{1+1+\cdots+1}_{n})a = (n1)a.$$
Thus, because\, $a \ne 0$\, and there are no zero divisors, an equation\, $na = 0$\, is \PMlinkname{equivalent}{Equivalent3} with the equation\, $n1 = 0$.\, So $a$ must have the same \PMlinkescapetext{order} as the unity of $D$.

\textbf{Note.}\, The \PMlinkescapetext{order} of the unity element is the \PMlinkname{characteristic}{Characteristic} of the integral domain, which is 0 or a positive prime number.
%%%%%
%%%%%
\end{document}
