\documentclass[12pt]{article}
\usepackage{pmmeta}
\pmcanonicalname{FormalPowerSeries}
\pmcreated{2013-03-22 12:49:30}
\pmmodified{2013-03-22 12:49:30}
\pmowner{AxelBoldt}{56}
\pmmodifier{AxelBoldt}{56}
\pmtitle{formal power series}
\pmrecord{14}{33148}
\pmprivacy{1}
\pmauthor{AxelBoldt}{56}
\pmtype{Topic}
\pmcomment{trigger rebuild}
\pmclassification{msc}{13H05}
\pmclassification{msc}{13B35}
\pmclassification{msc}{13J05}
\pmclassification{msc}{13F25}
\pmrelated{PowerSeries}
\pmrelated{SumOfKthPowersOfTheFirstNPositiveIntegers}
\pmrelated{PolynomialRingOverIntegralDomain}
\pmrelated{FiniteRingHasNoProperOverrings}
\pmdefines{formal power series}
\pmdefines{generating function}
\pmdefines{formal Laurent series}
\pmdefines{power series field}

% this is the default PlanetMath preamble.  as your knowledge
% of TeX increases, you will probably want to edit this, but
% it should be fine as is for beginners.

% almost certainly you want these
\usepackage{amssymb}
\usepackage{amsmath}
\usepackage{amsfonts}
\usepackage{amsthm}
\usepackage{mathrsfs}

% used for TeXing text within eps files
%\usepackage{psfrag}
% need this for including graphics (\includegraphics)
%\usepackage{graphicx}
% for neatly defining theorems and propositions
%\usepackage{amsthm}
% making logically defined graphics
%%%\usepackage{xypic}

% there are many more packages, add them here as you need them

% define commands here
\newtheorem{theorem}{Theorem}
\newtheorem{defn}{Definition}
\newtheorem{prop}{Proposition}
\newtheorem{lemma}{Lemma}
\newtheorem{cor}{Corollary}

\begin{document}
\PMlinkescapeword{natural}
\PMlinkescapeword{algebraic}
\PMlinkescapeword{order}
\PMlinkescapeword{closed}
\PMlinkescapeword{formulas}
\PMlinkescapeword{formula}
\PMlinkescapeword{information}
\PMlinkescapephrase{Laurent series}

\emph{Formal power series} allow one to employ much of the analytical
machinery of power series in settings which don't have natural notions
of convergence. They are also useful in order to compactly describe
sequences and to find closed formulas
for recursively described sequences; this is known as the method of
generating functions and will be illustrated below.

We start with a commutative ring $R$. We want to define the ring of
formal power series over $R$ in the variable $X$, denoted by $R[[X]]$;
each element of this ring can be written in a unique way as an
infinite sum of the form $\sum_{n=0}^\infty a_n X^n$, where the
coefficients $a_n$ are elements of $R$; any choice of coefficients
$a_n$ is allowed. $R[[X]]$ is actually a topological ring so that
these infinite sums are well-defined and convergent. The addition and
multiplication  of such sums follows the usual laws of power series.

\paragraph{Formal construction}

Start with the set $R^\Bbb{N}$ of all infinite sequences in $R$.
Define addition of two such sequences by 
$$(a_n) + (b_n) = (a_n + b_n)$$
and multiplication by
$$(a_n) (b_n) = (\sum_{k=0}^n a_kb_{n-k}).$$
This turns $R^\Bbb{N}$ into a commutative ring with multiplicative
identity (1,0,0,\ldots). We identify the element $a$ of $R$ with the
sequence ($a$,0,0,\ldots) and define $X:=(0,1,0,0,\ldots)$. Then every element
of $R^\Bbb{N}$ of the form $(a_0,a_1,a_2,\ldots,a_N,0,0,\ldots)$ can be written as the \emph{finite} sum
$$\sum_{n=0}^Na_n X^n.$$
In order to extend this equation to infinite series, we need a metric
on $R^\Bbb{N}$. We define $d((a_n),(b_n))=2^{-k}$, where $k$ is the
smallest natural number such that $a_k\not=b_k$ (if there is not such
$k$, then the two sequences are equal and we define their distance to
be zero). This is a metric which turns $R^\Bbb{N}$ into a topological
ring, and the equation
$$(a_n) = \sum_{n=0}^\infty a_n X^n$$
can now be rigorously proven using the notion of convergence arising
from $d$; in fact, any rearrangement of the series converges to the
same limit.

This topological ring is the ring of formal power series over $R$ and
is denoted by $R[[X]]$.

\paragraph{Properties}

$R[[X]]$ is an associative algebra over $R$ which contains the ring
$R[X]$ of polynomials over $R$; the polynomials correspond to the
sequences which end in zeros.

The geometric series formula is valid in $R[[X]]$:
$$(1-X)^{-1}=\sum_{n=0}^\infty X^n$$
An element $\sum a_n X^n$ of $R[[X]]$ is invertible in $R[[X]]$ if and only if its
constant coefficient $a_0$ is invertible in $R$ (see invertible formal power series).\, This implies that the
Jacobson radical of $R[[X]]$ is the ideal generated by $X$ and the
Jacobson radical of $R$.

Several algebraic properties of $R$ are inherited by $R[[X]]$:
\begin{itemize}
\item if $R$ is a local ring, then so is $R[[X]]$
\item if $R$ is Noetherian, then so is $R[[X]]$
\item if $R$ is an integral domain, then so is $R[[X]]$
\item if $R$ is a field, then $R[[X]]$ is a discrete valuation ring.
\end{itemize}

The metric space $(R[[X]],d)$ is complete. The topology on $R[[X]]$ is
equal to the product topology on $R^\Bbb{N}$ where $R$ is equipped
with the discrete topology. It follows from  Tychonoff's theorem that
$R[[X]]$ is compact if and only if $R$ is finite. The topology on
$R[[X]]$ can also be seen as the $I$-adic topology, where $I=(X)$ is
the ideal generated by $X$ (whose elements are precisely the formal
power series with zero constant coefficient).

If $R=K$ is a field, we can consider the quotient field of the
integral domain $K[[X]]$; it is denoted by $K((X))$ and called a (\emph{formal}) \emph{power series field}. It is a
topological field whose elements are called \emph{formal Laurent
series}; they can be uniquely written in the form
$$f = \sum_{n=-M}^\infty a_n X^n$$
where $M$ is an integer which depends on the Laurent series $f$. 

\paragraph{Formal power series as functions}

In analysis, every convergent power series defines a function with
values in the real or complex numbers. Formal power series can also be
interpreted as functions, but one has to be careful with the domain
and codomain. If $f=\sum a_n X^n$ is an element of $R[[X]]$, if $S$ is a
commutative associative algebra over $R$, if $I$ an ideal in $S$ such
that the $I$-adic topology on $S$ is complete, and if $x$ is an element
of $I$, then we can define
$$f(x) := \sum_{n=0}^\infty a_n x^n.$$
This latter series is guaranteed to converge in $S$ given the above
assumptions. Furthermore, we have
$$(f+g)(x) = f(x) + g(x)$$
and
$$(fg)(x) = f(x) g(x)$$
(unlike in the case of bona fide functions, these formulas are not
definitions but have to proved).

Since the topology on $R[[X]]$ is the $(X)$-adic topology and $R[[X]]$
is complete, we can in particular apply power series to other power
series, provided that the arguments don't have constant coefficients:
$f(0)$, $f(X^2-X)$ and $f((1-X)^{-1}-1)$ are all well-defined for any
formal power series $f\in R[[X]]$.

With this formalism, we can give an explicit formula for the
multiplicative inverse of a power series $f$ whose constant
coefficient $a=f(0)$ is invertible in $R$:
$$f^{-1} = \sum_{n=0}^\infty a^{-n-1} (a-f)^n$$

\paragraph{Differentiating formal power series}

If $f=\sum_{n=0}^\infty a_n X^n\in R[[X]]$, we define the formal derivative of $f$ as
$$\operatorname{D}f = \sum_{n=1}^{\infty} a_n n X^{n-1}.$$
This operation is $R$-linear, obeys the product rule
$$\operatorname{D}(f\cdot g) = (\operatorname{D}f)\cdot g + f\cdot (\operatorname{D} g)$$
and the chain rule:
$$\operatorname{D}(f(g)) = (\operatorname{D}f)(g)\cdot \operatorname{D}g$$
(in case g(0)=0).

In a sense, all formal power series are Taylor series, because if
$f=\sum a_n X^n$, then 
$$(\operatorname{D}^kf)(0) = k!\; a_k$$
(here $k!$ denotes the element $1\times (1+1)\times(1+1+1)\times\ldots\in R$.

One can also define differentiation for formal Laurent series in a
natural way, and then the quotient rule, in addition to the rules
listed above, will also be valid.

\paragraph{Power series in several variables}

The fastest way to define the ring $R[[X_1,\ldots,X_r]]$ of formal power
series over $R$ in $r$ variables starts with the ring $S =
R[X_1,\ldots,X_r]$ of polynomials over $R$. Let $I$ be the ideal in $S$
generated by $X_1,\ldots,X_r$, consider the $I$-adic topology on $S$, and
form its completion. This results in a complete topological ring
containing $S$ which is denoted by $R[[X_1,\ldots,X_r]]$. 

For $\mathbf{n}=(n_1,\ldots,n_r)\in\Bbb{N}^r$, we write
$\mathbf{X}^\mathbf{n} = X_1^{n_1}\cdots X_r^{n_r}$. Then every element of
$R[[X_1,\ldots,X_r]]$ can be written in a unique was as a sum
$$\sum_{\mathbf{n}\in\Bbb{N}^r} a_{\mathbf{n}} \mathbf{X}^\mathbf{n}$$
where the sum extends over all $\mathbf{n}\in\Bbb{N}^r$.  These sums converge
for any choice of the coefficients $ a_{\mathbf{n}}\in R$ and the order
in which the summation is carried out does not matter.

If $J$ is the ideal in $R[[X_1,\ldots,X_r]]$ generated by $X_1,\ldots,X_r$
(i.e.\ $J$ consists of those power series with zero constant
coefficient), then the topology on $R[[X_1,\ldots,X_r]]$ is the $J$-adic
topology.

Since $R[[X_1]]$ is a commutative ring, we can define its power
series ring, say $R[[X_1]][[X_2]]$. This ring is naturally isomorphic to
the ring $R[[X_1,X_2]]$ just defined, but as topological rings the two
are different.

If $K=R$ is a field, then $K[[X_1,\ldots,X_r]]$ is a unique factorization
domain.

Similar to the situation described above, we can ``apply'' power
series in several variables to other power series with zero constant
coefficients. It is also possible to define partial derivatives for
formal power series in a straightforward way. Partial derivatives
commute, as they do for continuously differentiable functions.

\paragraph{Uses}

One can use formal power series to prove several relations familar
from analysis in a purely algebraic setting. Consider for instance the
following elements of $\Bbb{Q}[[X]]$:
$$\operatorname{sin}(X) := \sum_{n=0}^\infty \frac{(-1)^n}{(2n+1)!} X^{2n+1}$$
$$\operatorname{cos}(X) := \sum_{n=0}^\infty \frac{(-1)^n}{(2n)!} X^{2n}$$
Then one can easily show that
$$ \operatorname{sin}^2(X) + \operatorname{cos}^2(X) = 1$$
and
$$ D \operatorname{sin} = \operatorname{cos}$$
as well as
$$ \operatorname{sin}(X+Y) =
\operatorname{sin}(X)\operatorname{cos}(Y) +
\operatorname{cos}(X)\operatorname{sin}(Y)
$$
(the latter being valid in the ring $\Bbb{Q}[[X,Y]]$).

As an example of the method of generating functions, consider the
problem of finding a closed formula for the Fibonacci numbers $f_n$
defined by $f_{n+2}=f_{n+1}+f_n$, $f_0=0$, and $f_1=1$. We work in the
ring $\Bbb{R}[[X]]$ and define the power series
$$ f=\sum_{n=0}^\infty f_n X^n;$$
$f$ is called the \emph{generating function} for the sequence $(f_n)$.
The generating function for the sequence $(f_{n-1})$ is $Xf$ while
that for $(f_{n-2})$ is $X^2f$. From the recurrence relation, we
therefore see that the power series $Xf + X^2f$ agrees with $f$ except
for the first two coefficients. Taking these into account, we find
that
$$f=Xf+X^2f+X$$
(this is the crucial step; recurrence relations can almost always be
translated into equations for the generating functions). Solving this
equation for $f$, we get
$$f=\frac{X}{1-X-X^2}.$$
Using the golden ratio $\phi_1=(1+\sqrt{5})/2$ and
$\phi_2=(1-\sqrt{5})/2$, we can write the latter expression as
$$\frac{1}{\sqrt{5}}\left(\frac{1}{1-\phi_1X}-\frac{1}{1-\phi_2X}\right).$$
These two power series are known explicitly because they are geometric
series; comparing coefficients, we find the explicit formula
$$f_n = \frac{1}{\sqrt{5}}\left(\phi_1^n-\phi_2^n\right).$$

In algebra, the ring $K[[X_1,\ldots,X_r]]$ (where $K$ is a field) is often
used as the ``standard, most general'' complete local ring over $K$.

\paragraph{Universal property}

The power series ring $R[[X_1,\ldots,X_r]]$ can be characterized by the
following universal property: if $S$ is a commutative associative
algebra over $R$, if $I$ is an ideal in $S$ such that the $I$-adic
topology on $S$ is complete, and if $x_1,\ldots,x_r\in I$ are given, then
there exists a \emph{unique} $\Phi : R[[X_1,\ldots,X_r]] \to S$ with the
following properties:
\begin{itemize}
\item $\Phi$ is an $R$-algebra homomorphism
\item $\Phi$ is \PMlinkname{continuous}{Continuous}
\item $\Phi(X_i)=x_i$ for $i=1,\ldots,r$.
\end{itemize}
%%%%%
%%%%%
\end{document}
