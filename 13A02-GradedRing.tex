\documentclass[12pt]{article}
\usepackage{pmmeta}
\pmcanonicalname{GradedRing}
\pmcreated{2013-03-22 11:45:03}
\pmmodified{2013-03-22 11:45:03}
\pmowner{aplant}{12431}
\pmmodifier{aplant}{12431}
\pmtitle{graded ring}
\pmrecord{19}{30192}
\pmprivacy{1}
\pmauthor{aplant}{12431}
\pmtype{Definition}
\pmcomment{trigger rebuild}
\pmclassification{msc}{13A02}
\pmclassification{msc}{16W30}
\pmclassification{msc}{14L15}
\pmclassification{msc}{14L05}
\pmclassification{msc}{12F10}
\pmclassification{msc}{11S31}
\pmclassification{msc}{11S15}
\pmclassification{msc}{11R33}
\pmsynonym{S-graded ring}{GradedRing}
\pmsynonym{G-graded ring}{GradedRing}
%\pmkeywords{algebra ring groupoid homogeneous}
\pmrelated{HomogeneousIdeal}
\pmrelated{SupportGradedRing}
\pmdefines{groupoid graded ring}
\pmdefines{semigroup graded ring}
\pmdefines{group graded ring}
\pmdefines{homogeneous element}
\pmdefines{strongly graded}

\endmetadata

\usepackage{amssymb}
\usepackage{amsmath}
\usepackage{amsfonts}
\usepackage{graphicx}
%%%%\usepackage{xypic}

\newcommand{\supp}{\,{\rm supp}\,}
\begin{document}
Let $S$ be a groupoid (semigroup,group) and let $R$ be a ring (not necessarily with unity) which can be expressed as a \PMlinkescapetext{direct sum} $R = {\bigoplus}_{s \in S} R_{s}$ of additive subgroups $R_{s}$ of $R$ with $s \in S$.  If $R_{s} R_{t} \subseteq R_{st}$ for all $s,t \in S$ then we say that $R$ is {\em groupoid graded} (semigroup-graded, group-graded) ring.

We refer to $R = \bigoplus_{s\in S} R_{s}$ as an $S$-grading of
$R$ and the subgroups $R_{s}$ as the
$s$-components of $R$. If we have the stronger
condition that $R_{s}R_{t} = R_{st}$ for all $s,t \in S$, then we say that the ring $R$ is {\em
strongly} graded by
$S$. 

Any element $r_{s}$ in $R_{s}$ (where $s\in S$) is said to be {\em homogeneous of degree
$s$}. Each element $r \in R$ can be expressed as a unique and finite sum $r =
\sum_{s \in S} r_{s}$ of homogeneous elements $r_{s} \in R_{s}$. 

%%We define the {\em
%%support} of $r$ to be the set $\supp(r) = \{ s \in S \st
%%r_{s} \neq 0 \}$. We can extend this definition to 
%%$\supp(R) = \bigcup \supp(r) = \{ s \in S \st
%%R_{s} \neq 0 \}$.  If $\supp(R)$ is a finite set then we say that the ring 
%%$R$ has {\em finite
%%support}.

For any subset $G \subseteq S$ we have $R_{G} = \sum_{g \in G} R_{g}$.
Similarly $r_{G} = \sum_{g \in G} r_{g}$.  If $G$ is a subsemigroup of $S$ then
$R_{G}$ is a subring of $R$.  If $G$ is a left (right, two-sided) ideal of $S$
then $R_{G}$ is a left (right, two-sided) ideal of $R$.

Some examples of graded rings include:\\
Polynomial rings\\
Ring of symmetric functions\\
Generalised matrix rings\\
Morita contexts\\
Ring of Hirota derivatives\\
group rings\\
filtered algebras\\


%%%%%
%%%%%
%%%%%
%%%%%
\end{document}
