\documentclass[12pt]{article}
\usepackage{pmmeta}
\pmcanonicalname{BoundOnTheKrullDimensionOfPolynomialRings}
\pmcreated{2013-03-22 15:22:11}
\pmmodified{2013-03-22 15:22:11}
\pmowner{mathcam}{2727}
\pmmodifier{mathcam}{2727}
\pmtitle{bound on the Krull dimension of polynomial rings}
\pmrecord{9}{37196}
\pmprivacy{1}
\pmauthor{mathcam}{2727}
\pmtype{Theorem}
\pmcomment{trigger rebuild}
\pmclassification{msc}{13C15}

% this is the default PlanetMath preamble.  as your knowledge
% of TeX increases, you will probably want to edit this, but
% it should be fine as is for beginners.

% almost certainly you want these
\usepackage{amssymb}
\usepackage{amsmath}
\usepackage{amsfonts}
\usepackage{amsthm}

% used for TeXing text within eps files
%\usepackage{psfrag}
% need this for including graphics (\includegraphics)
%\usepackage{graphicx}
% for neatly defining theorems and propositions
%\usepackage{amsthm}
% making logically defined graphics
%%%\usepackage{xypic}

% there are many more packages, add them here as you need them

% define commands here

\newcommand{\mc}{\mathcal}
\newcommand{\mb}{\mathbb}
\newcommand{\mf}{\mathfrak}
\newcommand{\ol}{\overline}
\newcommand{\ra}{\rightarrow}
\newcommand{\la}{\leftarrow}
\newcommand{\La}{\Leftarrow}
\newcommand{\Ra}{\Rightarrow}
\newcommand{\nor}{\vartriangleleft}
\newcommand{\Gal}{\text{Gal}}
\newcommand{\GL}{\text{GL}}
\newcommand{\Z}{\mb{Z}}
\newcommand{\R}{\mb{R}}
\newcommand{\Q}{\mb{Q}}
\newcommand{\C}{\mb{C}}
\newcommand{\<}{\langle}
\renewcommand{\>}{\rangle}
\begin{document}
If $A$ is a commutative ring, and $\operatorname{dim}$ denotes Krull dimension, then
\begin{align*}
\operatorname{dim}(A)+1\leq \operatorname{dim}(A[x])\leq 2\operatorname{dim}(A)+1.
\end{align*}

It is known (see \cite{Seid},\cite{Seid2}) that for any $k\geq 0$ and $n$ with $k+1\leq n\leq 2k+1$, there exists a ring $A$ such that $\dim A=k$ and $\dim A[x]=n$.

\begin{thebibliography}{9}
\bibitem[Seid]{Seid} A. Seidenberg, \emph{A note on the dimension theory of rings.} Pacific J. of Mathematics, Volume 3 (1953), 505-512.
\bibitem[Seid2]{Seid2} A. Seidenberg, \emph{On the dimension theory of rings (II).} Pacific J. of Mathematics, Volume 4 (1954), 603-614.

\end{thebibliography}
%%%%%
%%%%%
\end{document}
