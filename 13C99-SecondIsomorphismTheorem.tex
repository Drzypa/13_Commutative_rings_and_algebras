\documentclass[12pt]{article}
\usepackage{pmmeta}
\pmcanonicalname{SecondIsomorphismTheorem}
\pmcreated{2013-03-22 12:08:46}
\pmmodified{2013-03-22 12:08:46}
\pmowner{djao}{24}
\pmmodifier{djao}{24}
\pmtitle{second isomorphism theorem}
\pmrecord{9}{31334}
\pmprivacy{1}
\pmauthor{djao}{24}
\pmtype{Theorem}
\pmcomment{trigger rebuild}
\pmclassification{msc}{13C99}
\pmclassification{msc}{20A05}

\usepackage{amssymb}
\usepackage{amsmath}
\usepackage{amsfonts}
\usepackage{graphicx}
%%%\usepackage{xypic}
\begin{document}
Let $(G,*)$ be a group. Let $H$ be a subgroup of $G$ and let $K$ be a normal subgroup of $G$. Then
\begin{itemize}
\item $HK := \{ h*k \mid h \in H,\ k \in K \}$ is a subgroup of $G$,
\item $K$ is a normal subgroup of $HK$,
\item $H \cap K$ is a normal subgroup of $H$,
\item There is a natural group isomorphism $H/(H \cap K) = HK/K$.
\end{itemize}

The same statement also holds in the category of modules over a fixed ring (where normality is neither needed nor relevant), and indeed can be formulated so as to hold in any abelian category.
%%%%%
%%%%%
%%%%%
\end{document}
