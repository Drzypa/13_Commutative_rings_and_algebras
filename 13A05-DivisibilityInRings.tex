\documentclass[12pt]{article}
\usepackage{pmmeta}
\pmcanonicalname{DivisibilityInRings}
\pmcreated{2015-05-06 15:18:14}
\pmmodified{2015-05-06 15:18:14}
\pmowner{pahio}{2872}
\pmmodifier{pahio}{2872}
\pmtitle{divisibility in rings}
\pmrecord{24}{36322}
\pmprivacy{1}
\pmauthor{pahio}{2872}
\pmtype{Definition}
\pmcomment{trigger rebuild}
\pmclassification{msc}{13A05}
\pmclassification{msc}{11A51}
%\pmkeywords{divide}
%\pmkeywords{divisor}
%\pmkeywords{factor}
\pmrelated{PrimeElement}
\pmrelated{Irreducible}
\pmrelated{GroupOfUnits}
\pmrelated{DivisibilityByPrimeNumber}
\pmrelated{GcdDomain}
\pmrelated{CorollaryOfBezoutsLemma}
\pmrelated{ExistenceAndUniquenessOfTheGcdOfTwoIntegers}
\pmrelated{MultiplicationRing}
\pmrelated{IdealDecompositionInDedekindDomain}
\pmrelated{IdealMultiplicationLaws}
\pmrelated{UnityPlusNilpotentIsUnit}
\pmdefines{divisible}
\pmdefines{divisibility}
\pmdefines{divisibility of ideals}

% this is the default PlanetMath preamble.  as your knowledge
% of TeX increases, you will probably want to edit this, but
% it should be fine as is for beginners.

% almost certainly you want these
\usepackage{amssymb}
\usepackage{amsmath}
\usepackage{amsfonts}

% used for TeXing text within eps files
%\usepackage{psfrag}
% need this for including graphics (\includegraphics)
%\usepackage{graphicx}
% for neatly defining theorems and propositions
%\usepackage{amsthm}
% making logically defined graphics
%%%\usepackage{xypic}

% there are many more packages, add them here as you need them

% define commands here
\begin{document}
Let\, $(A,\,+,\,\cdot)$\, be a commutative ring with a non-zero 
unity 1.\, If $a$ and $b$ are two elements of $A$ and if there 
is an element $q$ of $A$ such that\, $b = qa$,\, then $b$ is 
said to be {\em divisible} by $a$; it may be denoted by\, 
$a\mid b$.\, (If $A$ has no zero divisors and\, $a \neq 0$,\, 
then $q$ is uniquely determined.)

When $b$ is divisible by $a$, $a$ is said to be a 
{\it divisor} or 
{\it \PMlinkname{factor}{DivisibilityInRings}} 
of $b$.\, On the other hand, $b$ is not said to be 
a {\it multiple} of $a$ except in the case that $A$ is the 
ring $\mathbb{Z}$ of the integers.\, In some languages, e.g. in 
the Finnish, $b$ has a name which could be approximately be 
translated as `{\it containant}':\quad  $b$ is a {\it containant} 
of $a$ (``$b$ on $a$:n {\it sis\"alt\"aj\"a}'').


\textbf{\PMlinkescapetext{Properties}}
\begin{itemize}
\item $a\mid b$\; iff\; $(b)\subseteq (a)$\,\, [see the principal ideals].
\item Divisibility is a reflexive and transitive relation in $A$.
\item 0 is divisible by all elements of $A$.
\item $a\mid 1$\; iff\; $a$ is a unit of $A$.
\item All elements of $A$ are divisible by every unit of $A$.
\item If\; $a\mid b$\; then\; $a^n\mid b^n \;\; (n = 1,\,2,\,\ldots)$. 
\item If\; $a\mid b$\; then\; $a\mid bc$\; and\; $ac\mid bc$.
\item If\; $a\mid b$\; and\; $a\mid c$\; then\; $a\mid b\!+\!c$.
\item If\; $a\mid b$\; and\; $a\nmid c$\; then\; $a\nmid b\!+\!c$.
\end{itemize}

\textbf{Note.}\, The divisibility can be similarly defined if\, 
$(A,\,+,\,\cdot)$\, is only a semiring; then it also has the 
above properties except the first.\, This concerns especially 
the case that we have a ring $R$ with non-zero unity and $A$ is 
the set of the ideals of $R$ (see the ideal multiplication laws).\,
Thus one may speak of the {\em divisibility of ideals} in 
$R$:\, $\mathfrak{a\mid b\,\,\Leftrightarrow\,\, 
(\exists q)\,(b = qa)}$.\, Cf. multiplication ring.

%%%%%
%%%%%
\end{document}
