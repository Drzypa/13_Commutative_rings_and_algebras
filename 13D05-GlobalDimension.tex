\documentclass[12pt]{article}
\usepackage{pmmeta}
\pmcanonicalname{GlobalDimension}
\pmcreated{2013-03-22 14:51:51}
\pmmodified{2013-03-22 14:51:51}
\pmowner{CWoo}{3771}
\pmmodifier{CWoo}{3771}
\pmtitle{global dimension}
\pmrecord{5}{36539}
\pmprivacy{1}
\pmauthor{CWoo}{3771}
\pmtype{Definition}
\pmcomment{trigger rebuild}
\pmclassification{msc}{13D05}
\pmclassification{msc}{16E10}
\pmclassification{msc}{18G20}
\pmsynonym{homological dimension}{GlobalDimension}

\endmetadata

% this is the default PlanetMath preamble.  as your knowledge
% of TeX increases, you will probably want to edit this, but
% it should be fine as is for beginners.

% almost certainly you want these
\usepackage{amssymb,amscd}
\usepackage{amsmath}
\usepackage{amsfonts}

% used for TeXing text within eps files
%\usepackage{psfrag}
% need this for including graphics (\includegraphics)
%\usepackage{graphicx}
% for neatly defining theorems and propositions
%\usepackage{amsthm}
% making logically defined graphics
%%%\usepackage{xypic}

% there are many more packages, add them here as you need them

% define commands here
\begin{document}
\PMlinkescapeword{terms}
\PMlinkescapeword{mean}

For any ring $R$, the \emph{left global dimension} of $R$ is defined to be the supremum of projective dimensions of left modules of $R$:
$$l.\operatorname{Gd}(R):=\operatorname{sup}\lbrace\operatorname{pd}_R(M)\mid M\mbox{ is a left R-module }\rbrace.$$

Similarly, the \emph{right global dimension} of $R$ is:
$$r.\operatorname{Gd}(R):=\operatorname{sup}\lbrace\operatorname{pd}_R(M)\mid M\mbox{ is a right R-module }\rbrace.$$

If $R$ is commutative, then $l.\operatorname{Gd}(R)=r.\operatorname{Gd}(R)$ and we may drop $l$ and $r$ and simply use $\operatorname{Gd}(R)$ to mean the \emph{global dimension} of $R$.

\textbf{Remarks}.  
\begin{enumerate}
\item For a ring $R$, $l.\operatorname{Gd}(R)=0$ iff $r.\operatorname{Gd}(R)=0$ (see the first example below).  However, in general, $l.\operatorname{Gd}(R)$ is not necessarily the same as $r.\operatorname{Gd}(R)$.
\item The left (right) global dimension of a ring can also be defined in terms of injective dimensions.  For example, for right global dimension of $R$, we have: $r.\operatorname{Gd}(R)=\operatorname{sup}\lbrace\operatorname{id}_R(M)\mid M\mbox{ is a right R-module }\rbrace$.  This definition turns out to be equivalent to the one using projective dimensions.
\end{enumerate}

\textbf{Examples}.
\begin{enumerate}
\item $l.\operatorname{Gd}(R)=0$ iff $R$ is a semisimple ring iff $r.\operatorname{Gd}(R)=0$.
\item $r.\operatorname{Gd}(R)=1$ iff $R$ is a right hereditary ring that is not semisimple.
\end{enumerate}
%%%%%
%%%%%
\end{document}
