\documentclass[12pt]{article}
\usepackage{pmmeta}
\pmcanonicalname{ExampleOfFreeModule}
\pmcreated{2013-03-22 13:48:41}
\pmmodified{2013-03-22 13:48:41}
\pmowner{mathcam}{2727}
\pmmodifier{mathcam}{2727}
\pmtitle{example of free module}
\pmrecord{5}{34534}
\pmprivacy{1}
\pmauthor{mathcam}{2727}
\pmtype{Example}
\pmcomment{trigger rebuild}
\pmclassification{msc}{13C10}

% this is the default PlanetMath preamble.  as your knowledge
% of TeX increases, you will probably want to edit this, but
% it should be fine as is for beginners.

% almost certainly you want these
\usepackage{amssymb}
\usepackage{amsmath}
\usepackage{amsfonts}
\usepackage{amsthm}

% used for TeXing text within eps files
%\usepackage{psfrag}
% need this for including graphics (\includegraphics)
%\usepackage{graphicx}
% for neatly defining theorems and propositions
%\usepackage{amsthm}
% making logically defined graphics
%%%\usepackage{xypic}

% there are many more packages, add them here as you need them

% define commands here
\newtheorem{Theo}{Theorem}
\newcommand{\mc}{\mathcal}
\newcommand{\mb}{\mathbb}
\newcommand{\mf}{\mathfrak}
\newcommand{\ol}{\overline}
\newcommand{\ra}{\rightarrow}
\newcommand{\la}{\leftarrow}
\newcommand{\La}{\Leftarrow}
\newcommand{\Ra}{\Rightarrow}
\newcommand{\nor}{\vartriangleleft}
\newcommand{\Gal}{\text{Gal}}
\newcommand{\GL}{\text{GL}}
\newcommand{\Z}{\mb{Z}}
\newcommand{\R}{\mb{R}}
\newcommand{\Q}{\mb{Q}}
\newcommand{\C}{\mb{C}}
\newcommand{\<}{\langle}
\renewcommand{\>}{\rangle}
\begin{document}
\PMlinkescapetext{Clearly} from the definition, $\Z^n$ is \PMlinkid{free}{FreeModule} as a $\Z$-module for any positive integer $n$.

A more interesting example is the following:

\begin{Theo}
The set of rational numbers $\Q$ do \emph{not} form a \PMlinkid{free}{FreeModule} $\Z$-module.
\end{Theo}

\begin{proof}
First note that any two elements in $\Q$
are $\Z$-linearly dependent.  If $x=\frac{p_1}{q_1}$ and
$y=\frac{p_2}{q_2}$, then $q_1p_2x-q_2p_1y=0$.  Since \PMlinkname{basis}{Basis} elements
must be linearly independent, this shows that any basis must consist
of only one element, say $\frac{p}{q}$, with $p$ and $q$ relatively prime, and without loss of generality, $q>0$.  The $\Z$-span of $\{\frac{p}{q}\}$ is the
set of rational numbers of the form $\frac{np}{q}$.  I claim that
$\frac{1}{q+1}$ is not in the set.  If it were, then we would have
$\frac{np}{q}=\frac{1}{q+1}$ for some $n$, but this implies that
$np=\frac{q}{q+1}$ which has no solutions for $n,p\in\Z$ ,$q\in\Z^+$, giving us
a contradiction.
\end{proof}
%%%%%
%%%%%
\end{document}
