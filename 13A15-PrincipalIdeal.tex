\documentclass[12pt]{article}
\usepackage{pmmeta}
\pmcanonicalname{PrincipalIdeal}
\pmcreated{2013-03-22 11:51:49}
\pmmodified{2013-03-22 11:51:49}
\pmowner{djao}{24}
\pmmodifier{djao}{24}
\pmtitle{principal ideal}
\pmrecord{7}{30437}
\pmprivacy{1}
\pmauthor{djao}{24}
\pmtype{Definition}
\pmcomment{trigger rebuild}
\pmclassification{msc}{13A15}
\pmclassification{msc}{16D25}
\pmclassification{msc}{81-00}
\pmclassification{msc}{82-00}
\pmclassification{msc}{83-00}
\pmclassification{msc}{46L05}

\endmetadata

\usepackage{amssymb}
\usepackage{amsmath}
\usepackage{amsfonts}
\usepackage{graphicx}
%%%%\usepackage{xypic}
\begin{document}
Let $R$ be a ring and let $a \in R$. The principal left (resp. right, 2-sided) ideal of $a$ is the smallest left (resp. right, 2-sided) ideal of $R$ containing the element $a$.

When $R$ is a commutative ring, the principal ideal of $a$ is denoted $(a)$.
%%%%%
%%%%%
%%%%%
%%%%%
\end{document}
