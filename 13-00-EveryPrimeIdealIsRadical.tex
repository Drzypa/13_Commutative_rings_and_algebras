\documentclass[12pt]{article}
\usepackage{pmmeta}
\pmcanonicalname{EveryPrimeIdealIsRadical}
\pmcreated{2013-03-22 13:56:54}
\pmmodified{2013-03-22 13:56:54}
\pmowner{alozano}{2414}
\pmmodifier{alozano}{2414}
\pmtitle{every prime ideal is radical}
\pmrecord{5}{34712}
\pmprivacy{1}
\pmauthor{alozano}{2414}
\pmtype{Theorem}
\pmcomment{trigger rebuild}
\pmclassification{msc}{13-00}
\pmclassification{msc}{14A05}
\pmsynonym{prime ideal is radical}{EveryPrimeIdealIsRadical}
%\pmkeywords{prime ideal}
%\pmkeywords{radical}
\pmrelated{RadicalOfAnIdeal}
\pmrelated{PrimeIdeal}
\pmrelated{HilbertsNullstellensatz}

% this is the default PlanetMath preamble.  as your knowledge
% of TeX increases, you will probably want to edit this, but
% it should be fine as is for beginners.

% almost certainly you want these
\usepackage{amssymb}
\usepackage{amsmath}
\usepackage{amsthm}
\usepackage{amsfonts}

% used for TeXing text within eps files
%\usepackage{psfrag}
% need this for including graphics (\includegraphics)
%\usepackage{graphicx}
% for neatly defining theorems and propositions
%\usepackage{amsthm}
% making logically defined graphics
%%%\usepackage{xypic}

% there are many more packages, add them here as you need them

% define commands here

\newtheorem{thm}{Theorem}
\newtheorem{defn}{Definition}
\newtheorem{prop}{Proposition}
\newtheorem{lemma}{Lemma}
\newtheorem{cor}{Corollary}

% Some sets
\newcommand{\Nats}{\mathbb{N}}
\newcommand{\Ints}{\mathbb{Z}}
\newcommand{\Reals}{\mathbb{R}}
\newcommand{\Complex}{\mathbb{C}}
\newcommand{\Rats}{\mathbb{Q}}
\begin{document}
Let $\mathcal{R}$ be a commutative ring and let $\mathfrak{P}$ be
a prime ideal of $\mathcal{R}$.

\begin{prop}
Every prime ideal $\mathfrak{P}$ of $\mathcal{R}$ is a radical
ideal, i.e.
$$\mathfrak{P}=\operatorname{Rad(\mathfrak{P})}$$
\end{prop}
\begin{proof}
Recall that $\mathfrak{P}\subsetneq \mathcal{R} $ is a prime ideal
if and only if for any $a,b\in \mathcal{R}$ $$a\cdot b\in
\mathfrak{P} \Rightarrow a\in \mathfrak{P} \text{ or } b\in
\mathfrak{P}$$ Also, recall that
$$\operatorname{Rad}(\mathfrak{P})=\{ r\in \mathcal{R} \mid
\exists n\in \Nats \text{ such that } r^n\in \mathfrak{P} \}$$

Obviously, we have $\mathfrak{P}\subseteq
\operatorname{Rad}(\mathfrak{P})$ (just take $n=1$), so it remains
to show the reverse inclusion.

Suppose $r\in \operatorname{Rad}(\mathfrak{P})$, so there exists
some $n\in \Nats$ such that $r^n\in \mathfrak{P}$. We want to
prove that $r$ must be an element of the prime ideal
$\mathfrak{P}$. For this, we use induction on $n$ to prove the
following proposition:

For all $n\in\Nats$, for all $r\in \mathcal{R}$,
$r^n\in\mathfrak{P} \Rightarrow r\in \mathfrak{P}$.

{\bf Case $n=1$:} This is clear, $r\in \mathfrak{P}
\Rightarrow r\in \mathfrak{P}$.

{\bf Case $n$ $\Rightarrow$ Case $n+1$: } Suppose we have proved
the proposition for the case $n$, so our induction hypothesis is
$$\forall r\in\mathcal{R}, \quad r^n\in\mathfrak{P} \Rightarrow r\in
\mathfrak{P}$$ and suppose $r^{n+1}\in \mathfrak{P}$. Then
$$r\cdot r^n = r^{n+1} \in \mathfrak{P}$$
and since $\mathfrak{P}$ is a prime ideal we have
$$r\in \mathfrak{P} \text{ or } r^n\in \mathfrak{P}$$
Thus we conclude, either directly or using the induction
hypothesis, that $r\in \mathfrak{P}$ as desired.

\end{proof}
%%%%%
%%%%%
\end{document}
