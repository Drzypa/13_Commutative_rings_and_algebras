\documentclass[12pt]{article}
\usepackage{pmmeta}
\pmcanonicalname{RulesOfCalculusForDerivativeOfPolynomial}
\pmcreated{2013-03-22 18:20:05}
\pmmodified{2013-03-22 18:20:05}
\pmowner{rspuzio}{6075}
\pmmodifier{rspuzio}{6075}
\pmtitle{rules of calculus for derivative of polynomial}
\pmrecord{10}{40967}
\pmprivacy{1}
\pmauthor{rspuzio}{6075}
\pmtype{Derivation}
\pmcomment{trigger rebuild}
\pmclassification{msc}{13P05}
\pmclassification{msc}{11C08}
\pmclassification{msc}{12E05}
\pmrelated{ProofOfPropertiesOfDerivativesByPureAlgebra}

\endmetadata

% this is the default PlanetMath preamble.  as your knowledge
% of TeX increases, you will probably want to edit this, but
% it should be fine as is for beginners.

% almost certainly you want these
\usepackage{amssymb}
\usepackage{amsmath}
\usepackage{amsfonts}

% used for TeXing text within eps files
%\usepackage{psfrag}
% need this for including graphics (\includegraphics)
%\usepackage{graphicx}
% for neatly defining theorems and propositions
\usepackage{amsthm}
% making logically defined graphics
%%%\usepackage{xypic}

% there are many more packages, add them here as you need them

% define commands here
\newtheorem{thm}{Theorem}
\newtheorem{dfn}{Definition}
\begin{document}
In this entry, we will derive the properties of derivatives
of polynomials in a rigorous fashion.  We begin by showing
that the derivative exists.

\begin{thm}  
If $A$ is a commutative ring and $p$ is a polynomial in 
$A[x]$, then there exist unique polynomials $q$ and $A$ 
such that $p(x+y) = p(x) + y \, q(x) + y^2 \, r(x,y)$.
\end{thm}

\begin{proof}
We will first show existence, then uniqueness.  Define $f(y) 
= p(x+y) - p(x)$.  Since $f$ is a polynomial in $y$ with
coefficients in the ring $A[x]$ and $f(0) = 0$, we must have
$y$ be a factor of $f(y)$, so $f(y) = y\,g(x,y)$ for some $g$ 
in $A[x,y]$.  By definition of $f$, this means that $p(x+y) -
p(x) = y\,g(x,y)$. \footnote{We are here making use of the
identification of $A[x][y]$ with $A[x,y]$ to write the
polynomial $g$ either as a polynomial in $y$ with coefficients
in $A[x]$ or as a polynomial in $x$ and $y$ with coefficients
in $A$.}  Define $q(x) = g(x,0)$ and $h(x,y) = g(x,y) - g(x,0)$.
Regarding $h$ as a polynomial in $y$ with coefficients in
$A[x]$, we may, similiarly to what we did earlier, note that,
since $h(0) = 0$ by construction, $y$ must be a factor of $h(y)$.
Hence there exists a polynomial $r$ with coefficients in $A[x,y]$ 
such that $h(y) = y \, r(x,y)$.  Combining our definitions, we
conclude that $p(x+y) = p(x) + y \, q(x) + y^2 \, r(x,y)$.

We will now show uniqueness.  Assume that there
exists polymonomials $q,\,r,\,Q,\,R$ such that $p(x+y) = p(x) + y \, 
q(x) + y^2 \, r(x,y)$ and $p(x+y) = p(x) + y \, Q(x) + y^2 \, 
R(x,y)$.  Subtracting and rearranging terms, $y (q(x) - Q(x)) =
y^2 (R(x,y) - r(x,y))$.  Cancelling $y$\footnote{Note that, in
general, the cancellation law need not hold.  However, even if
$A$ has divisors of zero, it still will be the case that the
polynomial $y$ cannot divide zero, so we may cancel it.}, 
we have $q(x) - Q(x) = y (R(x,y) - r(x,y))$.  Substituting 
$0$ for $y$, we have $q(x) - Q(x) = 0$.  Replacing this in our
equation, $y (R(x,y) - r(x,y)) = 0$.  Cancelling another $y$,
$R(x,y) - r(x,y) = 0$.  Hence, we conclude that $Q = q$ and
$R = r$, so our \PMlinkescapetext{representation} is unique.
\end{proof}

Hence, the following is well-defined:

\begin{dfn}
Let $A$ be a commutative ring and let $p$ be polynomial in
$A[x]$.  Then $p'$ is the unique element of $A[x]$ such that
$p(x+y) = p(x) + y \, p'(x) + y^2 \, r[x,y]$ for some
$r \in A[x,y]$
\end{dfn}

We will now derive some of the rules for manipulating
derivatives familiar form calculus for polynomials using
purely algebraic operations with no limits involved.

\begin{thm}
If $A$ is a commutative ring and $p,q \in A[x]$, then
$(p + q)' = p' + q'$.
\end{thm}

\begin{proof}
Let us write $p(x+y) = p(x) + y \, p'(x) + y^2 \, r(x,y)$
and $q(x+y) = q(x) + y \, q'(y) + y^2 \, s(x,y)$.  Adding,
we have 
\[
p(x,y) + q(x,y) = 
  p(x) + q(x) + y (p'(x) + q'(x)) + 
  y^2 (r(x,y) + s(x,y)).
\]  
By definition of derivative, this means that $(p+q)' = p' + q'$.
\end{proof}

\begin{thm}
If $A$ is a commutative ring and $p,q \in A[x]$, then
$(p \cdot q)' = p' \cdot q + p \cdot q'$.
\end{thm}

\begin{proof}
Let us write $p(x+y) = p(x) + y \, p'(x) + y^2 \, r(x,y)$
and $q(x+y) = q(x) + y \, q'(y) + y^2 \, s(x,y)$.  Multiplying,
grouping terms, and pulling out some common factors, we have
\begin{align*}
 p(x+y) q(x+y) &= p(x) q(y) + y (p'(x) q(x) + p(x) q'(x)) \\ &+ 
   y^2 (p(x) s(x,y) + q(x) r(x,y) + p'(x) q'(y) \\ & \quad+
   y\, p'(x) s(x,y) + y\, q'(x) r(x,y) + y^2 \, r(x,y) s(x,y)).
\end{align*}
By definition of derivative, this means that 
$(p \cdot q)' = p' \cdot q + p \cdot q'$.
\end{proof}

\begin{thm}
If $A$ is a commutative ring and $p,q \in A[x]$, then
$(p \circ q)' = (p' \circ q) \cdot q'$.
\end{thm}

\begin{proof}
Let us write $p(x+y) = p(x) + y \, p'(x) + y^2 \, r(x,y)$
and $q(x+y) = q(x) + y \, q'(y) + y^2 \, s(x,y)$.  Composing,
grouping terms, and pulling out some common factors, we have
\begin{align*}
p(q(x+y)) &= p \left( q(x) + y\,q'(y) + y^2 \, s(x,y) \right) \\
 &= p(q(x)) + \left( y\,q'(y) + y^2 \, s(x,y) \right) p'(q(x)) \\
&\quad+ \left( y\,q'(y) + y^2 \, s(x,y) \right)^2
                r \left( q(x), y\,q'(y) + y^2 \, s(x,y) \right) \\
&= p(q(x)) + y \, p'(q(x)) q'(y) \\ &\quad+
 y^2 \left( s(x,y) p'(q(x)) + \left( q'(y) + y \, s(x,y) \right)^2 
 r \left( q(x), y\,q'(y) + y^2 \, s(x,y) \right) \right)
\end{align*}
By definition of derivative, this means that 
$(p \circ q)' = (p' \circ q) \cdot q'$.
\end{proof}
%%%%%
%%%%%
\end{document}
