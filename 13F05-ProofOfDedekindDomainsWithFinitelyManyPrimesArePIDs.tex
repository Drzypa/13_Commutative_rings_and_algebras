\documentclass[12pt]{article}
\usepackage{pmmeta}
\pmcanonicalname{ProofOfDedekindDomainsWithFinitelyManyPrimesArePIDs}
\pmcreated{2013-03-22 18:35:24}
\pmmodified{2013-03-22 18:35:24}
\pmowner{rm50}{10146}
\pmmodifier{rm50}{10146}
\pmtitle{proof of Dedekind domains with finitely many primes are PIDs}
\pmrecord{4}{41317}
\pmprivacy{1}
\pmauthor{rm50}{10146}
\pmtype{Proof}
\pmcomment{trigger rebuild}
\pmclassification{msc}{13F05}
\pmclassification{msc}{11R04}

% this is the default PlanetMath preamble.  as your knowledge
% of TeX increases, you will probably want to edit this, but
% it should be fine as is for beginners.

% almost certainly you want these
\usepackage{amssymb}
\usepackage{amsmath}
\usepackage{amsfonts}

% used for TeXing text within eps files
%\usepackage{psfrag}
% need this for including graphics (\includegraphics)
%\usepackage{graphicx}
% for neatly defining theorems and propositions
\usepackage{amsthm}
% making logically defined graphics
%%%\usepackage{xypic}

% there are many more packages, add them here as you need them

% define commands here
\newcommand{\smp}{\mathfrak{p}}
\begin{document}
\begin{proof} Let $\smp_1,\ldots,\smp_k$ be all the primes of a Dedekind domain $R$. If $I$ is any ideal of $R$, then by the Weak Approximation Theorem we can choose $x\in R$ such that $\nu_{\smp_i}((x))=\nu_{\smp_i}(I)$ for all $i$ (where $\nu_{\smp}$ is the $\smp$-adic valuation). But since $R$ is Dedekind, ideals have unique factorization; since $(x)$ and $I$ have identical factorizations, we must have $(x)=I$ and $I$ is principal.
\end{proof}

%%%%%
%%%%%
\end{document}
