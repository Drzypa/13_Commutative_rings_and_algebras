\documentclass[12pt]{article}
\usepackage{pmmeta}
\pmcanonicalname{SpecialReduciblePolynomialsOverAFieldWithPositiveCharacteristic}
\pmcreated{2013-03-22 18:31:05}
\pmmodified{2013-03-22 18:31:05}
\pmowner{joking}{16130}
\pmmodifier{joking}{16130}
\pmtitle{special reducible polynomials over a field with positive characteristic}
\pmrecord{10}{41205}
\pmprivacy{1}
\pmauthor{joking}{16130}
\pmtype{Theorem}
\pmcomment{trigger rebuild}
\pmclassification{msc}{13F07}

% this is the default PlanetMath preamble.  as your knowledge
% of TeX increases, you will probably want to edit this, but
% it should be fine as is for beginners.

% almost certainly you want these
\usepackage{amssymb}
\usepackage{amsmath}
\usepackage{amsfonts}

% used for TeXing text within eps files
%\usepackage{psfrag}
% need this for including graphics (\includegraphics)
%\usepackage{graphicx}
% for neatly defining theorems and propositions
%\usepackage{amsthm}
% making logically defined graphics
%%%\usepackage{xypic}

% there are many more packages, add them here as you need them

% define commands here

\begin{document}
Let $k$ be an arbitrary field such that $\mathrm{char}(k) = p> 0$. We will assume that $0\not\in\mathbb{N}$.

\textbf{Proposition}. Let $m\in\mathbb{N}$. Then for any $a\in k$ the polynomial $W(X)=X^{p^{m}}-a$ is reducible if and only if there exist $c\in k$ and $n\in\mathbb{N}$ such that $c^{p^{n}}=a$. Moreover the factorization of $W(X)$ is given by the formula
$$W(X)=(X^{p^{m-n}}-c)^{p^n},$$
where $n$ is a maximal natural number such that $0\leq n\leq m$ and $a=c^{p^n}$ for some $c\in k$.

\textit{Proof}. ``$\Leftarrow$'' Assume that $a=c^{p^{n}}$ for some $c\in k$ and $n\in\mathbb{N}$. It is well known that if $\mathrm{char}(k) = p> 0$ and $t\in\mathbb{N}$ then for any $x,y\in k$ we have $(x+y)^{p^t}=x^{p^t}+y^{p^t}$. Therefore
$$W(X)=X^{p^{m}}-a=X^{p^{m}}-c^{p^{n}}=(X^{p^{m-1}})^p-(c^{p^{n-1}})^p=(X^{p^{m-1}}-c^{p^{n-1}})^p=(V(X))^p.$$
Note that $p^m > \mathrm{deg}(V(X))=p^{m-1} > 0$ and therefore $W(X)$ is reducible. $\square$

``$\Rightarrow$'' Assume that $W(X)$ is reducible. Therefore there exist $V(X),U(X)\in k[X]$ such that $W(X)=V(X)\cdot U(X)$ and both $\mathrm{deg}(V(X))>0$ and $\mathrm{deg}(U(X))>0$.

Recall that there exists an algebraically closed field $\overline{k}$ such that $k$ is a subfield of $\overline{k}$ (generally it is true for any field). Therefore there exists $c_{0}\in\overline{k}$ such that $c_{0}^{p^m}=a$ and thus we have:
$$W(X)=X^{p^{m}}-a=X^{p^{m}}-c_{0}^{p^{m}}=(X-c_{0})^{p^m}$$
in $\overline{k}[X]$. Now $V(X)\cdot U(X)=W(X)=(X-c_{0})^{p^m}$ and since $\overline{k}[X]$ is a unique factorization domain then for $n=\mathrm{deg}(V(X))>0$ we have:
$$V(X)=(X-c_{0})^n.$$
But $V(X)\in k[X]$ (the factorization was assumed to be over $k$) and therefore $c_{0}^n\in k$. It is easy to see that since $c_{0}^{n}\in k$ and $c_{0}^{p^m}\in k$ then $c_0^{\mathrm{gcd}(n,p^m)}\in k$, but $\mathrm{gcd}(n,p^m)=p^s$ for some $s\in\mathbb{N}$. Thus if we put $c=c_{0}^{p^s}$ we gain that $c^{p^{m-s}}=a$. But $m>s$ (since $n<p^m$ because we assumed that both $\mathrm{deg}(V(X))>0$ and $\mathrm{deg}(U(X))>0$), which completes the proof of the first part. $\square$

Now let $n\in\mathbb{N}$ be a maximal natural number such that $n\leq m$ and $a=c^{p^n}$ for some $c\in k$. Then we have
$$W(X)=(X^{p^{m-n}}-c)^{p^n}.$$
Note that the polynomial $X^{p^{m-n}}-c$ is irreducible. Indeed, assume that $X^{p^{m-n}}-c$ is reducible. Then (due to first part of the proposition) $c=u^{p^k}$ for some $k\in\mathbb{N}$ and $u\in k$. But then $a=(u^{p^k})^{p^n}=u^{p^{n+k}}$. Contradiction, since $n+k>n$ and $n$ was assumed to be maximal. $\square$
%%%%%
%%%%%
\end{document}
