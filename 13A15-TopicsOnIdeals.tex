\documentclass[12pt]{article}
\usepackage{pmmeta}
\pmcanonicalname{TopicsOnIdeals}
\pmcreated{2013-03-22 18:03:12}
\pmmodified{2013-03-22 18:03:12}
\pmowner{pahio}{2872}
\pmmodifier{pahio}{2872}
\pmtitle{topics on ideals}
\pmrecord{16}{40579}
\pmprivacy{1}
\pmauthor{pahio}{2872}
\pmtype{Definition}
\pmcomment{trigger rebuild}
\pmclassification{msc}{13A15}
\pmclassification{msc}{16D25}
\pmrelated{EntriesOnFinitelyGeneratedIdeals}

\endmetadata

% this is the default PlanetMath preamble.  as your knowledge
% of TeX increases, you will probably want to edit this, but
% it should be fine as is for beginners.

% almost certainly you want these
\usepackage{amssymb}
\usepackage{amsmath}
\usepackage{amsfonts}

% used for TeXing text within eps files
%\usepackage{psfrag}
% need this for including graphics (\includegraphics)
%\usepackage{graphicx}
% for neatly defining theorems and propositions
 \usepackage{amsthm}
% making logically defined graphics
%%%\usepackage{xypic}

% there are many more packages, add them here as you need them

% define commands here

\theoremstyle{definition}
\newtheorem*{thmplain}{Theorem}

\begin{document}
Below is a list of main concepts in the part of ring theory concerning ideals:

\begin{itemize}
\item \PMlinkname{ideal}{Ideal}
\item zero ideal
\item proper ideal
\item ideal of elements with finite order
\item modular ideal
\item regular ideal
\item nil ideal
\item nilpotent ideal
\item primitive ideal
\item primary ideal
\item semiprime ideal
\item prime ideal, minimal prime ideal
\item characterization of prime ideals
\item large ideal
\item maximal ideal
\item maximal ideal is prime
\item comaximal ideal
\item principal ideal
\item \PMlinkname{finitely generated ideal}{EntriesOnFinitelyGeneratedIdeals}
\item ideal generated by
\item sum of ideals
\item product of ideals
\item quotient of ideals
\item ideal multiplication laws
\item product of finitely generated ideals
\item ideal included in union of prime ideals
\item ideals contained in a union of ideals
\item dense ideals/subsets of a ring
\item cancellation ideal
\item radical ideal
\item fractional ideal of commutative ring
\item integral ideal
\item invertible ideal
\item invertibility of regularly generated ideal
\item inverse ideal
\item generators of inverse ideal
\item ideal generators in Pr\"ufer ring
\item ideal inverting in Pr\"ufer ring
\item fractional ideal (integral domain)
\item image ideal of divisor
\item ideals in a Dedekind domain

\end{itemize}
%%%%%
%%%%%
\end{document}
