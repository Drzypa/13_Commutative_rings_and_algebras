\documentclass[12pt]{article}
\usepackage{pmmeta}
\pmcanonicalname{FiniteRingHasNoProperOverrings}
\pmcreated{2013-03-22 15:11:12}
\pmmodified{2013-03-22 15:11:12}
\pmowner{pahio}{2872}
\pmmodifier{pahio}{2872}
\pmtitle{finite ring has no proper overrings}
\pmrecord{10}{36942}
\pmprivacy{1}
\pmauthor{pahio}{2872}
\pmtype{Result}
\pmcomment{trigger rebuild}
\pmclassification{msc}{13G05}
%\pmkeywords{regular element}
%\pmkeywords{unit}
\pmrelated{ExtensionByLocalization}
\pmrelated{ClassicalRingOfQuotients}
\pmrelated{AFiniteIntegralDomainIsAField}
\pmrelated{RingAdjunction}
\pmrelated{FormalPowerSeries}

\endmetadata

% this is the default PlanetMath preamble.  as your knowledge
% of TeX increases, you will probably want to edit this, but
% it should be fine as is for beginners.

% almost certainly you want these
\usepackage{amssymb}
\usepackage{amsmath}
\usepackage{amsfonts}

% used for TeXing text within eps files
%\usepackage{psfrag}
% need this for including graphics (\includegraphics)
%\usepackage{graphicx}
% for neatly defining theorems and propositions
 \usepackage{amsthm}
% making logically defined graphics
%%%\usepackage{xypic}

% there are many more packages, add them here as you need them

% define commands here

\theoremstyle{definition}
\newtheorem*{thmplain}{Theorem}
\begin{document}
The regular elements of a finite commutative ring $R$ are the units of the ring (see the \PMlinkname{parent}{NonZeroDivisorsOfFiniteRing} of this entry). \,Generally, the largest overring of $R$, the total ring of fractions $T$, is obtained by forming $S^{-1}R$, the extension by localization, using as the multiplicative set $S$ the set of all regular elements, which in this case is the unit group of $R$. \,The ring $R$ may be considered as a subring of $T$, which consists formally of the fractions \,$\frac{a}{s} = as^{-1}$\, with \,$a\in R$\, and \,$s\in S$. \,Since every $s$ has its own group inverse $s^{-1}$ in $S$ and so in $R$, it's evident that $T$ \PMlinkescapetext{contains} no other elements than the elements of $R$. \,Consequently, \,$T = R$,\, and therefore also any overring of $R$ coincides with $R$.

Accordingly, one can not extend a finite commutative ring by using a localization. \,Possible extensions must be made via some kind of \PMlinkname{adjunction}{RingAdjunction}. \,A more known special case is a \PMlinkname{finite integral domain}{AFiniteIntegralDomainIsAField} --- it is always a field and thus closed under the divisions.
%%%%%
%%%%%
\end{document}
