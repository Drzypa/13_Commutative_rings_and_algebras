\documentclass[12pt]{article}
\usepackage{pmmeta}
\pmcanonicalname{PrincipalIdealRing}
\pmcreated{2013-03-22 14:33:16}
\pmmodified{2013-03-22 14:33:16}
\pmowner{Wkbj79}{1863}
\pmmodifier{Wkbj79}{1863}
\pmtitle{principal ideal ring}
\pmrecord{7}{36106}
\pmprivacy{1}
\pmauthor{Wkbj79}{1863}
\pmtype{Definition}
\pmcomment{trigger rebuild}
\pmclassification{msc}{13F10}
\pmclassification{msc}{13A15}
\pmsynonym{principal ring}{PrincipalIdealRing}
\pmrelated{CriterionForCyclicRingsToBePrincipalIdealRings}

% this is the default PlanetMath preamble.  as your knowledge
% of TeX increases, you will probably want to edit this, but
% it should be fine as is for beginners.

% almost certainly you want these
\usepackage{amssymb}
\usepackage{amsmath}
\usepackage{amsfonts}

% used for TeXing text within eps files
%\usepackage{psfrag}
% need this for including graphics (\includegraphics)
%\usepackage{graphicx}
% for neatly defining theorems and propositions
%\usepackage{amsthm}
% making logically defined graphics
%%%\usepackage{xypic}

% there are many more packages, add them here as you need them

% define commands here
\begin{document}
A commutative ring $R$ in which all ideals are \PMlinkname{principal}{PrincipalIdeal}, \PMlinkname{i.e.}{Ie} \PMlinkname{generated by}{IdealGeneratedBy} a single ring element, is called a {\em principal ideal ring}.\, If $R$ is also an integral domain, it is a principal ideal domain.

Some well-known principal ideal rings are the ring $\mathbb{Z}$ of integers, its factor rings $\mathbb{Z}/n\mathbb{Z}$, and any polynomial ring over a field.
%%%%%
%%%%%
\end{document}
