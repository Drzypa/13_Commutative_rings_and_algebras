\documentclass[12pt]{article}
\usepackage{pmmeta}
\pmcanonicalname{TopicsOnPolynomials}
\pmcreated{2013-03-22 15:20:57}
\pmmodified{2013-03-22 15:20:57}
\pmowner{matte}{1858}
\pmmodifier{matte}{1858}
\pmtitle{topics on polynomials}
\pmrecord{31}{37170}
\pmprivacy{1}
\pmauthor{matte}{1858}
\pmtype{Topic}
\pmcomment{trigger rebuild}
\pmclassification{msc}{13P05}
\pmclassification{msc}{11C08}
\pmclassification{msc}{12E05}
\pmrelated{PropertiesOfOrthogonalPolynomials}

% this is the default PlanetMath preamble.  as your knowledge
% of TeX increases, you will probably want to edit this, but
% it should be fine as is for beginners.

% almost certainly you want these
\usepackage{amssymb}
\usepackage{amsmath}
\usepackage{amsfonts}
\usepackage{amsthm}

\usepackage{mathrsfs}

% used for TeXing text within eps files
%\usepackage{psfrag}
% need this for including graphics (\includegraphics)
%\usepackage{graphicx}
% for neatly defining theorems and propositions
%
% making logically defined graphics
%%%\usepackage{xypic}

% there are many more packages, add them here as you need them

% define commands here

\newcommand{\sR}[0]{\mathbb{R}}
\newcommand{\sC}[0]{\mathbb{C}}
\newcommand{\sN}[0]{\mathbb{N}}
\newcommand{\sZ}[0]{\mathbb{Z}}

 \usepackage{bbm}
 \newcommand{\Z}{\mathbbmss{Z}}
 \newcommand{\C}{\mathbbmss{C}}
 \newcommand{\F}{\mathbbmss{F}}
 \newcommand{\R}{\mathbbmss{R}}
 \newcommand{\Q}{\mathbbmss{Q}}



\newcommand*{\norm}[1]{\lVert #1 \rVert}
\newcommand*{\abs}[1]{| #1 |}



\newtheorem{thm}{Theorem}
\newtheorem{defn}{Definition}
\newtheorem{prop}{Proposition}
\newtheorem{lemma}{Lemma}
\newtheorem{cor}{Corollary}
\begin{document}
\subsubsection*{Definitions}
\begin{enumerate}
\item polynomial ring 
\item derivative of polynomial
\item polynomial ring over integral domain
\item polynomial ring over a field
\item polynomial ring which is PID
\item skew polynomial ring
\item $q$ skew polynomial ring
\item factor theorem
\item order and degree of polynomial
\item length of a polynomial
\item polynomial growth
\end{enumerate}

\subsubsection*{Polynomials over $\R$}
\begin{enumerate}
\item polynomial long division
\item polynomial interpolation, Lagrange interpolation formula
\item Weierstrass approximation theorem
\item Sturm's theorem 
\item properties of quadratic equation 
\item cubic formula 
\item quartic formula 
\item Abel's theorem
\item algebraic equation
\item square root of polynomial
\item {\em casus irreducibilis}
\item conditional congruence
\end{enumerate}

\subsubsection*{Roots of polynomials}
\begin{enumerate}
\item zero of polynomial
\item fundamental theorem of algebra
\item continuous dependence on coefficients
\item Tchirnhaus transformations
\item rational root theorem  
\item polynomial equation of odd degree
\item Viete formulas
\item Descartes' rule of signs
\item Eisenstein criterion 
\item zeros of polynomial derivative
\item homogeneous equation
\end{enumerate}

\subsubsection*{Polynomials with special properties}
\begin{enumerate}
\item zero polynomial
\item opposite polynomial
\item symmetric polynomial
\item monic polynomial
\item minimal polynomial
\item all one polynomial
\item \PMlinkname{prime factors of $x^n-1$}{PrimeFactorsOfXn1}
\item primitive polynomial
\item homogeneous polynomial
\item weighted homogeneous polynomial
\item irreducible polynomial
\item irreducibility of binomials with unity coefficients
\item reciprocal polynomial
\item orthogonal polynomials
\end{enumerate}


\subsection*{Classical orthogonal polynomials}
\begin{enumerate}
\item Chebyshev polynomials
\item Hermite polynomials
\item Laguerre polynomials
\item Legendre polynomials
\item Jacobi polynomials
\item properties of orthogonal polynomials
\end{enumerate}

\subsection*{Other polynomial families}
\begin{enumerate}
\item Bernoulli polynomials and numbers
\item Bernoulli polynomials
\item Bernstein polynomials
\item \PMlinkid{Boubaker polynomials}{12200}
\item Euler polynomials
\item Lam\'e polynomials


\end{enumerate}

\subsection*{Miscellaneous applications}
\begin{enumerate}
\item Taylor polynomial 
\item Characteristic polynomial in linear algebra (see also 
Cayley-Hamilton theorem)
\end{enumerate}


\subsection*{Related concepts}
\begin{enumerate}
\item division algorithm for polynomials
\item since sets of polynomial are rings, commutative algebra concepts like PID, UFD and Euclidean domains are heavily related
\item Gr\"obner basis
\end{enumerate}
%%%%%
%%%%%
\end{document}
