\documentclass[12pt]{article}
\usepackage{pmmeta}
\pmcanonicalname{NonisomorphicCompletionsOfmathbbQ}
\pmcreated{2013-03-22 14:58:17}
\pmmodified{2013-03-22 14:58:17}
\pmowner{pahio}{2872}
\pmmodifier{pahio}{2872}
\pmtitle{non-isomorphic completions of $\mathbb{Q}$}
\pmrecord{9}{36671}
\pmprivacy{1}
\pmauthor{pahio}{2872}
\pmtype{Theorem}
\pmcomment{trigger rebuild}
\pmclassification{msc}{13J10}
\pmclassification{msc}{13A18}
\pmclassification{msc}{12J20}
\pmclassification{msc}{13F30}
\pmrelated{PAdicCanonicalForm}
\pmdefines{$p$-adic numbers}

% this is the default PlanetMath preamble.  as your knowledge
% of TeX increases, you will probably want to edit this, but
% it should be fine as is for beginners.

% almost certainly you want these
\usepackage{amssymb}
\usepackage{amsmath}
\usepackage{amsfonts}

% used for TeXing text within eps files
%\usepackage{psfrag}
% need this for including graphics (\includegraphics)
%\usepackage{graphicx}
% for neatly defining theorems and propositions
%\usepackage{amsthm}
% making logically defined graphics
%%%\usepackage{xypic}

% there are many more packages, add them here as you need them

% define commands here
\begin{document}
No field $\mathbb{Q}_p$ of the {\em $p$-adic numbers} (\PMlinkname{$p$-adic rationals}{PAdicIntegers}) is isomorphic with the field $\mathbb{R}$ of the real numbers.

{\em Proof.} \,Let's assume the existence of a field isomorphism \,$f:\,\mathbb{R}\to \mathbb{Q}_p$\, for some positive prime number $p$. \,If we denote \,$f(\sqrt{p}) = a$, \,then we obtain 
         $$a^2 = (f(\sqrt{p}))^2 = f((\sqrt{p})^2) = f(p) = p,$$
because the isomorphism maps the elements of the prime subfield on themselves. \,Thus, if \,$|\cdot|_p$\, is the \PMlinkname{normed $p$-adic valuation}{PAdicValuation} of $\mathbb{Q}$ and of $\mathbb{Q}_p$, we get 
 $$|a|_p = \sqrt{|a^2|_p} = \sqrt{|p|_p} = \sqrt{\frac{1}{p}},$$
which value is an irrational number as a \PMlinkname{square root of a non-square}{SquareRootOf2IsIrrationalProof} rational. \,But this is impossible, since the value group of the completion $\mathbb{Q}_p$ must be the same as the value group $|\mathbb{Q}\setminus\{0\}|_p$ which consists of all integer powers of $p$. \,So we conclude that there can not exist such an isomorphism.
%%%%%
%%%%%
\end{document}
