\documentclass[12pt]{article}
\usepackage{pmmeta}
\pmcanonicalname{IntegralClosureIsRing}
\pmcreated{2013-03-22 19:15:40}
\pmmodified{2013-03-22 19:15:40}
\pmowner{pahio}{2872}
\pmmodifier{pahio}{2872}
\pmtitle{integral closure is ring}
\pmrecord{6}{42190}
\pmprivacy{1}
\pmauthor{pahio}{2872}
\pmtype{Theorem}
\pmcomment{trigger rebuild}
\pmclassification{msc}{13B22}
\pmrelated{PolynomialRing}
\pmrelated{RingAdjunction}
\pmrelated{IntegralClosuresInSeparableExtensionsAreFinitelyGenerated}

\endmetadata

% this is the default PlanetMath preamble.  as your knowledge
% of TeX increases, you will probably want to edit this, but
% it should be fine as is for beginners.

% almost certainly you want these
\usepackage{amssymb}
\usepackage{amsmath}
\usepackage{amsfonts}

% used for TeXing text within eps files
%\usepackage{psfrag}
% need this for including graphics (\includegraphics)
%\usepackage{graphicx}
% for neatly defining theorems and propositions
 \usepackage{amsthm}
% making logically defined graphics
%%%\usepackage{xypic}

% there are many more packages, add them here as you need them

% define commands here

\theoremstyle{definition}
\newtheorem*{thmplain}{Theorem}

\begin{document}
\textbf{Theorem.}\, Let $A$ be a subring of a commutative ring $B$ having nonzero unity.\, Then the integral closure of 
$A$ in $B$ is a subring of $B$ containing $A$.\\

\emph{Proof.}\, Let $x$ be an arbitrary element of the integral closure $A'$ of $A$ in $B$.\, Then there are the elements $a_0,\,a_1,\,\ldots,\,a_{n-1}$ of $A$ such that 
$$a_0\!+\!a_1x\!+\ldots+\!a_{n-1}x^{n-1}\!+\!x^n \;=\; 0$$
where\, $n > 0$.\, If\, $f(X) = c_0\!+\!c_1X\!+\ldots+\!c_mX^m$\, is a polynomial in $A[X]$ with degree\, $m > n$,\, we have
\begin{align*}
f(x) &\;=\; c_0\!+\!c_1x\!+\ldots+\!c_{m-1}x^{m-1}\!+\!c_mx^{m-n}(-a_0\!-\!a_1x\!-\ldots-\!a_{n-1}x^{n-1})\\
     &\;=\; c_0'\!+\!c_1'x\!+\ldots+\!c_{m-1}'x^{m-1}
\end{align*}
where\, the elements $c'_i$ belong to $A$.\, This procedure may be repeated until we see that $f(x)$ is an element of the $A$-module generated by $1,\,x,\,\ldots,\,x^n$.\, Accordingly, 
$$A[x] \;=\; A+Ax+\ldots+Ax^n$$
is a finitely generated $A$-module.

Now we have evidently\, $A \subseteq A'$.\, Let $y$ be another element of $A'$.\, Then
$$A[x,\,y] \;=\; A[x][y]$$
is a finitely generated $A[x]$-module, whence\, $A[x,\,y]$\, is a finitely generated $A$-module.\, Because the elements 
$x\!-\!y$ and $xy$ belong to\, $A[x,\,y]$,\, they are integral over $A$ and thus belong to $A'$.\, Consequently, $A'$ is a subring of $B$ (see the \PMlinkid{subring condition}{2738}).

\begin{thebibliography}{8}
\bibitem{LM}{\sc M. Larsen \& P. McCarthy}: {\em Multiplicative theory of ideals}.\, Academic Press, New York (1971).
\end{thebibliography}
%%%%%
%%%%%
\end{document}
