\documentclass[12pt]{article}
\usepackage{pmmeta}
\pmcanonicalname{IntegralityIsTransitive}
\pmcreated{2013-03-22 17:01:25}
\pmmodified{2013-03-22 17:01:25}
\pmowner{rm50}{10146}
\pmmodifier{rm50}{10146}
\pmtitle{integrality is transitive}
\pmrecord{6}{39309}
\pmprivacy{1}
\pmauthor{rm50}{10146}
\pmtype{Theorem}
\pmcomment{trigger rebuild}
\pmclassification{msc}{13B21}

\endmetadata

% this is the default PlanetMath preamble.  as your knowledge
% of TeX increases, you will probably want to edit this, but
% it should be fine as is for beginners.

% almost certainly you want these
\usepackage{amssymb}
\usepackage{amsmath}
\usepackage{amsfonts}

% used for TeXing text within eps files
%\usepackage{psfrag}
% need this for including graphics (\includegraphics)
%\usepackage{graphicx}
% for neatly defining theorems and propositions
%\usepackage{amsthm}
% making logically defined graphics
%%%\usepackage{xypic}

% there are many more packages, add them here as you need them

% define commands here

\begin{document}
Let $C\subset B\subset A$ be rings. If $B$ is integral over $C$ and $A$ is integral over $B$, then $A$ is integral over $C$.

\textbf{Proof. }
Choose $u\in A$. Then $u^n+b_1 u^{n-1}+\cdots+b_n=0, b_i\in B$. Thus $C[b_1,\ldots,b_n,u]$ is integral and thus module-finite over $C[b_1,\ldots,b_n]$. Each $b_i$ is integral over $C$, so $C[b_1,\ldots,b_n]$ is integral hence module-finite over $C$. Thus $C[b_1,\ldots,b_n,u]$ is module-finite, hence integral, over $C$, so $u$ is integral over $C$.

%%%%%
%%%%%
\end{document}
