\documentclass[12pt]{article}
\usepackage{pmmeta}
\pmcanonicalname{KrullValuationDomain}
\pmcreated{2013-03-22 14:55:01}
\pmmodified{2013-03-22 14:55:01}
\pmowner{pahio}{2872}
\pmmodifier{pahio}{2872}
\pmtitle{Krull valuation domain}
\pmrecord{8}{36603}
\pmprivacy{1}
\pmauthor{pahio}{2872}
\pmtype{Theorem}
\pmcomment{trigger rebuild}
\pmclassification{msc}{13F30}
\pmclassification{msc}{13A18}
\pmclassification{msc}{12J20}
\pmclassification{msc}{11R99}
\pmrelated{ValuationDeterminedByValuationDomain}

% this is the default PlanetMath preamble.  as your knowledge
% of TeX increases, you will probably want to edit this, but
% it should be fine as is for beginners.

% almost certainly you want these
\usepackage{amssymb}
\usepackage{amsmath}
\usepackage{amsfonts}

% used for TeXing text within eps files
%\usepackage{psfrag}
% need this for including graphics (\includegraphics)
%\usepackage{graphicx}
% for neatly defining theorems and propositions
 \usepackage{amsthm}
% making logically defined graphics
%%%\usepackage{xypic}

% there are many more packages, add them here as you need them

% define commands here
\theoremstyle{definition}
\newtheorem*{thmplain}{Theorem}
\begin{document}
\begin{thmplain}
 \, Any Krull valuation \,$|\cdot|$\, of a field $K$ determines a unique valuation domain \,$R = \{a\in K: \,\,|x|\leqq 1\}$, whose field of fraction is $K$.
\end{thmplain}

{\em Proof.} \,We first see that \,$1\in R$\, since \,$|1| = 1$. \,Let then \,$a,\,b$\, be any two elements of $R$. \,The non-archimedean triangle inequality shows that \,$|a-b| \leqq \max\{|a|,\,|b|\} \leqq 1$, \,i.e. that the difference \,$a-b$\, belongs to $R$. \,Using the \PMlinkname{multiplication rule}{OrderedGroup} 4 of inequalities we obtain
       $$|ab| = |a|\cdot|b| \leqq 1\cdot 1 = 1$$
which shows that also the product $ab$ is element of $R$. \,Thus, $R$ is a subring of the field $K$, and so an integral domain. \,Let now $c$ be an arbitrary element of $K$ not belonging to $R$. \,This implies that \,$1 < |c|$, \,whence \,$|c^{-1}| = |c|^{-1} < 1$ (see the \PMlinkname{inverse rule}{OrderedGroup} 5). \,Consequently, the inverse $c^{-1}$ belongs to $R$, and we conclude that $R$ is a valuation domain. \, The \PMlinkescapetext{presentations} \,$a = \frac{a}{1}$\, and \,$c = \frac{1}{c^{-1}}$\, make evident that $K$ is the field of fractions of $R$.
%%%%%
%%%%%
\end{document}
