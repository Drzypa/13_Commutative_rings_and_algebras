\documentclass[12pt]{article}
\usepackage{pmmeta}
\pmcanonicalname{ProofThatADomainIsDedekindIfItsIdealsAreProductsOfPrimes}
\pmcreated{2013-03-22 18:35:07}
\pmmodified{2013-03-22 18:35:07}
\pmowner{gel}{22282}
\pmmodifier{gel}{22282}
\pmtitle{proof that a domain is Dedekind if its ideals are products of primes}
\pmrecord{5}{41311}
\pmprivacy{1}
\pmauthor{gel}{22282}
\pmtype{Proof}
\pmcomment{trigger rebuild}
\pmclassification{msc}{13A15}
\pmclassification{msc}{13F05}
%\pmkeywords{Dedekind domain}
%\pmkeywords{prime ideal}
%\pmkeywords{invertible ideal}
\pmrelated{DedekindDomain}
\pmrelated{PrimeIdeal}

% this is the default PlanetMath preamble.  as your knowledge
% of TeX increases, you will probably want to edit this, but
% it should be fine as is for beginners.

% almost certainly you want these
\usepackage{amssymb}
\usepackage{amsmath}
\usepackage{amsfonts}

% used for TeXing text within eps files
%\usepackage{psfrag}
% need this for including graphics (\includegraphics)
%\usepackage{graphicx}
% for neatly defining theorems and propositions
\usepackage{amsthm}
% making logically defined graphics
%%%\usepackage{xypic}

% there are many more packages, add them here as you need them

% define commands here
\newtheorem*{theorem*}{Theorem}
\newtheorem*{lemma*}{Lemma}
\newtheorem*{corollary*}{Corollary}
\newtheorem{theorem}{Theorem}
\newtheorem{lemma}{Lemma}
\newtheorem{corollary}{Corollary}


\begin{document}
\PMlinkescapeword{maximal}
\PMlinkescapeword{invertible}
\PMlinkescapeword{prime factorizations}
\PMlinkescapeword{prime factorization}
\PMlinkescapeword{primes}
 We show that for an integral domain $R$, the following are equivalent.
\begin{enumerate}
\item $R$ is a Dedekind domain.\label{dedekind}
\item every nonzero proper ideal is a product of maximal ideals.\label{max prop}
\item every nonzero proper ideal is a product of prime ideals.\label{prime prop}
\end{enumerate}
For the equivalence of \ref{dedekind} and \ref{max prop} see proof that a domain is Dedekind if its ideals 
are products of maximals. Also, as every maximal ideal is prime, it is immediate that \ref{max prop} implies \ref{prime prop}. So, we just need to consider the case where \ref{prime prop} is satisfied and show that \ref{max prop} follows, for which it enough to show that every nonzero prime ideal is maximal.

We first suppose that $\mathfrak{p}$ is an \PMlinkname{invertible}{FractionalIdeal} prime ideal and show that it is maximal.
To do this it is enough to show that any $a\in R\setminus\mathfrak{p}$ gives $\mathfrak{p}+(a)=R$. First, we have the following inclusions,
\begin{equation*}
\mathfrak{p}\subsetneq\mathfrak{p}+(a^2)\subseteq\mathfrak{p}+(a).
\end{equation*}
Then, consider the prime factorizations
\begin{align}
&\mathfrak{p}+(a)=\mathfrak{p}_1\cdots\mathfrak{p}_m,\label{eq:2}\\
&\mathfrak{p}+(a^2)=\mathfrak{q}_1\cdots\mathfrak{q}_n.\label{eq:3}
\end{align}
We write $\bar a$ for the image of $a$ under the \PMlinkname{natural homorphism}{NaturalHomomorphism} $R\rightarrow R/\mathfrak{p}$ and $\mathfrak{\bar a}$ for the image of any ideal $\mathfrak{a}$. Equations (\ref{eq:2}) and (\ref{eq:3}) give
\begin{equation}\label{eq:4}
\mathfrak{\bar q}_1\cdots\mathfrak{\bar q}_n=(\bar a)^2=(\mathfrak{\bar p}_1\cdots\mathfrak{\bar p}_m)^2.
\end{equation}
As $\mathfrak{p}$ is strictly contained in $\mathfrak{p}+(a)$ and $\mathfrak{p}+(a^2)$, it must also be strictly contained in $\mathfrak{p}_k$ and $\mathfrak{q}_k$. So, $\mathfrak{\bar p}_k,\mathfrak{\bar q}_k$ are nonzero prime ideals, and by uniqueness of prime factorization (see, prime ideal factorization is unique) Equation (\ref{eq:4}) gives $n=2m$ and $\mathfrak{\bar p}_k=\mathfrak{\bar q}_k=\mathfrak{\bar q}_{k+m}$, after reordering of the factors. So $\mathfrak{p}_k=\mathfrak{q}_k=\mathfrak{q}_{k+m}$ and,
\begin{equation}\label{eq:5}
\mathfrak{p}+(a^2)
=\mathfrak{q}_1\cdots\mathfrak{q}_n
= \left(\mathfrak{p}_1\cdots\mathfrak{p}_m\right)^2
=\left(\mathfrak{p}+(a)\right)^2 = \mathfrak{p}^2+a\mathfrak{p}+(a^2).
\end{equation}
Then, $a\not\in\mathfrak{p}$ gives $\mathfrak{p}\cap(a^2)\subseteq a\mathfrak{p}$ and taking the intersection of both sides of (\ref{eq:5}) with $\mathfrak{p}$,
\begin{equation*}
\mathfrak{p}=\mathfrak{p}^2+a\mathfrak{p}+\mathfrak{p}\cap(a^2)\subseteq\mathfrak{p}^2+a\mathfrak{p}=\left(\mathfrak{p}+(a)\right)\mathfrak{p}.
\end{equation*}
But $\mathfrak{p}$ was assumed to be invertible, and can be cancelled giving $R\subseteq\mathfrak{p}+(a)$, showing that $\mathfrak{p}$ is maximal.

Now let $\mathfrak{p}$ be any prime ideal and $a\in\mathfrak{p}\setminus\{0\}$. Factoring into a product of primes
\begin{equation}\label{eq:6}
\mathfrak{p}_1\cdots\mathfrak{p}_n=(a)\subseteq\mathfrak{p},
\end{equation}
each of the $\mathfrak{p}_k$ is invertible and, by the above argument, must be maximal. Finally, as $\mathfrak{p}$ is prime, (\ref{eq:6}) gives $\mathfrak{p}_k\subseteq\mathfrak{p}$ for some $k$, so $\mathfrak{p}=\mathfrak{p}_k$ is maximal.

%%%%%
%%%%%
\end{document}
