\documentclass[12pt]{article}
\usepackage{pmmeta}
\pmcanonicalname{CounterExampleToNakayamasLemmaForNonfinitelyGeneratedModules}
\pmcreated{2013-03-22 18:03:55}
\pmmodified{2013-03-22 18:03:55}
\pmowner{sjm}{20613}
\pmmodifier{sjm}{20613}
\pmtitle{counter example to Nakayama's lemma for non-finitely generated modules}
\pmrecord{9}{40597}
\pmprivacy{1}
\pmauthor{sjm}{20613}
\pmtype{Example}
\pmcomment{trigger rebuild}
\pmclassification{msc}{13C99}

\endmetadata

% this is the default PlanetMath preamble.  as your knowledge
% of TeX increases, you will probably want to edit this, but
% it should be fine as is for beginners.

% almost certainly you want these
\usepackage{amssymb}
\usepackage{amsmath}
\usepackage{amsfonts}

% used for TeXing text within eps files
%\usepackage{psfrag}
% need this for including graphics (\includegraphics)
%\usepackage{graphicx}
% for neatly defining theorems and propositions
%\usepackage{amsthm}
% making logically defined graphics
%%%\usepackage{xypic}

% there are many more packages, add them here as you need them

% define commands here
\newcommand{\Q}{\mathbb{Q}}
\newcommand{\Z}{\mathbb{Z}}

\begin{document}
The hypothesis that the module $M$ be finitely generated is really
necessary. For example, the field of $p$-adic numbers $\Q_p$ is
not finitely generated over its ring of integers $\Z_p$ and
$(p)\Q_p = \Q_p$. \\

In one sense, the reason why $\Q_p$ is ``bad'' is that is has no
proper sub module which is also maximal. Thus $\Q_p$ has no non-zero simple
quotient. This explains why the following
\PMlinkname{Proof of Nakayama's Lemma}{ProofOfNakayamasLemma2}
does not work for non-finitely generated modules.
%%%%%
%%%%%
\end{document}
