\documentclass[12pt]{article}
\usepackage{pmmeta}
\pmcanonicalname{ProofOfTheRingOfIntegersOfANumberFieldIsFinitelyGeneratedOvermathbbZ}
\pmcreated{2013-03-22 16:03:07}
\pmmodified{2013-03-22 16:03:07}
\pmowner{rm50}{10146}
\pmmodifier{rm50}{10146}
\pmtitle{proof of the ring of integers of a number field is finitely generated over $\mathbb{Z}$}
\pmrecord{9}{38103}
\pmprivacy{1}
\pmauthor{rm50}{10146}
\pmtype{Proof}
\pmcomment{trigger rebuild}
\pmclassification{msc}{13B22}

\endmetadata

% this is the default PlanetMath preamble.  as your knowledge
% of TeX increases, you will probably want to edit this, but
% it should be fine as is for beginners.

% almost certainly you want these
\usepackage{amssymb}
\usepackage{amsmath}
\usepackage{amsfonts}

% used for TeXing text within eps files
%\usepackage{psfrag}
% need this for including graphics (\includegraphics)
%\usepackage{graphicx}
% for neatly defining theorems and propositions
%\usepackage{amsthm}
% making logically defined graphics
%%%\usepackage{xypic}

% there are many more packages, add them here as you need them

% define commands here
\newcommand{\Nats}{\mathbb{N}}
\newcommand{\Ints}{\mathbb{Z}}
\newcommand{\Reals}{\mathbb{R}}
\newcommand{\Complex}{\mathbb{C}}
\newcommand{\Rats}{\mathbb{Q}}



\begin{document}
\textbf{Proof:}\,
Choose any basis $\alpha_1,\ldots,\alpha_n$ of $K$ over $\Rats$. Using the theorem in the entry multiples of an algebraic number, we can multiply each element of the basis by an integer to get a new basis $\alpha_1,\ldots,\alpha_n$ with each $\alpha_i\in\mathcal{O}_K$.
\newline
Consider the group homomorphism
\[\varphi:K\rightarrow \Rats^n:\gamma\mapsto(\operatorname{Tr}_\Rats^K(\gamma\alpha_1),\ldots,\operatorname{Tr}_\Rats^K(\gamma\alpha_n))\]
where $\operatorname{Tr}_\Rats^K$ is the \PMlinkname{trace}{trace2} from $K$ to $\Rats$. Note that $\varphi$ is $1-1$, since if $\gamma\neq 0$ and $\varphi(\gamma)=0$, then 
\[n=\operatorname{Tr}_\Rats^K(1)=\operatorname{Tr}_\Rats^K(\gamma\gamma^{-1})=\operatorname{Tr}_\Rats^K(\gamma\sum r_i\alpha_i)=\sum r_i \operatorname{Tr}_\Rats^K(\gamma\alpha_i)=0\]
where the last equality holds since $\gamma\in\ker\varphi$.

Hence $\varphi:\mathcal{O}_K\hookrightarrow \Ints^n$, so $\mathcal{O}_K$ is finitely generated and torsion-free. It has rank $\geq n$ since the $\alpha_i$ are linearly independent, and rank $\leq n$ since it injects into $\Ints^n$, so it has rank $n$.

%%%%%
%%%%%
\end{document}
