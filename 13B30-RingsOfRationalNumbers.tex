\documentclass[12pt]{article}
\usepackage{pmmeta}
\pmcanonicalname{RingsOfRationalNumbers}
\pmcreated{2014-03-18 15:35:06}
\pmmodified{2014-03-18 15:35:06}
\pmowner{pahio}{2872}
\pmmodifier{pahio}{2872}
\pmtitle{rings of rational numbers}
\pmrecord{17}{40600}
\pmprivacy{1}
\pmauthor{pahio}{2872}
\pmtype{Theorem}
\pmcomment{trigger rebuild}
\pmclassification{msc}{13B30}
\pmclassification{msc}{11A99}
\pmsynonym{subrings of rationals}{RingsOfRationalNumbers}
\pmsynonym{subrings of $\mathbb{Q}$}{RingsOfRationalNumbers}
\pmrelated{Localization}
\pmrelated{ThereforeSign}
\pmdefines{dyadic fraction}
\pmdefines{p-integral rational numbers}
\pmdefines{$p$-integral rational number}

\endmetadata

% this is the default PlanetMath preamble.  as your knowledge
% of TeX increases, you will probably want to edit this, but
% it should be fine as is for beginners.

% almost certainly you want these
\usepackage{amssymb}
\usepackage{amsmath}
\usepackage{amsfonts}

% used for TeXing text within eps files
%\usepackage{psfrag}
% need this for including graphics (\includegraphics)
%\usepackage{graphicx}
% for neatly defining theorems and propositions
 \usepackage{amsthm}
% making logically defined graphics
%%%\usepackage{xypic}

% there are many more packages, add them here as you need them

% define commands here

\theoremstyle{definition}
\newtheorem*{thmplain}{Theorem}

\begin{document}
The criterion for a non-empty subset $R$ of a given ring $Q$ for being a subring of $Q$, is that $R$ contains always along with its two elements also their difference and product.\, Since the field $\mathbb{Q}$ of the rational numbers is (isomorphic to) the total ring of quotients of the ring $\mathbb{Z}$ of the integers, any rational number is a quotient $\displaystyle\frac{m}{n}$ of two integers $m$ and $n$.\, If now $R$ is an arbitrary subring of $\mathbb{Q}$ and 
$$\frac{m_1}{n_1},\, \frac{m_2}{n_2} \in R$$
with\, $m_1,\,n_1,\,m_2,\,n_2 \in \mathbb{Z}$\, (and\, $n_1n_2 \neq 0$), then one must have
$$\frac{m_1n_2-m_2n_1}{n_1n_2} \in R, \qquad \frac{m_1m_2}{n_1n_2} \in R.$$
Therefore, the set of possible denominators of the elements of $R$ is closed under multiplication, i.e. it forms a multiplicative set.\, We can of course confine us to subsets $S$ containing only positive integers.\, But along with any positive integer $n_0$, the set $S$ has to contain also all positive \PMlinkname{divisors}{Divisibility}, inclusive 1 and the \PMlinkname{prime divisors}{FundamentalTheoremOfArithmetics} of the number $n_0$, since the factorisation\, $n_0 = uv$\, of the denominator of an element $\displaystyle\frac{m}{n_0}$ of $R$ implies that the \PMlinkname{multiple}{GeneralAssociativity}\, $\displaystyle u\cdot\frac{m}{uv} = \frac{m}{v}$\, belongs to $R$.\, Accordingly, $S$ consists of 1, a certain set of positive prime numbers and all finite products of these, thus being a free monoid on the set of those prime numbers.

Since $R$ contains all \PMlinkescapetext{multiples} of each of its elements, it is apparent that the set of possible numerators form an ideal of $\mathbb{Z}$.

$\therefore$\; \textbf{Theorem.}\, If $R$ is a subring of $\mathbb{Q}$, then there are a principal ideal $(k)$ of $\mathbb{Z}$ and a multiplicative subset $S$ of $\mathbb{Z}$ such that $S$ is a free monoid on certain set of prime numbers and any element $\displaystyle\frac{m}{n}$ of $R$ is characterised by 
\begin{align*}
\begin{cases}
m \in (k),\\
n \in S.
\end{cases}
\end{align*}
The positive generator $k$ of $(k)$ does not belong to $S$ except when it is 1.\\

\textbf{Note.}\, Since $k$ may be greater than 1, the ring $R$ is not necessarily the ring of quotients $S^{-1}\mathbb{Z}$, e.g. in the case
$$R = \left\{\frac{2a}{3^s}\,\vdots\;\; a \in \mathbb{Z},\;\, s \in \mathbb{Z}_+\right\}.\\$$

\textbf{Examples.}\\

1.\, The ring\, $R := S^{-1}\mathbb{Z}$\, of the {\em \PMlinkname{p-integral rational numbers}{PAdicValuation}} where\\
$S = \{\mathrm{the\;power\;products\;of\;all\;positive\;primes\;except\;} p\}$.\, E.g. the 2-integral rational numbers consist of fractions with arbitrary integer numerators and odd denominators, for example $\frac{1000}{1001}$.\\

2.\, The ring\, $R := S^{-1}\mathbb{Z}$\, of the decimal fractions \,where\, 
$S = \{\mathrm{the\;power\;products\;of\;2\;and\;5}\}$.\\

3.\, The ring of the \PMlinkescapetext{{\em terminating binary}} or {\em dyadic fractions} with any integer numerators but denominators from the set\, $S = \{1,\,2,\,4,\,8,\,\ldots\}$.\\

4.\, If\, $S = \{1\}$,\, the subring of $\mathbb{Q}$ is simply some ideal $(k)$ of the ring $\mathbb{Z}$.\\

All the subrings of $\mathbb{Q}$ (except the trivial ring $\{0\}$) have $\mathbb{Q}$ as their total ring of quotients.

%%%%%
%%%%%
\end{document}
