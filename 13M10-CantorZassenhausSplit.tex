\documentclass[12pt]{article}
\usepackage{pmmeta}
\pmcanonicalname{CantorZassenhausSplit}
\pmcreated{2013-03-22 14:54:24}
\pmmodified{2013-03-22 14:54:24}
\pmowner{mathwizard}{128}
\pmmodifier{mathwizard}{128}
\pmtitle{Cantor-Zassenhaus split}
\pmrecord{8}{36591}
\pmprivacy{1}
\pmauthor{mathwizard}{128}
\pmtype{Algorithm}
\pmcomment{trigger rebuild}
\pmclassification{msc}{13M10}
\pmclassification{msc}{13P05}
\pmclassification{msc}{11C99}
\pmclassification{msc}{11Y99}
\pmrelated{SquarefreeFactorization}
\pmdefines{distinct degree factorization}

% this is the default PlanetMath preamble.  as your knowledge
% of TeX increases, you will probably want to edit this, but
% it should be fine as is for beginners.

% almost certainly you want these
\usepackage{amssymb}
\usepackage{amsmath}
\usepackage{amsfonts}

% used for TeXing text within eps files
%\usepackage{psfrag}
% need this for including graphics (\includegraphics)
%\usepackage{graphicx}
% for neatly defining theorems and propositions
%\usepackage{amsthm}
% making logically defined graphics
%%%\usepackage{xypic}

% there are many more packages, add them here as you need them

% define commands here
\newtheorem{theorem}{Lemma}
\begin{document}
Assume we want to factor a polynomial $A\in\mathbb{F}_p[X]$, where $p$ is a prime and $\mathbb{F}_p$ is the field with $p$ elements. By using squarefree factorization we can assume that $A$ is squarefree. The algorithm presented here now first splits $A$ into polynomials $A_i$, where each irreducible factor of $A_i$ has degree $i$. The main part will then be to factor these polynomials. The Cantor-Zassenhaus split is an efficient algorithm to achieve that. It uses random numbers, nevertheless it always returns a correct factorization.

\section{The distinct degree factorization}
To completely factor $A$ we will first find polynomials $A_d$ with $A=\prod A_d$, such that all irreducible factors of $A_d$ have degree $d$. This is called the \textit{distinct degree factorization}. Recall that if
$P\in\mathbb{F}_p$ is an irreducible polynomial of degree $d$, then
$K:=\mathbb{F}_p[X]/P(X)\mathbb{F}_p[X]$ is a finite field with $p^d$ elements.
So every $x\in K^*=K\backslash\{0\}$ satisfies $x^{p^d-1}=1$, so every $x\in K$ satisfies 
$x^{p^d}=x$, so $P$ is a divisor of the polynomial $X^{p^d}-X\in\mathbb{F}_p[X]$ and every irreducible factor of $X^{p^d}-X$, which does not divide $X^{p^e}-X$ for some $e<d$. So we can easily find these $A_d$ by introducing the series $B_i$ with $B_1=A$ and 
$$B_{k+1}=\frac{A}{\gcd\left(B_k,X^{p^k}-X\right)}.$$
Then $A_d=\gcd(B_d,X^{p^d}-X)$.

\section{The final splitting}
We can now assume that we want to factor a polynomial $A$ that has only irreducible factors of the known degree $d$. 
\subsection{Splitting for odd $p$}
First assume that $p$ is odd. Then we have the following
\begin{theorem}\label{thm:split.odd}
If $A$ is as above, then for any $T\in\mathbb{F}_p[X]$ we have
$$A=\gcd(A,T)\gcd(A,T^\frac{p^d-1}{2}-1)\gcd(A,T^\frac{p^d-1}{2}+1).$$
\end{theorem}
This is true because the roots of $X^{p^d}-X$ are exactly the elements of $\mathbb{F}_{p^d}$ and are all distinct in that field. So for any polynomial $T\in\mathbb{F}_p[X]$, the polynomial $T^{p^d}-T$ has also all elements of $\mathbb{F}_{p^d}$ as roots, so $X^{p^d}-X|T^{p^d}-T$. So as we have seen it is divisible by all irreducible polynomials of degree $d$, so it is divisible by $A$ since $A$ is squarefree. By noting that
$$T^{p^d}-T=T\left(T^\frac{p^d-1}{2}-1\right)\left(T^\frac{p^d-1}{2}+1\right)$$
is a decomposition with pairwise coprime factors, the claimed identity follows.

Now one can simply choose a random polynomial $T$ of degree less than $2d$. Then it is likely that $B:=\gcd(A,T^\frac{p^d-1}{2}-1)$ is a non-trivial divisor of $A$. We can then start over with $B$ and $A/B$ instead of $A$.

In this algorithm quite large powers of $T$ need to be computed. It is of course sufficient (and useful) to compute these powers modulo $A$, in order to keep the degree of the appearing polynomials low.

To illustrate this idea, we choose $d=1$. This means that $A$ is made up of different linear factors. Factorization is the achieved by finding zeroes. For this we could split $\mathbb{F}_p$ into three disjoint sets $M$, $N$ and $\{0\}$, such that $M\cup N\cup \{0\}=\mathbb{F}_p$.Then we construct the polynomial 
$$S=\prod_{\alpha\in M}(X-\alpha).$$
Obvioulsy it is now sufficient to factor the polynomials $B:=\gcd(A,S)$ and $\frac{A}{B}$. If $M$ and $N$ were chosen wisely, these polynomials have lower degree than $A$. Now what is left is the choice of $M$ and $N$. For the start it might be a good idea to consider the set of quadratic residues in $\mathbb{F}_p$, so choosing
$$M=\left\{x\in\mathbb{F}_p:\left(\frac{x}{p}\right)=1\right\},$$
where $\left(\frac{x}{p}\right)$ denotes the Legendre symbol. This choice is almost what we want. It is known that $M$ and $N$ are now of equal size and also we know that 
$$S=X^\frac{p-1}{2}-1.$$
But if we want to apply the same method again to $B$ and $\frac{A}{B}$, we need some randomness. To achieve this, we simply do not consider the set of all quadratic residues, but the set
$$M=\left\{x\in\mathbb{F}_p:\left(\frac{x-t}{p}\right)=1\right\},$$
where $t\in\mathbb{F}_p$ is some random element. This gives us the polynomial
$$S=(X-t)^\frac{p-1}{2}-1.$$
Actually we have now chosen a random monic polynomial $T$ of degree $1=2d-1$ and are reducing the problem of factoring $A$ to the problem of factoring $B:=\gcd(A,T^\frac{p-1}{2}-1)$ and $\frac{A}{B}$, as it was described above.

\subsection{Splitting for $p=2$}
Let $p=2$. Lemma \ref{thm:split.odd} does not work here because in the proof we use that
$$T^{p^d-1}-1=\left(T^\frac{p^d-1}{2}-1\right)\left(T^\frac{p^d-1}{2}+1\right),$$
which requires $p$ to be odd. However we can find something similar:
\begin{theorem}\label{thm:split.even}
Let
$$U(X)=X+X^2+X^4+\dots+X^{2^{d-1}}$$
and $A$ as above. Then for any $T\in\mathbb{F}_2[X]$ we have
$$A=\gcd(A,U\circ T)\cdot\gcd(A,U\circ T+1).$$
\end{theorem}
This is because $(U\circ T)^2=T^2+T^4+\dots+T^{2^d}$, so
$$(U\circ T)(U\circ T+1)=T^{2^d}-T$$
(we are in characteristic 2). Now this is a multiple of $A$ and the identity follows.

This gives us an algorithm for factorization of $A$ in the same way as above.
%%%%%
%%%%%
\end{document}
