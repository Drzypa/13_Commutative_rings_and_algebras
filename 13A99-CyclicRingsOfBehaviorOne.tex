\documentclass[12pt]{article}
\usepackage{pmmeta}
\pmcanonicalname{CyclicRingsOfBehaviorOne}
\pmcreated{2013-03-22 16:03:10}
\pmmodified{2013-03-22 16:03:10}
\pmowner{Wkbj79}{1863}
\pmmodifier{Wkbj79}{1863}
\pmtitle{cyclic rings of behavior one}
\pmrecord{9}{38104}
\pmprivacy{1}
\pmauthor{Wkbj79}{1863}
\pmtype{Theorem}
\pmcomment{trigger rebuild}
\pmclassification{msc}{13A99}
\pmclassification{msc}{16U99}
\pmclassification{msc}{13F10}
\pmrelated{MultiplicativeIdentityOfACyclicRingMustBeAGenerator}
\pmrelated{CriterionForCyclicRingsToBePrincipalIdealRings}

\endmetadata

\usepackage{amssymb}
\usepackage{amsmath}
\usepackage{amsfonts}

\usepackage{psfrag}
\usepackage{graphicx}
\usepackage{amsthm}
%%\usepackage{xypic}

\newtheorem*{thm*}{Theorem}
\begin{document}
\begin{thm*}
A cyclic ring has a multiplicative identity if and only if it has behavior one.
\end{thm*}

\begin{proof}
For a proof that a cyclic ring with a multiplicative identity has behavior one, see \PMlinkname{this theorem}{MultiplicativeIdentityOfACyclicRingMustBeAGenerator}.

Let $R$ be a cyclic ring with behavior one.  Let $r$ be a \PMlinkname{generator}{Generator} of the additive group of $R$ such that $r^2=r$.  Let $s \in R$.  Then there exists $a \in R$ with $s=ar$.  Since $rs=r(ar)=ar^2=ar=s$ and multiplication in cyclic rings is commutative, then $r$ is a multiplicative identity.
\end{proof}
%%%%%
%%%%%
\end{document}
