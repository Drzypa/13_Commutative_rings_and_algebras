\documentclass[12pt]{article}
\usepackage{pmmeta}
\pmcanonicalname{ExampleOfPID}
\pmcreated{2013-03-22 13:33:56}
\pmmodified{2013-03-22 13:33:56}
\pmowner{sleske}{997}
\pmmodifier{sleske}{997}
\pmtitle{example of PID}
\pmrecord{5}{34175}
\pmprivacy{1}
\pmauthor{sleske}{997}
\pmtype{Example}
\pmcomment{trigger rebuild}
\pmclassification{msc}{13G05}

% this is the default PlanetMath preamble.  as your knowledge
% of TeX increases, you will probably want to edit this, but
% it should be fine as is for beginners.

% almost certainly you want these
\usepackage{amssymb}
\usepackage{amsmath}
\usepackage{amsfonts}

% used for TeXing text within eps files
%\usepackage{psfrag}
% need this for including graphics (\includegraphics)
%\usepackage{graphicx}
% for neatly defining theorems and propositions
%\usepackage{amsthm}
% making logically defined graphics
%%%\usepackage{xypic}

% there are many more packages, add them here as you need them

% define commands here
\begin{document}
Important examples of principal ideal domains:

\begin{itemize}
\item The ring of the integers $\mathbb Z$.
\item The ring of polynomials in one variable over a field, i.e. a ring of
the form $\mathbb F[X]$, where $\mathbb F$ is a field. Note that the ring of polynomials in more than one variable over a field is never a PID.
\end{itemize}

Both of these examples are actually examples of Euclidean rings, which are 
always PIDs. There are, however, more complicated examples of PIDs which are
not Euclidean rings.
%%%%%
%%%%%
\end{document}
