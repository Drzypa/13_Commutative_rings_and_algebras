\documentclass[12pt]{article}
\usepackage{pmmeta}
\pmcanonicalname{ZeroOfPolynomial}
\pmcreated{2013-03-22 18:19:50}
\pmmodified{2013-03-22 18:19:50}
\pmowner{pahio}{2872}
\pmmodifier{pahio}{2872}
\pmtitle{zero of polynomial}
\pmrecord{8}{40962}
\pmprivacy{1}
\pmauthor{pahio}{2872}
\pmtype{Definition}
\pmcomment{trigger rebuild}
\pmclassification{msc}{13P05}
\pmclassification{msc}{11C08}
\pmclassification{msc}{12E05}
%\pmkeywords{zero}
%\pmkeywords{order}
%\pmkeywords{simple}
\pmrelated{PolynomialFunction}
\pmrelated{ZerosAndPolesOfRationalFunction}
\pmdefines{zero of polynomial}
\pmdefines{order of zero}
\pmdefines{order}
\pmdefines{simple zero}
\pmdefines{simple}

\endmetadata

% this is the default PlanetMath preamble.  as your knowledge
% of TeX increases, you will probably want to edit this, but
% it should be fine as is for beginners.

% almost certainly you want these
\usepackage{amssymb}
\usepackage{amsmath}
\usepackage{amsfonts}

% used for TeXing text within eps files
%\usepackage{psfrag}
% need this for including graphics (\includegraphics)
%\usepackage{graphicx}
% for neatly defining theorems and propositions
 \usepackage{amsthm}
% making logically defined graphics
%%%\usepackage{xypic}

% there are many more packages, add them here as you need them

% define commands here

\theoremstyle{definition}
\newtheorem*{thmplain}{Theorem}

\begin{document}
Let $R$ be a subring of a commutative ring $S$.\, If $f$ is a polynomial in $R[X]$, it defines an evaluation homomorphism from $S$ to $S$.\, Any element $\alpha$ of $S$ satisfying
$$f(\alpha) \;=\; 0$$
is a {\em zero of the polynomial} $f$.

If $R$ also is equipped with a non-zero unity, then the polynomial $f$ is in $S[X]$ divisible by the binomial 
\,$X\!-\!\alpha$ (cf. the factor theorem).\, In this case, if $f$ is divisible by $(X\!-\!\alpha)^n$ but not by 
$(X\!-\!\alpha)^{n+1}$, then $\alpha$ is a zero of the \emph{order} $n$ of the polynomial $f$.\, If this order is 1, then $\alpha$ is a \emph{simple zero} of $f$.

For example, the real number $\sqrt{2}$ ($\in \mathbb{R}$) is a zero of the polynomial $X^2\!-\!2$ of the polynomial ring $\mathbb{Q}[X]$.
%%%%%
%%%%%
\end{document}
