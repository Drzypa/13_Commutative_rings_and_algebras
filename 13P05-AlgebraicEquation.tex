\documentclass[12pt]{article}
\usepackage{pmmeta}
\pmcanonicalname{AlgebraicEquation}
\pmcreated{2013-03-22 15:14:07}
\pmmodified{2013-03-22 15:14:07}
\pmowner{PrimeFan}{13766}
\pmmodifier{PrimeFan}{13766}
\pmtitle{algebraic equation}
\pmrecord{7}{37006}
\pmprivacy{1}
\pmauthor{PrimeFan}{13766}
\pmtype{Definition}
\pmcomment{trigger rebuild}
\pmclassification{msc}{13P05}
\pmclassification{msc}{11C08}
\pmclassification{msc}{12E05}
\pmrelated{Equation}
\pmrelated{PolynomialEquationOfOddDegree}
\pmrelated{SymmetricQuarticEquation}
\pmdefines{degree of equation}

% this is the default PlanetMath preamble.  as your knowledge
% of TeX increases, you will probably want to edit this, but
% it should be fine as is for beginners.

% almost certainly you want these
\usepackage{amssymb}
\usepackage{amsmath}
\usepackage{amsfonts}

% used for TeXing text within eps files
%\usepackage{psfrag}
% need this for including graphics (\includegraphics)
%\usepackage{graphicx}
% for neatly defining theorems and propositions
 \usepackage{amsthm}
% making logically defined graphics
%%%\usepackage{xypic}

% there are many more packages, add them here as you need them

% define commands here

\theoremstyle{definition}
\newtheorem*{thmplain}{Theorem}
\begin{document}
The equation 
$$f(x_1,\,x_2,\,...,\,x_m) = 0,$$
where the left hand \PMlinkescapetext{side} is a polynomial in $x_1$, $x_2$, \ldots, $x_m$ with coefficients in a certain field, is called an {\em algebraic equation} over that field.\, Often the field in question is $\mathbb{Q}$; then the coefficients may be assumed to be integers.

By the {\em degree} of an algebraic equation is meant the degree of the polynomial.

E.g.\, $3x^2-1 = 0$\, and\, $x^3+x^2y+xy^2+y^3 = 0$\, are algebraic equations over the field $\mathbb{Q}$, the degrees of which are 2 and 3.
%%%%%
%%%%%
\end{document}
