\documentclass[12pt]{article}
\usepackage{pmmeta}
\pmcanonicalname{EveryPIDIsAUFDAlternativeProof}
\pmcreated{2013-03-22 19:04:26}
\pmmodified{2013-03-22 19:04:26}
\pmowner{joking}{16130}
\pmmodifier{joking}{16130}
\pmtitle{every PID is a UFD - alternative proof}
\pmrecord{5}{41959}
\pmprivacy{1}
\pmauthor{joking}{16130}
\pmtype{Theorem}
\pmcomment{trigger rebuild}
\pmclassification{msc}{13F07}
\pmclassification{msc}{16D25}
\pmclassification{msc}{13G05}
\pmclassification{msc}{11N80}
\pmclassification{msc}{13A15}

% this is the default PlanetMath preamble.  as your knowledge
% of TeX increases, you will probably want to edit this, but
% it should be fine as is for beginners.

% almost certainly you want these
\usepackage{amssymb}
\usepackage{amsmath}
\usepackage{amsfonts}

% used for TeXing text within eps files
%\usepackage{psfrag}
% need this for including graphics (\includegraphics)
%\usepackage{graphicx}
% for neatly defining theorems and propositions
%\usepackage{amsthm}
% making logically defined graphics
%%%\usepackage{xypic}

% there are many more packages, add them here as you need them

% define commands here

\begin{document}
\textbf{Proposition.} If $R$ is a principal ideal domain, then $R$ is a unique factorization domain.

\textit{Proof.} Recall, that due to Kaplansky Theorem (see \PMlinkname{this article}{EquivalentDefinitionsForUFD} for details) it is enough to show that every nonzero prime ideal in $R$ contains a prime element.

On the other hand, recall that an element $p\in R$ is prime if and only if an ideal $(p)$ generated by $p$ is nonzero and prime.

Thus, if $P$ is a nonzero prime ideal in $R$, then (since $R$ is a PID) there exists $p\in R$ such that $P=(p)$. This completes the proof. $\square$
%%%%%
%%%%%
\end{document}
