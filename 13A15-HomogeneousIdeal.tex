\documentclass[12pt]{article}
\usepackage{pmmeta}
\pmcanonicalname{HomogeneousIdeal}
\pmcreated{2013-03-22 11:45:00}
\pmmodified{2013-03-22 11:45:00}
\pmowner{archibal}{4430}
\pmmodifier{archibal}{4430}
\pmtitle{homogeneous ideal}
\pmrecord{11}{30190}
\pmprivacy{1}
\pmauthor{archibal}{4430}
\pmtype{Definition}
\pmcomment{trigger rebuild}
\pmclassification{msc}{13A15}
\pmclassification{msc}{33C75}
\pmclassification{msc}{33E05}
\pmclassification{msc}{86A30}
\pmclassification{msc}{14H52}
\pmclassification{msc}{14J27}
%\pmkeywords{commutative algebra}
%\pmkeywords{algebraic geometry}
\pmrelated{GradedRing}
\pmrelated{ProjectiveVariety}
\pmrelated{HomogeneousElementsOfAGradedRing}
\pmrelated{HomogeneousPolynomial}
\pmdefines{homogeneous}
\pmdefines{homogeneous element}

\endmetadata

\usepackage{amssymb}
\usepackage{amsmath}
\usepackage{amsfonts}
\usepackage{graphicx}
%%%%\usepackage{xypic}
\begin{document}
Let $R = \oplus_{g\in G} R_g$ be a graded ring.  Then an element $r$ of $R$ is said to be \emph{homogeneous} if it is an element of some $R_g$.  An ideal $I$ of $R$ is said to be homogeneous if it can be generated by a set of homogeneous elements, or equivalently if it is the ideal generated by the set of elements $\bigcup_{g\in G} I\cap R_g$.

One observes that if $I$ is a homogeneous ideal and $r=\sum_i r_{g_i}$ is the sum of homogeneous elements $r_{g_i}$ for distinct $g_i$, then each $r_{g_i}$ must be in $I$. 

To see some examples, let $k$ be a field, and take $R=k[X_1,X_2,X_3]$ with the usual grading by total degree.  Then the ideal generated by $X_1^n+X_2^n-X_3^n$ is a homogeneous ideal.  It is also a radical ideal.  One reason homogeneous ideals in $k[X_1,\ldots,X_n]$ are of interest is because (if they are radical) they define projective varieties; in this case the projective variety is the \PMlinkname{Fermat}{FermatsLastTheorem} curve.  For contrast, the ideal generated by $X_1+X_2^2$ is not homogeneous.
%%%%%
%%%%%
%%%%%
%%%%%
\end{document}
