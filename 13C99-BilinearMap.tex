\documentclass[12pt]{article}
\usepackage{pmmeta}
\pmcanonicalname{BilinearMap}
\pmcreated{2013-03-22 15:35:47}
\pmmodified{2013-03-22 15:35:47}
\pmowner{yark}{2760}
\pmmodifier{yark}{2760}
\pmtitle{bilinear map}
\pmrecord{11}{37510}
\pmprivacy{1}
\pmauthor{yark}{2760}
\pmtype{Definition}
\pmcomment{trigger rebuild}
\pmclassification{msc}{13C99}
\pmsynonym{bilinear function}{BilinearMap}
\pmsynonym{bilinear operation}{BilinearMap}
\pmsynonym{bilinear mapping}{BilinearMap}
\pmsynonym{bilinear operator}{BilinearMap}
\pmsynonym{bilinear pairing}{BilinearMap}
\pmsynonym{pairing}{BilinearMap}
\pmrelated{Multilinear}
\pmrelated{BilinearForm}
\pmrelated{ScalarMap}
\pmdefines{bilinear}


\begin{document}
\PMlinkescapeword{operation}

Let $R$ be a ring, and let $M$, $N$ and $P$ be modules over $R$.
A function $f\colon M\times N\to P$
is said to be a \emph{bilinear map}
if for each $b\in M$ the function $h\colon N\to P$
defined by $h(y)=f(b,y)$ for all $y\in N$ is \PMlinkname{linear}{LinearTransformation}
(that is, an $R$-module homomorphism),
and for each $c\in N$ the function $g\colon M\to P$
defined by $g(x)=f(x,c)$ for all $x\in M$ is linear.
Sometimes we may say that the function is \emph{$R$-bilinear},
\PMlinkescapetext{in order to make the base ring clear}.

A common case is a bilinear map $V\times V\to V$,
where $V$ is a vector space over a field $K$;
the vector space with this operation then forms an algebra over $K$.

If $R$ is a commutative ring, then every $R$-bilinear map $M\times N\to P$
corresponds in a natural way to a linear map $M\otimes N\to P$,
where $M\otimes N$ is the tensor product of $M$ and $N$ (over $R$).
%%%%%
%%%%%
\end{document}
