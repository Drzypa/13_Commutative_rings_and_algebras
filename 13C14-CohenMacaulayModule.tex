\documentclass[12pt]{article}
\usepackage{pmmeta}
\pmcanonicalname{CohenMacaulayModule}
\pmcreated{2013-03-22 14:14:58}
\pmmodified{2013-03-22 14:14:58}
\pmowner{mathcam}{2727}
\pmmodifier{mathcam}{2727}
\pmtitle{Cohen-Macaulay module}
\pmrecord{6}{35696}
\pmprivacy{1}
\pmauthor{mathcam}{2727}
\pmtype{Definition}
\pmcomment{trigger rebuild}
\pmclassification{msc}{13C14}
\pmclassification{msc}{16E65}
\pmdefines{Cohen-Macaulay ring}
\pmdefines{C-M module}
\pmdefines{C-M ring}

\endmetadata

\usepackage{amssymb}
\usepackage{amsmath}
\usepackage{amsfonts}
\usepackage{amsthm}
\begin{document}
A module $M$ over a ring $R$ is a Cohen-Macaulay module if its depth is defined and equals its Krull dimension.  A ring is said to be Cohen-Macaulay (or just C-M) if it is a Cohen-Macaulay module viewed as a module over itself.

Cohen-Macaulay rings are used extensively in combinatorial geometry and commutative ring theory, and has applications to algebraic geometry as well.  For instance, a variety all of whose local rings are Cohen-Macaulay has, in a sense, nicer behaviour than an arbitrary singular variety.
%%%%%
%%%%%
\end{document}
