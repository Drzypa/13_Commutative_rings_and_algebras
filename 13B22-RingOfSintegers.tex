\documentclass[12pt]{article}
\usepackage{pmmeta}
\pmcanonicalname{RingOfSintegers}
\pmcreated{2013-03-22 15:57:27}
\pmmodified{2013-03-22 15:57:27}
\pmowner{alozano}{2414}
\pmmodifier{alozano}{2414}
\pmtitle{ring of $S$-integers}
\pmrecord{4}{37970}
\pmprivacy{1}
\pmauthor{alozano}{2414}
\pmtype{Definition}
\pmcomment{trigger rebuild}
\pmclassification{msc}{13B22}
\pmsynonym{ring of S-integers}{RingOfSintegers}

% this is the default PlanetMath preamble.  as your knowledge
% of TeX increases, you will probably want to edit this, but
% it should be fine as is for beginners.

% almost certainly you want these
\usepackage{amssymb}
\usepackage{amsmath}
\usepackage{amsthm}
\usepackage{amsfonts}

% used for TeXing text within eps files
%\usepackage{psfrag}
% need this for including graphics (\includegraphics)
%\usepackage{graphicx}
% for neatly defining theorems and propositions
%\usepackage{amsthm}
% making logically defined graphics
%%%\usepackage{xypic}

% there are many more packages, add them here as you need them

% define commands here

\newtheorem{thm}{Theorem}
\newtheorem*{defn}{Definition}
\newtheorem{prop}{Proposition}
\newtheorem{lemma}{Lemma}
\newtheorem{cor}{Corollary}

\theoremstyle{definition}
\newtheorem*{exa}{Example}

% Some sets
\newcommand{\Nats}{\mathbb{N}}
\newcommand{\Ints}{\mathbb{Z}}
\newcommand{\Reals}{\mathbb{R}}
\newcommand{\Complex}{\mathbb{C}}
\newcommand{\Rats}{\mathbb{Q}}
\newcommand{\Gal}{\operatorname{Gal}}
\newcommand{\Cl}{\operatorname{Cl}}
\begin{document}
\begin{defn}
Let $K$ be a number field and let $S$ be a finite set of absolute values of $K$, containing all archimedean valuations. The ring of $S$-integers of $K$, usually denoted by $R_S$, is the ring:
$$R_S=\{ k\in K : \nu(k)\geq 0 \text{ for all valuations } \nu \notin S \}.$$
\end{defn}

Notice that, for any set $S$ as above, the ring of integers of $K$, $\mathcal{O}_K$, is always contained in $R_S$.

\begin{exa}
Let $K=\Rats$ and let $S=\{\nu_p,|\cdot|\}$ where $p$ is a prime and $\nu_p$ is the usual $p$-adic valuation, and $|\cdot|$ is the usual absolute value. Then
$$R_S=\Ints\left[\frac{1}{p}\right]$$
, i.e. $R_S$ is the result of adjoining (as a new ring element) $1/p$ to $\Ints$ (i.e. we allow to invert $p$).  
\end{exa}
%%%%%
%%%%%
\end{document}
