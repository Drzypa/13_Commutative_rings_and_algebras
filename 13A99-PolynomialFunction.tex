\documentclass[12pt]{article}
\usepackage{pmmeta}
\pmcanonicalname{PolynomialFunction}
\pmcreated{2013-03-22 15:40:34}
\pmmodified{2013-03-22 15:40:34}
\pmowner{pahio}{2872}
\pmmodifier{pahio}{2872}
\pmtitle{polynomial function}
\pmrecord{13}{37617}
\pmprivacy{1}
\pmauthor{pahio}{2872}
\pmtype{Definition}
\pmcomment{trigger rebuild}
\pmclassification{msc}{13A99}
\pmsynonym{ring of polynomial functions}{PolynomialFunction}
\pmrelated{NotationInSetTheory}
\pmrelated{ProductAndQuotientOfFunctionsSum}
\pmrelated{ZeroOfPolynomial}
\pmrelated{PolynomialFunctionIsAProperMap}
\pmrelated{DerivativeOfPolynomial}
\pmdefines{polynomial function}

% this is the default PlanetMath preamble.  as your knowledge
% of TeX increases, you will probably want to edit this, but
% it should be fine as is for beginners.

% almost certainly you want these
\usepackage{amssymb}
\usepackage{amsmath}
\usepackage{amsfonts}

% used for TeXing text within eps files
%\usepackage{psfrag}
% need this for including graphics (\includegraphics)
%\usepackage{graphicx}
% for neatly defining theorems and propositions
 \usepackage{amsthm}
% making logically defined graphics
%%%\usepackage{xypic}

% there are many more packages, add them here as you need them

% define commands here

\theoremstyle{definition}
\newtheorem*{thmplain}{Theorem}
\begin{document}
\textbf{Definition.}\, Let $R$ be a commutative ring.\, A function \,$f: R\to R$\, is called a {\em polynomial function of $R$}, if there are some elements \,$a_0,\,a_1,\,\ldots,\,a_m$ of $R$ such that
   $$f(x) \;=\; a_0\!+\!a_1x\!+\cdots+\!a_mx^m \,\,\,\, \forall x\in R.$$

\textbf{Remark.}\, The coefficients $a_i$ in a polynomial function need not be unique; e.g. if\, $R = \{0,\,1\}$\, is the ring (and field) of two elements, then the polynomials $X$ and $X^2$ both may be used for the same polynomial function.\, However, if we stipulate that $R$ is an infinite integral domain, the coefficients are guaranteed to be unique.

The set of all polynomial functions of $R$, being a subset of the set $R^R$ of all functions from $R$ to $R$, is here denoted by\, $R/^R$.

\begin{thmplain}
If $R$ is a commutative ring, then the set $R/^R$ of all polynomial functions of $R$, equipped with the operations 
\begin{align}
(f\!+\!g)(x) \;:=\; f(x)\!+\!g(x), \quad (f\!\cdot\!g)(x) \;:=\; f(x)g(x) \quad \forall x\in R,
\end{align}
is a commutative ring.
\end{thmplain}

{\em Proof.}\, It's straightforward to show that the function set $R^R$ forms a commutative ring when equipped with the operations ``$+$'' and ``$\cdot$'' defined as (1).\, We show now that $R/^R$ forms a subring of $R^R$.\, Let $f$ and $g$ be any two polynomial functions given by
$$f(x) \;=\; a_0\!+\!a_1x\!+\cdots+\!a_mx^m, \,\,\, g(x) \;=\; b_0\!+\!b_1x\!+\cdots+\!b_nx^n.$$
Then we can give $f\!+\!g$ by
  $$(f\!+\!g)(x) \;=\; \sum_{i=0}^k(a_i\!+\!b_i)x^i$$
where\, $k = \max\{m,\,n\}$\, and\, $a_i = 0$ (resp.\, $b_i = 0$) for\, $i > m$ (resp.\, $i > n$).\, This means that\, $f\!+\!g \in R/^R$.\, Secondly, the equation
$$(f\!\cdot\!g)(x) \;=\; a_0b_0+(a_0b_1\!+\!a_1b_0)x+(a_0b_2\!+\!a_1b_1\!+\!a_2b_0)x^2\!+\cdots+\!a_mb_nx^{m+n}$$
signifies that\, $f\!\cdot\!g \in R/^R$.\, Because also the function $-\!f$ given by
  $$(-\!f)(x) \;=\; -\!a_0\!-\!a_1x\!-\cdots-\!a_mx^m$$
and satisfying\, $-\!f\!+\!f = 0:x\mapsto 0$\, belongs to $R/^R$, the subset 
$R/^R$ is a subring of $R^R$.
%%%%%
%%%%%
\end{document}
