\documentclass[12pt]{article}
\usepackage{pmmeta}
\pmcanonicalname{CharacterizationOfPrimeIdeals}
\pmcreated{2013-03-22 15:22:01}
\pmmodified{2013-03-22 15:22:01}
\pmowner{GrafZahl}{9234}
\pmmodifier{GrafZahl}{9234}
\pmtitle{characterization of prime ideals}
\pmrecord{9}{37192}
\pmprivacy{1}
\pmauthor{GrafZahl}{9234}
\pmtype{Result}
\pmcomment{trigger rebuild}
\pmclassification{msc}{13C05}
\pmclassification{msc}{16D25}
\pmsynonym{characterisation of prime ideals}{CharacterizationOfPrimeIdeals}
%\pmkeywords{prime}
%\pmkeywords{ideal}
%\pmkeywords{monoid}
%\pmkeywords{semigroup}
\pmrelated{Localization}
\pmrelated{QuotientRingModuloPrimeIdeal}

% this is the default PlanetMath preamble.  as your knowledge
% of TeX increases, you will probably want to edit this, but
% it should be fine as is for beginners.

% almost certainly you want these
\usepackage{amssymb}
\usepackage{amsmath}
\usepackage{amsfonts}

% used for TeXing text within eps files
%\usepackage{psfrag}
% need this for including graphics (\includegraphics)
%\usepackage{graphicx}
% for neatly defining theorems and propositions
\usepackage{amsthm}
% making logically defined graphics
%%%\usepackage{xypic}

% there are many more packages, add them here as you need them

% define commands here
\newcommand{\<}{\langle}
\renewcommand{\>}{\rangle}
\newcommand{\Bigcup}{\bigcup\limits}
\newcommand{\DirectSum}{\bigoplus\limits}
\newcommand{\Prod}{\prod\limits}
\newcommand{\Sum}{\sum\limits}
\newcommand{\h}{\widehat}
\newcommand{\mbb}{\mathbb}
\newcommand{\mbf}{\mathbf}
\newcommand{\mc}{\mathcal}
\newcommand{\mmm}[9]{\left(\begin{array}{rrr}#1&#2&#3\\#4&#5&#6\\#7&#8&#9\end{array}\right)}
\newcommand{\mf}{\mathfrak}
\newcommand{\ol}{\overline}

% Math Operators/functions
\DeclareMathOperator{\Aut}{Aut}
\DeclareMathOperator{\End}{End}
\DeclareMathOperator{\Frob}{Frob}
\DeclareMathOperator{\cwe}{cwe}
\DeclareMathOperator{\id}{id}
\DeclareMathOperator{\mult}{mult}
\DeclareMathOperator{\we}{we}
\DeclareMathOperator{\wt}{wt}
\begin{document}
\PMlinkescapeword{type}
\newtheorem{thm}{Theorem}
This entry gives a number of equivalent \PMlinkid{characterizations}{5865}
of prime ideals in rings of different generality.

We start with a general ring $R$.

\begin{thm}
\label{thm:nounit}
Let $R$ be a ring and $P\subsetneq R$ a two-sided ideal. Then the
following statements are equivalent:
\begin{enumerate}
\item\label{i:a1} Given (left, right or two-sided) ideals $I,J$ of $P$
  such that the product of ideals $IJ\subseteq P$, then $I\subseteq P$
  or $J\subseteq P$.
\item\label{i:a2} If $x,y\in R$ such that $xRy\subseteq P$, then $x\in
  P$ or $y\in P$.
\end{enumerate}
\end{thm}
\begin{proof}
\begin{itemize}
\item ``\ref{i:a1}$\Rightarrow$\ref{i:a2}'':

Let $x,y\in R$ such that $xRy\subseteq P$. Let $(x)$ and $(y)$ be the
(left, right or two-sided) ideals generated by $x$ and $y$,
respectively. Then each element of the product of ideals $(x)R(y)$ can
be expanded to a finite sum of products each of which contains or is a
factor of the form $\pm xry$ for a suitable $r\in R$. Since $P$ is an
ideal and $xRy\subseteq P$, it follows that $(x)R(y)\subseteq
P$. Assuming statement~\ref{i:a1}, we have $(x)\subseteq P$,
$R\subseteq P$ or $(y)\subseteq P$. But $P\subsetneq R$, so we have
$(x)\subseteq P$ or $(y)\subseteq P$ and hence $x\in P$ or $y\in P$.

\item ``\ref{i:a2}$\Rightarrow$\ref{i:a1}'':

Let $I,J$ be (left, right or two-sided) ideals, such that the product
of ideals $IJ\subseteq P$. Now $RJ\subseteq J$ or $IR\subseteq I$
(depending on what type of ideal we consider), so $IRJ\subseteq
IJ\subseteq P$. If $I\subseteq P$, nothing remains to be shown. Otherwise,
let $i\in I\setminus P$, then $iRj\subseteq P$ for all $j\in J$. Since
$i\notin P$ we have by statement~\ref{i:a2} that $j\in P$ for all $j\in
J$, hence $J\subseteq P$.
\end{itemize}
\end{proof}


There are some additional properties if our ring is commutative.

\begin{thm}
\label{thm:comm}
Let $R$ a commutative ring and $P\subsetneq R$ an ideal. Then the
following statements are equivalent:
\begin{enumerate}
\item\label{i:b1}Given ideals $I,J$ of $P$ such that the product of
ideals $IJ\subseteq P$, then $I\subseteq P$ or $J\subseteq P$.
\item\label{i:b2}The quotient ring $R/P$ is a cancellation ring.
\item\label{i:b3}The set $R\setminus P$ is a subsemigroup of the
multiplicative semigroup of $R$.
\item\label{i:b4}Given $x,y\in R$ such that $xy\in P$, then $x\in P$
or $y\in P$.
\item\label{i:b5} The ideal $P$ is maximal in the set of such ideals of $R$ which do not intersect a subsemigroup $S$ of the multiplicative semigroup of $R$.

\end{enumerate}
\end{thm}

\begin{proof}
\begin{itemize}
\item ``\ref{i:b1}$\Rightarrow$\ref{i:b2}'':

Let $\bar{x},\bar{y}\in R/P$ be arbitrary nonzero elements. Let $x$
and $y$ be representatives of $\bar{x}$ and $\bar{y}$, respectively,
then $x\notin P$ and $y\notin P$. Since $R$ is commutative, each
element of the product of ideals $(x)(y)$ can be written as a product
involving the factor $xy$. Since $P$ is an ideal, we would have
$(x)(y)\subseteq P$ if $xy\in P$ which by statement~\ref{i:b1} would
imply $(x)\subseteq P$ or $(y)\subseteq P$ in contradiction with
$x\notin P$ and $y\notin P$. Hence, $xy\notin P$ and thus
$\bar{x}\bar{y}\neq 0$.

\item ``\ref{i:b2}$\Rightarrow$\ref{i:b3}'':

Let $x,y\in R\setminus P$. Let $\pi\colon R\to R/P$ be the canonical
projection. Then $\pi(x)$ and $\pi(y)$ are nonzero elements of
$R/P$. Since $\pi$ is a homomorphism and due to statement~\ref{i:b2},
$\pi(x)\pi(y)=\pi(xy)\neq 0$. Therefore $xy\notin P$, that is
$R\setminus P$ is closed under multiplication. The associative
property is inherited from $R$.

\item ``\ref{i:b3}$\Rightarrow$\ref{i:b4}'':

Let $x,y\in R$ such that $xy\in P$. If both $x,y$ were not elements of
$P$, then by statement~\ref{i:b3} $xy$ would not be an element of
$P$. Therefore at least one of $x,y$ is an element of $P$.

\item ``\ref{i:b4}$\Rightarrow$\ref{i:b1}'':

Let $I,J$ be ideals of $R$ such that $IJ\subseteq P$. If $I\subseteq
P$, nothing remains to be shown. Otherwise, let $i\in I\setminus
P$. Then for all $j\in J$ the product $ij\in IJ$, hence $ij\in P$. It
follows by statement~\ref{i:b4} that $j\in P$, and therefore
$J\subseteq P$.

\item ``\ref{i:b4}$\Rightarrow$\ref{i:b5}'':

The condition~\ref{i:b4} \PMlinkescapetext{means} that the set\, $S = R\setminus P$\, is a multiplicative semigroup.\, Now $P$ is trivially the greatest ideal which does not intersect $S$.

\item ``\ref{i:b5}$\Rightarrow$\ref{i:b4}'':

We presume that $P$ is maximal of the ideals of $R$ which do not intersect a semigroup $S$ and that\, $xy\in P$.\, Assume the contrary of the assertion, i.e. that\, $x\notin P$\, and\, $y\notin P$.\, Therefore, $P$ is a proper subset of both\, $(P,\,x)$\, and\, $(P,\,y)$.\, Thus the maximality of $P$ implies that
   $$(P,\,x)\cap S \neq\{\}, \quad (P,\,y)\cap S \neq\{\}.$$
So we can choose the elements $s_1$ and $s_2$ of $S$ such that
     $$s_1 = p_1+r_1x+n_1x, \quad s_2 = p_2+r_2y+n_2y,$$
where\, $p_1,\,p_2\in P$,\,\, $r_1,\,r_2\in R$\, and\, $n_1,\,n_2\in \mathbb{Z}$.\, Then we see that the product
$$s_1s_2 = (p_1+r_2y+n_2y)p_1+(r_1x+n_1x)p_2+(r_1r_2+n_2r_1+n_1r_2)xy+(n_1n_2)xy$$
would belong to the ideal $P$.\, But this is impossible because $s_1s_2$ is an element of the multiplicative semigroup $S$ and $P$ does not intersect $S$.\, Thus we can conclude that either $x$ or $y$ belongs to the ideal $P$.

\end{itemize}
\end{proof}



If $R$ has an identity element $1$, statements~\ref{i:b2} and~\ref{i:b3} of
the preceding theorem become stronger:

\begin{thm}
Let $R$ be a commutative ring with identity element $1$. Then an ideal
$P$ of $R$ is a prime ideal if and only if $R/P$ is an integral
domain. Furthermore, $P$ is prime if and only if $R\setminus P$ is a
monoid with identity element $1$ with respect to the multiplication in
$R$.
\end{thm}
\begin{proof}
Let $P$ be prime, then $1\notin P$ since otherwise $P$ would be equal
to $R$. Now by theorem~\ref{thm:comm} $R/P$ is a cancellation
ring. The canonical projection $\pi\colon R\to R/P$ is a homomorphism,
so $\pi(1)$ is the identity element of $R/P$. This in turn implies that
the semigroup $R\setminus P$ is a monoid with identity element $1$.
\end{proof}
%%%%%
%%%%%
\end{document}
