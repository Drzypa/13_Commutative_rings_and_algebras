\documentclass[12pt]{article}
\usepackage{pmmeta}
\pmcanonicalname{PlaceOfField}
\pmcreated{2013-03-22 14:56:51}
\pmmodified{2013-03-22 14:56:51}
\pmowner{pahio}{2872}
\pmmodifier{pahio}{2872}
\pmtitle{place of field}
\pmrecord{16}{36640}
\pmprivacy{1}
\pmauthor{pahio}{2872}
\pmtype{Theorem}
\pmcomment{trigger rebuild}
\pmclassification{msc}{13F30}
\pmclassification{msc}{13A18}
\pmclassification{msc}{12E99}
\pmsynonym{place}{PlaceOfField}
\pmsynonym{spot of field}{PlaceOfField}
\pmrelated{KrullValuation}
\pmrelated{ValuationDeterminedByValuationDomain}
\pmrelated{IntegrityCharacterizedByPlaces}
\pmrelated{RamificationOfArchimedeanPlaces}
\pmdefines{place of field}

\endmetadata

% this is the default PlanetMath preamble.  as your knowledge
% of TeX increases, you will probably want to edit this, but
% it should be fine as is for beginners.

% almost certainly you want these
\usepackage{amssymb}
\usepackage{amsmath}
\usepackage{amsfonts}

% used for TeXing text within eps files
%\usepackage{psfrag}
% need this for including graphics (\includegraphics)
%\usepackage{graphicx}
% for neatly defining theorems and propositions
 \usepackage{amsthm}
% making logically defined graphics
%%%\usepackage{xypic}

% there are many more packages, add them here as you need them

% define commands here
\theoremstyle{definition}
\newtheorem*{thmplain}{Theorem}
\begin{document}
Let $F$ be a field and $\infty$ an element not belonging to $F$.\, The mapping
               $$\varphi: \,k\to F\cup\{\infty\},$$
where $k$ is a field, is called a {\em place of the field} $k$, if it satisfies the following conditions.
\begin{itemize}
 \item The preimage\, $\varphi^{-1}(F) = \mathfrak{o}$\, is a subring of $k$.
 \item The restriction\, $\varphi|_\mathfrak{o}$\, is a ring homomorphism from $\mathfrak{o}$ to $F$.
 \item If\, $\varphi(a) = \infty$,\, then\, $\varphi(a^{-1}) = 0$.
\end{itemize}

It is easy to see that the subring $\mathfrak{o}$ of the field $k$ is a valuation domain; so any place of a field determines a unique valuation domain in the field.\, Conversely, every valuation domain $\mathfrak{o}$ with field of fractions $k$ determines a place of $k$:

\begin{thmplain}
\,Let $\mathfrak{o}$ be a valuation domain with field of fractions $k$ and $\mathfrak{p}$ the maximal ideal of $\mathfrak{o}$, consisting of the non-units of $\mathfrak{o}$.\, Then the mapping
   $$\varphi: \,k\to \mathfrak{o/p}\cup\{\infty\}$$
defined by
$$
\varphi(x):=
\begin{cases}
x+\mathfrak{p} \quad \mathrm{when} \,\,\, x \in\mathfrak{o}, \\
\infty \quad \mathrm{when} \,\,\, x \in k\smallsetminus\mathfrak{o},
\end{cases}
$$
is a place of the field $k$.
\end{thmplain}

{\em Proof.}\, Apparently,\, $\varphi^{-1}(\mathfrak{o/p}) = \mathfrak{o}$\, and the restriction\, $\varphi|_\mathfrak{o}$\, is the canonical homomorphism from the ring $\mathfrak{o}$ onto the residue-class ring $\mathfrak{o/p}$.\, Moreover, if\, $\varphi(x) = \infty$,\, then $x$ does not belong to the valuation domain $\mathfrak{o}$ and thus the inverse element $x^{-1}$ must belong to it without being its unit.\, Hence $x^{-1}$ belongs to the ideal $\mathfrak{p}$ which is the kernel of the homomorphism\, 
$\varphi|\mathfrak{o}$.\, So we see that\, $\varphi(x^{-1}) = 0$.

\begin{thebibliography}{7}
\bibitem{artin} Emil Artin: {\em \PMlinkescapetext{Theory of Algebraic Numbers}}.\, Lecture notes.\, Mathematisches Institut, G\"ottingen (1959).
\end{thebibliography}
%%%%%
%%%%%
\end{document}
