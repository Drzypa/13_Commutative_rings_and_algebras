\documentclass[12pt]{article}
\usepackage{pmmeta}
\pmcanonicalname{AnyDivisorIsGcdOfTwoPrincipalDivisors}
\pmcreated{2013-03-22 17:59:37}
\pmmodified{2013-03-22 17:59:37}
\pmowner{pahio}{2872}
\pmmodifier{pahio}{2872}
\pmtitle{any divisor is gcd of two principal divisors}
\pmrecord{5}{40505}
\pmprivacy{1}
\pmauthor{pahio}{2872}
\pmtype{Theorem}
\pmcomment{trigger rebuild}
\pmclassification{msc}{13A05}
\pmclassification{msc}{13A18}
\pmclassification{msc}{12J20}
\pmrelated{TwoGeneratorProperty}
\pmrelated{SumOfIdeals}

% this is the default PlanetMath preamble.  as your knowledge
% of TeX increases, you will probably want to edit this, but
% it should be fine as is for beginners.

% almost certainly you want these
\usepackage{amssymb}
\usepackage{amsmath}
\usepackage{amsfonts}

% used for TeXing text within eps files
%\usepackage{psfrag}
% need this for including graphics (\includegraphics)
%\usepackage{graphicx}
% for neatly defining theorems and propositions
 \usepackage{amsthm}
% making logically defined graphics
%%%\usepackage{xypic}

% there are many more packages, add them here as you need them

% define commands here

\theoremstyle{definition}
\newtheorem*{thmplain}{Theorem}

\begin{document}
Using the exponent valuations, one can easily prove the

\textbf{Theorem.}\, In any divisor theory, each divisor is the greatest common divisor of two principal divisors.

{\em Proof.}\, Let\, $\mathcal{O}^* \to \mathfrak{D}$\, be a divisor theory and $\mathfrak{d}$ an arbitrary divisor in $\mathfrak{D}$.\, We may suppose that $\mathfrak{d}$ is not a principal divisor (if $\mathfrak{D}$ contains exclusively principal divisors, then\, $\mathfrak{d} = \gcd(\mathfrak{d},\,\mathfrak{d})$\, and the proof is ready).\, Let 
$$\mathfrak{d} = \prod_{i=1}^r\mathfrak{p}_i^{k_i}$$
where the $\mathfrak{p}_i$'s are pairwise distinct prime divisors and every $k_i > 0$.\, Then third condition in the theorem concerning divisors and exponents allows to choose an element $\alpha$ of the ring $\mathcal{O}$ such that
$$\nu_{\mathfrak{p}_1}(\alpha) = k_1,\;\;\ldots,\;\;\nu_{\mathfrak{p}_r}(\alpha) = k_r.$$
Let the principal divisor corresponding to $\alpha$ be
$$(\alpha) = \prod_{i=1}^r\mathfrak{p}_i^{k_i}\prod_{j=1}^s\mathfrak{q}_j^{l_j} = \mathfrak{dd}',$$
where the prime divisors $\mathfrak{q}_j$ are pairwise different among themselves and with the divisors $\mathfrak{p}_i$.\, We can then choose another element $\beta$ of $\mathcal{O}$ such that
$$\nu_{\mathfrak{p}_1}(\beta) = k_1,\;\;\ldots,\;\;\nu_{\mathfrak{p}_r}(\beta) = k_r,\;\; 
\nu_{\mathfrak{q}_1}(\beta) = \ldots = \nu_{\mathfrak{q}_s}(\beta) = 0.$$
Then we have\; $(\beta) = \mathfrak{dd}''$,\, where\, $\mathfrak{d}'' \in \mathfrak{D}$\, and
$$\gcd(\mathfrak{d}',\,\mathfrak{d}'') = \mathfrak{q}^0\cdots\mathfrak{q}^0 = \mathfrak{e} = (1).$$
The gcd of the principal divisors $(\alpha)$ and $(\beta)$ is apparently $\mathfrak{d}$, whence the proof is settled.
%%%%%
%%%%%
\end{document}
