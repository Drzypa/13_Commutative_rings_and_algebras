\documentclass[12pt]{article}
\usepackage{pmmeta}
\pmcanonicalname{PolynomialRingWhichIsPID}
\pmcreated{2013-03-22 17:53:04}
\pmmodified{2013-03-22 17:53:04}
\pmowner{pahio}{2872}
\pmmodifier{pahio}{2872}
\pmtitle{polynomial ring which is PID}
\pmrecord{8}{40367}
\pmprivacy{1}
\pmauthor{pahio}{2872}
\pmtype{Theorem}
\pmcomment{trigger rebuild}
\pmclassification{msc}{13P05}
\pmrelated{PolynomialRingOverFieldIsEuclideanDomain}

% this is the default PlanetMath preamble.  as your knowledge
% of TeX increases, you will probably want to edit this, but
% it should be fine as is for beginners.

% almost certainly you want these
\usepackage{amssymb}
\usepackage{amsmath}
\usepackage{amsfonts}

% used for TeXing text within eps files
%\usepackage{psfrag}
% need this for including graphics (\includegraphics)
%\usepackage{graphicx}
% for neatly defining theorems and propositions
 \usepackage{amsthm}
% making logically defined graphics
%%%\usepackage{xypic}

% there are many more packages, add them here as you need them

% define commands here

\theoremstyle{definition}
\newtheorem*{thmplain}{Theorem}

\begin{document}
\textbf{Theorem.}\, If a polynomial ring $D[X]$ over an integral domain $D$ is a principal ideal domain, then coefficient ring $D$ is a field. (Cf. the corollary 4 in the entry polynomial ring over a field.)\\

{\em Proof.}\, Let $a$ be any non-zero element of $D$.\, Then the ideal \,$(a,\,X)$\, of $D[X]$ is a principal ideal $(f(X))$ with $f(X)$ a \PMlinkname{non-zero polynomial}{ZeroPolynomial2}.\, Therefore, 
$$a \;=\; f(X)g(X), \quad X \;=\; f(X)h(X)$$
with $g(X)$ and $h(X)$ certain polynomials in $D[X]$.\, From these equations one infers that $f(X)$ is a \PMlinkescapetext{constant} polynomial $c$ and $h(X)$ is a first degree polynomial $b_0\!+\!b_1X$ ($b_1 \neq 0$).\, Thus we obtain the equation
$$cb_0+cb_1X \;=\; X,$$
which shows that $cb_1$ is the unity 1 of $D$.\, Thus\, $c = f(X)$\, is a unit of $D$, whence
$$(a,\,X) \;=\; (f(X)) \;=\; (1) \;=\; D[X].$$
So we can write 
$$1 \;=\; a\!\cdot\!u(X)+X\!\cdot\!v(X),$$
where\, $u(X),\,v(X) \in D[X]$.\, This equation cannot be possible without that $a$ times the constant term of $u(X)$ is the unity.\, Accordingly, $a$ has a multiplicative inverse in $D$.\, Because $a$ was arbitrary non-zero elenent of the integral domain $D$, $D$ is a field.

\begin{thebibliography}{9}
\bibitem{D.B.}{\sc David M. Burton}: {\em A first course in rings and ideals}. Addison-Wesley Publishing Company. Reading, Menlo Park, London, Don Mills (1970).
\end{thebibliography}
%%%%%
%%%%%
\end{document}
