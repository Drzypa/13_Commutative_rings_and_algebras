\documentclass[12pt]{article}
\usepackage{pmmeta}
\pmcanonicalname{ZeroDivisor}
\pmcreated{2013-03-22 12:49:59}
\pmmodified{2013-03-22 12:49:59}
\pmowner{cvalente}{11260}
\pmmodifier{cvalente}{11260}
\pmtitle{zero divisor}
\pmrecord{9}{33157}
\pmprivacy{1}
\pmauthor{cvalente}{11260}
\pmtype{Definition}
\pmcomment{trigger rebuild}
\pmclassification{msc}{13G05}
\pmrelated{CancellationRing}
\pmrelated{IntegralDomain}
\pmrelated{Unity}
\pmdefines{left zero divisor}
\pmdefines{right zero divisor}
\pmdefines{regular element}

\endmetadata

% this is the default PlanetMath preamble.  as your knowledge
% of TeX increases, you will probably want to edit this, but
% it should be fine as is for beginners.

% almost certainly you want these
\usepackage{amssymb}
\usepackage{amsmath}
\usepackage{amsfonts}

% used for TeXing text within eps files
%\usepackage{psfrag}
% need this for including graphics (\includegraphics)
%\usepackage{graphicx}
% for neatly defining theorems and propositions
%\usepackage{amsthm}
% making logically defined graphics
%%%\usepackage{xypic}

% there are many more packages, add them here as you need them

% define commands here
\begin{document}
Let $a$ be a nonzero element of a ring $R$.

The element $a$ is a {\em left zero divisor} if there exists a nonzero element $b \in R$ such that $a \cdot b = 0$.  Similarly, $a$ is a {\em right zero divisor} if there exists a nonzero element $c \in R$ such that $c \cdot a = 0$.

The element $a$ is said to be a {\em zero divisor} if it is both a left and right zero divisor.  A nonzero element $a \in R$ is said to be a {\em regular element} if it is neither a left nor a right zero divisor.

{\bf Example:}
Let $R = \mathbb{Z}_6$.  Then the elements $2$ and $3$ are zero divisors, since $2 \cdot 3 \equiv 6 \equiv 0 \pmod 6$.
%%%%%
%%%%%
\end{document}
