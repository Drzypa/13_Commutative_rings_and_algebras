\documentclass[12pt]{article}
\usepackage{pmmeta}
\pmcanonicalname{ProofThatAEuclideanDomainIsAPID}
\pmcreated{2013-03-22 12:43:11}
\pmmodified{2013-03-22 12:43:11}
\pmowner{rm50}{10146}
\pmmodifier{rm50}{10146}
\pmtitle{proof that a Euclidean domain is a PID}
\pmrecord{7}{33015}
\pmprivacy{1}
\pmauthor{rm50}{10146}
\pmtype{Result}
\pmcomment{trigger rebuild}
\pmclassification{msc}{13F07}
\pmrelated{PID}
\pmrelated{UFD}
\pmrelated{IntegralDomain}
\pmrelated{EuclideanValuation}

\endmetadata

% this is the default PlanetMath preamble.  as your knowledge
% of TeX increases, you will probably want to edit this, but
% it should be fine as is for beginners.

% almost certainly you want these
\usepackage{amssymb}
\usepackage{amsmath}
\usepackage{amsfonts}

% used for TeXing text within eps files
%\usepackage{psfrag}
% need this for including graphics (\includegraphics)
%\usepackage{graphicx}
% for neatly defining theorems and propositions
%\usepackage{amsthm}
% making logically defined graphics
%%%\usepackage{xypic}

% there are many more packages, add them here as you need them

% define commands here
\begin{document}
Let $D$ be a Euclidean domain, and let $\mathfrak{a} \subseteq D$ be a nonzero ideal. We show that $\mathfrak{a}$ is principal. Let
\[ A = \{\nu(x) : x \in \mathfrak{a}, x \neq 0\} \]
be the set of Euclidean valuations of the non-zero elements of $\mathfrak{a}$. Since $A$ is a non-empty set of non-negative integers, it has a minimum $m$. Choose $d\in \mathfrak{a}$ such that $\nu(d) = m$. Claim that $\mathfrak{a} = (d)$. Clearly $(d) \subseteq \mathfrak{a}$. To see the reverse inclusion, choose $x\in \mathfrak{a}$. Since $D$ is a Euclidean domain, there exist elements $y,r\in D$ such that
\[ x = yd + r \]
with $\nu(r) < \nu(d)$ or $r = 0$. Since $r \in \mathfrak{a}$ and $\nu(d)$ is minimal in $A$, we must have $r = 0$. Thus $d \lvert x$ and $x\in(d)$.
%%%%%
%%%%%
\end{document}
