\documentclass[12pt]{article}
\usepackage{pmmeta}
\pmcanonicalname{ValuationDomainIsLocal}
\pmcreated{2013-03-22 14:54:49}
\pmmodified{2013-03-22 14:54:49}
\pmowner{pahio}{2872}
\pmmodifier{pahio}{2872}
\pmtitle{valuation domain is local}
\pmrecord{10}{36599}
\pmprivacy{1}
\pmauthor{pahio}{2872}
\pmtype{Theorem}
\pmcomment{trigger rebuild}
\pmclassification{msc}{13F30}
\pmclassification{msc}{13G05}
\pmclassification{msc}{16U10}
\pmrelated{ValuationRing}
\pmrelated{ValuationDeterminedByValuationDomain}
\pmrelated{HenselianField}

\endmetadata

% this is the default PlanetMath preamble.  as your knowledge
% of TeX increases, you will probably want to edit this, but
% it should be fine as is for beginners.

% almost certainly you want these
\usepackage{amssymb}
\usepackage{amsmath}
\usepackage{amsfonts}

% used for TeXing text within eps files
%\usepackage{psfrag}
% need this for including graphics (\includegraphics)
%\usepackage{graphicx}
% for neatly defining theorems and propositions
 \usepackage{amsthm}
% making logically defined graphics
%%%\usepackage{xypic}

% there are many more packages, add them here as you need them

% define commands here
\theoremstyle{definition}
\newtheorem*{thmplain}{Theorem}
\begin{document}
\begin{thmplain}
\, \,Every valuation domain is a local ring.
\end{thmplain}

{\em Proof.} \,Let $R$ be a valuation domain and $K$ its field of fractions. \,We shall show that the set of all non-units of $R$ is the only maximal ideal of $R$.

Let $a$ and $b$ first be such elements of $R$ that $a-b$ is a unit of $R$; we may suppose that \,$ab \neq 0$\, since otherwise one of $a$ and $b$ is instantly stated to be a unit. \,Because $R$ is a valuation domain in $K$, therefore e.g. \,$\frac{a}{b}\in R$. \,Because now \,$\frac{a-b}{b} = 1-\frac{a}{b}$\, and \,$(a-b)^{-1}$\, belong to $R$, so does also the product \,$\frac{a-b}{b}\cdot(a-b)^{-1} = \frac{1}{b}$, \,i.e. $b$ is a unit of $R$. \,We can conclude that the difference $a-b$ must be a non-unit whenever $a$ and $b$ are non-units.

Let $a$ and $b$ then be such elements of $R$ that $ab$ is its unit, i.e. \,$a^{-1}b^{-1}\in R$. \,Now we see that
 $$a^{-1} = b\cdot a^{-1}b^{-1}\in R,\,\,\,b^{-1} = a\cdot a^{-1}b^{-1}\in R ,$$
and consequently $a$ and $b$ both are units. \,So we conclude that the product $ab$ must be a non-unit whenever $a$ is an element of $R$ and $b$ is a non-unit.

Thus the non-units form an ideal $\mathfrak{m}$. \,Suppose now that there is another ideal $\mathfrak{n}$ of $R$ such that \,$\mathfrak{m}\subset\mathfrak{n}\subseteq R$. \,Since $\mathfrak{m}$ contains all non-units, we can take a unit $\varepsilon$ in $\mathfrak{n}$. \,Thus also the product $\varepsilon^{-1}\varepsilon$, i.e. 1, belongs to $\mathfrak{n}$, or \,$R\subseteq\mathfrak{n}$. \,So we see that $\mathfrak{m}$ is a maximal ideal. \,On the other hand, any maximal ideal of $R$ contains no units and hence is contained in $\mathfrak{m}$; therefore $\mathfrak{m}$ is the only maximal ideal.
%%%%%
%%%%%
\end{document}
