\documentclass[12pt]{article}
\usepackage{pmmeta}
\pmcanonicalname{TheRingOfIntegersOfANumberFieldIsFinitelyGeneratedOvermathbbZ}
\pmcreated{2013-03-22 15:08:22}
\pmmodified{2013-03-22 15:08:22}
\pmowner{alozano}{2414}
\pmmodifier{alozano}{2414}
\pmtitle{the ring of integers of a number field is finitely generated over $\mathbb{Z}$}
\pmrecord{7}{36883}
\pmprivacy{1}
\pmauthor{alozano}{2414}
\pmtype{Theorem}
\pmcomment{trigger rebuild}
\pmclassification{msc}{13B22}

% this is the default PlanetMath preamble.  as your knowledge
% of TeX increases, you will probably want to edit this, but
% it should be fine as is for beginners.

% almost certainly you want these
\usepackage{amssymb}
\usepackage{amsmath}
\usepackage{amsthm}
\usepackage{amsfonts}

% used for TeXing text within eps files
%\usepackage{psfrag}
% need this for including graphics (\includegraphics)
%\usepackage{graphicx}
% for neatly defining theorems and propositions
%\usepackage{amsthm}
% making logically defined graphics
%%%\usepackage{xypic}

% there are many more packages, add them here as you need them

% define commands here

\newtheorem*{thm}{Theorem}
\newtheorem{defn}{Definition}
\newtheorem{prop}{Proposition}
\newtheorem{lemma}{Lemma}
\newtheorem*{cor}{Corollary}

\theoremstyle{definition}
\newtheorem{exa}{Example}

% Some sets
\newcommand{\Nats}{\mathbb{N}}
\newcommand{\Ints}{\mathbb{Z}}
\newcommand{\Reals}{\mathbb{R}}
\newcommand{\Complex}{\mathbb{C}}
\newcommand{\Rats}{\mathbb{Q}}
\newcommand{\Gal}{\operatorname{Gal}}
\newcommand{\Cl}{\operatorname{Cl}}
\begin{document}
\begin{thm}
Let $K$ be a number field of degree $n$ over $\Rats$ and let $\mathcal{O}_K$ be the ring of integers of $K$. The ring $\mathcal{O}_K$ is a free abelian group of rank $n$. In other words, there exists a finite integral basis (with $n$ elements) for $K$, i.e. there exist algebraic integers $\alpha_1,\ldots,\ \alpha_n$ such that every element of $\mathcal{O}_K$ can be expressed uniquely as a $\Ints$-linear combination of the $\alpha_i$. 
\end{thm}

\begin{cor}
Every ideal of $\mathcal{O}_K$ is finitely generated.
\end{cor}
\begin{proof}[Proof of the corollary]
By the theorem, $\mathcal{O}_K$ is a free abelian group of rank $n$, and therefore it is finitely generated. Notice that an ideal is an additive subgroup. Finally a subgroup of a finitely generated free abelian group is also finitely generated.
\end{proof}

This is the first step to prove that $\mathcal{O}_K$ is a Dedekind domain. Notice that the field of fractions of $\mathcal{O}_K$ is the field $K$ itself. Therefore, by definition, $\mathcal{O}_K$ is integrally closed in $K$.
%%%%%
%%%%%
\end{document}
