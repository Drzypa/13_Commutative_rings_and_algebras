\documentclass[12pt]{article}
\usepackage{pmmeta}
\pmcanonicalname{FormalPowerSeriesOverField}
\pmcreated{2015-10-19 9:13:35}
\pmmodified{2015-10-19 9:13:35}
\pmowner{pahio}{2872}
\pmmodifier{pahio}{2872}
\pmtitle{formal power series over field}
\pmrecord{7}{42087}
\pmprivacy{1}
\pmauthor{pahio}{2872}
\pmtype{Theorem}
\pmcomment{trigger rebuild}
\pmclassification{msc}{13H05}
\pmclassification{msc}{13J05}
\pmclassification{msc}{13F25}

% this is the default PlanetMath preamble.  as your knowledge
% of TeX increases, you will probably want to edit this, but
% it should be fine as is for beginners.

% almost certainly you want these
\usepackage{amssymb}
\usepackage{amsmath}
\usepackage{amsfonts}

% used for TeXing text within eps files
%\usepackage{psfrag}
% need this for including graphics (\includegraphics)
%\usepackage{graphicx}
% for neatly defining theorems and propositions
 \usepackage{amsthm}
% making logically defined graphics
%%%\usepackage{xypic}

% there are many more packages, add them here as you need them

% define commands here

\theoremstyle{definition}
\newtheorem*{thmplain}{Theorem}

\begin{document}
\textbf{Theorem.}\, If $K$ is a field, then the ring $K[[X]]$ of formal power series is a discrete valuation ring with 
$(X)$ its unique maximal ideal.\\

\emph{Proof.}\, We show first that an arbitrary ideal $I$ of $K[[X]]$ is a principal ideal.\, If\, 
$I = (0)$,\, the thing is ready.\, Therefore, let\, $I \neq (0)$.\, Take an element 
$$f(X) \;:=\; \sum_{i=0}^\infty a_iX^i$$
of $I$ such that it has the least possible amount of successive zero coefficients in its beginning; let its first non-zero coefficient be $a_k$.\, Then
$$f(X) \;=\; X^k(a_k\!+\!a_{k+1}X\!+\ldots).$$
Here we have in the parentheses an invertible formal power series $g(X)$, whence get the equation
$$X^k \;=\; f(X)[g(X)]^{-1}$$
implying\, $X^k \in I$\, and consequently\, $(X^k) \subseteq I$.\\
For obtaining the reverse inclusion, suppose that
$$h(X) \;:=\; b_nX^n\!+\!b_{n+1}X^{n+1}\!+\ldots$$
is an arbitrary nonzero element of $I$ where\, $b_n \neq 0$.\, Because\, $n \ge k$,\, we may write
$$h(X) \;=\; X^k(b_nX^{n-k}\!+\!b_{n+1}X^{n-k+1}\!+\ldots).$$
This equation says that\, $h(X) \in (X^k)$,\, whence\, $I \subseteq (X^k)$.\\
Thus we have seen that $I$ is the principal ideal $(X^k)$, so that $K[[X]]$ is a principal ideal domain.\\
Now, all ideals of the ring $K[[X]]$ form apparently the strictly descending chain
$$(X) \;\supset\; (X^2) \;\supset\; (X^3) \;\supset\; \ldots \;\supset\; (0),$$
whence the ring has the unique maximal ideal $(X)$.\, A principal ideal domain with only one maximal ideal is a discrete valuation ring.





%%%%%
%%%%%
\end{document}
