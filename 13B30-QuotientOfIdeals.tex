\documentclass[12pt]{article}
\usepackage{pmmeta}
\pmcanonicalname{QuotientOfIdeals}
\pmcreated{2013-03-22 14:48:36}
\pmmodified{2013-03-22 14:48:36}
\pmowner{pahio}{2872}
\pmmodifier{pahio}{2872}
\pmtitle{quotient of ideals}
\pmrecord{20}{36468}
\pmprivacy{1}
\pmauthor{pahio}{2872}
\pmtype{Definition}
\pmcomment{trigger rebuild}
\pmclassification{msc}{13B30}
\pmsynonym{residual}{QuotientOfIdeals}
\pmsynonym{quotient ideal}{QuotientOfIdeals}
%\pmkeywords{fractional ideal}
\pmrelated{SumOfIdeals}
\pmrelated{ProductOfIdeals}
\pmrelated{Submodule}
\pmrelated{ArithmeticalRing}

% this is the default PlanetMath preamble.  as your knowledge
% of TeX increases, you will probably want to edit this, but
% it should be fine as is for beginners.

% almost certainly you want these
\usepackage{amssymb}
\usepackage{amsmath}
\usepackage{amsfonts}

% used for TeXing text within eps files
%\usepackage{psfrag}
% need this for including graphics (\includegraphics)
%\usepackage{graphicx}
% for neatly defining theorems and propositions
%\usepackage{amsthm}
% making logically defined graphics
%%%\usepackage{xypic}

% there are many more packages, add them here as you need them

% define commands here
\begin{document}
Let $R$ be a commutative ring having regular elements and let $T$ be its total ring of fractions.\, If $\mathfrak{a}$ and $\mathfrak{b}$ are fractional ideals of $R$, then one can define two different \PMlinkescapetext{{\em quotients}} or {\em residuals} of $\mathfrak{a}$ by $\mathfrak{b}$:
\begin{itemize}
 \item \, $\mathfrak{a\!:\!b}\, \;:=\; 
     \{r\in R|\quad r\mathfrak{b} \subseteq \mathfrak{a}\}$
 \item $[\mathfrak{a\!:\!b}] \;:=\; 
     \{t\in T|\quad t\mathfrak{b} \subseteq \mathfrak{a}\}$
\end{itemize}

They both are fractional ideals of $R$, and the former in fact an integral ideal of $R$.\, It is clear that 
$$\mathfrak{a\!:\!b} \;=\; [\mathfrak{a\!:\!b}]\cap\!R.$$
In the special case that $R$ has non-zero unity and $\mathfrak{b}$ has the inverse ideal $\mathfrak{b}^{-1}$, we have
$$[\mathfrak{a\!:\!b}] \;=\; \mathfrak{a}\mathfrak{b}^{-1},$$
in particular
$$[R\!:\!\mathfrak{b}] \;=\; \mathfrak{b}^{-1}.$$

Some rules concerning the former \PMlinkescapetext{type} of quotient (the corresponding rules are valid also for the latter \PMlinkescapetext{type}):
\begin{enumerate}
 \item $\mathfrak{a}\subseteq\mathfrak{b}\,\,\,\,\Rightarrow\,\,\,
\mathfrak{a}:\mathfrak{c}\subseteq\mathfrak{b}:\mathfrak{c}\,\,
\land\,\,\mathfrak{c}:\mathfrak{a}\supseteq\mathfrak{c}:\mathfrak{b}$
 \item $\mathfrak{a}:(\mathfrak{b}\mathfrak{c}) =    (\mathfrak{a}:\mathfrak{b}):\mathfrak{c}$ 
\item $\mathfrak{a}:(\mathfrak{b}+\mathfrak{c})= (\mathfrak{a}:\mathfrak{b})\cap(\mathfrak{a}:\mathfrak{c})$
 \item $(\mathfrak{a}\cap\mathfrak{b}):\mathfrak{c} = (\mathfrak{a}:\mathfrak{c})\cap(\mathfrak{b}:\mathfrak{c})$
\end{enumerate}


\textbf{Remark.} \,In a Pr\"ufer ring $R$ the \PMlinkname{addition}{SumOfIdeals} and intersection of ideals are dual operations of each other in the sense that there we have the duals

$\quad\quad \mathfrak{a}:(\mathfrak{b}\cap\mathfrak{c}) = (\mathfrak{a}:\mathfrak{b})+(\mathfrak{a}:\mathfrak{c})$

$\quad\quad (\mathfrak{a}+\mathfrak{b}):\mathfrak{c} = (\mathfrak{a}:\mathfrak{c})+(\mathfrak{b}:\mathfrak{c})$

of the two last rules if the \PMlinkescapetext{divisor ideals} are finitely generated.
%%%%%
%%%%%
\end{document}
