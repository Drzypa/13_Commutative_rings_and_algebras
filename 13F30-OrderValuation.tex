\documentclass[12pt]{article}
\usepackage{pmmeta}
\pmcanonicalname{OrderValuation}
\pmcreated{2013-03-22 16:53:28}
\pmmodified{2013-03-22 16:53:28}
\pmowner{pahio}{2872}
\pmmodifier{pahio}{2872}
\pmtitle{order valuation}
\pmrecord{19}{39147}
\pmprivacy{1}
\pmauthor{pahio}{2872}
\pmtype{Definition}
\pmcomment{trigger rebuild}
\pmclassification{msc}{13F30}
\pmclassification{msc}{13A18}
\pmclassification{msc}{12J20}
\pmclassification{msc}{11R99}
\pmsynonym{additive valuation}{OrderValuation}
\pmrelated{KrullValuation}
\pmrelated{Valuation}
\pmrelated{PAdicValuation}
\pmrelated{DiscreteValuation}
\pmrelated{ZerosAndPolesOfRationalFunction}
\pmrelated{AlternativeDefinitionOfValuation2}
\pmrelated{StrictDivisibility}
\pmrelated{ExponentValuation2}
\pmrelated{DivisibilityOfPrimePowerBinomialCoefficients}
\pmdefines{exponent of field}
\pmdefines{zero}
\pmdefines{zero of an element}
\pmdefines{pole}
\pmdefines{pole of an element}

\endmetadata

% this is the default PlanetMath preamble.  as your knowledge
% of TeX increases, you will probably want to edit this, but
% it should be fine as is for beginners.

% almost certainly you want these
\usepackage{amssymb}
\usepackage{amsmath}
\usepackage{amsfonts}

% used for TeXing text within eps files
%\usepackage{psfrag}
% need this for including graphics (\includegraphics)
%\usepackage{graphicx}
% for neatly defining theorems and propositions
 \usepackage{amsthm}
% making logically defined graphics
%%%\usepackage{xypic}

% there are many more packages, add them here as you need them

% define commands here

\theoremstyle{definition}
\newtheorem*{thmplain}{Theorem}

\begin{document}
Given a Krull valuation $|.|$ of a field $K$ as a mapping of $K$ to an ordered group $G$ (with operation ``$\cdot$'') equipped with $0$, one may use for the \PMlinkescapetext{valuation} an alternative notation ``ord'':

The \PMlinkescapetext{order} ``$<$'' of $G$ is reversed and the operation of $G$ is denoted by ``$+$''.\,  The element $0$ of $G$ is denoted as $\infty$, thus $\infty$ is greater than any other element of $G$.\, When we still call the valuation the {\em order} of $K$ and instead of 
$|x|$ write\, $\mathrm{ord}\,x$, the valuation postulates read as follows.
\begin{enumerate}
\item $\mathrm{ord}\,x \,=\, \infty$\,\, iff\,\, $x = 0$;
\item $\mathrm{ord}\,xy \,=\, \mathrm{ord}\,x+\mathrm{ord}\,y$;
\item $\mathrm{ord}(x+y) \,\geqq\, \min\{\mathrm{ord}\,x,\,\mathrm{ord}\,y\}$.
\end{enumerate}

We must emphasize that the order valuation is nothing else than a Krull valuation.\, The name {\em order} comes from complex analysis, where the ``places'' \PMlinkname{zero}{ZeroOfAFunction} and \PMlinkname{pole}{Pole} of a meromorphic function with their orders have a fully analogous meaning as have the corresponding concepts \PMlinkname{place}{PlaceOfField} and order valuation in the valuation theory.\, Thus also a place $\varphi$ of a field is called a {\em zero} of an element $a$ of the field, if\, $\varphi(a) = 0$,\, and a {\em pole} of an element $b$ of the field, if\, $\varphi(b) = \infty$;\, then e.g. the equation\, $\varphi(a) = 0$\, implies always that\, $\mathrm{ord}\,a > 0$.\\

\textbf{Example.}\, Let $p$ be a given positive prime number.\, Any non-zero rational number $x$ can be uniquely expressed in the form
$$x = p^nu,$$
in which $n$ is an integer and the rational number $u$ is by $p$ indivisible, i.e. when reduced to lowest terms, $p$ divides neither its numerator nor its denominator.\, If we define
\begin{align*}
\mathrm{ord}_px \;=\; 
\begin{cases}
\infty\,\,\, \mathrm{for}\,\,\, x = 0,\\
 n\,\,\, \mathrm{for}\,\,\, x = p^nu \neq 0,
\end{cases}
\end{align*}
then the function $\mathrm{ord}_p$, defined in $\mathbb{Q}$, clearly satisfies the above postulates of the order valuation.


In [2], an order valuation having only integer values is called the {\em exponent of the field} ({\em der Exponent des K\"orpers}); this name apparently motivated by the exponent $n$ of $p$.\, Such an order valuation is a special case of the discrete valuation.\, Note, that an arbitrary order valuation need not be a discrete valuation, since the values need not be real numbers.


\begin{thebibliography}{9}
\bibitem{Artin} {\sc E. Artin}: {\em Theory of Algebraic Numbers}.\, Lecture notes.\, Mathematisches Institut, G\"ottingen (1959).
\bibitem{BS}{\sc S. Borewicz \& I. Safarevic}: {\em Zahlentheorie}.\, Birkh\"auser Verlag. Basel und Stuttgart (1966).

\end{thebibliography}
%%%%%
%%%%%
\end{document}
