\documentclass[12pt]{article}
\usepackage{pmmeta}
\pmcanonicalname{HilbertsNullstellensatz}
\pmcreated{2013-03-22 13:03:59}
\pmmodified{2013-03-22 13:03:59}
\pmowner{rmilson}{146}
\pmmodifier{rmilson}{146}
\pmtitle{Hilbert's Nullstellensatz}
\pmrecord{8}{33474}
\pmprivacy{1}
\pmauthor{rmilson}{146}
\pmtype{Theorem}
\pmcomment{trigger rebuild}
\pmclassification{msc}{13A10}
\pmsynonym{Nullstellensatz}{HilbertsNullstellensatz}
%\pmkeywords{Nullstellensatz}
\pmrelated{RadicalOfAnIdeal}
\pmrelated{AlgebraicSetsAndPolynomialIdeals}
\pmdefines{zero set}
\pmdefines{Hilbert's Nullstellensatz}
\pmdefines{weak Nullstellensatz}

% this is the default PlanetMath preamble.  as your knowledge
% of TeX increases, you will probably want to edit this, but
% it should be fine as is for beginners.

% almost certainly you want these
\usepackage{amssymb}
\usepackage{amsmath}
\usepackage{amsfonts}

% used for TeXing text within eps files
%\usepackage{psfrag}
% need this for including graphics (\includegraphics)
%\usepackage{graphicx}
% for neatly defining theorems and propositions
%\usepackage{amsthm}
% making logically defined graphics
%%%\usepackage{xypic}

% there are many more packages, add them here as you need them

% define commands here
\begin{document}
Let $K$ be an algebraically closed field, and let $I$ be an ideal in $K[x_1,\ldots,x_n]$, the polynomial ring in $n$ indeterminates.

Define $V(I)$, the {\em zero set} of $I$, by 
\[ V(I) = \{(a_1,\ldots,a_n) \in K^n \mid f(a_1,\ldots,a_n)=0 \text{ for all } f \in I\}\]

{\bf Weak Nullstellensatz}:\\
If $V(I)=\emptyset$, then $I=K[x_1,\ldots,x_n]$.  In other words, the zero set of any proper ideal of $K[x_1,\ldots,x_n]$ is nonempty.

{\bf Hilbert's (Strong) Nullstellensatz}:\\
Suppose $f \in K[x_1,\ldots,x_n]$ satisfies $f(a_1,\ldots,a_n)=0$ for every $(a_1,\ldots,a_n) \in V(I)$.  Then $f^r \in I$ for some integer $r>0$.

In the \PMlinkescapetext{language} of algebraic geometry, the latter result is equivalent to the statement that $\operatorname{Rad}(I)=I(V(I))$, that is, the radical of $I$ is equal to the ideal of $V(I)$.
%%%%%
%%%%%
\end{document}
