\documentclass[12pt]{article}
\usepackage{pmmeta}
\pmcanonicalname{Nilradical}
\pmcreated{2013-03-22 12:47:52}
\pmmodified{2013-03-22 12:47:52}
\pmowner{djao}{24}
\pmmodifier{djao}{24}
\pmtitle{nilradical}
\pmrecord{4}{33114}
\pmprivacy{1}
\pmauthor{djao}{24}
\pmtype{Definition}
\pmcomment{trigger rebuild}
\pmclassification{msc}{13A10}
\pmrelated{PrimeRadical}
\pmrelated{JacobsonRadical}
\pmdefines{nilpotent}

\endmetadata

% this is the default PlanetMath preamble.  as your knowledge
% of TeX increases, you will probably want to edit this, but
% it should be fine as is for beginners.

% almost certainly you want these
\usepackage{amssymb}
\usepackage{amsmath}
\usepackage{amsfonts}

% used for TeXing text within eps files
%\usepackage{psfrag}
% need this for including graphics (\includegraphics)
%\usepackage{graphicx}
% for neatly defining theorems and propositions
%\usepackage{amsthm}
% making logically defined graphics
%%%\usepackage{xypic} 

% there are many more packages, add them here as you need them

% define commands here
\begin{document}
Let $R$ be a commutative ring. An element $x \in R$ is said to be {\em nilpotent} if $x^n = 0$ for some positive integer $n$. The set of all nilpotent elements of $R$ is an ideal of $R$, called the {\em nilradical} of $R$ and denoted $\operatorname{Nil}(R)$. The nilradical is so named because it is the radical of the zero ideal.

The nilradical of $R$ equals the prime radical of $R$, although proving that the two are equivalent requires the axiom of choice.
%%%%%
%%%%%
\end{document}
