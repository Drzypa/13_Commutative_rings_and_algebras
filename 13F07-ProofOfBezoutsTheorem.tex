\documentclass[12pt]{article}
\usepackage{pmmeta}
\pmcanonicalname{ProofOfBezoutsTheorem}
\pmcreated{2013-03-22 13:19:58}
\pmmodified{2013-03-22 13:19:58}
\pmowner{Thomas Heye}{1234}
\pmmodifier{Thomas Heye}{1234}
\pmtitle{proof of Bezout's Theorem}
\pmrecord{7}{33846}
\pmprivacy{1}
\pmauthor{Thomas Heye}{1234}
\pmtype{Proof}
\pmcomment{trigger rebuild}
\pmclassification{msc}{13F07}

% this is the default PlanetMath preamble.  as your knowledge
% of TeX increases, you will probably want to edit this, but
% it should be fine as is for beginners.

% almost certainly you want these
\usepackage{amssymb}
\usepackage{amsmath}
\usepackage{amsfonts}

% used for TeXing text within eps files
%\usepackage{psfrag}
% need this for including graphics (\includegraphics)
%\usepackage{graphicx}
% for neatly defining theorems and propositions
%\usepackage{amsthm}
% making logically defined graphics
%%%\usepackage{xypic}

% there are many more packages, add them here as you need them

% define commands here
\begin{document}
Let $D$ be an integral domain with an Euclidean valuation. Let $a,b \in D$ not both 0. Let $(a,b) =\{ax +by \vert x,y \in D\}$. $(a,b)$ is an ideal in $D \ne \{0\}$. We choose $d \in (a,b)$ such that $\mu(d)$ is the smallest positive value. Then $(a,b)$ is generated by $d$ and has the property $d \vert a$ and $d \vert b$. Two elements $x$ and $y$ in $D$ are associate if and only if $\mu(x) =\mu(y)$. So $d$ is unique up to a unit in $D$. Hence $d$ is the greatest common divisor of $a$ and $b$.
%%%%%
%%%%%
\end{document}
