\documentclass[12pt]{article}
\usepackage{pmmeta}
\pmcanonicalname{HomogeneousSystemOfParameters}
\pmcreated{2013-03-22 14:14:55}
\pmmodified{2013-03-22 14:14:55}
\pmowner{mathcam}{2727}
\pmmodifier{mathcam}{2727}
\pmtitle{homogeneous system of parameters}
\pmrecord{5}{35695}
\pmprivacy{1}
\pmauthor{mathcam}{2727}
\pmtype{Definition}
\pmcomment{trigger rebuild}
\pmclassification{msc}{13A02}
\pmdefines{partial homogeneous system of parameters}
\pmdefines{complete homogeneous system of parameters}
\pmdefines{homogeneous $M$-sequence}
\pmdefines{depth}
\pmdefines{depth of a module}

\endmetadata

% this is the default PlanetMath preamble.  as your knowledge
% of TeX increases, you will probably want to edit this, but
% it should be fine as is for beginners.

% almost certainly you want these
\usepackage{amssymb}
\usepackage{amsmath}
\usepackage{amsfonts}
\usepackage{amsthm}

% used for TeXing text within eps files
%\usepackage{psfrag}
% need this for including graphics (\includegraphics)
%\usepackage{graphicx}
% for neatly defining theorems and propositions
%\usepackage{amsthm}
% making logically defined graphics
%%%\usepackage{xypic}

% there are many more packages, add them here as you need them

% define commands here

\newcommand{\mc}{\mathcal}
\newcommand{\mb}{\mathbb}
\newcommand{\mf}{\mathfrak}
\newcommand{\ol}{\overline}
\newcommand{\ra}{\rightarrow}
\newcommand{\la}{\leftarrow}
\newcommand{\La}{\Leftarrow}
\newcommand{\Ra}{\Rightarrow}
\newcommand{\nor}{\vartriangleleft}
\newcommand{\Gal}{\text{Gal}}
\newcommand{\GL}{\text{GL}}
\newcommand{\Z}{\mb{Z}}
\newcommand{\R}{\mb{R}}
\newcommand{\Q}{\mb{Q}}
\newcommand{\C}{\mb{C}}
\newcommand{\<}{\langle}
\renewcommand{\>}{\rangle}
\begin{document}
Let $k$ be a field, let $R$ be an $\mb{N}^m$-\PMlinkname{graded}{GradedAlgebra} $k$-algebra, and let $M$ be a $\Z^m$-graded $R$-module.

Let $\mathcal{H}(R_+)$ be the homogeneous union of the irrelevant ideal of $R$.

A \emph{partial homogeneous system of parameters} for $M$ is a finite sequence of elements $\theta_1, \theta_2, \ldots, \theta_r\in\mathcal{H}(R_+)$ such that 

\begin{align*}
\dim\left(M/\left(\sum_{i=1}^r \theta_iM\right)\right)=\dim(M)-r,
\end{align*}

where $\dim$ gives the Krull dimension.

A (\PMlinkescapetext{complete}) \emph{homogeneous system of parameters} is a partial homogeneous system of parameters such that $r=\dim(M)$.

A sequence $\theta_1,\ldots,\theta_r\in\mathcal{H}(R_+)$ is a \emph{\PMlinkescapetext{homogeneous} $M$-sequence} if for all $i$ with $0\leq i<r$, we have that $\theta_{i+1}$ is not a zero-divisor in 

\begin{align*}
M/\left(\sum_{j=1}^i \theta_iM\right).
\end{align*}

Finally, view $M$ as being $\Z$-graded by using any specialization of the above $\Z^m$-grading.  Then we define the \emph{depth} of $M$ to be the length of the longest homogeneous $M$-sequence.

\begin{thebibliography}{9}
\bibitem{Stan} Richard P. Stanley, {\em Combinatorics and Commutative Algebra}, Second edition, Birkhauser Press.  Boston, MA.  1986.
\end{thebibliography}
%%%%%
%%%%%
\end{document}
