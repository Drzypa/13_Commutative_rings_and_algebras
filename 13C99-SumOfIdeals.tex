\documentclass[12pt]{article}
\usepackage{pmmeta}
\pmcanonicalname{SumOfIdeals}
\pmcreated{2013-03-22 14:39:26}
\pmmodified{2013-03-22 14:39:26}
\pmowner{pahio}{2872}
\pmmodifier{pahio}{2872}
\pmtitle{sum of ideals}
\pmrecord{22}{36250}
\pmprivacy{1}
\pmauthor{pahio}{2872}
\pmtype{Definition}
\pmcomment{trigger rebuild}
\pmclassification{msc}{13C99}
\pmclassification{msc}{16D25}
\pmclassification{msc}{08A99}
\pmsynonym{greatest common divisor of ideals}{SumOfIdeals}
\pmrelated{QuotientOfIdeals}
\pmrelated{ProductOfIdeals}
\pmrelated{LeastCommonMultiple}
\pmrelated{TwoGeneratorProperty}
\pmrelated{Submodule}
\pmrelated{AlgebraicLattice}
\pmrelated{LatticeOfIdeals}
\pmrelated{MaximalIdealIsPrime}
\pmrelated{AnyDivisorIsGcdOfTwoPrincipalDivisors}
\pmrelated{GcdDomain}
\pmdefines{sum ideal}
\pmdefines{sum of the ideals}
\pmdefines{addition of ideals}
\pmdefines{factor of ideal}
\pmdefines{greatest common divisor of ideals}
\pmdefines{least common multiple of ideals}

% this is the default PlanetMath preamble.  as your knowledge
% of TeX increases, you will probably want to edit this, but
% it should be fine as is for beginners.

% almost certainly you want these
\usepackage{amssymb}
\usepackage{amsmath}
\usepackage{amsfonts}

% used for TeXing text within eps files
%\usepackage{psfrag}
% need this for including graphics (\includegraphics)
%\usepackage{graphicx}
% for neatly defining theorems and propositions
%\usepackage{amsthm}
% making logically defined graphics
%%%\usepackage{xypic}

% there are many more packages, add them here as you need them

% define commands here
\begin{document}
\textbf{Definition.}\, Let's consider some set of ideals (left, right or two-sided) of a ring.\, The {\em sum of the ideals} is the smallest ideal of the ring containing all those ideals.\, The sum of ideals is denoted by using ``+'' and ``$\sum$'' as usually.\\


It is not difficult to be persuaded of the following:
\begin{itemize}
 \item The sum of a finite amount of ideals is
   $$\mathfrak{a}_1+\mathfrak{a}_2+\cdots+\mathfrak{a}_k \;=\; 
   \{a_1\!+\!a_2\!+\!\cdots\!+\!a_k\,\vdots \quad a_i \in \mathfrak{a}_i 
    \,\,\forall i\}.$$
 \item The sum of any set of ideals consists of all finite sums 
$\displaystyle\sum_j a_j$ where every $a_j$ belongs to one $\mathfrak{a}_j$ of those ideals.
\end{itemize}
Thus, one can say that the sum ideal is {\em generated by} the set of all elements of the individual ideals; in fact it suffices to use all generators of these ideals.\\

Let\, $\mathfrak{a}+\mathfrak{b} = \mathfrak{d}$\, in a ring $R$.\, Because\, $\mathfrak{a} \subseteq \mathfrak{d}$\, and\, $\mathfrak{b} \subseteq \mathfrak{d}$,\, we can say that $\mathfrak{d}$ is a \PMlinkescapetext{{\em factor} or {\em divisor}} of both $\mathfrak{a}$ and $\mathfrak{b}$.\footnote{This may be motivated by the situation in $\mathbb{Z}$:\, $(n) \subseteq (m)$\, iff\, $m$ is a factor of $n$.}\, Moreover, $\mathfrak{d}$ is contained in every common factor $\mathfrak{c}$ of $\mathfrak{a}$ and $\mathfrak{b}$ by virtue of its minimality.\, Hence, $\mathfrak{d}$ may be called the {\em greatest common divisor} of the ideals $\mathfrak{a}$ and $\mathfrak{b}$.\, The notations
$$\mathfrak{a}+\mathfrak{b} \;=\; \gcd(\mathfrak{a},\,\mathfrak{b}) \;=\; (\mathfrak{a}, \,\mathfrak{b})$$
are used, too.

In an analogous way, the intersection of ideals may be designated as the {\em least common \PMlinkescapetext{multiple}} of the ideals.\\

The by ``$\subseteq$'' partially ordered set of all ideals of a ring forms a lattice, where the least upper bound of $\mathfrak{a}$ and $\mathfrak{b}$ is\, $\mathfrak{a+b}$\, and the greatest lower bound is\, $\mathfrak{a\cap b}$.\, See also the example 3 in algebraic lattice.

%%%%%
%%%%%
\end{document}
