\documentclass[12pt]{article}
\usepackage{pmmeta}
\pmcanonicalname{DiscreteValuation}
\pmcreated{2013-03-22 13:59:14}
\pmmodified{2013-03-22 13:59:14}
\pmowner{djao}{24}
\pmmodifier{djao}{24}
\pmtitle{discrete valuation}
\pmrecord{6}{34761}
\pmprivacy{1}
\pmauthor{djao}{24}
\pmtype{Definition}
\pmcomment{trigger rebuild}
\pmclassification{msc}{13F30}
\pmclassification{msc}{12J20}
\pmsynonym{rank one valuations}{DiscreteValuation}
\pmsynonym{rank-one valuations}{DiscreteValuation}
\pmrelated{DiscreteValuationRing}
\pmrelated{Valuation}

\endmetadata

% this is the default PlanetMath preamble.  as your knowledge
% of TeX increases, you will probably want to edit this, but
% it should be fine as is for beginners.

% almost certainly you want these
\usepackage{amssymb}
\usepackage{amsmath}
\usepackage{amsfonts}

% used for TeXing text within eps files
%\usepackage{psfrag}
% need this for including graphics (\includegraphics)
%\usepackage{graphicx}
% for neatly defining theorems and propositions
%\usepackage{amsthm}
% making logically defined graphics
%%%\usepackage{xypic} 

% there are many more packages, add them here as you need them

% define commands here
\newcommand{\R}{\mathbb{R}}
\begin{document}
A {\em discrete valuation} on a field $K$ is a valuation $|\cdot|: K \to \R$ whose image is a discrete subset of $\R$.

For any field $K$ with a discrete valuation $|\cdot|$, the set
$$
R := \{x \in K : |x| \leq 1\}
$$
is a subring of $K$ with sole maximal ideal
$$
M := \{x \in K : |x| < 1\},
$$
and hence $R$ is a discrete valuation ring. Conversely, given any discrete valuation ring $R$, the field of fractions $K$ of $R$ admits a discrete valuation sending each element $x \in R$ to $c^n$, where $0 < c < 1$ is some arbitrary fixed constant and $n$ is the order of $x$, and extending multiplicatively to $K$.

{\bf Note:} Discrete valuations are often written additively instead of multiplicatively; under this alternate viewpoint, the element $x$ maps to $\log_c|x|$ (in the above notation) instead of just $|x|$. This transformation reverses the order of the absolute values (since $c < 1$), and sends the element $0 \in K$ to $\infty$. It has the advantage that every valuation can be normalized by a suitable scalar multiple to take values in the integers.
%%%%%
%%%%%
\end{document}
