\documentclass[12pt]{article}
\usepackage{pmmeta}
\pmcanonicalname{RingOfExponent}
\pmcreated{2013-03-22 17:59:43}
\pmmodified{2013-03-22 17:59:43}
\pmowner{pahio}{2872}
\pmmodifier{pahio}{2872}
\pmtitle{ring of exponent}
\pmrecord{13}{40507}
\pmprivacy{1}
\pmauthor{pahio}{2872}
\pmtype{Definition}
\pmcomment{trigger rebuild}
\pmclassification{msc}{13F30}
\pmclassification{msc}{13A18}
\pmclassification{msc}{12J20}
\pmclassification{msc}{11R99}
\pmrelated{DiscreteValuationRing}
\pmrelated{ValuationRingOfAField}
\pmrelated{LocalRing}
\pmdefines{ring of an exponent}
\pmdefines{ring of the exponent}
\pmdefines{integral with respect to an exponent}

% this is the default PlanetMath preamble.  as your knowledge
% of TeX increases, you will probably want to edit this, but
% it should be fine as is for beginners.

% almost certainly you want these
\usepackage{amssymb}
\usepackage{amsmath}
\usepackage{amsfonts}

% used for TeXing text within eps files
%\usepackage{psfrag}
% need this for including graphics (\includegraphics)
%\usepackage{graphicx}
% for neatly defining theorems and propositions
 \usepackage{amsthm}
% making logically defined graphics
%%%\usepackage{xypic}

% there are many more packages, add them here as you need them

% define commands here

\theoremstyle{definition}
\newtheorem*{thmplain}{Theorem}

\begin{document}
\PMlinkescapeword{exponents} \PMlinkescapeword{exponent}

\textbf{Definition.}\, Let $\nu$ be an exponent valuation of the field $K$.\,
The subring
$$\mathcal{O}_\nu \;:=\; \{\alpha \in K\,\vdots\;\; \nu(\alpha) \geqq 0\}$$
of $K$ is called the \PMlinkescapetext{{\em ring of the exponent}} $\nu$.\, It is, naturally, an integral domain.\, Its elements are called \PMlinkescapetext{{\em integral with respect to}} $\nu$.\\

\textbf{Theorem 1.}\, The ring of the exponent $\nu$ of the field $K$ is integrally closed in $K$.\\

\textbf{Theorem 2.}\, The ring $\mathcal{O}_\nu$ \PMlinkescapetext{contains} only one prime element $\pi$, when one does not regard associated elements as different.\, Any non-zero element $\alpha$ can be represented uniquely with a \PMlinkescapetext{fixed} $\pi$ in the form
$$\alpha \;=\; \varepsilon\pi^m,$$
where $\varepsilon$ is a unit of $\mathcal{O}_\nu$ and\, $m = \nu(\alpha) \geqq 0$.\, This means that $\mathcal{O}$ is a UFD.\\

\textbf{Remark 1.}\, The prime elements $\pi$ of the ring $\mathcal{O}_\nu$ are characterised by the equation \,$\nu(\pi) = 1$\, and the units \,$\varepsilon$ the equation \,$\nu(\varepsilon) = 0$.

\textbf{Remark 2.}\, In an algebraically closed field $\Omega$, there are no \PMlinkname{exponents}{ExponentValuation}.\, In fact, if there were an exponent $\nu$ of $\Omega$ and if $\pi$ were a prime element of the ring of the exponent, then, since the equation\, $x^2\!-\!\pi = 0$\, would have a \PMlinkname{root}{Equation} $\varrho$ in $\Omega$, we would obtain\; $2\nu(\varrho) = \nu(\varrho^2) =  \nu(\pi) = 1$;\; this is however impossible, because an exponent attains only integer values.\\

\textbf{Theorem 3.}\, Let\, $\mathfrak{O}_1,\,\ldots,\,\mathfrak{O}_r$ be the rings of the different exponent valuations $\nu_1,\,\ldots,\,\nu_r$ of the field $K$.\, Then also the intersection
$$\mathfrak{O} \;:=\; \bigcap_{i=1}^r\mathfrak{O}_i$$
is a subring of $K$ with \PMlinkname{unique factorisation}{UFD}.\, To be precise, any non-zero element $\alpha$ of $\mathfrak{O}$ may be uniquely represented in the form
$$\alpha \;=\; \varepsilon\pi_1^{n_1}\cdots\pi_r^{n_r},$$
in which $\varepsilon$ is a unit of $\mathfrak{O}$,\, the integers $n_1,\,\ldots,\,n_r$ are nonnegative and 
$\pi_1,\,\ldots,\,\pi_r$ are \PMlinkescapetext{fixed} coprime prime elements of $\mathfrak{O}$ satisfying
\[
\nu_i(\pi_j) \;=\; \delta_{ij} \;=\;  
\begin{cases}
& 1 \;\;\mbox{for  }\, i = j,\\
& 0 \;\;\mbox{for  }\, i \neq j.
\end{cases} 
\]

%%%%%
%%%%%
\end{document}
