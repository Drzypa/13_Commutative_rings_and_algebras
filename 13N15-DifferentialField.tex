\documentclass[12pt]{article}
\usepackage{pmmeta}
\pmcanonicalname{DifferentialField}
\pmcreated{2013-03-22 14:18:47}
\pmmodified{2013-03-22 14:18:47}
\pmowner{CWoo}{3771}
\pmmodifier{CWoo}{3771}
\pmtitle{differential field}
\pmrecord{10}{35778}
\pmprivacy{1}
\pmauthor{CWoo}{3771}
\pmtype{Definition}
\pmcomment{trigger rebuild}
\pmclassification{msc}{13N15}
\pmclassification{msc}{12H05}
\pmrelated{DifferentialPropositionalCalculus}
\pmdefines{differential ring}
\pmdefines{partial differential field}
\pmdefines{partial differential ring}
\pmdefines{field of constants}
\pmdefines{ring of constants}

\endmetadata

% this is the default PlanetMath preamble.  as your knowledge
% of TeX increases, you will probably want to edit this, but
% it should be fine as is for beginners.

% almost certainly you want these
\usepackage{amssymb}
\usepackage{amsmath}
\usepackage{amsfonts}

% used for TeXing text within eps files
%\usepackage{psfrag}
% need this for including graphics (\includegraphics)
%\usepackage{graphicx}
% for neatly defining theorems and propositions
%\usepackage{amsthm}
% making logically defined graphics
%%%\usepackage{xypic}

% there are many more packages, add them here as you need them

% define commands here
\def\sse{\subseteq}
\def\bigtimes{\mathop{\mbox{\Huge $\times$}}}
\def\impl{\Rightarrow}
\def\del{\partial}
\begin{document}
\PMlinkescapeword{term}%
\PMlinkescapeword{constant}%
\PMlinkescapeword{inversion}%
\PMlinkescapeword{place}%
%
Let $F$ be a field (ring) together with a derivation $(\cdot)' \colon F \to F$.
The derivation must satisfy two properties:
\begin{description}
  \item[Additivity] $(a+b)' = a' + b'$;
  \item[Leibniz' Rule] $(ab)' = a'b+ab'$.
\end{description}
A derivation is the algebraic abstraction of a derivative from ordinary
calculus. Thus the terms \emph{derivation}, \emph{derivative}, and
\emph{differential} are often used interchangeably.

Together, $(F,{}')$ is referred to as a \emph{differential field (ring)}.
The subfield (subring) of all elements with vanishing derivative,
$K=\{ a\in F \mid a'=0 \}$,
is called the \emph{field (ring) of constants}. Clearly, $(\cdot)'$ is $K$-linear.

There are many notations for the derivation symbol, for example $a'$ may
also be denoted as $da$, $\delta a$, $\del a$, etc. When there is more
than one derivation $\del_i$, $(F,\{\del_i\})$ is referred to as a
\emph{partial differential field (ring)}.

\section{Examples}
Differential fields and rings (together under the name of differential algebra)
are a natural setting for the study of algebraic properties of derivatives
and anti-derivatives (indefinite integrals), as well as ordinary and partial differential
equations and their solutions. There is an abundance of examples drawn
from these areas.

\begin{itemize}
\item
  The trivial example is a field $F$ with $a'=0$ for each $a\in F$. Here,
  nothing new is gained by introducing the derivation.
\item
  The most common example is the field of rational functions $\mathbb{R}(z)$
  over an indeterminant satisfying $z'=1$. The field of constants is
  $\mathbb{R}$. This is the setting for ordinary calculus.
\item
  Another example is $\mathbb{R}(x,y)$ with two derivations $\del_x$
  and $\del_y$. The field of constants is $\mathbb{R}$ and the
  derivations are extended to all elements from the properties $\del_x x=1$,
  $\del_y y = 1$, and $\del_x y = \del_y x = 0$.
\item
  Consider the set of smooth functions $C^\infty(M)$ on a manifold $M$. They
  form a ring (or a field if we allow formal inversion of functions
  vanishing in some places). Vector fields on $M$ act naturally as
  derivations on $C^\infty(M)$.
\item
  Let $A$ be an algebra and $U_t = \exp(tu)$ be a one-parameter subgroup of
  automorphisms of $A$. Here $u$ is the infinitesimal generator of these
  automorphisms. From the properties of $U_t$, $u$ must be a linear operator
  on $A$ that satisfies the Leibniz rule $u(ab)=u(a)b+au(b)$. So
  $(A,u)$ can be considered a differential ring.
\end{itemize}
%%%%%
%%%%%
\end{document}
