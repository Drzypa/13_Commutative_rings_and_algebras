\documentclass[12pt]{article}
\usepackage{pmmeta}
\pmcanonicalname{UFDsAreIntegrallyClosed}
\pmcreated{2013-03-22 15:49:25}
\pmmodified{2013-03-22 15:49:25}
\pmowner{rm50}{10146}
\pmmodifier{rm50}{10146}
\pmtitle{UFD's are integrally closed}
\pmrecord{6}{37790}
\pmprivacy{1}
\pmauthor{rm50}{10146}
\pmtype{Theorem}
\pmcomment{trigger rebuild}
\pmclassification{msc}{13G05}

\endmetadata

% this is the default PlanetMath preamble.  as your knowledge
% of TeX increases, you will probably want to edit this, but
% it should be fine as is for beginners.

% almost certainly you want these
\usepackage{amssymb}
\usepackage{amsmath}
\usepackage{amsfonts}

% used for TeXing text within eps files
%\usepackage{psfrag}
% need this for including graphics (\includegraphics)
%\usepackage{graphicx}
% for neatly defining theorems and propositions
%\usepackage{amsthm}
% making logically defined graphics
%%%\usepackage{xypic}

% there are many more packages, add them here as you need them

% define commands here
\begin{document}
\textbf{Theorem:} Every UFD is integrally closed.

\textbf{Proof:} Let $R$ be a UFD, $K$ its field of fractions, $u\in K, u$ integral over $R$. Then for some $c_0,\ldots,c_{n-1}\in R$,
\[u^n+c_{n-1}u^{n-1}+\ldots+c_0=0\]
Write $u=\frac{a}{b}, a,b\in R$, where $a,b$ have no non-unit common divisor (which we can assume since $R$ is a UFD). Multiply the above equation by $b^n$ to get
\[a^n+c_{n-1}ba^{n-1}+\ldots+c_0b^n=0\]
Let $d$ be an irreducible divisor of $b$. Then $d$ is prime since $R$ is a UFD. Now, $d\lvert a^n$ since it divides all the other terms and thus (since $d$ is prime) $d\lvert a$. But $a, b$ have no non-unit common divisors, so $d$ is a unit. Thus $b$ is a unit and hence $u\in R$.
%%%%%
%%%%%
\end{document}
