\documentclass[12pt]{article}
\usepackage{pmmeta}
\pmcanonicalname{FactoringAllonePolynomialsUsingTheGroupingMethod}
\pmcreated{2013-03-22 15:06:52}
\pmmodified{2013-03-22 15:06:52}
\pmowner{rspuzio}{6075}
\pmmodifier{rspuzio}{6075}
\pmtitle{factoring all-one polynomials using the grouping method}
\pmrecord{13}{36851}
\pmprivacy{1}
\pmauthor{rspuzio}{6075}
\pmtype{Example}
\pmcomment{trigger rebuild}
\pmclassification{msc}{13P05}
\pmrelated{AllOnePolynomial}
\pmrelated{CyclotomicPolynomial}

\endmetadata

% this is the default PlanetMath preamble.  as your knowledge
% of TeX increases, you will probably want to edit this, but
% it should be fine as is for beginners.

% almost certainly you want these
\usepackage{amssymb}
\usepackage{amsmath}
\usepackage{amsfonts}

% used for TeXing text within eps files
%\usepackage{psfrag}
% need this for including graphics (\includegraphics)
%\usepackage{graphicx}
% for neatly defining theorems and propositions
%\usepackage{amsthm}
% making logically defined graphics
%%%\usepackage{xypic}

% there are many more packages, add them here as you need them

% define commands here
\begin{document}
The method of grouping terms can be used to factor all-one polynomials, i.e. polynomials of the form
\[
\sum_{m=0}^{n-1} x^m
\]
when $n$ is composite. (When $n$ is prime, these polynomials are irreducible, so there is nothing to do in that case.)

Let us consider a few examples:

$n = 4$:
\begin{eqnarray*}
1 + x + x^2 + x^3 = \\
(1 + x) + (x^2 + x^3) = \\
(1 + x) + x^2 (1 + x) = \\
(1 + x) (1 + x^2)
\end{eqnarray*}

$n = 6$:
\begin{eqnarray*}
1 + x + x^2 + x^3 + x^4 + x^5 = \\
(1 + x + x^2) + (x^3 + x^4 + x^5) = \\
(1 + x + x^2) + x^3 (1 + x + x^2) = \\
(1 + x^3) (1 + x + x^2)
\end{eqnarray*}

$n = 8$:
\begin{eqnarray*}
1 + x + x^2 + x^3 + x^4 + x^5 + x^6 + x^7 = \\
(1 + x + x^2 + x^3) + (x^4 + (x^5 + x^6 + x^7) = \\
(1 + x + x^2 + x^3) + x^4 (1 + x + x^2 + x^3) =\\
(1 + x^4) (1 + x + x^2 + x^3)
\end{eqnarray*}
Combining this result with the factorization we have for the case $n=4$, we obtain the following:
\begin{eqnarray*}
1 + x + x^2 + x^3 + x^4 + x^5 + x^6 + x^7 = \\
(1 + x) (1 + x^2) (1 + x^4)
\end{eqnarray*}

$n = 9$:
\begin{eqnarray*}
1 + x + x^2 + x^3 + x^4 + x^5 + x^6 + x^7 + x^8 = \\
(1 + x + x^2) + (x^3 + x^4 + x^5) + (x^6 + x^7  + x^8) = \\
(1 + x + x^2) + x^3 (1 + x + x^2) + x^6 (1 + x + x^2) = \\
(1 + x + x^2) (1 + x^3 + x^6)
\end{eqnarray*}

$n = 12$:
\begin{eqnarray*}
1 + x + x^2 + x^3 + x^4 + x^5 + x^6 + x^7 + x^8 + x^9 + x^{10} + x^{11} = \\
(1 + x + x^2) + (x^3 + x^4 + x^5) + (x^6 + x^7 + x^8) + (x^9 + x^{10} + x^{11}) = \\
(1 + x + x^2) + x^3 (1 + x + x^2) + x^6 (1 + x + x^2) + x^9 (1 + x + x^2) = \\
(1 + x + x^2) (1 + x^3 + x^6 + x^9) = \\
(1 + x + x^2) ((1 + x^3) + (x^6 + x^9)) = \\
(1 + x + x^2) ((1 + x^3) + x^6 (1 + x^3)) = \\
(1 + x + x^2) (1 + x^3) (1 + x^6)
\end{eqnarray*}

It might be worth pointing out that the polynomials produced by this factorization are not all irreducible.  For instance,
\[ 1 + x^3 = (1 + x) (1 - x + x^2). \]
However, to obtain this factorization, one needs to use some techique other than the grouping method.  Likewise. the polynomial $1 + x^6$ is also reducible.
%%%%%
%%%%%
\end{document}
