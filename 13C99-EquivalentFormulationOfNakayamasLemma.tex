\documentclass[12pt]{article}
\usepackage{pmmeta}
\pmcanonicalname{EquivalentFormulationOfNakayamasLemma}
\pmcreated{2013-03-22 19:11:47}
\pmmodified{2013-03-22 19:11:47}
\pmowner{rm50}{10146}
\pmmodifier{rm50}{10146}
\pmtitle{equivalent formulation of Nakayama's lemma}
\pmrecord{4}{42109}
\pmprivacy{1}
\pmauthor{rm50}{10146}
\pmtype{Theorem}
\pmcomment{trigger rebuild}
\pmclassification{msc}{13C99}

\endmetadata

\usepackage{amssymb}
\usepackage{amsmath}
\usepackage{amsfonts}

% used for TeXing text within eps files
%\usepackage{psfrag}
% need this for including graphics (\includegraphics)
%\usepackage{graphicx}
% for neatly defining theorems and propositions
%\usepackage{amsthm}
% making logically defined graphics
%%%\usepackage{xypic}

% there are many more packages, add them here as you need them

% define commands here
\newcommand{\BQ}{\mathbb{Q}}
\newcommand{\BR}{\mathbb{R}}
\newcommand{\BZ}{\mathbb{Z}}
\newcommand{\sma}{\mathfrak{a}}
\begin{document}
The following is equivalent to Nakayama's lemma.

Let $A$ be a ring, $M$ be a finitely-generated $A$-module, $N$ a submodule of $M$, and $\sma$ an ideal of $A$ contained in its Jacobson radical. Then $M = \sma M + N \Rightarrow M=N$.

Clearly this statement implies Nakayama's Lemma, by setting $N$ to $0$. To see that it follows from Nakayama's Lemma, note first that by the second isomorphism theorem for modules,
\[
  \frac{\sma M+N}{N} = \frac{\sma M}{\sma M\cap N}
\]
and the obvious map
\[
  \sma M\to \sma\frac{M}{N} : am\mapsto a(m+N)
\]
is surjective; the kernel is clearly $\sma M\cap N$. Thus
\[
  \frac{\sma M+N}{N} \cong \sma\frac{M}{N}
\]
So from $M=\sma M+N$ we get $M/N = \sma(M/N)$. Since $\sma$ is contained in the Jacobson radical of $M$, it is contained in the Jacobson radical of $M/N$, so by Nakayama, $M/N=0$, i.e. $M=N$.


%%%%%
%%%%%
\end{document}
