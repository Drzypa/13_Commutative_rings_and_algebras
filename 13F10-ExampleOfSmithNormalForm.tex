\documentclass[12pt]{article}
\usepackage{pmmeta}
\pmcanonicalname{ExampleOfSmithNormalForm}
\pmcreated{2013-03-22 14:22:51}
\pmmodified{2013-03-22 14:22:51}
\pmowner{aoh45}{5079}
\pmmodifier{aoh45}{5079}
\pmtitle{example of Smith normal form}
\pmrecord{4}{35873}
\pmprivacy{1}
\pmauthor{aoh45}{5079}
\pmtype{Example}
\pmcomment{trigger rebuild}
\pmclassification{msc}{13F10}

\usepackage{amssymb}
\usepackage{amsmath}
\usepackage{amsfonts}

% need this for including graphics (\includegraphics)
%\usepackage{graphicx}
% for neatly defining theorems and propositions
%\usepackage{amsthm}
% making logically defined graphics
%%%\usepackage{xypic}
\begin{document}
As an example, we will find the Smith normal form of the following matrix over the integers.
\begin{equation*}
\left(\begin{array}{ccc}
2  & 4  & 4  \\
-6 & 6  & 12 \\
10 & -4 & -16
\end{array}\right)
\end{equation*}
The following matrices are the intermediate steps as the algorithm is applied to the above matrix.

\begin{equation*}
\left(\begin{array}{ccc}
2  & 0  & 0  \\
-6 & 18 & 24 \\
10 & -24& -36
\end{array}\right)
\to
\left(\begin{array}{ccc}
2  & 0  & 0  \\
0  & 18 & 24 \\
0  & -24& -36
\end{array}\right)
\end{equation*}

\begin{equation*}
\to
\left(\begin{array}{ccc}
2  & 0  & 0  \\
0  & 18 & 24 \\
0  & -6 & -12
\end{array}\right)
\to
\left(\begin{array}{ccc}
2  & 0  & 0  \\
0  & 6  & 12 \\
0  & 18 & 24
\end{array}\right)
\end{equation*}

\begin{equation*}
\to
\left(\begin{array}{ccc}
2  & 0  & 0  \\
0  & 6  & 12 \\
0  & 0  & -12
\end{array}\right)
\to
\left(\begin{array}{ccc}
2  & 0  & 0  \\
0  & 6  & 0  \\
0  & 0  & 12
\end{array}\right)
\end{equation*}

So the Smith normal form is

\begin{equation*}
\left(\begin{array}{ccc}
2  & 0  & 0  \\
0  & 6  & 0  \\
0  & 0  & 12
\end{array}\right)
\end{equation*}

and the elementary divisors are $2$, $6$ and $12$.
%%%%%
%%%%%
\end{document}
