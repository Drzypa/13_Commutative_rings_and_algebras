\documentclass[12pt]{article}
\usepackage{pmmeta}
\pmcanonicalname{IdealsOfADiscreteValuationRingArePowersOfItsMaximalIdeal}
\pmcreated{2013-03-22 18:00:47}
\pmmodified{2013-03-22 18:00:47}
\pmowner{rm50}{10146}
\pmmodifier{rm50}{10146}
\pmtitle{ideals of a discrete valuation ring are powers of its maximal ideal}
\pmrecord{9}{40528}
\pmprivacy{1}
\pmauthor{rm50}{10146}
\pmtype{Theorem}
\pmcomment{trigger rebuild}
\pmclassification{msc}{13H10}
\pmclassification{msc}{13F30}
\pmrelated{PAdicCanonicalForm}
\pmrelated{IdealDecompositionInDedekindDomain}

% this is the default PlanetMath preamble.  as your knowledge
% of TeX increases, you will probably want to edit this, but
% it should be fine as is for beginners.

% almost certainly you want these
\usepackage{amssymb}
\usepackage{amsmath}
\usepackage{amsfonts}

% used for TeXing text within eps files
%\usepackage{psfrag}
% need this for including graphics (\includegraphics)
%\usepackage{graphicx}
% for neatly defining theorems and propositions
\usepackage{amsthm}
% making logically defined graphics
%%%\usepackage{xypic}

% there are many more packages, add them here as you need them

% define commands here
\newcommand{\smm}{\mathfrak{m}}
\newtheorem{thm}{Theorem}
\newtheorem{cor}{Corollary}
\begin{document}
\begin{thm} Let $R$ be a discrete valuation ring. Then all nonzero ideals of $R$ are powers of its maximal ideal $\smm$.
\end{thm}

\textbf{Proof. } Let $\smm = (\pi)$ (that is, $\pi$ is a uniformizer for $R$). Assume that $R$ is not a field (in which case the result is trivial), so that $\pi\neq 0$. 
Let $I=(\alpha)\subset R$ be any ideal; claim $(\alpha)=\smm^k$ for some $k$. By the Krull intersection theorem, we have
\[\bigcap_{n\geq 0}\smm^n=(0)\]
so that we may choose $k\geq 0$ with $\alpha\in \smm^k-\smm^{k+1}$. Since $\alpha\in\smm^k$, we have $\alpha = u\pi^k$ for $u\in R$. $u\notin \smm$, since otherwise $\alpha\in\smm^{k+1}$, so that $\alpha$ is a unit (in a DVR, the maximal ideal consists precisely of the nonunits). Thus $(\alpha)=(\pi)^k$.

\begin{cor} Let $R$ be a Noetherian local ring with a principal maximal ideal. Then all nonzero ideals are powers of the maximal ideal $\smm$.
\end{cor}

\textbf{Proof. } Let $I=(\alpha_1,\ldots,\alpha_n)$ be an ideal of $R$. Then by the above argument, for each $i$, $\alpha_i = u_i\pi^{k_i}$ for $u_i$ a unit, and thus $I=(\pi^{k_1},\ldots,\pi^{k_n}) = (\pi^k)$ for $k=\min(k_1,\ldots,k_n)$.
%%%%%
%%%%%
\end{document}
