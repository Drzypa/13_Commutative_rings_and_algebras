\documentclass[12pt]{article}
\usepackage{pmmeta}
\pmcanonicalname{ImageIdealOfDivisor}
\pmcreated{2013-03-22 18:02:40}
\pmmodified{2013-03-22 18:02:40}
\pmowner{pahio}{2872}
\pmmodifier{pahio}{2872}
\pmtitle{image ideal of divisor}
\pmrecord{9}{40568}
\pmprivacy{1}
\pmauthor{pahio}{2872}
\pmtype{Theorem}
\pmcomment{trigger rebuild}
\pmclassification{msc}{13A15}
\pmclassification{msc}{13A05}
\pmclassification{msc}{11A51}
\pmdefines{image ideal}
\pmdefines{ideal determined by the divisor}

\endmetadata

% this is the default PlanetMath preamble.  as your knowledge
% of TeX increases, you will probably want to edit this, but
% it should be fine as is for beginners.

% almost certainly you want these
\usepackage{amssymb}
\usepackage{amsmath}
\usepackage{amsfonts}

% used for TeXing text within eps files
%\usepackage{psfrag}
% need this for including graphics (\includegraphics)
%\usepackage{graphicx}
% for neatly defining theorems and propositions
 \usepackage{amsthm}
 \usepackage[T2A]{fontenc}
 \usepackage[russian, english]{babel}

% making logically defined graphics
%%%\usepackage{xypic}

% there are many more packages, add them here as you need them

% define commands here

\theoremstyle{definition}
\newtheorem*{thmplain}{Theorem}
\begin{document}
\textbf{Theorem.}\, If an integral domain $\mathcal{O}$ has a divisor theory \,$\mathcal{O}^* \to \mathfrak{D}$,\, then the subset $[\mathfrak{a}]$ of $\mathcal{O}$, consisting of 0 and all elements divisible by a divisor $\mathfrak{a}$, is an ideal of $\mathcal{O}$.\, The mapping
$$\mathfrak{a} \mapsto [\mathfrak{a}]$$
from the set $\mathfrak{D}$ of divisors into the set of ideals of $\mathcal{O}$ is injective and maps any principal divisor $(\alpha)$ to the principal ideal $(\alpha)$.\\

{\em Proof.}\, Let\, $\alpha,\,\beta \in [\mathfrak{a}]$\, and\, $\lambda \in \mathcal{O}$.\, Then, by the postulate 2 of \PMlinkname{divisor theory}{DivisorTheory}, $\alpha\!-\!\beta$ is divisible by $\mathfrak{a}$ or is 0, and in both cases belongs to $[\mathfrak{a}]$.\, When\, $\lambda\alpha \neq 0$,\, we can write\, $(\alpha) = \mathfrak{ac}$\, with $\mathfrak{c}$ a divisor.\, According to the homomorphicity of the mapping \,$\mathcal{O}^* \to \mathfrak{D}$,\, we have
$$(\lambda\alpha) = (\lambda)(\alpha) = (\lambda)\mathfrak{ac},$$
and therefore the element $\lambda\alpha$ is divisible by $\mathfrak{a}$, i.e. $\lambda\alpha \in [\mathfrak{a}]$.\, Thus, $[\mathfrak{a}]$ is an ideal of $\mathcal{O}$.

The injectivity of the mapping\, $\mathfrak{a} \mapsto [\mathfrak{a}]$\, follows from the postulate 3 of \PMlinkname{divisor theory}{DivisorTheory}.\\



The ideal $[\mathfrak{a}]$ may be called the {\em image ideal} of $\mathfrak{a}$ or the {\em ideal determined by the divisor} $\mathfrak{a}$.\\

\textbf{Remark.}\, There are integral domains $\mathcal{O}$ having a divisor theory but also having ideals which are not of the form $[\mathfrak{a}]$ (for example a polynomial ring in two indeterminates and its ideal formed by the polynomials without constant term).\, Such rings have ``too many ideals''.\; On the other hand, in some integral domains the monoid of principal ideals cannot be embedded into a free monoid; thus those rings cannot have a divisor theory.

\begin{thebibliography}{9}
\bibitem{MMP} \CYRM. \CYRM. \CYRP\cyro\cyrs\cyrt\cyrn\cyri\cyrk\cyro\cyrv: 
{\em \CYRV\cyrv\cyre\cyrd\cyre\cyrn\cyri\cyre\, \cyrv\, \cyrt\cyre\cyro\cyrr\cyri\cyryu\, \cyra\cyrl\cyrg\cyre\cyrb\cyrr\cyra\cyri\cyrch\cyre\cyrs\cyrk\cyri\cyrh \,
\cyrch\cyri\cyrs\cyre\cyrl}. \,\CYRI\cyrz\cyrd\cyra\cyrt\cyre\cyrl\cyrsftsn\cyrs\cyrt\cyrv\cyro \,
``\CYRN\cyra\cyru\cyrk\cyra''. \CYRM\cyro\cyrs\cyrk\cyrv\cyra \,(1982).
\end{thebibliography}

%%%%%
%%%%%
\end{document}
