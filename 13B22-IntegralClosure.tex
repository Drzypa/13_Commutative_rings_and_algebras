\documentclass[12pt]{article}
\usepackage{pmmeta}
\pmcanonicalname{IntegralClosure}
\pmcreated{2013-03-22 12:07:53}
\pmmodified{2013-03-22 12:07:53}
\pmowner{djao}{24}
\pmmodifier{djao}{24}
\pmtitle{integral closure}
\pmrecord{8}{31299}
\pmprivacy{1}
\pmauthor{djao}{24}
\pmtype{Definition}
\pmcomment{trigger rebuild}
\pmclassification{msc}{13B22}
\pmrelated{IntegrallyClosed}
\pmdefines{ring of integers}

\usepackage{amssymb}
\usepackage{amsmath}
\usepackage{amsfonts}
\usepackage{graphicx}
%%%\usepackage{xypic}
\begin{document}
Let $B$ be a ring with a subring $A$. The {\em integral closure} of $A$ in $B$ is the set $A' \subset B$ consisting of all elements of $B$ which are integral over $A$.

It is a theorem that the integral closure of $A$ in $B$ is itself a ring. In the special case where $A = \mathbb{Z}$, the integral closure $A'$ of $\mathbb{Z}$ is often called the {\em ring of integers} in $B$.
%%%%%
%%%%%
%%%%%
\end{document}
