\documentclass[12pt]{article}
\usepackage{pmmeta}
\pmcanonicalname{ProofOfOstrowskisValuationTheorem}
\pmcreated{2013-03-22 17:58:26}
\pmmodified{2013-03-22 17:58:26}
\pmowner{rm50}{10146}
\pmmodifier{rm50}{10146}
\pmtitle{proof of Ostrowski's valuation theorem}
\pmrecord{4}{40482}
\pmprivacy{1}
\pmauthor{rm50}{10146}
\pmtype{Proof}
\pmcomment{trigger rebuild}
\pmclassification{msc}{13A18}

\endmetadata

% this is the default PlanetMath preamble.  as your knowledge
% of TeX increases, you will probably want to edit this, but
% it should be fine as is for beginners.

% almost certainly you want these
\usepackage{amssymb}
\usepackage{amsmath}
\usepackage{amsfonts}

% used for TeXing text within eps files
%\usepackage{psfrag}
% need this for including graphics (\includegraphics)
%\usepackage{graphicx}
% for neatly defining theorems and propositions
\usepackage{amsthm}
% making logically defined graphics
%%%\usepackage{xypic}

% there are many more packages, add them here as you need them

% define commands here
\newcommand{\Ints}{\mathbb{Z}}
\newcommand{\Rats}{\mathbb{Q}}
\newcommand{\suchthat}{\ \mid\ }
\newcommand{\smm}{\mathfrak{m}}
\newcommand{\U}[1]{#1^{\star}}
\newcommand{\Order}[1]{\lvert #1\rvert}
\newcommand{\Abs}[1]{\left\lvert #1\right\rvert}
% \theoremstyle{plain} %% This is the default
\newtheorem{thm}{Theorem}
\newtheorem{lem}[thm]{Lemma}

\begin{document}
This article proves Ostrowski's theorem on valuations of $\Rats$, which states:
\begin{thm} (Ostrowski) Over $\Rats$, every nontrivial absolute value is equivalent either to $\Abs{\cdot}_p$ for some prime $p$, or to the usual absolute value $\Abs{\cdot}_{\infty}$.
\end{thm}

We start with an estimation lemma:
\begin{lem} If $m,n>1$ are integers and $\Abs{\cdot}$ any nontrivial absolute value on $\Rats$, then $\Abs{m}\leq\max(1,\Abs n)^{\log m/\log n}$.
\end{lem}
\textbf{Proof. } Write $m=a_0+a_1n+\cdots+a_rn^r$ for $a_i\in\Ints, 0\leq a_i\leq n-1$, and with $a_r\neq 0$. Then clearly
\[
\Abs{a_i} = \Abs{\underset{a_i}{\underbrace{1+\cdots+1}}}\leq a_i\Abs 1 = a_i\leq n
\]
by the triangle inequality; also, $r < \frac{\log m}{\log n}$.

Thus
\begin{align*}
\Abs m &= \Abs{a_0+a_1n+\cdots+a_rn^r}\leq (r+1)n\max(1,\Abs n)^r\\
&\leq \left(1+\frac{\log m}{\log n}\right)n\max(1,\Abs n)^{\log m/\log n}
\end{align*}
Replace $m$ by $m^t$ for $t$ a positive integer, and take $t^{\mathrm{th}}$ roots of the resulting inequality, to get
\[\Abs m \leq \left(1+t\frac{\log m}{\log n}\right)^{1/t}n^{1/t}\max(1,\Abs n)^{\log m/\log n}\]
Now let $t\to\infty$; the first two factors each approach $1$, and the lemma follows.

\textbf{Proof of Ostrowski's theorem: }
\newline
First assume that for every $n>1$ we have $\Abs n>1$. Then by the lemma, $\Abs m\leq \Abs{n}^{\log m/\log n}$, so that for every $m,n$ we have
\[\Abs{m}^{1/\log m} \leq \Abs{n}^{1/\log n}\]
Since this holds for every $m,n>0$, after reversing the roles of $m,n$, we see that in fact equality holds, so that for every $m$, $\Abs{m}^{1/\log m}=c$ and $\Abs m=c^{\log m}$ for some constant $c$; this absolute value is obviously equivalent to $\Abs{m}_{\infty}=e^{\log m}$.

If instead, for some $n>1$ we have $\Abs n<1$, then by the lemma, for every $m$, $\Abs m \leq 1$. Thus the absolute value is nonarchimedean. Define $A=\{x\in \Rats\suchthat \Abs x\leq 1\}$ and let $\smm\subset A$ be the (unique) maximal ideal defined by $\smm = \{x\in\Rats\suchthat \Abs x<1\}$. Then $\Ints\subset A$ since $\Abs m \leq 1$ for every $m$, and $\smm\cap\Ints$ is nonzero since otherwise the valuation would be trivial (we would have $\Abs m = 1$ for every $m$). Thus $\smm\cap\Ints$ is prime since $\smm$ is, so is equal to $(p)$ for some rational prime $p$. Now, if $p\nmid a$ for an integer $a$, then $\Abs a$ cannot be strictly less than $1$ (else it would be in $(p)$), so $\Abs a =1$ and $a\in\U{A}$. But given any $x\in\Rats$, we can write $x=\frac{ap^t}{b}$ with $a,b$ prime to $p$, so that
\[\Abs x = \frac{\Abs a \cdot \Abs{p}^t}{\Abs b} = \Abs{p}^t\]
so that the valuation is obviously equivalent to the $p$-adic valuation.
%%%%%
%%%%%
\end{document}
