\documentclass[12pt]{article}
\usepackage{pmmeta}
\pmcanonicalname{ProductOfFinitelyGeneratedIdeals}
\pmcreated{2015-05-05 19:19:39}
\pmmodified{2015-05-05 19:19:39}
\pmowner{pahio}{2872}
\pmmodifier{pahio}{2872}
\pmtitle{product of finitely generated ideals}
\pmrecord{29}{37217}
\pmprivacy{1}
\pmauthor{pahio}{2872}
\pmtype{Definition}
\pmcomment{trigger rebuild}
\pmclassification{msc}{13A15}
\pmclassification{msc}{16D25}
\pmclassification{msc}{16D10}
\pmsynonym{special cases of ideal product}{ProductOfFinitelyGeneratedIdeals}
\pmrelated{PruferRing}
\pmrelated{IdealGeneratorsInPruferRing}
\pmrelated{IdealDecompositionInDedekindDomain}
\pmrelated{EntriesOnFinitelyGeneratedIdeals}
\pmrelated{UniqueFactorizationAndIdealsInRingOfIntegers}
\pmrelated{ContentOfAPolynomial}
\pmrelated{WellDefinednessOfProductOfFinitelyGeneratedIdeals}
\pmdefines{Dedekind--Mertens lemma}

% this is the default PlanetMath preamble.  as your knowledge
% of TeX increases, you will probably want to edit this, but
% it should be fine as is for beginners.

% almost certainly you want these
\usepackage{amssymb}
\usepackage{amsmath}
\usepackage{amsfonts}

% used for TeXing text within eps files
%\usepackage{psfrag}
% need this for including graphics (\includegraphics)
%\usepackage{graphicx}
% for neatly defining theorems and propositions
 \usepackage{amsthm}
% making logically defined graphics
%%%\usepackage{xypic}

% there are many more packages, add them here as you need them

% define commands here

\theoremstyle{definition}
\newtheorem*{thmplain}{Theorem}
\begin{document}
Let $R$ be a commutative ring having at least one regular element 
and $T$ its total ring of fractions.\, Let\, $\mathfrak{a} := 
(a_0,\,a_1,\,\ldots,\,a_{m-1})$\, and\, $\mathfrak{b} := 
(b_0,\,b_1,\,\ldots,\,b_{n-1})$\, be two fractional ideals of $R$ 
(see the entry ``fractional ideal of commutative ring'').\, Then 
the product submodule \,$\mathfrak{ab}$\, of $T$ is also a 
\PMlinkescapetext{fractional ideal} of $R$ and is generated by all the elements $a_ib_j$, thus having a generating set of\, $mn$\, elements.

Such a generating set may be condensed in the case of any Dedekind domain, especially for the \PMlinkescapetext{fractional ideals} of any algebraic number field one has the multiplication formula
\begin{align}
\mathfrak{ab} = (a_0b_0,\,a_0b_1\!+\!a_1b_0,\,a_0b_2\!+\!a_1b_1\!+\!a_2b_0,\,\ldots,\,a_{m-1}b_{n-1}).
\end{align}
Here, the number of generators is only\, $m\!+\!n\!-\!1$ (in principle, every ideal of a Dedekind domain has a \PMlinkname{generating system of two elements}{TwoGeneratorProperty}).\, The formula is \PMlinkname{characteristic}{Characterization} still for a wider class of rings $R$ which may contain zero divisors, viz. for the Pr\"ufer rings (see [1]), but then at least one of $\mathfrak{a}$ and $\mathfrak{b}$ must be a regular ideal.

Note that the generators in (1) are formed similarly as the coefficients in the product of the polynomials 
\,$f(X) := f_0\!+\!f_1X\!+\cdots+\!f_{m-1}X^{m-1}$\, and\, 
$g(X) := g_0\!+\!g_1X\!+\cdots+\!g_{n-1}X^{n-1}$.\, Thus we may call the fractional ideals $\mathfrak{a}$ and $\mathfrak{b}$ of $R$ the {\em coefficient modules} $\mathfrak{m}_f$ and $\mathfrak{m}_g$ of the polynomials $f$ and $g$ (they are $R$-modules).\, Hence the formula (1) may be rewritten as
\begin{align}
\mathfrak{m}_f\mathfrak{m}_g = \mathfrak{m}_{fg}.
\end{align}
This formula says the same as Gauss's lemma I for a unique factorization domain $R$.

Arnold and Gilmer [2] have presented and proved the following generalisation of (2) which is valid under much less stringent assumptions than the ones requiring $R$ to be a Pr\"ufer ring (initially: a Pr\"ufer domain); the proof is somewhat simplified in [1].

\textbf{Theorem (Dedekind--Mertens lemma)}.
\, Let $R$ be a subring of a commutative ring $T$.\, If $f$ and 
$g$ are two arbitrary polynomials in the polynomial ring $T[X]$, 
then there exists a non-negative integer $n$ such that the 
$R$-submodules of $T$ generated by the coefficients of the 
polynomials $f$, $g$ and $fg$ satisfy the equality
\begin{align}
\mathfrak{m}_f^{n+1}\,\mathfrak{m}_g = \mathfrak{m}_f^n\,\mathfrak{m}_{fg}.
\end{align}


\begin{thebibliography}{9}
 \bibitem{PJ}{\sc J. Pahikkala:} ``{Some formulae for multiplying and inverting ideals}''. \,-- {\em Ann. Univ. Turkuensis} \textbf{183} (A) (1982).
 \bibitem{A+G}{\sc J. Arnold \& R. Gilmer:} ``{On the contents of polynomials}''. \,-- {\em Proc. Amer. Math. Soc.} \textbf{24} (1970).
\end{thebibliography}
%%%%%
%%%%%
\end{document}
