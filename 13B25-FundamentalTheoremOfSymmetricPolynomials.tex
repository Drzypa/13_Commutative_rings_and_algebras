\documentclass[12pt]{article}
\usepackage{pmmeta}
\pmcanonicalname{FundamentalTheoremOfSymmetricPolynomials}
\pmcreated{2013-03-22 19:07:40}
\pmmodified{2013-03-22 19:07:40}
\pmowner{pahio}{2872}
\pmmodifier{pahio}{2872}
\pmtitle{fundamental theorem of symmetric polynomials}
\pmrecord{7}{42023}
\pmprivacy{1}
\pmauthor{pahio}{2872}
\pmtype{Theorem}
\pmcomment{trigger rebuild}
\pmclassification{msc}{13B25}
\pmclassification{msc}{12F10}
\pmsynonym{fundamental theorem of symmetric functions}{FundamentalTheoremOfSymmetricPolynomials}

% this is the default PlanetMath preamble.  as your knowledge
% of TeX increases, you will probably want to edit this, but
% it should be fine as is for beginners.

% almost certainly you want these
\usepackage{amssymb}
\usepackage{amsmath}
\usepackage{amsfonts}

% used for TeXing text within eps files
%\usepackage{psfrag}
% need this for including graphics (\includegraphics)
%\usepackage{graphicx}
% for neatly defining theorems and propositions
 \usepackage{amsthm}
% making logically defined graphics
%%%\usepackage{xypic}

% there are many more packages, add them here as you need them

% define commands here

\theoremstyle{definition}
\newtheorem*{thmplain}{Theorem}

\begin{document}
Every symmetric polynomial \,$P(x_1,\,x_2,\,\ldots,\,x_n)$\, in the indeterminates $x_1,\,x_2,\,\ldots,\,x_n$ can be expressed as a polynomial \,$Q(p_1,\,p_2,\,\ldots,\,p_n)$\, in the elementary symmetric polynomials 
$p_1,\,p_2,\,\ldots,\,p_n$ of $x_1,\,x_2,\,\ldots,\,x_n$.\, The polynomial $Q$ is unique, its coefficients are elements of the ring determined by the coefficients of $P$ and its degree with respect to $p_1,\,p_2,\,\ldots,\,p_n$ is same as the degree of $P$ with respect to $x_1$.
%%%%%
%%%%%
\end{document}
