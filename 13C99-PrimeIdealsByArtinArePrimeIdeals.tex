\documentclass[12pt]{article}
\usepackage{pmmeta}
\pmcanonicalname{PrimeIdealsByArtinArePrimeIdeals}
\pmcreated{2013-03-22 18:44:55}
\pmmodified{2013-03-22 18:44:55}
\pmowner{pahio}{2872}
\pmmodifier{pahio}{2872}
\pmtitle{prime ideals by Artin are prime ideals}
\pmrecord{10}{41525}
\pmprivacy{1}
\pmauthor{pahio}{2872}
\pmtype{Theorem}
\pmcomment{trigger rebuild}
\pmclassification{msc}{13C99}
\pmclassification{msc}{06A06}
\pmrelated{IdealGeneratedByASet}
\pmrelated{PrimeIdeal}

% this is the default PlanetMath preamble.  as your knowledge
% of TeX increases, you will probably want to edit this, but
% it should be fine as is for beginners.

% almost certainly you want these
\usepackage{amssymb}
\usepackage{amsmath}
\usepackage{amsfonts}

% used for TeXing text within eps files
%\usepackage{psfrag}
% need this for including graphics (\includegraphics)
%\usepackage{graphicx}
% for neatly defining theorems and propositions
 \usepackage{amsthm}
% making logically defined graphics
%%%\usepackage{xypic}

% there are many more packages, add them here as you need them

% define commands here

\theoremstyle{definition}
\newtheorem*{thmplain}{Theorem}

\begin{document}
\textbf{Theorem.}\, Due to Artin, a prime ideal of a commutative ring $R$ is the maximal element among the ideals not intersecting a multiplicative subset $S$ of $R$.\, This is \PMlinkname{equivalent}{Equivalent3} to the usual criterion 
\begin{align}
ab \in \mathfrak{p} \quad \Rightarrow \quad a \in \mathfrak{p} \;\lor\; b \in \mathfrak{p}
\end{align}
of prime ideal (see the entry \PMlinkname{prime ideal}{PrimeIdeal}).


{\em Proof.}\, $1^{\underline{o}}$.\, Let $\mathfrak{p}$ be a prime ideal by Artin, corresponding the semigroup $S$, and let the ring product $ab$ belong to $\mathfrak{p}$.\, Assume, contrary to the assertion, that\, neither of $a$ and $b$ lies in $\mathfrak{p}$.\, When\, $(\mathfrak{p},\,x)$\, generally means the least ideal containing $\mathfrak{p}$ and an element $x$, the antithesis implies that
$$\mathfrak{p} \subset (\mathfrak{p},\,a) \;\; \land \;\; \mathfrak{p} \subset (\mathfrak{p},\,a),$$
whence by the maximality of $\mathfrak{p}$ we have
$$(\mathfrak{p},\,a)\cap S \neq \varnothing \;\; \land \;\; (\mathfrak{p},\,b)\cap S \neq \varnothing.$$
Therefore we can chose such elements \,$s_i = p_i+r_ia+n_ia$\, of $S$ (N.B. the multiples) that
$$p_i \in \mathfrak{p},\;\, r_i \in R,\;\, n_i \in \mathbb{Z} \quad (i \,=\, 1,\,2).$$
But then
$$s_1s_2 \;=\; (p_2+r_2b+n_2b)p_1+(r_1a+n_1a)p_2+(r_1r_2+n_2r_1+n_1r_2)ab+(n_1n_2)ab \in \mathfrak{p}.$$
This is however impossible, since the product $s_1s_2$ belongs to the semigroup $S$ and\, 
$\mathfrak{p}\cap S = \varnothing$.\, Because the antithesis thus is wrong, we must have\, $a \in \mathfrak{p}$\, or\, $b \in \mathfrak{p}$.

$2^{\underline{o}}$.\, Let us then suppose that an ideal $\mathfrak{p}$ satisfies the condition (1) for all\, 
$a,\,b \in R$.\, It means that the set\, $S = R\!\smallsetminus\!\mathfrak{p}$\, is a multiplicative semigroup.\, Accordingly, the $\mathfrak{p}$ is the greatest ideal not intersecting the semigroup $S$, Q.E.D.\\

\textbf{Remark.}\, It follows easily from the theorem, that if $\mathfrak{p}$ is a prime ideal of the commutative ring 
$\mathfrak{O}$ and $\mathfrak{o}$ is a subring of $\mathfrak{O}$, then $\mathfrak{p\cap o}$ is a prime ideal of 
$\mathfrak{o}$.


%%%%%
%%%%%
\end{document}
