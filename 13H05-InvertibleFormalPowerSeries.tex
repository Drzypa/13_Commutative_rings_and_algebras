\documentclass[12pt]{article}
\usepackage{pmmeta}
\pmcanonicalname{InvertibleFormalPowerSeries}
\pmcreated{2016-04-27 10:47:14}
\pmmodified{2016-04-27 10:47:14}
\pmowner{pahio}{2872}
\pmmodifier{pahio}{2872}
\pmtitle{invertible formal power series}
\pmrecord{8}{42086}
\pmprivacy{1}
\pmauthor{pahio}{2872}
\pmtype{Theorem}
\pmcomment{trigger rebuild}
\pmclassification{msc}{13H05}
\pmclassification{msc}{13F25}
\pmclassification{msc}{13J05}
\pmrelated{RulesOfCalculusForDerivativeOfFormalPowerSeries}

% this is the default PlanetMath preamble.  as your knowledge
% of TeX increases, you will probably want to edit this, but
% it should be fine as is for beginners.

% almost certainly you want these
\usepackage{amssymb}
\usepackage{amsmath}
\usepackage{amsfonts}

% used for TeXing text within eps files
%\usepackage{psfrag}
% need this for including graphics (\includegraphics)
%\usepackage{graphicx}
% for neatly defining theorems and propositions
 \usepackage{amsthm}
% making logically defined graphics
%%%\usepackage{xypic}

% there are many more packages, add them here as you need them

% define commands here

\theoremstyle{definition}
\newtheorem*{thmplain}{Theorem}

\begin{document}
\PMlinkescapeword{invertible} \PMlinkescapeword{element}

\textbf{Theorem.}\, Let $R$ be a commutative ring with non-zero unity.\, A formal power series 
\begin{align}
f(X) \;:=\; \sum_{i=0}^\infty a_iX^i
\end{align}
is invertible in the ring $R[[X]]$\, iff\, $a_0$ is invertible in the ring $R$.\\


\emph{Proof.}\, $1^\circ$.\, Let $f(X)$ have the multiplicative inverse \,$g(X) := \sum_{i=0}^\infty b_iX^i$.\, Since
$$f(X)g(X) \;=\; \sum_{i=0}^\infty\sum_{j=0}^ia_jb_{i-j}X^i \;=\; 1,$$ 
we see that\, $a_0b_0 = 1$, i.e. $a_0$ is an invertible element (unit) of $R$.\\

$2^\circ$.\, Assume conversely that $a_0$ is invertible in $R$.\, For making from a formal power series
\begin{align}
g(X) := \sum_{i=0}^\infty b_iX^i
\end{align}
the inverse of $f(X) = \sum_{i=0}^\infty a_iX^i$, we first choose\, $b_0 := a_0^{-1}$.\, For all already defined coefficients $b_0,\,b_1,\,\ldots,\,b_{i-1}$ let the next coefficient be defined as
$$b_i \;:=\; -a_0^{-1}(a_1b_{i-1}\!+\!a_2b_{i-2}\!+\ldots+\!a_ib_0).$$
This equation means that 
$$\sum_{j=0}^ia_jb_{i-j} \;=\; a_0b_i\!+\!a_1b_{i-1}\!+\!a_2b_{i-2}\!+\ldots+\!a_ib_0$$
vanishes for all\, $i = 1,\,2,\,\ldots$;\, since\, $a_0b_0 = 1$,\, the product of the formal power series (1) and (2) becomes simply equal to 1.\, Accordingly, $f(x)$ is invertible.
%%%%%
%%%%%
\end{document}
