\documentclass[12pt]{article}
\usepackage{pmmeta}
\pmcanonicalname{ExamplesOfRingOfIntegersOfANumberField}
\pmcreated{2013-03-22 15:08:09}
\pmmodified{2013-03-22 15:08:09}
\pmowner{alozano}{2414}
\pmmodifier{alozano}{2414}
\pmtitle{examples of ring of integers of a number field}
\pmrecord{7}{36879}
\pmprivacy{1}
\pmauthor{alozano}{2414}
\pmtype{Example}
\pmcomment{trigger rebuild}
\pmclassification{msc}{13B22}
\pmrelated{NumberField}
\pmrelated{AlgebraicNumberTheory}
\pmrelated{CanonicalBasis}
\pmrelated{IntegralBasisOfQuadraticField}

\endmetadata

% this is the default PlanetMath preamble.  as your knowledge
% of TeX increases, you will probably want to edit this, but
% it should be fine as is for beginners.

% almost certainly you want these
\usepackage{amssymb}
\usepackage{amsmath}
\usepackage{amsthm}
\usepackage{amsfonts}

% used for TeXing text within eps files
%\usepackage{psfrag}
% need this for including graphics (\includegraphics)
%\usepackage{graphicx}
% for neatly defining theorems and propositions
%\usepackage{amsthm}
% making logically defined graphics
%%%\usepackage{xypic}

% there are many more packages, add them here as you need them

% define commands here

\newtheorem{thm}{Theorem}
\newtheorem{defn}{Definition}
\newtheorem{prop}{Proposition}
\newtheorem{lemma}{Lemma}
\newtheorem{cor}{Corollary}

\theoremstyle{definition}
\newtheorem{exa}{Example}

% Some sets
\newcommand{\Nats}{\mathbb{N}}
\newcommand{\Ints}{\mathbb{Z}}
\newcommand{\Reals}{\mathbb{R}}
\newcommand{\Complex}{\mathbb{C}}
\newcommand{\Rats}{\mathbb{Q}}
\newcommand{\Gal}{\operatorname{Gal}}
\newcommand{\Cl}{\operatorname{Cl}}
\begin{document}
\begin{defn}
Let $K$ be a number field. The ring of integers of $K$, usually denoted by $\mathcal{O}_K$, is the set of all elements $\alpha\in K$ which are roots of some monic polynomial with coefficients in $\Ints$, i.e. those $\alpha\in K$ which are integral over $\Ints$. In other words, $\mathcal{O}_K$ is the integral closure of $\Ints$ in $K$.
\end{defn}

\begin{exa}
Notice that the only rational numbers which are roots of monic polynomials with integer coefficients are the integers themselves. Thus, the ring of integers of $\Rats$ is $\Ints$.
\end{exa}

\begin{exa}
Let $\mathcal{O}_K$ denote the ring of integers of $K=\Rats(\sqrt{d})$, where $d$ is a square-free integer. Then:
$$\mathcal{O}_K\cong \begin{cases}
\Ints\oplus \frac{1+\sqrt{d}}{2}\Ints, \text{ if } d\equiv 1 \ \operatorname{mod}\ 4,\\
\Ints\oplus \sqrt{d}\ \Ints, \text{ if } d\equiv 2,3
\operatorname{mod}\ 4. \end{cases}
$$
In other words, if we let 
$$\alpha = \begin{cases}
\frac{1+\sqrt{d}}{2}, \text{ if } d\equiv 1 \ \operatorname{mod}\ 4,\\
\sqrt{d}, \text{ if } d\equiv 2,3
\operatorname{mod}\ 4. \end{cases}
$$
then
$$\mathcal{O}_K=\{ n+m\alpha : n,m \in \Ints \}.$$
\end{exa}

\begin{exa}
Let $K=\Rats(\zeta_n)$ be a cyclotomic extension of $\Rats$, where $\zeta_n$ is a primitive $n$th root of unity. Then the ring of integers of $K$ is $\mathcal{O}_K=\Ints[\zeta_n]$, i.e.
$$\mathcal{O}_K=\{ a_0 +a_1\zeta_n +a_2\zeta_n^2+\ldots+a_{n-1}\zeta_n^{n-1} : a_i \in \Ints\}.$$
\end{exa}

\begin{exa}
Let $\alpha$ be an algebraic integer and let $K=\Rats(\alpha)$. It is {\it not true in general} that $\mathcal{O}_K=\Ints[\alpha]$ (as we saw in Example $2$, for $d\equiv 1 \mod 4$).
\end{exa}

\begin{exa}
Let $p$ be a prime number and let $F=\Rats(\zeta_p)$ be a cyclotomic extension of $\Rats$, where $\zeta_p$ is a primitive $p$th root of unity. Let $F^+$ be the maximal real subfield of $F$. It can be shown that:
$$F^+=\Rats(\zeta_p+\zeta_p^{-1}).$$
Moreover, it can also be shown that the ring of integers of $F^+$ is $\mathcal{O}_{F^+}=\Ints[\zeta_p+\zeta_p^{-1}]$. 
\end{exa}
%%%%%
%%%%%
\end{document}
