\documentclass[12pt]{article}
\usepackage{pmmeta}
\pmcanonicalname{GeneratorsOfInverseIdeal}
\pmcreated{2015-05-06 14:52:30}
\pmmodified{2015-05-06 14:52:30}
\pmowner{pahio}{2872}
\pmmodifier{pahio}{2872}
\pmtitle{generators of inverse ideal}
\pmrecord{18}{35934}
\pmprivacy{1}
\pmauthor{pahio}{2872}
\pmtype{Theorem}
\pmcomment{trigger rebuild}
\pmclassification{msc}{13A15}
%\pmkeywords{invertible ideal}
%\pmkeywords{inverse ideal}
\pmrelated{FractionalIdealOfCommutativeRing}
\pmrelated{IdealGeneratedByASet}
\pmrelated{PruferRing}

% this is the default PlanetMath preamble.  as your knowledge
% of TeX increases, you will probably want to edit this, but
% it should be fine as is for beginners.

% almost certainly you want these
\usepackage{amssymb}
\usepackage{amsmath}
\usepackage{amsfonts}

% used for TeXing text within eps files
%\usepackage{psfrag}
% need this for including graphics (\includegraphics)
%\usepackage{graphicx}
% for neatly defining theorems and propositions
 \usepackage{amsthm}
% making logically defined graphics
%%%\usepackage{xypic}

% there are many more packages, add them here as you need them

% define commands here

\theoremstyle{definition}
\newtheorem*{thmplain}{Theorem}
\begin{document}
\textbf{Theorem.}\, Let $R$ be a commutative ring with non-zero 
unity and let $T$ be the total ring of fractions of $R$.\, If\, 
$\mathfrak{a} = (a_1,\,\ldots,\,a_n)$\, is an invertible 
\PMlinkname{{\em fractional ideal}}{FractionalIdealOfCommutativeRing} of $R$ with \,$\mathfrak{ab} = R$,\, then also the inverse 
ideal $\mathfrak{b}$ can be generated by $n$ elements of $T$.\\

{\em Proof.} \,The equation\, $\mathfrak{ab} = (1)$\, implies the existence of the elements $a_i'$ of $\mathfrak{a}$ and $b_i'$ of $\mathfrak{b}$\, $(i = 1, \ldots,\,m)$ such that\, $a_1'b_1'\!+\cdots+\!a_m'b_m' = 1$.\, Because the $a_i'$'s are
in $\mathfrak{a}$, they may be expressed as
            $$a_i' = \sum_{j=1}^{n}r_{ij}a_j \qquad(i = 1, \ldots, m),$$
where the $r_{ij}$'s are some elements of $R$.\, Now the unity acquires the 
form
 $$1 = \sum_{i=1}^{m}a_i'b_i' =
   \sum_{i=1}^{m}\sum_{j=1}^{n}r_{ij}a_jb_i' =
   \sum_{j=1}^{n}a_j\sum_{i=1}^{m}r_{ij}b_i' = \sum_{j=1}^{n}a_jb_j,$$
in which
  $$b_j = \sum_{i=1}^{m}r_{ij}b_i' \,\in R\mathfrak{b} = \mathfrak{b}
\qquad (j = 1, \ldots, n).$$
Thus an arbitrary element $b$ of the \PMlinkescapetext{fractional ideal} $\mathfrak{b}$ satisfies the condition
 $$b = b\!\cdot\!1 = \sum_{j=1}^{n}(a_jb)b_j \, \in Rb_1\!+\cdots+\!Rb_n = (b_1,\,\ldots,\,b_n).$$
Consequently,\, $\mathfrak{b} \subseteq (b_1,\,\ldots,\,b_n)$.\, Since the inverse inclusion is apparent, we have the equality
     $$\mathfrak{a}^{-1} = \mathfrak{b} = (b_1,\,\ldots,\,b_n).$$
%%%%%
%%%%%
\end{document}
