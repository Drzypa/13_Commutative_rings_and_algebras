\documentclass[12pt]{article}
\usepackage{pmmeta}
\pmcanonicalname{SeparabilityIsRequiredForIntegralClosuresToBeFinitelyGenerated}
\pmcreated{2013-03-22 17:02:15}
\pmmodified{2013-03-22 17:02:15}
\pmowner{rm50}{10146}
\pmmodifier{rm50}{10146}
\pmtitle{separability is required for integral closures to be finitely generated}
\pmrecord{6}{39324}
\pmprivacy{1}
\pmauthor{rm50}{10146}
\pmtype{Example}
\pmcomment{trigger rebuild}
\pmclassification{msc}{13B21}
\pmclassification{msc}{12F05}

\endmetadata

% this is the default PlanetMath preamble.  as your knowledge
% of TeX increases, you will probably want to edit this, but
% it should be fine as is for beginners.

% almost certainly you want these
\usepackage{amssymb}
\usepackage{amsmath}
\usepackage{amsfonts}

% used for TeXing text within eps files
%\usepackage{psfrag}
% need this for including graphics (\includegraphics)
%\usepackage{graphicx}
% for neatly defining theorems and propositions
%\usepackage{amsthm}
% making logically defined graphics
%%%\usepackage{xypic}

% there are many more packages, add them here as you need them

% define commands here
\newcommand{\Nats}{\mathbb{N}}
\newcommand{\Ints}{\mathbb{Z}}
\newcommand{\Reals}{\mathbb{R}}
\newcommand{\Complex}{\mathbb{C}}
\newcommand{\Proj}{\mathbb{P}}
\newcommand{\Rats}{\mathbb{Q}}
\newcommand{\Gal}{\operatorname{Gal}}
\newcommand{\Cl}{\operatorname{Cl}}
\newcommand{\Alg}{\mathcal{O}}
\newcommand{\ol}{\overline}
\newcommand{\Leg}[2]{\left(\frac{#1}{#2}\right)}
\newcommand{\Spec}[1]{\text{Spec }#1}
\newcommand{\Pic}[1]{\text{Pic }#1}
\newcommand{\kx}{k[x_1,\ldots,x_n]}
\newcommand{\Order}[1]{\left\lvert #1 \right\rvert}
\renewcommand{\frak}[1]{\mathfrak{#1}}
\newcommand{\ip}[2]{(#1,#2)}
\newcommand{\conv}[2]{(#1*#2)}
\newcommand{\Hom}{\mathrm{Hom}}

\begin{document}
The parent theorem assumed that $L$ was a separable extension of $K$. Here is an example that shows that separability is in fact a necessary condition for the result to hold.


Let $k=(\Ints/p\Ints)(b_1,b_2,\ldots)$ where the $b_i$ are indeterminates.

Let $B$ be the subring of $k[[t]]$ (power series in $t$ over $k$) consisting of
\[\left\{\sum_{i=0}^{\infty} c_i t^i\right\}\]
such that $\{c_0,c_1,c_2,\ldots\}$ generates a finite extension of $(\Ints/p\Ints)(b^p_1,b^p_2,\ldots)$. Then every element of $B$ is a power of $t$ times a unit (first factor out the largest power of $t$. What's left is $a(1+c_1 t+c_2 t^2+\cdots)$; its inverse is $a^{-1}(1-c_1 t-\cdots)$). Hence the ideals of $B$ are powers of $t$, so $B$ is a PID (in fact, it is a DVR).

Now, let $u=b_0 + b_1 t+b_2 t^2+\cdots$. Now, $u\notin B$ because it uses all of the $b_i$ and thus the coefficients do not define a finite extension of $(\Ints/p\Ints)(b^p_1,b^p_2,\ldots)$. However, $u$ is integral over $B$: $b^p_i\in B\Rightarrow u^p\in B$ which implies that the degree of $u$ over the field of fractions is $p$. Hence a basis for $K(u)/K$ is $\{1,u,u^2,\ldots,u^{p-1}\}$. There are other elements integral over $B$:
\begin{gather*}
b_1+b_2 t+b_3 t^2+\cdots = \frac{u-b_0}{t}=\frac{-b_0}{t}+\frac{1}{t}u\\
b_2+b_3 t+b_4 t^2+\cdots = \frac{u-b_0-b_1 t}{t^2}=\frac{-b_0-b_1t}{t^2}+\frac{1}{t^2}u\\
\vdots
\end{gather*}
Clearly the denominators are getting bigger, so the integral closure of $B$ cannot be finitely generated as a $B$-module.

%%%%%
%%%%%
\end{document}
