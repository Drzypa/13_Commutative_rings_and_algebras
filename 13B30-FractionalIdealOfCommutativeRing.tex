\documentclass[12pt]{article}
\usepackage{pmmeta}
\pmcanonicalname{FractionalIdealOfCommutativeRing}
\pmcreated{2015-05-06 14:40:32}
\pmmodified{2015-05-06 14:40:32}
\pmowner{pahio}{2872}
\pmmodifier{pahio}{2872}
\pmtitle{fractional ideal of commutative ring}
\pmrecord{16}{36986}
\pmprivacy{1}
\pmauthor{pahio}{2872}
\pmtype{Definition}
\pmcomment{trigger rebuild}
\pmclassification{msc}{13B30}
%\pmkeywords{regular ideal}
\pmrelated{FractionalIdeal}
\pmrelated{GeneratorsOfInverseIdeal}
\pmrelated{IdealClassesFormAnAbelianGroup}
\pmdefines{fractional ideal}
\pmdefines{integral ideal}
\pmdefines{invertible ideal}
\pmdefines{invertible}
\pmdefines{inverse ideal}
\pmdefines{class group of a ring}
\pmdefines{unit ideal}

% this is the default PlanetMath preamble.  as your knowledge
% of TeX increases, you will probably want to edit this, but
% it should be fine as is for beginners.

% almost certainly you want these
\usepackage{amssymb}
\usepackage{amsmath}
\usepackage{amsfonts}

% used for TeXing text within eps files
%\usepackage{psfrag}
% need this for including graphics (\includegraphics)
%\usepackage{graphicx}
% for neatly defining theorems and propositions
 \usepackage{amsthm}
% making logically defined graphics
%%%\usepackage{xypic}

% there are many more packages, add them here as you need them

% define commands here

\theoremstyle{definition}
\newtheorem*{thmplain}{Theorem}
\begin{document}


\textbf{Definition.}\, Let $R$ be a commutative ring having a regular element and let $T$ be the total ring of fractions of $R$.\, An \PMlinkname{$R$-submodule}{Submodule} $\mathfrak{a}$ of $T$ is called {\em fractional ideal} of $R$, provided that there exists a regular element $d$ of $R$ such that\, $\mathfrak{a}d \subseteq R$.\, If a fractional ideal is contained in $R$, it is a usual ideal of $R$, and we can call it an {\em integral ideal} of $R$. \\

Note that a fractional ideal of $R$ is not necessarily a subring of $T$.\, The set of all fractional ideals of $R$ form under the multiplication an commutative semigroup with identity element\, $R' = R\!+\!\mathbb{Z}e$,\, where $e$ is the unity of $T$.

An ideal $\mathfrak{a}$ (\PMlinkescapetext{integral} or fractional) of $R$ is called {\em invertible}, if there exists another ideal $\mathfrak{a}^{-1}$ of $R$ such that\, $\mathfrak{aa}^{-1} = R'$.\, It is not hard to show that any invertible ideal $\mathfrak{a}$ is finitely generated and \PMlinkname{regular}{RegularIdeal}, moreover that the {\em inverse ideal} $\mathfrak{a}^{-1}$ is uniquely determined (see the entry ``\PMlinkname{invertible ideal is finitely generated}{InvertibleIdealIsFinitelyGenerated}'') and may be generated by the \PMlinkname{same amount of generators}{GeneratorsOfInverseIdeal} as $\mathfrak{a}$.

The set of all invertible fractional ideals of $R$ forms an Abelian group under the multiplication.\, This group has a normal subgroup consisting of all regular principal fractional ideals; the corresponding factor group is called the \PMlinkescapetext{{\em class group}} of the ring $R$. \\

\textbf{Note.}\, In the special case that the ring $R$ has a unity 1, $R$ itself is the principal ideal (1), being the identity element of the semigroup of fractional ideals and the group of invertible fractional ideals.\, It is called the {\em unit ideal}.\, The unit ideal is the only integral ideal containing units of the ring.
%%%%%
%%%%%
\end{document}
