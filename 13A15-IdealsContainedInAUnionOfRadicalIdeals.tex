\documentclass[12pt]{article}
\usepackage{pmmeta}
\pmcanonicalname{IdealsContainedInAUnionOfRadicalIdeals}
\pmcreated{2013-03-22 19:04:23}
\pmmodified{2013-03-22 19:04:23}
\pmowner{joking}{16130}
\pmmodifier{joking}{16130}
\pmtitle{ideals contained in a union of radical ideals}
\pmrecord{4}{41958}
\pmprivacy{1}
\pmauthor{joking}{16130}
\pmtype{Corollary}
\pmcomment{trigger rebuild}
\pmclassification{msc}{13A15}

% this is the default PlanetMath preamble.  as your knowledge
% of TeX increases, you will probably want to edit this, but
% it should be fine as is for beginners.

% almost certainly you want these
\usepackage{amssymb}
\usepackage{amsmath}
\usepackage{amsfonts}

% used for TeXing text within eps files
%\usepackage{psfrag}
% need this for including graphics (\includegraphics)
%\usepackage{graphicx}
% for neatly defining theorems and propositions
%\usepackage{amsthm}
% making logically defined graphics
%%%\usepackage{xypic}

% there are many more packages, add them here as you need them

% define commands here

\begin{document}
Let $R$ be a commutative ring and $I\subseteq R$ an ideal. Recall that \textit{the radical of} $I$ is defined as
$$r(I)=\{x\in R\ |\ \exists_{n\in\mathbb{N}}\ x^n\in I\}.$$
It can be shown, that $r(I)$ is again an ideal and $I\subseteq r(I)$. Let 
$$V(I)=\{P\subseteq R\ |\ P\mbox{ is a prime ideal and }I\subseteq P\}.$$
Of course $V(I)\neq\emptyset$ (because $I$ is contained in at least one maximal ideal) and it can be shown, that
$$r(I)=\bigcap_{P\in V(I)} P.$$
Finaly, recall that an ideal $I$ is called \textit{radical}, if $I=r(I)$.

\textbf{Proposition.} Let $I,R_1,\ldots,R_n$ be ideals in $R$, such that each $R_i$ is radical. If $$I\subseteq R_1\cup\cdots\cup R_n,$$ then there exists $i\in\{1,\ldots,n\}$ such that $I\subseteq R_i$.

\textit{Proof.} Assume that this not true, i.e. for every $i$ we have $I\not\subseteq R_i$. Then for any $i\in\{1,\ldots,n\}$ there exists $P_i\in V(R_i)$ such that $I\not\subseteq P_i$ (this follows from the fact, that $R_i=r(R_i)$ and the characterization of radicals via prime ideals). But for any $i$ we have $R_i\subseteq P_i$ and thus
$$I\subseteq P_1\cup\cdots\cup P_n.$$
Contradiction, since each $P_i$ is prime (see the parent object for details). $\square$
%%%%%
%%%%%
\end{document}
