\documentclass[12pt]{article}
\usepackage{pmmeta}
\pmcanonicalname{EquivalentValuations}
\pmcreated{2013-03-22 14:25:27}
\pmmodified{2013-03-22 14:25:27}
\pmowner{pahio}{2872}
\pmmodifier{pahio}{2872}
\pmtitle{equivalent valuations}
\pmrecord{18}{35932}
\pmprivacy{1}
\pmauthor{pahio}{2872}
\pmtype{Definition}
\pmcomment{trigger rebuild}
\pmclassification{msc}{13A18}
\pmrelated{DiscreteValuation}
\pmrelated{IndependenceOfTheValuations}
\pmdefines{equivalence of valuations}

% this is the default PlanetMath preamble.  as your knowledge
% of TeX increases, you will probably want to edit this, but
% it should be fine as is for beginners.

% almost certainly you want these
\usepackage{amssymb}
\usepackage{amsmath}
\usepackage{amsfonts}

% used for TeXing text within eps files
%\usepackage{psfrag}
% need this for including graphics (\includegraphics)
%\usepackage{graphicx}
% for neatly defining theorems and propositions
 \usepackage{amsthm}
% making logically defined graphics
%%%\usepackage{xypic}

% there are many more packages, add them here as you need them

% define commands here
\theoremstyle{definition}
\newtheorem*{thmplain}{Theorem}
\begin{document}
Let $K$ be a field. \,The {\em equivalence of valuations} $|\cdot|_1$ and  $|\cdot|_2$ of $K$ may be defined so that
\begin{enumerate}
\item $|\cdot|_1$ is not the trivial valuation;
\item if \, $|a|_1 < 1$ then $|a|_2 < 1 \qquad \forall a \in K.$
\end{enumerate}

It it easy to see that these conditions imply \PMlinkescapetext{symmetry} for both valuations (use $\frac{1}{a}$). \,Also, we have always
     $$|a|_1 \leqq 1 \, \Leftrightarrow \, |a|_2 \leqq 1;$$
so both valuations have a common valuation ring in the case they are non-archimedean. \,(The \PMlinkescapetext{equivalence} of the more general Krull valuations is defined to \PMlinkescapetext{mean} that they have common valuation rings.) \,Further, both valuations determine a common metric on $K$. 

\begin{thmplain}
\,Two valuations (of \PMlinkname{rank}{KrullValuation} one)  \,$|\cdot|_1$\, and \,$|\cdot|_2$\, of $K$ are \PMlinkescapetext{equivalent} iff one of them is a positive power of the other,
       $$|a|_1 = |a|_2^c \qquad \forall a \in K,$$
where $c$ is a positive \PMlinkescapetext{constant}.
\end{thmplain}
%%%%%
%%%%%
\end{document}
