\documentclass[12pt]{article}
\usepackage{pmmeta}
\pmcanonicalname{ScalarMap}
\pmcreated{2013-03-22 17:24:22}
\pmmodified{2013-03-22 17:24:22}
\pmowner{Algeboy}{12884}
\pmmodifier{Algeboy}{12884}
\pmtitle{scalar map}
\pmrecord{7}{39778}
\pmprivacy{1}
\pmauthor{Algeboy}{12884}
\pmtype{Definition}
\pmcomment{trigger rebuild}
\pmclassification{msc}{13C99}
\pmsynonym{outer linear}{ScalarMap}
\pmrelated{BilinearMap}
\pmdefines{scalar map}

\usepackage{latexsym}
\usepackage{amssymb}
\usepackage{amsmath}
\usepackage{amsfonts}
\usepackage{amsthm}

%%\usepackage{xypic}

%-----------------------------------------------------

%       Standard theoremlike environments.

%       Stolen directly from AMSLaTeX sample

%-----------------------------------------------------

%% \theoremstyle{plain} %% This is the default

\newtheorem{thm}{Theorem}

\newtheorem{coro}[thm]{Corollary}

\newtheorem{lem}[thm]{Lemma}

\newtheorem{lemma}[thm]{Lemma}

\newtheorem{prop}[thm]{Proposition}

\newtheorem{conjecture}[thm]{Conjecture}

\newtheorem{conj}[thm]{Conjecture}

\newtheorem{defn}[thm]{Definition}

\newtheorem{remark}[thm]{Remark}

\newtheorem{ex}[thm]{Example}



%\countstyle[equation]{thm}



%--------------------------------------------------

%       Item references.

%--------------------------------------------------


\newcommand{\exref}[1]{Example-\ref{#1}}

\newcommand{\thmref}[1]{Theorem-\ref{#1}}

\newcommand{\defref}[1]{Definition-\ref{#1}}

\newcommand{\eqnref}[1]{(\ref{#1})}

\newcommand{\secref}[1]{Section-\ref{#1}}

\newcommand{\lemref}[1]{Lemma-\ref{#1}}

\newcommand{\propref}[1]{Prop\-o\-si\-tion-\ref{#1}}

\newcommand{\corref}[1]{Cor\-ol\-lary-\ref{#1}}

\newcommand{\figref}[1]{Fig\-ure-\ref{#1}}

\newcommand{\conjref}[1]{Conjecture-\ref{#1}}


% Normal subgroup or equal.

\providecommand{\normaleq}{\unlhd}

% Normal subgroup.

\providecommand{\normal}{\lhd}

\providecommand{\rnormal}{\rhd}
% Divides, does not divide.

\providecommand{\divides}{\mid}

\providecommand{\ndivides}{\nmid}


\providecommand{\union}{\cup}

\providecommand{\bigunion}{\bigcup}

\providecommand{\intersect}{\cap}

\providecommand{\bigintersect}{\bigcap}










\begin{document}
Given a ring $R$, a left $R$-module $U$, a right $R$-module $V$ and a two-sided 
$R$-module $W$
then a map $b:U\times V\to W$ is an $R$-\emph{scalar map} if
\begin{enumerate}
\item $b$ is biadditive, that is $b(u+u',v)=b(u,v)+b(u',v)$ and $b(u,v+v')=b(u,v)+b(u,v')$
for all $u,u'\in U$ and $v,v'\in V$;
\item $b(ru,v)=rb(u,v)$ and $b(u,vr)=b(u,v)r$ for all $u\in U$, $v\in V$ and $r\in R$.
\end{enumerate}

Such maps can also be called \emph{outer linear}.

Unlike bilinear maps, scalar maps do not force a commutative multiplication 
on $R$ even when the map is non-degenerate and the modules are faithful.
For example, if $A$ is an associative ring then the multiplication of $A$,
$b:A\times A\to A$ is a $A$-outer linear:
\[b(xy,z)=(xy)z=x(yz)=xb(y,z)\]
and likewise $b(x,yz)=b(x,y)z$.  Using a non-commutative ring $A$ confirms 
the claim.

It is immediate however that $\langle b(U,V)\rangle$ is in fact an $R$-bimodule.  
This is because:
\[s(b(u,v)r)=sb(u,vr)=b(su,vr)=sb(u,vr)=(sb(u,v))r\]
for all $u\in U$, $v\in V$ and $s,r\in R$.  Therefore it is not uncommon to
require that indeed all of $W$ be an $R$-bimodule.
%%%%%
%%%%%
\end{document}
