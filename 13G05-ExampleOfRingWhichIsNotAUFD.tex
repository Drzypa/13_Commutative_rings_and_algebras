\documentclass[12pt]{article}
\usepackage{pmmeta}
\pmcanonicalname{ExampleOfRingWhichIsNotAUFD}
\pmcreated{2013-03-22 15:08:19}
\pmmodified{2013-03-22 15:08:19}
\pmowner{alozano}{2414}
\pmmodifier{alozano}{2414}
\pmtitle{example of ring which is not a UFD}
\pmrecord{7}{36882}
\pmprivacy{1}
\pmauthor{alozano}{2414}
\pmtype{Example}
\pmcomment{trigger rebuild}
\pmclassification{msc}{13G05}
\pmsynonym{example of a ring of integers which is not a UFD}{ExampleOfRingWhichIsNotAUFD}
\pmrelated{DeterminingTheContinuationsOfExponent}
\pmdefines{example of a number ring which is not a UFD}

% this is the default PlanetMath preamble.  as your knowledge
% of TeX increases, you will probably want to edit this, but
% it should be fine as is for beginners.

% almost certainly you want these
\usepackage{amssymb}
\usepackage{amsmath}
\usepackage{amsthm}
\usepackage{amsfonts}

% used for TeXing text within eps files
%\usepackage{psfrag}
% need this for including graphics (\includegraphics)
%\usepackage{graphicx}
% for neatly defining theorems and propositions
%\usepackage{amsthm}
% making logically defined graphics
%%%\usepackage{xypic}

% there are many more packages, add them here as you need them

% define commands here

\newtheorem{thm}{Theorem}
\newtheorem{defn}{Definition}
\newtheorem{prop}{Proposition}
\newtheorem{lemma}{Lemma}
\newtheorem{cor}{Corollary}

\theoremstyle{definition}
\newtheorem{exa}{Example}

% Some sets
\newcommand{\Nats}{\mathbb{N}}
\newcommand{\Ints}{\mathbb{Z}}
\newcommand{\Reals}{\mathbb{R}}
\newcommand{\Complex}{\mathbb{C}}
\newcommand{\Rats}{\mathbb{Q}}
\newcommand{\Gal}{\operatorname{Gal}}
\newcommand{\Cl}{\operatorname{Cl}}
\begin{document}
\begin{exa}
We define a ring $R=\Ints[\sqrt{-5}]=\{ n+m\sqrt{-5} : n,m\in\Ints\}$ with addition and multiplication inherited from $\Complex$ (notice that $R$ is the ring of integers of the quadratic number field $\Rats(\sqrt{-5})$). Notice that the only \PMlinkname{units}{UnitsOfQuadraticFields} of $R$ are $R^\times=\{ \pm 1 \}$. Then:
\begin{eqnarray} \label{eq1} 6=2\cdot 3 = (1+\sqrt{-5})\cdot (1-\sqrt{-5}).\end{eqnarray}
Moreover, $2,\ 3,\ 1+\sqrt{-5}$ and $1-\sqrt{-5}$ are irreducible elements of $R$ and they are not associates (to see this, one can compare the norm of every element). Therefore, $R$ is not a UFD. \\

However, the ideals of $R$ \PMlinkname{factor}{DivisibilityInRings} uniquely into prime ideals. For example:
$$(6)=(2,1+\sqrt{-5})^2\cdot (3,1+\sqrt{-5})\cdot (3,1-\sqrt{-5})$$
where $\mathfrak{P}=(2,1+\sqrt{-5})$, $\mathfrak{Q}=(3,1+\sqrt{-5})$, and $\overline{\mathfrak{Q}}=(3,1-\sqrt{-5})$ are all prime ideals (see \PMlinkname{prime ideal decomposition of quadratic extensions of $\mathbb{Q}$}{PrimeIdealDecompositionInQuadraticExtensionsOfMathbbQ}). Notice that:
$$\mathfrak{P}^2=(2),\quad \mathfrak{Q}\cdot\overline{\mathfrak{Q}}=(3),\quad  \mathfrak{P}\cdot\mathfrak{Q}=(1+\sqrt{-5}),\quad  \mathfrak{P}\cdot \overline{\mathfrak{Q}}=(1-\sqrt{-5}).$$
Thus, Eq. (\ref{eq1}) above is the outcome of different rearrangements of the product of prime ideals:
$$(6)=\mathfrak{P}^2\cdot(\mathfrak{Q}\cdot \overline{\mathfrak{Q}})=(\mathfrak{P}\cdot \mathfrak{Q})\cdot (\mathfrak{P}\cdot \overline{\mathfrak{Q}}).$$
Notice also that if $\mathfrak{P}$ was a principal ideal then there would be an element $\alpha \in R$ with $(\alpha)=\mathfrak{P}$ and $(\alpha)^2 = (2)$. Thus such a number $\alpha$ would have norm $2$, but the norm of $n+m\sqrt{-5}$ is $n^2+5m^2$ so it is clear that there are no algebraic integers of norm $2$. Therefore $\mathfrak{P}$ is not principal. Thus $R$ is not a PID. 
\end{exa}
%%%%%
%%%%%
\end{document}
