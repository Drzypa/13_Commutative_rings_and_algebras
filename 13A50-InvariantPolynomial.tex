\documentclass[12pt]{article}
\usepackage{pmmeta}
\pmcanonicalname{InvariantPolynomial}
\pmcreated{2013-03-22 13:40:19}
\pmmodified{2013-03-22 13:40:19}
\pmowner{Daume}{40}
\pmmodifier{Daume}{40}
\pmtitle{invariant polynomial}
\pmrecord{6}{34337}
\pmprivacy{1}
\pmauthor{Daume}{40}
\pmtype{Definition}
\pmcomment{trigger rebuild}
\pmclassification{msc}{13A50}

\endmetadata

% this is the default PlanetMath preamble.  as your knowledge
% of TeX increases, you will probably want to edit this, but
% it should be fine as is for beginners.

% almost certainly you want these
\usepackage{amssymb}
\usepackage{amsmath}
\usepackage{amsfonts}

% used for TeXing text within eps files
%\usepackage{psfrag}
% need this for including graphics (\includegraphics)
%\usepackage{graphicx}
% for neatly defining theorems and propositions
%\usepackage{amsthm}
% making logically defined graphics
%%%\usepackage{xypic} 

% there are many more packages, add them here as you need them

% define commands here
\begin{document}
An \emph{invariant polynomial} is a polynomial $P$ that is invariant under a \text{(compact)} Lie group $\Gamma$ acting on a vector space $V$.  Therefore $P$ is $\Gamma$-invariant polynomial if $P(\gamma x) = P(x)$ for all $\gamma \in \Gamma$ and $x \in V$.
\begin{thebibliography}{1}
\bibitem[GSS]{1} Golubitsky, Martin. Stewart, Ian. Schaeffer, G. David: Singularities and Groups in Bifurcation Theory \textit{(Volume II)}. Springer-Verlag, New York, 1988.
\end{thebibliography}
%%%%%
%%%%%
\end{document}
