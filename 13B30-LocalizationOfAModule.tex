\documentclass[12pt]{article}
\usepackage{pmmeta}
\pmcanonicalname{LocalizationOfAModule}
\pmcreated{2013-03-22 17:26:59}
\pmmodified{2013-03-22 17:26:59}
\pmowner{CWoo}{3771}
\pmmodifier{CWoo}{3771}
\pmtitle{localization of a module}
\pmrecord{7}{39831}
\pmprivacy{1}
\pmauthor{CWoo}{3771}
\pmtype{Definition}
\pmcomment{trigger rebuild}
\pmclassification{msc}{13B30}

\endmetadata

\usepackage{amssymb,amscd}
\usepackage{amsmath}
\usepackage{amsfonts}
\usepackage{mathrsfs}

% used for TeXing text within eps files
%\usepackage{psfrag}
% need this for including graphics (\includegraphics)
%\usepackage{graphicx}
% for neatly defining theorems and propositions
\usepackage{amsthm}
% making logically defined graphics
%%\usepackage{xypic}
\usepackage{pst-plot}
\usepackage{psfrag}

% define commands here
\newtheorem{prop}{Proposition}
\newtheorem{thm}{Theorem}
\newtheorem{ex}{Example}
\newcommand{\real}{\mathbb{R}}
\newcommand{\pdiff}[2]{\frac{\partial #1}{\partial #2}}
\newcommand{\mpdiff}[3]{\frac{\partial^#1 #2}{\partial #3^#1}}
\begin{document}
Let $R$ be a commutative ring and $M$ an $R$-module.  Let $S\subset R$ be a non-empty multiplicative set.  Form the Cartesian product $M\times S$, and define a binary relation $\sim$ on $M\times S$ as follows: 
\begin{quote}
$(m_1,s_1)\sim (m_2,s_2)$ if and only if  there is some $t\in S$ such that $t(s_2m_1-s_1m_2)=0$
\end{quote}

\begin{prop} $\sim$ on $M\times S$ is an equivalence relation. \end{prop}
\begin{proof}  Clearly $(m,s)\sim (m,s)$ as $t(sm-sm)=0$ for any $t\in S$, where $S\ne \varnothing$.  Also, $(m_1,s_1)\sim (m_2,s_2)$ implies that $(m_2,s_2)\sim (m_1,s_1)$, since $t(s_2m_1-s_1m_2)=0$ implies that $t(s_1m_2-s_2m_1)=0$.  Finally, given $(m_1,s_1)\sim (m_2,s_2)$ and $(m_2,s_2)\sim (m_3,s_3)$, we are led to two equations $t(s_2m_1-s_1m_2)=0$ and $u(s_3m_2-s_2m_3)=0$ for some $t,u\in S$.  Expanding and rearranging these, then multiplying the first equation by $us_3$ and the second by $ts_1$, we get $tus_2(s_3m_1-s_1m_3)=0$.  Since $tus_2\in S$, $(m_1,s_1)\sim (m_3,s_3)$ as required.
\end{proof}

Let $M_S$ be the set of equivalence classes in $M\times S$ under $\sim$.  For each $(m,s)\in M\times S$, write $$[(m,s)]\mbox{ or more commonly }\frac{m}{s}$$ the equivalence class in $M_S$ containing $(m,s)$.  Next, 
\begin{itemize}
\item define a binary operation $+$ on $M_S$ as follows: $$\frac{m_1}{s_1}+\frac{m_2}{s_2}:=\frac{s_2m_1+s_1m_2}{s_1s_2}.$$
\item define a function $\cdot: R_S\times M_S\to M_S$ as follows:
$$\frac{r}{s}\cdot \frac{m}{t}:=\frac{rm}{st}$$
where $R_S$ is the localization of $R$ over $S$.
\end{itemize}

\begin{prop} $M_S$ together with $+$ and $\cdot$ defined above is a unital module over $R_S$. \end{prop}
\begin{proof} That $+$ and $\cdot$ are well-defined is based on the following: if $(m_1,s_1)\sim (m_2,s_2)$, then 
$$\frac{m}{s}+\frac{m_1}{s_1}=\frac{m}{s}+\frac{m_2}{s_2},\qquad \frac{m_1}{s_1}+\frac{m}{s}=\frac{m_2}{s_2}+\frac{m}{s},\quad\mbox{and}\quad \frac{r}{s}\cdot \frac{m_1}{s_1}=\frac{r}{s}\cdot \frac{m_2}{s_2},$$
which are clear by Proposition $1$.  Furthermore $+$ is commutative and associative and that $\cdot$ distributes over $+$ on both sides, which are all properties inherited from $M$.  Next, $\displaystyle{\frac{0}{s}}$ is the additive identity in $M_S$ and $\displaystyle{\frac{-m}{s}}\in M_S$ is the additive inverse of $\displaystyle{\frac{m}{s}}$.  So $M_S$ is a module over $R_S$.  Finally, since $(mt,st)\sim (m,s)$ for any $t\in S$, $\displaystyle{\frac{t}{t}\cdot \frac{m}{s}=\frac{m}{s}}$ so that $M_S$ is unital.  \end{proof}

\textbf{Definition}.  $M_S$, as an $R_S$-module, is called the \emph{localization} of $M$ at $S$.  $M_S$ is also written $S^{-1}M$.

\textbf{Remarks}.  
\begin{itemize}
\item
The notion of the localization of a module generalizes that of a ring in the sense that $R_S$ is the localization of $R$ at $S$ as an $R_S$-module.
\item
If $S=R-\mathfrak{p}$, where $\mathfrak{p}$ is a prime ideal in $R$, then $M_S$ is usually written $M_{\mathfrak{p}}$.
\end{itemize}
%%%%%
%%%%%
\end{document}
