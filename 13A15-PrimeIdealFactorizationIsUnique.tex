\documentclass[12pt]{article}
\usepackage{pmmeta}
\pmcanonicalname{PrimeIdealFactorizationIsUnique}
\pmcreated{2013-03-22 18:34:24}
\pmmodified{2013-03-22 18:34:24}
\pmowner{gel}{22282}
\pmmodifier{gel}{22282}
\pmtitle{prime ideal factorization is unique}
\pmrecord{9}{41297}
\pmprivacy{1}
\pmauthor{gel}{22282}
\pmtype{Theorem}
\pmcomment{trigger rebuild}
\pmclassification{msc}{13A15}
\pmclassification{msc}{13F05}
%\pmkeywords{prime ideal}
%\pmkeywords{invertible ideal}
\pmrelated{DedekindDomain}
\pmrelated{FractionalIdeal}
\pmrelated{PrimeIdeal}
\pmrelated{FundamentalTheoremOfIdealTheory}

\endmetadata

% this is the default PlanetMath preamble.  as your knowledge
% of TeX increases, you will probably want to edit this, but
% it should be fine as is for beginners.

% almost certainly you want these
\usepackage{amssymb}
\usepackage{amsmath}
\usepackage{amsfonts}

% used for TeXing text within eps files
%\usepackage{psfrag}
% need this for including graphics (\includegraphics)
%\usepackage{graphicx}
% for neatly defining theorems and propositions
\usepackage{amsthm}
% making logically defined graphics
%%%\usepackage{xypic}

% there are many more packages, add them here as you need them

% define commands here
\newtheorem*{theorem*}{Theorem}
\newtheorem*{lemma*}{Lemma}
\newtheorem*{corollary*}{Corollary}
\newtheorem{theorem}{Theorem}
\newtheorem{lemma}{Lemma}
\newtheorem{corollary}{Corollary}


\begin{document}
\PMlinkescapeword{invertible}
\PMlinkescapeword{integral}
\PMlinkescapeword{prime factors}
The following theorem shows that the decomposition of an (integral) invertible ideal into its prime factors is unique, if it exists. This applies to the ring of integers in a number field or, more generally, to any Dedekind domain, in which every nonzero ideal is invertible.

\begin{theorem*}
Let $I$ be an invertible ideal in an integral domain $R$, and that
\begin{equation*}
I=\mathfrak{p}_1\mathfrak{p}_2\cdots\mathfrak{p}_m=\mathfrak{q}_1\mathfrak{q}_2\cdots\mathfrak{q}_n
\end{equation*}
are two factorizations of $I$ into a product of prime ideals. Then $m=n$ and, up to reordering of the factors, $\mathfrak{p}_k=\mathfrak{q}_k$ ($k=1,2,\ldots,n$).
\end{theorem*}

Here we allow the case where $m$ or $n$ is zero, in which case such an empty product is taken to be the full ring $R$.

\begin{proof}
We use induction on $m+n$. First, the case with $m+n=0$ is trivial, so suppose that $m+n>0$.
As the set of prime ideals $\mathfrak{p}_k$, $\mathfrak{q}_k$ is partially ordered by inclusion, there must be a minimal element. After reordering, without loss of generality we may suppose that it is $\mathfrak{p}_1$. Then
\begin{equation*}
\mathfrak{q}_1\mathfrak{q}_2\cdots\mathfrak{q}_n\subseteq\mathfrak{p}_1,
\end{equation*}
so $n\ge 1$. Furthermore, as $\mathfrak{p}_1$ is prime, this implies that $\mathfrak{q}_k\subseteq\mathfrak{p}_1$ for some $k$. After reordering the factors, we can take $k=1$, so that $\mathfrak{q}_1\subseteq\mathfrak{p}_1$.

As $\mathfrak{p}_1$ is minimal among the prime factors, we have $\mathfrak{q}_1=\mathfrak{p}_1$. Also, $\mathfrak{p}_1$ is a factor of the invertible ideal $I$ and so is itself invertible. Therefore, it can be cancelled from the products,
\begin{equation*}
\mathfrak{p}_2\cdots\mathfrak{p}_m=\mathfrak{q}_2\cdots\mathfrak{q}_n.
\end{equation*}
The induction hypothesis gives $m=n$ and, after reordering, $\mathfrak{p}_k=\mathfrak{q}_k$ for $k=2,\ldots,n$.
\end{proof}

%%%%%
%%%%%
\end{document}
