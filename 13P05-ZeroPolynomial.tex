\documentclass[12pt]{article}
\usepackage{pmmeta}
\pmcanonicalname{ZeroPolynomial}
\pmcreated{2013-03-22 14:46:58}
\pmmodified{2013-03-22 14:46:58}
\pmowner{pahio}{2872}
\pmmodifier{pahio}{2872}
\pmtitle{zero polynomial}
\pmrecord{13}{36431}
\pmprivacy{1}
\pmauthor{pahio}{2872}
\pmtype{Definition}
\pmcomment{trigger rebuild}
\pmclassification{msc}{13P05}
\pmclassification{msc}{11C08}
\pmclassification{msc}{12E05}
\pmrelated{PolynomialRingOverIntegralDomain}
\pmrelated{OrderAndDegreeOfPolynomial}
\pmrelated{MinimalPolynomialEndomorphism}

% this is the default PlanetMath preamble.  as your knowledge
% of TeX increases, you will probably want to edit this, but
% it should be fine as is for beginners.

% almost certainly you want these
\usepackage{amssymb}
\usepackage{amsmath}
\usepackage{amsfonts}

% used for TeXing text within eps files
%\usepackage{psfrag}
% need this for including graphics (\includegraphics)
%\usepackage{graphicx}
% for neatly defining theorems and propositions
%\usepackage{amsthm}
% making logically defined graphics
%%%\usepackage{xypic}

% there are many more packages, add them here as you need them

% define commands here
\begin{document}
The {\em zero polynomial} in a ring $R[X]$ of polynomials over a ring $R$ is the \PMlinkescapetext{additive} identity element $\textbf{0}$ of this polynomial ring:
$$f\!+\!\textbf{0} \;=\; \textbf{0}\!+\!f \;=\; f \quad\forall\, f\in R[X]$$
So the zero polynomial is also the absorbing element for the multiplication of polynomials.

All coefficients of the zero polynomial are equal to 0, i.e. 
$$\textbf{0} \;:=\; (0,\,0,\,0,\,...).$$

Because always
$$f\cdot\textbf{0} \;=\; \textbf{0}$$
and because in general \,$\deg(fg) = \deg(f)+\deg(g)$\, when $R$ has no zero divisors, one may define that that the zero polynomial has no \PMlinkname{degree}{Polynomial} at all, or alternatively that
$$\deg(\textbf{0}) \;=\; -\infty$$
(see the extended real numbers).
%%%%%
%%%%%
\end{document}
