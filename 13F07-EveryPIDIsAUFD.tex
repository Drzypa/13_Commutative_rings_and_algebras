\documentclass[12pt]{article}
\usepackage{pmmeta}
\pmcanonicalname{EveryPIDIsAUFD}
\pmcreated{2013-03-22 16:55:51}
\pmmodified{2013-03-22 16:55:51}
\pmowner{rm50}{10146}
\pmmodifier{rm50}{10146}
\pmtitle{every PID is a UFD}
\pmrecord{9}{39196}
\pmprivacy{1}
\pmauthor{rm50}{10146}
\pmtype{Theorem}
\pmcomment{trigger rebuild}
\pmclassification{msc}{13F07}
\pmclassification{msc}{16D25}
\pmclassification{msc}{11N80}
\pmclassification{msc}{13G05}
\pmclassification{msc}{13A15}
\pmrelated{UFD}
\pmrelated{UniqueFactorizationAndIdealsInRingOfIntegers}

% this is the default PlanetMath preamble.  as your knowledge
% of TeX increases, you will probably want to edit this, but
% it should be fine as is for beginners.

% almost certainly you want these
\usepackage{amssymb}
\usepackage{amsmath}
\usepackage{amsfonts}

% used for TeXing text within eps files
%\usepackage{psfrag}
% need this for including graphics (\includegraphics)
%\usepackage{graphicx}
% for neatly defining theorems and propositions
\usepackage{amsthm}
% making logically defined graphics
%%%\usepackage{xypic}

% there are many more packages, add them here as you need them

% define commands here
% Some sets
\newcommand{\Nats}{\mathbb{N}}
\newcommand{\Ints}{\mathbb{Z}}
\newcommand{\Reals}{\mathbb{R}}
\newcommand{\Complex}{\mathbb{C}}
\newcommand{\Rats}{\mathbb{Q}}
\newcommand{\Gal}{\operatorname{Gal}}
\newcommand{\Cl}{\operatorname{Cl}}
\newcommand{\Alg}{\mathcal{O}}
\newcommand{\ol}{\overline}
\newcommand{\Leg}[2]{\left(\frac{#1}{#2}\right)}
%
%% \theoremstyle{plain} %% This is the default
\newtheorem{thm}{Theorem}
\newtheorem{cor}[thm]{Corollary}
\newtheorem{lem}[thm]{Lemma}
\newtheorem{prop}[thm]{Proposition}
\newtheorem{ax}{Axiom}

\theoremstyle{definition}
\newtheorem{defn}{Definition}
\begin{document}
\begin{thm} Every Principal Ideal Domain (PID) is a Unique Factorization Domain (UFD).
\end{thm}

The first step of the proof shows that any PID is a Noetherian ring in which every irreducible is prime. The second step is to show that any Noetherian ring in which every irreducible is prime is a UFD.

We will need the following
\begin{lem} Every PID $R$ is a gcd domain. Any two gcd's of a pair of elements $a,b$ are associates of each other.
\end{lem}
\begin{proof} Suppose $a,b\in R$. Consider the ideal generated by $a$ and $b$, $(a,b)$. Since $R$ is a PID, there is an element $d\in R$ such that $(a,b)=(d)$. But $a,b\in(a,b)$, so $d\mid a, d\mid b$. So $d$ is a common divisor of $a$ and $b$. Now suppose $c\mid a, c\mid b$. Then $(d)=(a,b)\subset (c)$ and hence $c\mid d$.

The second part of the lemma follows since if $c,d$ are two such gcd's, then $(c)=(a,b)=(d)$, so $c\mid d$ and $d\mid c$ so that $c,d$ are associates.
\end{proof}

\begin{thm} If $R$ is a PID, then $R$ is Noetherian and every irreducible element of $R$ is prime.
\end{thm}
\begin{proof}

Let $I_1\subset I_2\subset I_3\subset \ldots$ be a chain of (principal) ideals
in $R$. Then $I_\infty = \cup_k I_k$ is also an ideal. Since $R$ is a PID, there is $a\in R$ such that $I_\infty=(a)$, and thus $a\in I_n$ for some $n$. Then for each $m>n$, $I_m=I_n$. So $R$ satisfies the ascending chain condition and thus is Noetherian.

To show that each irreducible in $R$ is prime, choose some irreducible $a\in R$, and suppose $a=bc$. Let $d=\gcd(a,b)$. Now, $d\mid a$, but $a$ is irreducible. Thus either $d$ is a unit, or $d$ is an associate of $a$. If $d$ is an associate of $a$, then $a\mid d\mid b$ so that $a\mid b$ and $c$ is a unit. If $d$ is itself a unit, then we can assume by the lemma that $d=1$. Then $1\in(a,b)$ so that there are $x,y\in R$ such that $xa+yb=1$. Multiplying through by $c$, we see that $xac+ybc=c$. But $a\mid xac$ and $a\mid ybc=ya$. Thus $a\mid c$ so that $b$ is a unit. In either case, $a$ is prime.
\end{proof}

\begin{thm} If $R$ is Noetherian, and if every irreducible element of $R$ is prime, then $R$ is a UFD.
\end{thm}

\begin{proof} We show that any nonzero nonunit is $R$ is expressible as a product of irreducibles (and hence as a product of primes), and then show that the factorization is unique.

Let $\mathcal{U}\subset R$ be the set of ideals generated by each element of $R$ that cannot be written as a product of irreducible elements of $R$. If $\mathcal{U}\neq\emptyset$, then $\mathcal{U}$ has a maximal element $(r)$ since $R$ is Noetherian. $r$ is not irreducible by construction and thus not prime, so $(r)$ is not prime and thus not maximal. So there is a proper maximal ideal $(s)$ with $(r)\subsetneq (s)$, and $s\mid r$.

Since $(r)$ is maximal in $\mathcal{U}$, it follows that $(s)\notin \mathcal{U}$ and thus that $s$ is a product of irreducibles. Choose some irreducible $a\mid s$; then $a\mid r$ and
\[r=ab\]
for some $b\in R$. If $(b)\notin \mathcal{U}$ (note that this includes the case where $b$ is a unit), then $b$ and hence $r$ is a product of irreducibles, a contradiction. If $(b)\in \mathcal{U}$ then $(r)\subset (b)$ (since $b\mid r$). $(r)\neq (b)$ since $a$ is not a unit, and thus $(r)\subsetneq (b)$. This contradicts the presumed maximality of $(r)$ in $\mathcal{U}$. Thus $\mathcal{U}=\emptyset$ and each element of $R$ can be written as a product of irreducibles (primes).

The proof of uniqueness is identical to the standard proof for the integers. Suppose
\[a = p_1\cdot \ldots \cdot p_n = q_1\cdot \ldots \cdot q_m\]
where the $p_i$ and $q_j$ are primes. Then $p_1\mid q_1\cdot\ldots\cdot q_m$; since $p_1$ is prime, it must divide some $q_j$. Reordering if necessary, assume $j=1$. Then $p_1=u\cdot q_1$ where $u$ is a unit. Factoring out these terms since $R$ is a domain, we get
\[p_2\cdot\ldots\cdot p_n=u\cdot q_2\cdot\ldots\cdot q_m\]
We may continue the process, matching prime factors from the two sides.
\end{proof}
%%%%%
%%%%%
\end{document}
