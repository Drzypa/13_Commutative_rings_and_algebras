\documentclass[12pt]{article}
\usepackage{pmmeta}
\pmcanonicalname{SchreierDomain}
\pmcreated{2013-03-22 14:50:41}
\pmmodified{2013-03-22 14:50:41}
\pmowner{CWoo}{3771}
\pmmodifier{CWoo}{3771}
\pmtitle{Schreier domain}
\pmrecord{7}{36514}
\pmprivacy{1}
\pmauthor{CWoo}{3771}
\pmtype{Definition}
\pmcomment{trigger rebuild}
\pmclassification{msc}{13G05}
\pmsynonym{pre-Schreier}{SchreierDomain}
\pmdefines{pre-Schreier domain}

% this is the default PlanetMath preamble.  as your knowledge
% of TeX increases, you will probably want to edit this, but
% it should be fine as is for beginners.

% almost certainly you want these
\usepackage{amssymb,amscd}
\usepackage{amsmath}
\usepackage{amsfonts}

% used for TeXing text within eps files
%\usepackage{psfrag}
% need this for including graphics (\includegraphics)
%\usepackage{graphicx}
% for neatly defining theorems and propositions
%\usepackage{amsthm}
% making logically defined graphics
%%%\usepackage{xypic}

% there are many more packages, add them here as you need them

% define commands here
\begin{document}
An integral domain $D$ is a \emph{pre-Schreier domain} if every non-zero element of $D$ is primal.  If in addition $D$ is integrally closed, then $D$ is called a \emph{Schreier domain}.

\textbf{Remarks.}
\begin{enumerate}
\item Every irreducible element of a pre-Schreier domain is prime.
\item A gcd domain is a Schreier domain (a proof of this can be found \PMlinkname{here}{ProofThatAGcdDomainIsIntegrallyClosed}).
\end{enumerate}
%%%%%
%%%%%
\end{document}
