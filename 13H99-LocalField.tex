\documentclass[12pt]{article}
\usepackage{pmmeta}
\pmcanonicalname{LocalField}
\pmcreated{2013-03-22 12:48:07}
\pmmodified{2013-03-22 12:48:07}
\pmowner{djao}{24}
\pmmodifier{djao}{24}
\pmtitle{local field}
\pmrecord{7}{33119}
\pmprivacy{1}
\pmauthor{djao}{24}
\pmtype{Definition}
\pmcomment{trigger rebuild}
\pmclassification{msc}{13H99}
\pmclassification{msc}{12J99}
\pmclassification{msc}{11S99}

\endmetadata

% this is the default PlanetMath preamble.  as your knowledge
% of TeX increases, you will probably want to edit this, but
% it should be fine as is for beginners.

% almost certainly you want these
\usepackage{amssymb}
\usepackage{amsmath}
\usepackage{amsfonts}

% used for TeXing text within eps files
%\usepackage{psfrag}
% need this for including graphics (\includegraphics)
%\usepackage{graphicx}
% for neatly defining theorems and propositions
%\usepackage{amsthm}
% making logically defined graphics
%%%\usepackage{xypic} 

% there are many more packages, add them here as you need them

% define commands here
\begin{document}
A {\em local field} is a topological field which is Hausdorff and locally compact as a topological space.

Examples of local fields include:
\begin{itemize}
\item Any field together with the discrete topology.
\item The field $\mathbb{R}$ of real numbers.
\item The field $\mathbb{C}$ of complex numbers.
\item The field $\mathbb{Q}_p$ of \PMlinkname{$p$--adic rationals}{PAdicIntegers}, or any finite extension thereof.
\item The field $\mathbb{F}_q((t))$ of formal Laurent series in one variable $t$ with coefficients in the finite field $\mathbb{F}_q$ of $q$ elements.
\end{itemize}
In fact, this list is complete---every local field is isomorphic as a topological field to one of the above fields.

\section{Acknowledgements}

This document is dedicated to those who made it all the way through Serre's book~\cite{serre} before realizing that nowhere within the book is there a definition of the term ``local field.''

\begin{thebibliography}{9}
\bibitem{serre} Jean--Pierre Serre, {\em Local Fields}, Springer--Verlag, 1979 (GTM {\bf 67}).
\end{thebibliography}
%%%%%
%%%%%
\end{document}
