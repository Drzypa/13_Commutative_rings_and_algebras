\documentclass[12pt]{article}
\usepackage{pmmeta}
\pmcanonicalname{Valuation}
\pmcreated{2013-03-22 12:35:07}
\pmmodified{2013-03-22 12:35:07}
\pmowner{djao}{24}
\pmmodifier{djao}{24}
\pmtitle{valuation}
\pmrecord{17}{32835}
\pmprivacy{1}
\pmauthor{djao}{24}
\pmtype{Definition}
\pmcomment{trigger rebuild}
\pmclassification{msc}{13F30}
\pmclassification{msc}{13A18}
\pmclassification{msc}{12J20}
\pmclassification{msc}{11R99}
\pmsynonym{absolute value}{Valuation}
\pmrelated{DiscreteValuationRing}
\pmrelated{DiscreteValuation}
\pmrelated{Ultrametric}
\pmrelated{HenselianField}
\pmdefines{infinite prime}
\pmdefines{finite prime}
\pmdefines{archimedean}
\pmdefines{non-archimedean}
\pmdefines{real prime}
\pmdefines{complex prime}
\pmdefines{prime}

% this is the default PlanetMath preamble.  as your knowledge
% of TeX increases, you will probably want to edit this, but
% it should be fine as is for beginners.

% almost certainly you want these
\usepackage{amssymb}
\usepackage{amsmath}
\usepackage{amsfonts}

% used for TeXing text within eps files
%\usepackage{psfrag}
% need this for including graphics (\includegraphics)
%\usepackage{graphicx}
% for neatly defining theorems and propositions
%\usepackage{amsthm}
% making logically defined graphics
%%%\usepackage{xypic} 

% there are many more packages, add them here as you need them

% define commands here
\newcommand{\lra}{\longrightarrow}
\newcommand{\R}{\mathbb{R}}
\newcommand{\C}{\mathbb{C}}
\newcommand{\p}{\mathfrak{p}}
\begin{document}
Let $K$ be a field. A \emph{valuation} or \emph{absolute value} on $K$ is a function $|\cdot|\colon K \to \R$ satisfying the properties:
\begin{enumerate}
\item $|x| \geq 0$ for all $x \in K$, with equality if and only if $x=0$
\item $|xy| = |x|\cdot |y|$ for all $x,y \in K$
\item $|x+y| \leq |x| + |y|$
\end{enumerate}
If a valuation satisfies $|x+y| \leq \max(|x|, |y|)$, then we say that it is a \emph{non-archimedean valuation}. Otherwise we say that it is an \emph{archimedean valuation}.

Every valuation on $K$ defines a metric on $K$, given by $d(x,y) := |x-y|$. This metric is an ultrametric if and only if the valuation is non-archimedean. Two valuations are \emph{equivalent} if their corresponding metrics induce the same topology on $K$. An equivalence class $v$ of valuations on $K$ is called a \emph{prime} of $K$. If $v$ consists of archimedean valuations, we say that $v$ is an \emph{infinite prime}, or \emph{archimedean prime}. Otherwise, we say that $v$ is a \emph{finite prime}, or \emph{non-archimedean prime}.

In the case where $K$ is a number field, primes as defined above generalize the notion of prime ideals in the following way. Let $\p \subset K$ be a nonzero prime ideal\footnote{By ``prime ideal'' we mean ``prime fractional ideal of $K$'' or equivalently ``prime ideal of the ring of integers of $K$''. We do not mean literally a prime ideal of the ring $K$, which would be the zero ideal.}, considered as a fractional ideal. For every nonzero element $x \in K$, let $r$ be the unique integer such that $x \in \p^r$ but $x \notin \p^{r+1}$. Define
$$
|x|_\p :=
\begin{cases}
1/N(\p)^r & x \neq 0, \\
0 & x=0,
\end{cases}
$$
where $N(\p)$ denotes the absolute norm of $\p$. Then $|\cdot|_\p$ is a non--archimedean valuation on $K$, and furthermore every non-archimedean valuation on $K$ is equivalent to $|\cdot|_\p$ for some prime ideal $\p$. Hence, the prime ideals of $K$ correspond bijectively with the finite primes of $K$, and it is in this sense that the notion of primes as valuations generalizes that of a prime ideal.

As for the archimedean valuations, when $K$ is a number field every embedding of $K$ into $\R$ or $\C$ yields a valuation of $K$ by way of the standard absolute value on $\R$ or $\C$, and one can show that every archimedean valuation of $K$ is equivalent to one arising in this way. Thus the infinite primes of $K$ correspond to embeddings of $K$ into $\R$ or $\C$.  Such a prime is called real or complex according to whether the valuations comprising it arise from real or complex embeddings.
%%%%%
%%%%%
\end{document}
