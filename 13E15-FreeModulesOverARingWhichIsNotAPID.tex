\documentclass[12pt]{article}
\usepackage{pmmeta}
\pmcanonicalname{FreeModulesOverARingWhichIsNotAPID}
\pmcreated{2013-03-22 18:50:08}
\pmmodified{2013-03-22 18:50:08}
\pmowner{joking}{16130}
\pmmodifier{joking}{16130}
\pmtitle{free modules over a ring which is not a PID}
\pmrecord{5}{41640}
\pmprivacy{1}
\pmauthor{joking}{16130}
\pmtype{Definition}
\pmcomment{trigger rebuild}
\pmclassification{msc}{13E15}

% this is the default PlanetMath preamble.  as your knowledge
% of TeX increases, you will probably want to edit this, but
% it should be fine as is for beginners.

% almost certainly you want these
\usepackage{amssymb}
\usepackage{amsmath}
\usepackage{amsfonts}

% used for TeXing text within eps files
%\usepackage{psfrag}
% need this for including graphics (\includegraphics)
%\usepackage{graphicx}
% for neatly defining theorems and propositions
%\usepackage{amsthm}
% making logically defined graphics
%%%\usepackage{xypic}

% there are many more packages, add them here as you need them

% define commands here

\begin{document}
Let $R$ be a unital ring. In the following modules will be left modules.

We will say that $R$ has \textit{the free submodule property} if for any free module $F$ over $R$ and any submodule $F'\subseteq F$ we have that $F'$ is also free. It is well known, that if $R$ is a PID, then $R$ has the free submodule property. One can ask whether the converse is also true? We will try to answer this question.

\textbf{Proposition.} If $R$ is a commutative ring, which is not a PID, then $R$ does not have the free submodule property.

\textit{Proof.} Assume that $R$ is not a PID. Then there are two possibilities: either $R$ is not a domain or there is an ideal $I\subseteq R$ which is not principal. Assume that $R$ is not a domain and let $a,b\in R$ be two zero divisors, i.e. $a\neq 0$, $b\neq 0$ and $a\cdot b=0$. Let $(b)\subseteq R$ be an ideal generated by $b$. Then obviously $(b)$ is a submodule of $R$ (regarded as a $R$-module). Assume that $(b)$ is free. In particular there exists $m \in (b)$, $m\neq 0$ such that $r\cdot m=0$ if and only if $r=0$. But $m$ is of the form $\lambda\cdot b$ and because $R$ is commutative we have
$$a\cdot m=a\cdot (\lambda\cdot b)=\lambda\cdot (a\cdot b)=0.$$
Contradiction, because $a\neq 0$. Thus $(b)$ is not free although $(b)$ is a submodule of a free module $R$.

Assume now that there is an ideal $I\subseteq R$ which is not principal and assume that $I$ is free as a $R$-module. Since $I$ is not principal, then there exist $a,b\in I$ such that $\{a,b\}$ is linearly independent. On the other hand $a,b\in R$ and $1$ is a free generator of $R$. Thus $\{1,a\}$ is linearly dependent, so
$$\lambda\cdot 1+ \alpha\cdot a=0$$
for some nonzero $\lambda, \alpha\in R$ (note that in this case both $\lambda,\alpha$ are nonzero, more precisely $\lambda=a$ and $\alpha=-1$). Multiply the equation by $b$. Thus we have
$$\lambda\cdot b + (\alpha\cdot b)\cdot a=0.$$
Note that here we used commutativity of $R$. Since $\{a,b\}$ is linearly independend (in $I$), then the last equation implies that $\lambda=0$. Contradiction. $\square$

\textbf{Corollary.} Commutative ring is a PID if and only if it has the free submodule property.
%%%%%
%%%%%
\end{document}
