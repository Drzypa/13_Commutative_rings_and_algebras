\documentclass[12pt]{article}
\usepackage{pmmeta}
\pmcanonicalname{UnityPlusNilpotentIsUnit}
\pmcreated{2013-03-22 15:11:54}
\pmmodified{2013-03-22 15:11:54}
\pmowner{Wkbj79}{1863}
\pmmodifier{Wkbj79}{1863}
\pmtitle{unity plus nilpotent is unit}
\pmrecord{21}{36956}
\pmprivacy{1}
\pmauthor{Wkbj79}{1863}
\pmtype{Theorem}
\pmcomment{trigger rebuild}
\pmclassification{msc}{13A10}
\pmclassification{msc}{16U60}
\pmrelated{DivisibilityInRings}

% this is the default PlanetMath preamble.  as your knowledge
% of TeX increases, you will probably want to edit this, but
% it should be fine as is for beginners.

% almost certainly you want these
\usepackage{amssymb}
\usepackage{amsmath}
\usepackage{amsfonts}

% used for TeXing text within eps files
%\usepackage{psfrag}
% need this for including graphics (\includegraphics)
%\usepackage{graphicx}
% for neatly defining theorems and propositions
 \usepackage{amsthm}
% making logically defined graphics
%%%\usepackage{xypic}

% there are many more packages, add them here as you need them

% define commands here

\theoremstyle{definition}
\newtheorem*{thmplain}{Theorem}
\begin{document}
\begin{thmplain}
If $x$ is a nilpotent element of a ring with unity 1 (which may be 0), then the sum $1\!+\!x$ is a unit of the ring.
\end{thmplain}

\begin{proof}
If $x=0$, then $1\!+\!x=1$, which is a unit.  Thus, we may assume that $x \neq 0$.

Since $x$ is nilpotent, there is a positive integer $n$ such that $x^n=0$. We multiply $1\!+\!x$ by another ring element:
\begin{eqnarray*}
(1\!+\!x)\cdot\sum_{j=0}^{n-1}(-1)^jx^j
&=& \sum_{j=0}^{n-1}(-1)^jx^j\!+\!\sum_{k=0}^{n-1}(-1)^kx^{k+1}\\
&=& \sum_{j=0}^{n-1}(-1)^jx^j\!-\!\sum_{k=1}^n(-1)^kx^k\\
&=& 1\!+\!\sum_{j=1}^{n-1}(-1)^jx^j\!-\!\sum_{k=1}^{n-1}(-1)^kx^k\!-\!(-1)^nx^n\\
&=& 1\!+\!0\!+\!0\\
&=& 1
\end{eqnarray*}

(Note that the summations include the term\, $(-1)^0x^0$, which is why $x=0$ is excluded from this case.)

The reversed multiplication gives the same result. Therefore, $1\!+\!x$ has a multiplicative inverse and thus is a unit.
\end{proof}

Note that there is a \PMlinkescapetext{similarity between} this proof and geometric series:  The goal was to produce a multiplicative inverse of $1\!+\!x$, and geometric series yields that

$$\displaystyle \frac{1}{1\!+\!x}=\sum_{n=0}^{\infty} (-1)^nx^n,$$

provided that the summation \PMlinkname{converges}{AbsoluteConvergence}.  Since $x$ is nilpotent, the summation has a finite number of nonzero terms and thus \PMlinkescapetext{converges}.
%%%%%
%%%%%
\end{document}
