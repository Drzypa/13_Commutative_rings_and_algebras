\documentclass[12pt]{article}
\usepackage{pmmeta}
\pmcanonicalname{InvertibilityOfRegularlyGeneratedIdeal}
\pmcreated{2015-05-06 15:27:47}
\pmmodified{2015-05-06 15:27:47}
\pmowner{pahio}{2872}
\pmmodifier{pahio}{2872}
\pmtitle{invertibility of regularly generated ideal}
\pmrecord{17}{36984}
\pmprivacy{1}
\pmauthor{pahio}{2872}
\pmtype{Theorem}
\pmcomment{trigger rebuild}
\pmclassification{msc}{13A15}
\pmclassification{msc}{11R04}
%\pmkeywords{invertible ideal}
\pmrelated{IdealMultiplicationLaws}
\pmrelated{PruferRing}
\pmrelated{InvertibleIdealIsFinitelyGenerated}

\endmetadata

% this is the default PlanetMath preamble.  as your knowledge
% of TeX increases, you will probably want to edit this, but
% it should be fine as is for beginners.

% almost certainly you want these
\usepackage{amssymb}
\usepackage{amsmath}
\usepackage{amsfonts}

% used for TeXing text within eps files
%\usepackage{psfrag}
% need this for including graphics (\includegraphics)
%\usepackage{graphicx}
% for neatly defining theorems and propositions
 \usepackage{amsthm}
% making logically defined graphics
%%%\usepackage{xypic}

% there are many more packages, add them here as you need them

% define commands here

\theoremstyle{definition}
\newtheorem*{thmplain}{Theorem}
\begin{document}
\textbf{Lemma.} \, Let $R$ be a commutative ring containing regular elements. \,If $\mathfrak{a}$, $\mathfrak{b}$ and $\mathfrak{c}$ are three ideals of $R$ such that \,$\mathfrak{b\!+\!c}$,\, $\mathfrak{c\!+\!a}$\, and\, $\mathfrak{a\!+\!b}$\, are \PMlinkname{invertible}{FractionalIdealOfCommutativeRing}, then also their sum ideal\, $\mathfrak{a\!+\!b\!+\!c}$\, is \PMlinkescapetext{invertible}.

{\em Proof.} \,We may assume that $R$ has a unity, therefore the product of an ideal and its \PMlinkname{inverse}{FractionalIdealOfCommutativeRing} is always $R$.\, Now, the ideals\, $\mathfrak{b+c}$,\, $\mathfrak{c+a}$\, and\, $\mathfrak{a+b}$\, have the \PMlinkescapetext{inverses} $\mathfrak{f_1}$, $\mathfrak{f_2}$ and $\mathfrak{f_3}$, respectively, so that
  $$\mathfrak{(b+c)f_1 \;=\; (c+a)f_2 \;=\; (a+b)f_3} \;=\; R.$$
Because\, $\mathfrak{af_2} \subseteq R$\, and\, $\mathfrak{cf_1} \subseteq R$,\, we obtain
\begin{align*} 
\mathfrak{(a+b+c)(af_2f_3+cf_1f_2)} &\;=\; \mathfrak{(a+b)af_2f_3+c(af_2)f_3+a(cf_1)f_2+(b+c)cf_1f_2}\\ 
                                   &\;=\; \mathfrak{af_2+cf_2 \;=\; (c+a)f_2}\\ 
                                   &\;=\; R.
\end{align*}\\

\textbf{Theorem.}\, Let $R$ be a commutative ring containing regular 
elements.\, If every ideal of $R$ generated by two regular elements is \PMlinkescapetext{invertible}, then in $R$ also every ideal generated by a finite set of regular elements is \PMlinkescapetext{invertible}.\\

{\em Proof.} \,We use induction on $n$, the number of the regular elements of the generating set.\, We thus assume that every ideal of $R$ generated by $n$ regular elements\, ($n \geqq 2)$\, is \PMlinkescapetext{invertible}.\, Let \,$\{r_1,\,r_2,\,\ldots,\,r_{n+1}\}$ be any set of regular elements of $R$.\, Denote 
 $$\mathfrak{a} \;=:\; (r_1),\quad \mathfrak{b} \;=:\; (r_2,\,\ldots,\,r_n),
                         \quad \mathfrak{c} \;=:\; (r_{n+1}).$$
The sums \,$\mathfrak{b+c}$, \,$\mathfrak{c+a}$\, and\, $\mathfrak{a+b}$\, are, by the assumptions, \PMlinkescapetext{invertible}.\, Then the ideal
    $$(r_1,\,r_2,\,\ldots,\,r_n,\,r_{n+1}) \;=\; \mathfrak{a+b+c}$$
is, by the lemma, \PMlinkescapetext{invertible}, and the induction 
proof is complete.

\begin{thebibliography}{9}
\bibitem{RG}{\sc R. Gilmer:} {\em Multiplicative ideal theory}. \,Queens University Press. Kingston, Ontario (1968).
\end{thebibliography}
%%%%%
%%%%%
\end{document}
