\documentclass[12pt]{article}
\usepackage{pmmeta}
\pmcanonicalname{PrimeElementIsIrreducibleInIntegralDomain}
\pmcreated{2013-03-22 17:15:29}
\pmmodified{2013-03-22 17:15:29}
\pmowner{me_and}{17092}
\pmmodifier{me_and}{17092}
\pmtitle{prime element is irreducible in integral domain}
\pmrecord{7}{39594}
\pmprivacy{1}
\pmauthor{me_and}{17092}
\pmtype{Theorem}
\pmcomment{trigger rebuild}
\pmclassification{msc}{13G05}
\pmrelated{IrreducibleOfAUFDIsPrime}

%\usepackage{amssymb}
%\usepackage{amsmath}
%\usepackage{amsfonts}
\usepackage{amsthm}

%Named sets
%\newcommand{\R}{\mathbb{R}} %Real numbers (amssymb or amsfonts)
%\newcommand{\C}{\mathbb{C}} %Complex numbers (amssymb or amsfonts)

%Functions & operators
%\newcommand{\modulus}[1]{\left|{#1}\right|} %|z|
\newcommand{\divs}[2]{{#1}\mid{#2}}

%Numbers
%\newcommand{\I}{\mathrm{i}} %sqrt{-1}
%\newcommand{\e}{\mathrm{e}} $exponential

%Greek
%\newcommand{\ve}{\varepsilon} %nice epsilon
\begin{document}
\newtheorem*{thm}{Theorem}

\begin{thm}
Every prime element of an integral domain is irreducible.
\end{thm}

\begin{proof}
Let $D$ be an integral domain, and let $a\in D$ be a prime element. Assume $a=bc$ for some $b,c\in D$.

Clearly $\divs{a}{bc}$, so since $a$ is prime, $\divs{a}{b}$ or $\divs{a}{c}$. Without loss of generality, assume $\divs{a}{b}$, and say $at=b$ for some $t\in D$.

If $1$ is the unity of $D$, then
\[
  1b=b=at=(bc)t=b(ct)
.\]
Since $D$ is an integral domain, $b$ can be cancelled, giving $1=ct$, so $c$ is a unit.
\end{proof}


%%%%%
%%%%%
\end{document}
