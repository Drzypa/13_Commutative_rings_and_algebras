\documentclass[12pt]{article}
\usepackage{pmmeta}
\pmcanonicalname{ProofOfNakayamasLemma}
\pmcreated{2013-03-22 13:16:50}
\pmmodified{2013-03-22 13:16:50}
\pmowner{nerdy2}{62}
\pmmodifier{nerdy2}{62}
\pmtitle{proof of Nakayama's lemma}
\pmrecord{6}{33764}
\pmprivacy{1}
\pmauthor{nerdy2}{62}
\pmtype{Proof}
\pmcomment{trigger rebuild}
\pmclassification{msc}{13C99}

\endmetadata

% this is the default PlanetMath preamble.  as your knowledge
% of TeX increases, you will probably want to edit this, but
% it should be fine as is for beginners.

% almost certainly you want these
\usepackage{amssymb}
\usepackage{amsmath}
\usepackage{amsfonts}

% used for TeXing text within eps files
%\usepackage{psfrag}
% need this for including graphics (\includegraphics)
%\usepackage{graphicx}
% for neatly defining theorems and propositions
%\usepackage{amsthm}
% making logically defined graphics
%%%\usepackage{xypic} 

% there are many more packages, add them here as you need them

% define commands here
\begin{document}
(This proof was taken from \cite{localalg}.)

If $M$ were not zero, it would have a simple quotient, isomorphic to $R/\mathfrak{m}$ for some maximal ideal $\mathfrak{m}$ of $R$.  Then we would have $\mathfrak{m}M\neq M$, so that $\mathfrak{a}M\neq M$ as $\mathfrak{a}\subseteq \mathfrak{m}$.

\begin{thebibliography}{9}
  \bibitem{localalg} Serre, J.-P. {\em Local Algebra.}  Springer-Verlag, 2000.
\end{thebibliography}
%%%%%
%%%%%
\end{document}
