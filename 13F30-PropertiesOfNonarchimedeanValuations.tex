\documentclass[12pt]{article}
\usepackage{pmmeta}
\pmcanonicalname{PropertiesOfNonarchimedeanValuations}
\pmcreated{2013-03-22 18:01:11}
\pmmodified{2013-03-22 18:01:11}
\pmowner{rm50}{10146}
\pmmodifier{rm50}{10146}
\pmtitle{properties of non-archimedean valuations}
\pmrecord{6}{40536}
\pmprivacy{1}
\pmauthor{rm50}{10146}
\pmtype{Theorem}
\pmcomment{trigger rebuild}
\pmclassification{msc}{13F30}
\pmclassification{msc}{13A18}
\pmclassification{msc}{12J20}
\pmclassification{msc}{11R99}
\pmrelated{CompleteUltrametricField}

% this is the default PlanetMath preamble.  as your knowledge
% of TeX increases, you will probably want to edit this, but
% it should be fine as is for beginners.

% almost certainly you want these
\usepackage{amssymb}
\usepackage{amsmath}
\usepackage{amsfonts}

% used for TeXing text within eps files
%\usepackage{psfrag}
% need this for including graphics (\includegraphics)
%\usepackage{graphicx}
% for neatly defining theorems and propositions
\usepackage{amsthm}
% making logically defined graphics
%%%\usepackage{xypic}

% there are many more packages, add them here as you need them

% define commands here
\newtheorem{thm}{Theorem}
\newcommand{\Abs}[1]{\left\lvert #1\right\rvert}

\begin{document}
If $K$ is a field, and $\Abs{\cdot}$ a nontrivial non-archimedean valuation (or absolute value) on $K$, then $\Abs{\cdot}$ has some properties that are counterintuitive (and that are false for archimedean valuations).
\begin{thm}
Let $K$ be a field with a non-archimedean absolute value $\Abs{\cdot}$. For $r>0$ a real number, $x\in K$, define
\begin{gather*}
B(x,r) = \{y\in K\ \mid\ \Abs{x-y}<r\},\ \text{the open ball of radius $r$ at $x$}\\
\bar{B}(x,r) = \{y\in K\ \mid\ \Abs{x-y}\leq r\},\ \text{the closed ball of radius $r$ at $x$}
\end{gather*}
Then
\begin{enumerate}
\item $B(x,r)$ is both open and closed;
\item $\bar{B}(x,r)$ is both open and closed;
\item If $y\in B(x,r)$ (resp. $\bar{B}(x,r)$) then $B(x,r)=B(y,r)$ (resp. $\bar{B}(x,r)=\bar{B}(y,r)$);
\item $B(x,r)$ and $B(y,r)$ (resp. $\bar{B}(x,r)$ and $\bar{B}(y,r)$) are either identical or disjoint;
\item If $B_1 = B(x,r)$ and $B_2 = B(y,s)$ are not disjoint, then either $B_1\subset B_2$ or $B_2\subset B_1$;
\item If $(x_n)$ is a sequence of elements of $K$ with $\lim_{n\to\infty} x_n=0$, then $\sum_{n=1}^{\infty} x_n$ is Cauchy (and thus if $K$ is complete, a sufficient condition for convergence of a series is that the terms tend to zero)
\end{enumerate}
\end{thm}

\textbf{Proof. }\ We start by proving (3). Suppose $y\in B(x,r)$. If $z\in B(x,r)$, then since the absolute value is non-archimedean, we have
\[\Abs{z-y} =\Abs{(z-x)+(x-y)} \leq \max(\Abs{z-x},\Abs{x-y}) < r\]
so that $z\in B(y,r)$. Clearly $x\in B(y,r)$, so reversing the roles of $x$ and $y$, we see that $B(x,r) = B(y,r)$. Finally, replacing $B$ by $\bar{B}$ and $<$ by $\leq$, we get equality of closed balls as well.

(4) is now trivial: If $B(x,r)\cap B(y,r)\neq\emptyset$, choose $z\in B(x,r)\cap B(y,r)$; then by (3), $B(x,r) = B(z,r) = B(y,r)$. An identical argument proves the result for closed balls.

To prove (5), choose $z\in B_1\cap B_2$. Assume first that $r\leq s$; then $B(z,r) = B_1$, and $B(z,r)\subset B(z,s) = B_2$, so that $B_1\subset B_2$. If $s\leq r$, then we have identically that $B_2\subset B_1$. (Note that (4) is a special case when $r=s$).

(1) and (2) now follow: for (1), note that $B(x,r)$ is obviously open; its complement consists of a union of open balls of radius $r$ disjoint with $B(x,r)$ and its complement is therefore open. Thus $B(x,r)$ is closed. For (2), $\bar{B}(x,r)$ is obviously closed; to see that it is open, take any $y\in \bar{B}(x,r)$; then $\bar{B}(x,r)=\bar{B}(y,r)$ and thus $B(y,s)\subset \bar{B}(y,r)$ for $s<r$ is an open neighborhood of $y$ contained in $\bar{B}(x,r)$, which is therefore open.

Finally, to prove (6), we must show that given $\epsilon$, we can find $N>0$ sufficiently large such that $\Abs{\sum_{i=m}^n x_i}<\epsilon$ whenever $m,n>N$. Simply choose $N$ such that $\Abs{x_i}<\epsilon$ for $i>N$; then
\[\Abs{\sum_{i=m}^n x_i}\leq \max(\Abs{x_m},\ldots,\Abs{x_n}) < \epsilon\]
%%%%%
%%%%%
\end{document}
