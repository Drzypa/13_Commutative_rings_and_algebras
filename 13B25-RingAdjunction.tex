\documentclass[12pt]{article}
\usepackage{pmmeta}
\pmcanonicalname{RingAdjunction}
\pmcreated{2014-02-18 14:13:46}
\pmmodified{2014-02-18 14:13:46}
\pmowner{pahio}{2872}
\pmmodifier{pahio}{2872}
\pmtitle{ring adjunction}
\pmrecord{17}{35874}
\pmprivacy{1}
\pmauthor{pahio}{2872}
\pmtype{Definition}
\pmcomment{trigger rebuild}
\pmclassification{msc}{13B25}
\pmclassification{msc}{13B02}
\pmrelated{GeneratedSubring}
\pmrelated{FiniteRingHasNoProperOverrings}
\pmrelated{GroundFieldsAndRings}
\pmrelated{PolynomialRingOverIntegralDomain}
\pmrelated{AConditionOfAlgebraicExtension}
\pmrelated{IntegralClosureIsRing}

% this is the default PlanetMath preamble.  as your knowledge
% of TeX increases, you will probably want to edit this, but
% it should be fine as is for beginners.

% almost certainly you want these
\usepackage{amssymb}
\usepackage{amsmath}
\usepackage{amsfonts}

% used for TeXing text within eps files
%\usepackage{psfrag}
% need this for including graphics (\includegraphics)
%\usepackage{graphicx}
% for neatly defining theorems and propositions
%\usepackage{amsthm}
% making logically defined graphics
%%%\usepackage{xypic}

% there are many more packages, add them here as you need them

% define commands here
\begin{document}
Let $R$ be a commutative ring and $E$ an extension ring of it.\, If\, $\alpha \in E$\, and commutes with all elements of $R$, then the smallest subring of $E$ containing $R$ and $\alpha$ is denoted by $R[\alpha]$.\, We say that $R[\alpha]$ is obtained from $R$ by adjoining  $\alpha$ to $R$ via {\em ring adjunction}. 

By the \PMlinkescapetext{Theorem 1} about ``evaluation 
homomorphism'', 
     $$R[\alpha] = \{f(\alpha)\mid \, f(X)\in R[X]\},$$
where $R[X]$ is the polynomial ring in one indeterminate over 
$R$.\, Therefore, $R[\alpha]$ consists of all expressions which 
can be formed of $\alpha$ and elements of the ring $R$ by using 
additions, subtractions and multiplications.

\textbf{Examples:}\, The polynomial rings $R[X]$, the ring $\mathbb{Z}[i]$ of the Gaussian integers, the ring $\mathbb{Z}[\frac{-1+i\sqrt{3}}{2}]$ of Eisenstein integers.
%%%%%
%%%%%
\end{document}
