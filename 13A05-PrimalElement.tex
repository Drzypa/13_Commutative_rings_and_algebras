\documentclass[12pt]{article}
\usepackage{pmmeta}
\pmcanonicalname{PrimalElement}
\pmcreated{2013-03-22 14:50:21}
\pmmodified{2013-03-22 14:50:21}
\pmowner{CWoo}{3771}
\pmmodifier{CWoo}{3771}
\pmtitle{primal element}
\pmrecord{8}{36508}
\pmprivacy{1}
\pmauthor{CWoo}{3771}
\pmtype{Definition}
\pmcomment{trigger rebuild}
\pmclassification{msc}{13A05}
\pmdefines{primal}

\endmetadata

% this is the default PlanetMath preamble.  as your knowledge
% of TeX increases, you will probably want to edit this, but
% it should be fine as is for beginners.

% almost certainly you want these
\usepackage{amssymb,amscd}
\usepackage{amsmath}
\usepackage{amsfonts}

% used for TeXing text within eps files
%\usepackage{psfrag}
% need this for including graphics (\includegraphics)
%\usepackage{graphicx}
% for neatly defining theorems and propositions
\usepackage{amsthm}
% making logically defined graphics
%%%\usepackage{xypic}

% there are many more packages, add them here as you need them

% define commands here
\begin{document}
An element $r$ in a commutative ring $R$ is called \emph{primal} if whenever $r\mid ab$, with $a,b\in R$, then there 
exist elements $s,t\in R$ such that
\begin{enumerate}
\item $r=st$,
\item $s\mid a$ and $t\mid b$.
\end{enumerate}

\textbf{Lemma}.  In a commutative ring, an element that is both irreducible and primal is a prime element.
\begin{proof}
Suppose $a$ is irreducible and primal, and $a\mid bc$.  Since $a$ is primal, there is $x,y\in R$ such that $a=xy$, with $x\mid b$ and $y\mid c$.  Since $a$ is irreducible, either $x$ or $y$ is a unit.  If $x$ is a unit, with $z$ as its inverse, then $za=zxy=y$, so that $a\mid y$.  But $y\mid c$, we have that $a\mid c$.
\end{proof}
%%%%%
%%%%%
\end{document}
