\documentclass[12pt]{article}
\usepackage{pmmeta}
\pmcanonicalname{Pring}
\pmcreated{2013-03-22 15:14:28}
\pmmodified{2013-03-22 15:14:28}
\pmowner{alozano}{2414}
\pmmodifier{alozano}{2414}
\pmtitle{p-ring}
\pmrecord{4}{37016}
\pmprivacy{1}
\pmauthor{alozano}{2414}
\pmtype{Definition}
\pmcomment{trigger rebuild}
\pmclassification{msc}{13J10}
\pmclassification{msc}{13K05}
\pmsynonym{$p$-ring}{Pring}
\pmsynonym{p-adic ring}{Pring}
\pmsynonym{$p$-adic ring}{Pring}
\pmsynonym{strict $p$-ring}{Pring}
\pmdefines{strict p-ring}

% this is the default PlanetMath preamble.  as your knowledge
% of TeX increases, you will probably want to edit this, but
% it should be fine as is for beginners.

% almost certainly you want these
\usepackage{amssymb}
\usepackage{amsmath}
\usepackage{amsthm}
\usepackage{amsfonts}

% used for TeXing text within eps files
%\usepackage{psfrag}
% need this for including graphics (\includegraphics)
%\usepackage{graphicx}
% for neatly defining theorems and propositions
%\usepackage{amsthm}
% making logically defined graphics
%%%\usepackage{xypic}

% there are many more packages, add them here as you need them

% define commands here

\newtheorem{thm}{Theorem}
\newtheorem{defn}{Definition}
\newtheorem{prop}{Proposition}
\newtheorem{lemma}{Lemma}
\newtheorem{cor}{Corollary}

\theoremstyle{definition}
\newtheorem{exa}{Example}

% Some sets
\newcommand{\Nats}{\mathbb{N}}
\newcommand{\Ints}{\mathbb{Z}}
\newcommand{\Reals}{\mathbb{R}}
\newcommand{\Complex}{\mathbb{C}}
\newcommand{\Rats}{\mathbb{Q}}
\newcommand{\Gal}{\operatorname{Gal}}
\newcommand{\Cl}{\operatorname{Cl}}
\newcommand{\mA}{\mathfrak{A}}
\begin{document}
\begin{defn}
Let $R$ be a commutative ring with identity element equipped with a topology defined by a decreasing sequence:
$$\ldots \subset \mA_3 \subset \mA_2 \subset \mA_1$$
of ideals such that $\mA_n\cdot \mA_m \subset \mA_{n+m}$. We say that $R$ is a $p$-ring if the following conditions are satisfied:
\begin{enumerate}
\item The residue ring $\overline{k}=R/\mA_1$ is a perfect ring of characteristic $p$.

\item The ring $R$ is Hausdorff and complete for its topology.
\end{enumerate}
\end{defn}

\begin{defn}
A $p$-ring $R$ is said to be strict (or a $p$-adic ring) if the topology is defined by the $p$-adic filtration $\mA_n=p^nR$, and $p$ is not a zero-divisor of $R$.
\end{defn}

\begin{exa}
The prototype of strict $p$-ring is the ring of \PMlinkname{$p$-adic integers}{PAdicIntegers} $\Ints_p$ with the usual profinite topology.
\end{exa}

\begin{thebibliography}{9}
\bibitem{serre} J. P. Serre, {\em Local Fields},
Springer-Verlag, New York.
\end{thebibliography}
%%%%%
%%%%%
\end{document}
