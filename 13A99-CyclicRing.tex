\documentclass[12pt]{article}
\usepackage{pmmeta}
\pmcanonicalname{CyclicRing}
\pmcreated{2013-03-22 13:30:13}
\pmmodified{2013-03-22 13:30:13}
\pmowner{Wkbj79}{1863}
\pmmodifier{Wkbj79}{1863}
\pmtitle{cyclic ring}
\pmrecord{33}{34084}
\pmprivacy{1}
\pmauthor{Wkbj79}{1863}
\pmtype{Definition}
\pmcomment{trigger rebuild}
\pmclassification{msc}{13A99}
\pmclassification{msc}{16U99}
\pmrelated{CyclicGroup}
\pmrelated{ProofOfTheConverseOfLagrangesTheoremForCyclicGroups}
\pmrelated{CriterionForCyclicRingsToBePrincipalIdealRings}
\pmrelated{MultiplicativeIdentityOfACyclicRingMustBeAGenerator}

\endmetadata

\usepackage{amssymb}
\usepackage{amsmath}
\usepackage{amsfonts}

\usepackage{psfrag}
\usepackage{graphicx}
\usepackage{amsthm}
%%\usepackage{xypic}
\begin{document}
A ring is a {\sl cyclic ring \/} if its additive group is cyclic.

Every cyclic ring is commutative under multiplication.  For if $R$ is a cyclic ring, $r$ is a \PMlinkname{generator}{Generator} of the additive group of $R$, and $s,t \in R$, then there exist $a,b \in {\mathbb Z}$ such that $s=ar$ and $t=br$.  As a result, $st=(ar)(br)=(ab)r^2=(ba)r^2=(br)(ar)=ts.$  (Note the disguised use of the \PMlinkname{distributive property}{Distributive}.)

A result of the \PMlinkname{fundamental theorem of finite abelian groups}{FundamentalTheoremOfFinitelyGeneratedAbelianGroups} is that every ring with squarefree order is a cyclic ring.

If $n$ is a positive integer, then, up to isomorphism, there are exactly $\tau (n)$ cyclic rings of order $n$, where $\tau$ refers to the tau function.  Also, if a cyclic ring has order $n$, then it has exactly $\tau (n)$ subrings.  This result mainly follows from Lagrange's theorem and its converse.  Note that the converse of Lagrange's theorem does not hold in general, but it does hold for finite cyclic groups.

Every subring of a cyclic ring is a cyclic ring.  Moreover, every subring of a cyclic ring is an ideal.

$R$ is a finite cyclic ring of order $n$ if and only if there exists a positive divisor $k$ of $n$ such that $R$ is isomorphic to $k{\mathbb Z}_{kn}$.  $R$ is an \PMlinkescapetext{infinite} cyclic ring that has no zero divisors if and only if there exists a positive integer $k$ such that $R$ is isomorphic to $k{\mathbb Z}$.  (See behavior and its attachments for details.)  Finally, $R$ is an \PMlinkescapetext{infinite} cyclic ring that has zero divisors if and only if it is isomorphic to the following subset of ${\mathbf M}_{2\operatorname{x}2}({\mathbb Z})$:

\begin{center}
$\left\{ \left. \left( \begin{array}{cc}
c & -c \\
c & -c \end{array} \right) \right| c \in {\mathbb Z} \right\}$
\end{center}

Thus, any \PMlinkescapetext{infinite} cyclic ring that has zero divisors is a zero ring.

\begin{thebibliography}{9}
\bibitem{buck} Buck, Warren.  \emph{\PMlinkexternal{Cyclic Rings}{http://planetmath.org/?op=getobj&from=papers&id=336}}.  Charleston, IL: Eastern Illinois University, 2004.

\bibitem{kruse} Kruse, Robert L. and Price, David T.  \emph{Nilpotent Rings}.  New York: Gordon and Breach, 1969.

\bibitem{maurer} Maurer, I. Gy. and Vincze, J.  ``Despre Inele Ciclice.''  \emph{Studia Universitatis Babe\c{s}-Bolyai.  Series Mathematica-Physica}, vol. 9 \#1.  Cluj, Romania: Universitatea Babe\c{s}-Bolyai, 1964, pp. 25-27.

\bibitem{peinado} Peinado, Rolando E.  ``On Finite Rings.''  \emph{Mathematics Magazine}, vol. 40 \#2.  Buffalo: The Mathematical Association of America, 1967, pp. 83-85.
\end{thebibliography}
%%%%%
%%%%%
\end{document}
