\documentclass[12pt]{article}
\usepackage{pmmeta}
\pmcanonicalname{WeakApproximationTheorem}
\pmcreated{2013-03-22 18:35:21}
\pmmodified{2013-03-22 18:35:21}
\pmowner{rm50}{10146}
\pmmodifier{rm50}{10146}
\pmtitle{weak approximation theorem}
\pmrecord{6}{41316}
\pmprivacy{1}
\pmauthor{rm50}{10146}
\pmtype{Theorem}
\pmcomment{trigger rebuild}
\pmclassification{msc}{13F05}
\pmclassification{msc}{11R04}
\pmrelated{IndependenceOfTheValuations}
\pmrelated{ChineseRemainderTheoremInTermsOfDivisorTheory}

\endmetadata

% this is the default PlanetMath preamble.  as your knowledge
% of TeX increases, you will probably want to edit this, but
% it should be fine as is for beginners.

% almost certainly you want these
\usepackage{amssymb}
\usepackage{amsmath}
\usepackage{amsfonts}

% used for TeXing text within eps files
%\usepackage{psfrag}
% need this for including graphics (\includegraphics)
%\usepackage{graphicx}
% for neatly defining theorems and propositions
\usepackage{amsthm}
% making logically defined graphics
%%%\usepackage{xypic}

% there are many more packages, add them here as you need them

% define commands here
\newcommand{\smp}{\mathfrak{p}}
\newcommand{\smq}{\mathfrak{q}}
\newtheorem{thm}{Theorem}

\begin{document}
The weak approximation theorem allows selection, in a Dedekind ring, of an element having specific valuations at a specific finite set of primes, and nonnegative valuations at all other primes. It is essentially a generalization of the Chinese Remainder theorem, as is evident from its proof.
\begin{thm}[Weak \PMlinkescapetext{Approximation Theorem}] Let $A$ be a Dedekind domain with fraction field $K$. Then for any finite set $\smp_1,\dotsc,\smp_k$ of primes of $A$ and integers $a_1,\dotsc,a_k$, there is $x\in K^{\star}$ such that $\nu_{\smp_i}((x))=a_i$ and for all other prime ideals $\smp$, $\nu_{\smp}((x))\geq 0$. Here $\nu_{\smp}$ is the $\smp$-adic valuation associated with a prime ideal $\smp$.
\end{thm}
\begin{proof}
Assume first that all $a_i\geq 0$. By the Chinese Remainder Theorem,
\[
  A/\smp_1^{a_1+1}\times\cdots A/\smp_k^{a_k+1}\cong A/\smp_1^{a_1+1}\cdots\smp_k^{a_k+1}
\]
Thus the map
\[
  A\to A/\smp_1^{a_1+1}\times\cdots A/\smp_k^{a_k+1}
\]
is surjective. Now choose $x_i\in p_i^{a_i}, x_i\notin p_i^{a_i+1}$; this is possible since these two ideals are unequal by unique factorization. Choose $x\in A$ with image $(x_1,\dotsc,x_k)$. Clearly $\nu_{\smp_i}((x))=a_i$. But $x\in A$, so all other valuations are nonnegative.

In the general case, assume wlog that we are given a set $\smp_1,\dotsc,\smp_r$ of primes of $A$ and integers $a_1,\dotsc,a_r\geq 0$, and a set $\smq_1,\dotsc,\smq_t$ of primes with integers $b_1,\dotsc,b_t<0$. First choose $y\in K^{\star}$ (using the case already proved above) so that
\[
  \begin{cases}
    \nu_{\smp}((y)) = 0    & \smp = \smp_i \\
    \nu_{\smp}((y)) = -b_i & \smp = \smq_j \\
    \nu_{\smp}((y)) \geq 0 & \text{otherwise}
  \end{cases}
\]
Now, there are only a finite number of primes $\smp'_k$ such that $\smp'_k$ is not the same as any of the $\smq_j$ and $\nu_{\smp'_k}((y))>0$. Let $\nu_{\smp'_k}((y)) = c_k>0$. Again using the case proved above, choose $x\in K^{\star}$ such that
\[
  \begin{cases}
    \nu_{\smp}((x)) = a_i  & \smp = \smp_i \\
    \nu_{\smp}((x)) = 0    & \smp = \smq_j \\
    \nu_{\smp}((x)) = c_k  & \smp = \smp'_k\\
    \nu_{\smp}((x)) \geq 0 & \text{otherwise}
  \end{cases}
\]
Then $x/y$ is the required element.
\end{proof}
%%%%%
%%%%%
\end{document}
