\documentclass[12pt]{article}
\usepackage{pmmeta}
\pmcanonicalname{FormalPowerSeriesConvergesIfAndOnlyIfItConvergesAlongEveryLine}
\pmcreated{2013-03-22 17:42:11}
\pmmodified{2013-03-22 17:42:11}
\pmowner{jirka}{4157}
\pmmodifier{jirka}{4157}
\pmtitle{formal power series converges if and only if it converges along every line}
\pmrecord{5}{40145}
\pmprivacy{1}
\pmauthor{jirka}{4157}
\pmtype{Theorem}
\pmcomment{trigger rebuild}
\pmclassification{msc}{13H05}
\pmclassification{msc}{13B35}
\pmclassification{msc}{13J05}
\pmclassification{msc}{13F25}

\endmetadata

% this is the default PlanetMath preamble.  as your knowledge
% of TeX increases, you will probably want to edit this, but
% it should be fine as is for beginners.

% almost certainly you want these
\usepackage{amssymb}
\usepackage{amsmath}
\usepackage{amsfonts}

% used for TeXing text within eps files
%\usepackage{psfrag}
% need this for including graphics (\includegraphics)
%\usepackage{graphicx}
% for neatly defining theorems and propositions
\usepackage{amsthm}
% making logically defined graphics
%%%\usepackage{xypic}

% there are many more packages, add them here as you need them

% define commands here
\theoremstyle{theorem}
\newtheorem*{thm}{Theorem}
\newtheorem*{lemma}{Lemma}
\newtheorem*{conj}{Conjecture}
\newtheorem*{cor}{Corollary}
\newtheorem*{example}{Example}
\newtheorem*{prop}{Proposition}
\theoremstyle{definition}
\newtheorem*{defn}{Definition}
\theoremstyle{remark}
\newtheorem*{rmk}{Remark}

\begin{document}
Suppose $T(x)$ denotes the formal power series 
$\sum_{\alpha} a_\alpha x^\alpha ,$ using the multi-index notation,
where $x = (x_1,\ldots,x_N)$ and $a_\alpha \in \mathbb{C}.$
Fixing $v \in {\mathbb{R}}^N$ and we can also talk of the formal power series in $t \in \mathbb{R}$
\begin{equation*}
\begin{split}
T(tv) & = \sum_{\alpha} a_\alpha (tv)^\alpha \\
 & = \sum_{\alpha} a_\alpha v^\alpha t^{\lvert \alpha \rvert} \\
 & = \sum_{k=0}^\infty \left( \sum_{\lvert \alpha \rvert = k} a_\alpha v^\alpha \right) t^k .
\end{split}
\end{equation*}

\begin{thm}
Suppose $T(x)$ is a formal power series in $x \in {\mathbb{R}}^N$.  Suppose 
$T(tv)$ is a convergent power series in $t \in \mathbb{R}$ for
all $v \in {\mathbb{R}}^N$.  Then $T$ is convergent.
\end{thm}

The other direction, if $T(x)$ converges then $t \mapsto T(tv)$ converges, is obvious.

%\begin{proof}
%Note that we only need to show this for $v \in S^{N-1}, $ that is, the unit sphere.
%Write
%\begin{equation*}
%T(x) := \sum_{k=0}^\infty T_k(x)
%\end{equation*}
%where $T_k$ are homogeneous polynomials of degree $k.$  As $T(tv)$ converges, then we have by standard
%Cauchy estimates that
%for every $v$ there exists a constant $C_v$ such that $\lvert T_k(tv) \rvert \leq C_v^k \lvert t \rvert^k ,$
%or since $T_k$ is homogeneous of degree $k,$
%\begin{equation*}
%\lvert T_k(v) \rvert \leq C_v^k .
%\end{equation*}
%
%%FIXME: finish (need bernstein's)
%\end{proof}


\begin{thebibliography}{9}
\bibitem{ber:submanifold}
M.\@ Salah Baouendi,
Peter Ebenfelt,
Linda Preiss Rothschild.
{\em \PMlinkescapetext{Real Submanifolds in Complex Space and Their Mappings}},
Princeton University Press,
Princeton, New Jersey, 1999.
\end{thebibliography}
%%%%%
%%%%%
\end{document}
