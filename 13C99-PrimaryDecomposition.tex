\documentclass[12pt]{article}
\usepackage{pmmeta}
\pmcanonicalname{PrimaryDecomposition}
\pmcreated{2013-03-22 14:15:05}
\pmmodified{2013-03-22 14:15:05}
\pmowner{mathcam}{2727}
\pmmodifier{mathcam}{2727}
\pmtitle{primary decomposition}
\pmrecord{8}{35698}
\pmprivacy{1}
\pmauthor{mathcam}{2727}
\pmtype{Definition}
\pmcomment{trigger rebuild}
\pmclassification{msc}{13C99}
\pmsynonym{shortest primary decomposition}{PrimaryDecomposition}
\pmdefines{decomposable ideal}
\pmdefines{minimal primary decomposition}

\endmetadata

% this is the default PlanetMath preamble.  as your knowledge
% of TeX increases, you will probably want to edit this, but
% it should be fine as is for beginners.

% almost certainly you want these
\usepackage{amssymb}
\usepackage{amsmath}
\usepackage{amsfonts}
\usepackage{amsthm}

% used for TeXing text within eps files
%\usepackage{psfrag}
% need this for including graphics (\includegraphics)
%\usepackage{graphicx}
% for neatly defining theorems and propositions
%\usepackage{amsthm}
% making logically defined graphics
%%%\usepackage{xypic}

% there are many more packages, add them here as you need them

% define commands here

\newcommand{\mc}{\mathcal}
\newcommand{\mb}{\mathbb}
\newcommand{\mf}{\mathfrak}
\newcommand{\ol}{\overline}
\newcommand{\ra}{\rightarrow}
\newcommand{\la}{\leftarrow}
\newcommand{\La}{\Leftarrow}
\newcommand{\Ra}{\Rightarrow}
\newcommand{\nor}{\vartriangleleft}
\newcommand{\Gal}{\text{Gal}}
\newcommand{\GL}{\text{GL}}
\newcommand{\Z}{\mb{Z}}
\newcommand{\R}{\mb{R}}
\newcommand{\Q}{\mb{Q}}
\newcommand{\C}{\mb{C}}
\newcommand{\<}{\langle}
\renewcommand{\>}{\rangle}
\begin{document}
Let $R$ be a commutative ring and $A$ be an ideal in $R$.  A \emph{\PMlinkescapetext{primary} decomposition} of $A$ is a way of writing $A$ as a finite intersection of primary ideals:
\begin{align*}
A=\bigcap_{i=1}^n Q_i,
\end{align*}

where the $Q_i$ are primary in $R$.

Not every ideal admits a primary decomposition, so we define a \emph{decomposable ideal} to be one that does.

\textbf{Example}.  Let $R=\Z$ and take $A=(180)$.  Then $A$ is decomposable, and a primary decomposition of $A$ is given by 
\begin{align*}
A=(4)\cap (9)\cap (5),
\end{align*}
since $(4)$, $(9)$, and $(5)$ are all primary ideals in $\Z$.

Given a primary decomposition $A=\cap Q_i$, we say that the decomposition is a \emph{minimal primary decomposition} if for all $i$, the prime ideals $P_i=\text{rad}(Q_i)$ (where rad denotes the radical of an ideal) are distinct, and for all $1\leq i\leq n$, we have
\begin{align*}
Q_i\not\subset \bigcap_{j\neq i} Q_j
\end{align*}

In the example above, the decomposition $(4)\cap (9)\cap (5)$ of $A$ is minimal, where as $A=(2)\cap (4) \cap (3) \cap (9) \cap (5)$ is not.

Every primary decomposition can be refined to admit a minimal primary decomposition.
%%%%%
%%%%%
\end{document}
