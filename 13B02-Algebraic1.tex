\documentclass[12pt]{article}
\usepackage{pmmeta}
\pmcanonicalname{Algebraic1}
\pmcreated{2013-03-22 12:07:50}
\pmmodified{2013-03-22 12:07:50}
\pmowner{djao}{24}
\pmmodifier{djao}{24}
\pmtitle{algebraic}
\pmrecord{8}{31297}
\pmprivacy{1}
\pmauthor{djao}{24}
\pmtype{Definition}
\pmcomment{trigger rebuild}
\pmclassification{msc}{13B02}
\pmrelated{AlgebraicExtension}
\pmdefines{transcendental}

\endmetadata

\usepackage{amssymb}
\usepackage{amsmath}
\usepackage{amsfonts}
\usepackage{graphicx}
%%%\usepackage{xypic}
\begin{document}
Let $B$ be a ring with a subring $A$. An element $x \in B$ is {\em algebraic} over $A$ if there exist elements $a_1, \dots, a_n \in A$, with $a_n \neq 0$, such that
$$
a_n x^n + a_{n-1} x^{n-1} + \cdots + a_1 x + a_0 = 0.
$$
An element $x \in B$ is {\em transcendental} over $A$ if it is not algebraic.

The ring $B$ is {\em algebraic} over $A$ if every element of $B$ is algebraic over $A$.
%%%%%
%%%%%
%%%%%
\end{document}
