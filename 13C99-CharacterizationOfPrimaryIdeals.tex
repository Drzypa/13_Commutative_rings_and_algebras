\documentclass[12pt]{article}
\usepackage{pmmeta}
\pmcanonicalname{CharacterizationOfPrimaryIdeals}
\pmcreated{2013-03-22 19:04:29}
\pmmodified{2013-03-22 19:04:29}
\pmowner{joking}{16130}
\pmmodifier{joking}{16130}
\pmtitle{characterization of primary ideals}
\pmrecord{5}{41960}
\pmprivacy{1}
\pmauthor{joking}{16130}
\pmtype{Derivation}
\pmcomment{trigger rebuild}
\pmclassification{msc}{13C99}

\endmetadata

% this is the default PlanetMath preamble.  as your knowledge
% of TeX increases, you will probably want to edit this, but
% it should be fine as is for beginners.

% almost certainly you want these
\usepackage{amssymb}
\usepackage{amsmath}
\usepackage{amsfonts}

% used for TeXing text within eps files
%\usepackage{psfrag}
% need this for including graphics (\includegraphics)
%\usepackage{graphicx}
% for neatly defining theorems and propositions
%\usepackage{amsthm}
% making logically defined graphics
%%%\usepackage{xypic}

% there are many more packages, add them here as you need them

% define commands here

\begin{document}
\textbf{Proposition.} Let $R$ be a commutative ring and $I\subseteq R$ an ideal. Then $I$ is primary if and only if every zero divisor in $R/I$ is nilpotent.

\textit{Proof.} ,,$\Rightarrow$'' Assume, that we have $x\in R$ such that $x+I$ is a zero divisor in $R/I$. In particular $x+I\neq 0+I$ and there is $y\in R$, $y+I\neq 0+I$ such that
$$0+I=(x+I)(y+I)=xy+I.$$
This is if and only if $xy\in I$. Thus either $y\in I$ or $x^n\in I$ for some $n\in\mathbb{N}$. Of course $y\not\in I$, because $y+I\neq 0+I$ and thus $x^n\in I$. Therefore $x^n+I=0+I$, which means that $x+I$ is nilpotent in $R/I$.

,,$\Leftarrow$'' Assume that for some $x,y\in R$ we have $xy\in I$ and $x,y\not\in I$. Then
$$(x+I)(y+I)=xy+I=0+I,$$
so both $x+I$ and $y+I$ are zero divisors in $R/I$. By our assumption both are nilpotent, and therefore there is $n,m\in\mathbb{N}$ such that $x^n+I=y^m+I=0+I$. This shows, that $x^n\in I$ and $y^m\in I$, which completes the proof. $\square$
%%%%%
%%%%%
\end{document}
