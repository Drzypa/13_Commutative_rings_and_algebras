\documentclass[12pt]{article}
\usepackage{pmmeta}
\pmcanonicalname{ProofOfNakayamasLemma}
\pmcreated{2013-03-22 13:07:46}
\pmmodified{2013-03-22 13:07:46}
\pmowner{mclase}{549}
\pmmodifier{mclase}{549}
\pmtitle{proof of Nakayama's lemma}
\pmrecord{5}{33564}
\pmprivacy{1}
\pmauthor{mclase}{549}
\pmtype{Proof}
\pmcomment{trigger rebuild}
\pmclassification{msc}{13C99}

\endmetadata

% this is the default PlanetMath preamble.  as your knowledge
% of TeX increases, you will probably want to edit this, but
% it should be fine as is for beginners.

% almost certainly you want these
\usepackage{amssymb}
\usepackage{amsmath}
\usepackage{amsfonts}

% used for TeXing text within eps files
%\usepackage{psfrag}
% need this for including graphics (\includegraphics)
%\usepackage{graphicx}
% for neatly defining theorems and propositions
%\usepackage{amsthm}
% making logically defined graphics
%%%\usepackage{xypic}

% there are many more packages, add them here as you need them

% define commands here
\begin{document}
Let $X = \{x_1, x_2, \dots, x_n\}$ be a minimal set of generators for $M$, in the sense that $M$ is not generated by any proper subset of $X$.

Elements of $\mathfrak{a}M$ can be written as linear combinations $\sum a_i x_i$, where $a_i \in \mathfrak{a}$.

Suppose that $|X| > 0$.  Since $M = \mathfrak{a}M$, we can express $x_1$ as a such a linear combination:
$$x_1 = \sum a_i x_i.$$
Moving the term involving $a_1$ to the left, we have
$$(1 - a_1)x_1 = \sum_{i > 1} a_i x_i.$$
But $a_1 \in J(R)$, so $1-a_1$ is invertible, say with inverse $b$.
Therefore,
$$x_1 = \sum_{i > 1} b a_i x_i.$$
But this means that $x_1$ is redundant as a generator of $M$, and so $M$ is generated by the subset $\{x_2, x_3, \dots, x_n\}$.  This contradicts the minimality of $X$.

We conclude that $|X| = 0$ and therefore $M = 0$.
%%%%%
%%%%%
\end{document}
